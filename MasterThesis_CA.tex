\documentclass[12pt,titlepage]{article}
\usepackage[ngerman]{babel}
\usepackage[utf8]{inputenc}
\usepackage{color}
\usepackage[a4paper,lmargin={4cm},rmargin={2cm},
tmargin={2.5cm},bmargin = {2.5cm}]{geometry}
\usepackage{amssymb}
\usepackage{amsthm}
\usepackage{graphicx}
% Makros für allgemein nützliche Dinge wie Korrekturfeedback :-)
% Dies hier auskommentieren, falls du in der Ansicht keine Kommentare sehen willst. 
%\newcommand{\dwi}[1]{}
% Dies hier mit % kommentieren, falls du in der Ansicht keine Kommentare sehen willst. 
\newcommand{\dwi}[1]{\textcolor{red}{\footnotesize#1}}

% Tests..
%\newcommand{\dwi}[1]{\marginpar{\textcolor{red}{\tiny#1}}}

\begin{document}
\title{DevOps meets OSS: Automatisierte Integration von OSS in die DevOps-Produktentwicklung\\}
\author{Claudia Arnold}

\maketitle

\begin{abstract}
   Unter dem Begriff OSS bekannt, bietet die Open Source Software eine Reihe an Möglichkeiten im Bereich der Softwareentwicklung. Durch die sinkenden Nutzungesbescchränkungen im Vergleich zur konventioneller Software gilt OSS als ein Instrument zur schnellen Integration von Software in die eigene Entwicklung.  
\end{abstract}

\newpage  
\begin{center}  
    \vspace*{\fill}
    \textit{‘Free software’ is a matter of liberty, not price. To understand the concept, you should think of ‘free’ as in ‘free speech,’ not as in ‘free beer’.}
    \newline
    \newline - Richard Matthew Stallman
    \vspace*{\fill}
\end{center}

\newpage
\section{Einleitung}
\begin{quote}
\textit{In real open source, you have the right to control your own destiny}\newline -Linus Torvalds
\end{quote}

Verändertes Konsumentenverhalten, schnelle Reaktion auf veränderte Kundenwuensche und Time-to-Market sind wesentliche Anforderungen, die für den wirtschaftlichen Erfolg eines Unternehmens insbesondere in Zeiten der Digitalisierung kennzeichnet sind und daher große Herausforderungen mit sich bringen. git staDie daraus resultierenden steigenden Anforderungen an neuer Funktionalität und die damit verbundene stetige Weiterentwicklung erfordern in erster Linie eine neue Dynamik innerhalb der Entwicklug, Konzepte und Tools. In diesem Zusammenhang wird der Softwareentwicklungsprozess nach agilen Methoden unablässig und gehört in den meisten Unternehmen bereits zum Standard einer IT-Organisation. 

In diesem Rahmen kommt der Ansatz des DevOps zum Tragen, nach dessen Grundsätzen der Softwareentwicklungsprozess, angefangen von der Idee, über die schnelle Entwicklung bishin zum Einsatz innerhalb der Produktivumgebung, durchgeführt wird. DevOps ist ein Kofferwort aus den Begriffen Development und IT-Operations und beschreibt im Wesentlichen die technische und organisatorische Verbindung zwischen der Softwareentwicklung und der Systemadministration. Charakterisch sind verkürzte Releasezyklen in einem inkrementellen Prozess und ein hoher Automatisierungsgrad, wodurch eine exakte Planung, Vermeidung von kritischen Fehlern und eine schnelle Reaktion auf Kundenanforderungen ermöglicht wird. Wesentliches Ziel von DevOps ist es, die gesamte Entwicklung transparent, flexibel und agil zu gestalten, wodurch die Produktivität und Effizenz maßgeblich gesteigert und das endgültige Produkt schneller zum Kunden ausgeliefert werden kann. 

Neben der agilen Softwareentwicklung ist ein weiteres wesentliches Kennzeichen des DevOps-Konzeptes die Grundsätze des Lean Manufactoring, nach denen die Minimierung von Verschwendung und die Maximierung der Produktivität im Vordergrund steht. Dabei spielt insbesondere die Wiederverwendung von Software eine wesentliche Rolle. Anstatt einer zeit- und kostenintensiven Neuentwicklung von Software, bleiben viele allgemeine oder wiederholende Funktionalitäten innerhalb eines neuen Projektes oftmals gleich und müssen demnach nicht verändert werden. Vor diesem Hintergrund sparen Entwickler viel Zeit und Ressourcen, indem Quellcode, Templates oder Algorithmen wiederverwendet werden, was wiederrum dem Grundgedanken von DevOps entspricht. 

Ausgehend hiervon ist die Integration von bestehenden Softwarekonzepten ein wesentlicher Vorteil, wodurch die Verwendung von Open Source Software zunehmend an Bedeutung im Unternehmensfeld gewinnt. Als eine realistische Alternative zu herkömlicher Software, erlangte die Open-Source-Bewegung mit der Idee von frei zugänglicher Software in den letzten Jahrzehnten immer mehr an Popularität. Unter Open Source Software versteht man einen öffentlich zugänglichen Quellcode, den jeder einsehen, verändern und für sich nutzen kann. Im Gegensatz zur herkömlicher proprietärer Software, erhalten Nutzer eine Open Source Software meist kostenlos und ohne weitere Kosten für Lizenzvereinbarungen. Durch den unmittelbaren Zugang zum Quellcode, haben Entwickler die Möglichkeit einerseits, am Quellcode mit nur wenigen Ressourcen zu experimentieren und zu überprüfen, ob sich dieser ein wiederkehrendes Problem innerhalb des Projektes beinhaltet oder anderseits nach der jeweligen Problematik zu individualisieren und zu verändern. Die Notwendigkeit einer kosten- und zeitspielige Entwicklung entfällt und die Verwendung von bereits gelösten Paradigmen rückt in den Vordergrund.   

% Häufige Vorgehensweise: Moduale Architektur von Open-Sorce projekten
% Beispiele an OSS aufzeigen: Betriebssysteme, Server-Anwendungen oder Office-Produkte
% Evtl hier nochmal auf Linux, als erstes populäres Produkt im OSS-Bereich eingehen 

Das daraus resultierende Innovationspotenzial kann, basierend auf der freien Gestaltungsfreiheit die OSS im Gegensatz zur kommentieller Lizenzsoftware ermöglicht, genutzt werden, Ideen umzusetzen ohne zu stark an unternehmensinterne Vorgaben gebunden zu sein oder unternehmensinterne Prozesse oder Prozessabläufe in die Entwicklung einzubinden. Vor diesem Hintergrund kann eine umfassende Integration von Open Source Software in vorhandene Unternehmenstrukturen im Bereich der Entwicklung, ausschließlich in einem agilen Umfeld durchgeführt werden, womit DevOps als ideale Plattform geeignet ist.  

Trotz der Freiheiten, die Open Source Software offensichtlich bietet, unterliegen die meisten Projekte rechtlichen Schutzmaßnahmen, einschließlich dem Marken- Patent- und Urheberrecht. Aufgrund dessen können, aus der Verwendung von OSS resultierende Lizenzfragen oder rechtliche Konsequenzen, zu einer maßgeblichen Einschränkung insbesondere im Hinblick auf eine mögliche Verteilung oder Weitergabe an Dritte führen, die von Unternehmen genau analysiert und überprüft werden muss.  

In dieser Thesis werden die grundsätzlichen Vorteile und Risiken, die sich aus den verschiedenen Lizenzvereinbarungen bei der Verwendung von OSS ergeben analysiert und die Migration in den Devops-Emntwicklungsprozeess dargestellt. Verdeutlicht wird die Themenstellung anhand der ausgeführten Analysen anahnd eines Beispiels bei der msg Systems. 




Durch die freie Verfügbarkeit des Quellcodes, die entfallenen Lizenzkosten für den Einsatz und die freie Weiterentwicklungsmöglichkeiten, erweist sich OSS als eine einfache und kostengünstige Innovationsquelle für Unternehmen. 

Dabei reicht das Anwendungsfeld für die Verwendung von OSS in diesem Zusammenhang von der Automobilindustrie bis zur Versicherungsbranche und steht daher im direkten Wettbewerb zu proprietären Softwareangeboten. 

\subsection{Problemstellung}
Laut einer aktuellen Studie des Bundesverband Informationswirtschaft, Telekommunikation und neue Medien e. V. (Bitkom) wird der Einsatz von Open Source Software in Unternehmen, im Jahre 2019 bei einer Unternehmensgröße von über 2.000 Mitarbeitern, bereits auf 86 Prozent geschätzt. Dabei reicht das Anwendungsfeld von der Automobilindustrie bis hin zu Banken und Versicherungen und steht daher im direkten Wettbewerb zu namenhaften proprietaeren Softwareangeboten. Ausgehend hiervon gehört die Etablierung von Open Source Software in die Softwareentwicklung bereits zur gängigen IT-Unternehmenskultur.  

Durch die freie Verfügbarkeit des Quellcodes, kann die entsprechende Software schnell und einfach eingebunden werden. 



Open-Source-Software gilt als zuverlässig und sicher. Für manche ist sie sogar ein Innovationsmotor mit ungeheurem Potenzial. Neben der Vielfalt der Anwendungsszenarien steht die Fülle 
der Anwendungen. Freie Software – ein gängiges anderes Etikett für Open-Source-Software – 
hat echte wirtschaftliche Bedeutung gewonnen. Die Serverinfrastruktur einiger großer Internetfirmen basiert zum größten Teil auf Linux. Google, Facebook oder Amazon sind bekennende 
Nutzer. Die Gründe dafür sind simpel: Unternehmen müssen für den Einsatz von Open-SourceSoftware keine Lizenzkosten einkalkulieren. Das vereinfacht den Aufbau skalierender Umgebungen signifikant. Außerdem können die Unternehmen die genutzte freie Software im Rahmen 
der jeweiligen Lizenzbestimmungen entsprechend eigener Zwecke und Ziele weiterentwickeln. 
Selbst wenn die modifizierte Version danach geschäftskritische Elemente enthält – und der 
wirklich geschäftskritische Anteil eines Softwaresystems ist in der Regel sehr klein –, gibt es 
immer noch gute Einsatzszenarien, die deren Schutz in Verbindung mit Open-Source-Software 
gewährleisten.


so selbstverständlich die Verwendung ist, so verbreitet sind 
oft auch Wissenslücken über Grundanforderungen im Umgang mit OSS, sei es als Basis für einen 
erfolgreichen Einsatz im Alltag oder als Voraussetzung für die Verwendung für eigene Softwareprojekte. Ein gängiges Missverständnis besteht z.B. darin, dass aus der einfachen und unentgeltlichen Verfügbarkeit von OSS auf das Fehlen jeglicher juristischen Einbettung geschlossen wird. 

Grundsätzlich unterliegt Software dem Urheberrecht. Daher 

Durch die freie Verfügbarkeit des Quellcodes, kann die entsprechende Software schnell und einfach eingebunden werden. 


Die implizierte Freiheit, die sich bei der Verwendung von OSS ergibt, beschränkt sich dementsprechend auf die Einsicht, Nutzung, Modifikation und Distribustion des Quellcodes.    

jedoch nicht uneingeschränkt lizenzfrei genutzt werden. 



Mittels einer Lizenzierung der entsprechenden Software soll sichergestellt werden, dass die Rechte zur Nutzung der Open Source Software nur mit Genehmigung des Urhebers geschützt bleiben. Ab dem Zeitpunkt der Weitergabe von veränderter Software, müssen entsprechenden Lizenzmodelle berücksichtigt werden, um vorrangig das geistige Eigentum des Urhebers zu schützen.  

Bereits bei groben Veränderungen im Laufe des Projektes, hat die Veränderungen von bestimmten SChutzmaßnahmen graviernede Auswirkungen auf das DevOps-Team. Viele erhalten nicht die benötigten Informationen, dass eine Veränderung vorliegt, die das Projekt maßgeblich beeinflußen könnte. 

Demgegenüber haben Urherber bei einer Verletzung an ihrer Open Source Software, die sich aus den einzelnen Lizenzmodellen ergibt, die Möglichkeit rechtliche Ansprüche geltend zu machen. Die Folge sind Unterlassungsansprüchen und kostenspieliger Aufwendungsersatz. Die Konsequenz wäre eine sofortige Einstellung der bearbeiteten Software und möglicherweise eine neue Entwicklung, der Funktionalitäten ohne das jeweilige Open Source als Basis./ Nutzer verliehrt das Recht die Software weiterhin zu nutzen.  


Eine
Lizenz (v lat: licere = erlauben) ist eine Erlaubnis oder Genehmigung zur Nutzung
eines Rechts durch den Urheber oder Inhaber dieses Rechts11. Im juristischen
Bereich stellt die Lizenz einen Vertrag dar, durch welchen einfache oder
ausschließliche Rechte eingeräumt werden können. 


Software ist in der Regel urheberrechtlich geschützt. Dies gilt auch für Freie Software oder Open-Source (OSS). Die für proprietäre Software eingeräumten Softwarelizenzen sind dahingehend ausgerichtet, die Freiheit der Nutzung, Verbreitung und Veränderung der Software einzuschränken. Der Softwarehersteller kann so durchsetzen, eine angemessene Vergütung für die Entwicklungsleistungen zu erhalten, Art und Umfang der Werknutzung zu bestimmen und die Software vor ungewollter Veränderung zu schützen. 


1. Freie Weitergabe
Die Lizenz darf niemanden in seinem Recht einschränken, die Software als Teil eines Software-Paketes, das Programme unterschiedlichen Ursprungs enthält, zu verschenken oder zu verkaufen. Die Lizenz darf für den Fall eines solchen Verkaufs keine Lizenz- oder sonstigen Gebühren festschreiben.

2. Quellcode
Das Programm muss den Quellcode beinhalten. Die Weitergabe muss sowohl für den Quellcode, als auch für die kompilierte Form zulässig sein. Wenn das Programm in irgendeiner Form ohne Quellcode weitergegeben wird, so muss es eine allgemein bekannte Möglichkeit geben, den Quellcode zum Selbstkostenpreis zu bekommen, vorzugsweise als gebührenfreien Download aus dem Internet. Der Quellcode soll die Form eines Programms haben, das ein Programmierer vorzugsweise bearbeitet. Ein absichtlich unverständlich geschriebener Quellcode ist daher nicht zulässig. Zwischenformen des Codes, so wie sie etwa ein Präprozessor oder ein Konverter („Translator”) erzeugt, sind unzulässig.

3. Abgeleitete Software
Die Lizenz muss Veränderungen und Derivate zulassen. Außerdem muss sie es zulassen, dass die solcher Art entstandenen Programme unter denselben Lizenzbestimmungen weiter vertrieben werden können wie die Ausgangssoftware.

4. Unversehrtheit des Quellcodes des Autors
Die Lizenz darf die Möglichkeit, den Quellcode in veränderter Form weiterzugeben, nur dann einschränken, wenn sie vorsieht, dass zusammen mit dem Quellcode so genannte „Patch files” weitergegeben werden dürfen, die den Programmcode bei der Kompilierung verändern. Die Lizenz muss die Weitergabe von Software, die aus einem veränderten Quellcode entstanden ist, ausdrücklich erlauben. Die Lizenz kann verlangen, dass die abgeleiteten Programme einen anderen Namen oder eine andere Versionsnummer als die Ausgangssoftware tragen.

5. Keine Diskriminierung von Personen oder Gruppen
Die Lizenz darf niemanden benachteiligen.

6. Keine Einschränkungen bezüglich des Einsatzfeldes
Die Lizenz darf niemanden daran hindern, das Programm in einem bestimmten Bereich einzusetzen. Beispielsweise darf sie den Einsatz des Programms in einem Geschäft oder in der Genforschung nicht ausschließen.

7. Weitergabe der Lizenz
Die Rechte an einem Programm müssen auf alle Personen übergehen, die diese Software erhalten, ohne dass für diese die Notwendigkeit besteht, eine eigene, zusätzliche Lizenz zu erwerben.

8. Die Lizenz darf nicht auf ein bestimmtes Produktpaket beschränkt sein
Die Rechte an dem Programm dürfen nicht davon abhängig sein, ob das Programm Teil eines bestimmten Software-Paketes ist. Wenn das Programm aus dem Paket herausgenommen und im Rahmen der zu diesem Programm gehörenden Lizenz benutzt oder weitergegeben wird, so sollen alle Personen, die dieses Programm dann erhalten, alle Rechte daran haben, die auch in Verbindung mit dem ursprünglichen Software-Paket gewährt wurden.

9. Die Lizenz darf die Weitergabe zusammen mit anderer Software nicht einschränken
Die Lizenz darf keine Einschränkungen enthalten bezüglich anderer Software, die zusammen mit der lizenzierten Software weitergegeben wird. So darf die Lizenz z. B. nicht verlangen, dass alle anderen Programme, die auf dem gleichen Medium weitergegeben werden, auch quelloffen sein müssen.


Um zu gewährleisten, dass EntwicklerInnen, die mit GNU-Software arbeiten, den Wissenspool ebenfalls nähren, verpflichtet die Lizenz ihre NutzerInnen zur Weitergabe der Derivate unter gleichen Bedingungen (Copyleft). Die Idee freier Lizenzen geht also auf die Software-Entwicklung zurück, die durch freien Code kollaborative Arbeiten ermöglichen wollte. 


Alle Open Source Lizenzen haben die Einräumung eines Vervielfältigungs- und Verarbeitungsrechts gemeinsam, welches meist bestimmten Voraussetzungen bzw. Beschränkungen unterliegt. Die einzelnen Lizenzen unterscheiden sich vor allem hinsichtlich der Nutzungsbedingungen und der Verpflichtungen, die dem Lizenznehmer zur Wahrnehmung des Vervielfältigungsrechts und des Verarbeitungsrechts aufgegeben werden.

Wichtigster Gesichtspunkt ist, welche Anforderungen an die Weiterverbreitung von veränderten Versionen der Software bzw. neuer Software gestellt werden, die auf Grundlage von Open Source Software entwickelt wurde.



Von Beginn an haben sich verschiedene Lizenzmodelle für OSS herausgebildet, die sich hinsichtlich der Nutzungsbedingungen teilweise recht deutlich unterscheiden. Zur Sichtung, Sammlung 
und Ordnung der vielfältigen Lizenzmodelle hat sich die Open Source Initiative (OSI) gegründet. 
Die OSI hat Kriterien aufgestellt, nach denen sie eine Lizenz als Open-Source-Lizenz klassifiziert 
und in die offizielle Liste der Open-Source-Lizenzen aufnimmt. Die Klassifikation einer Softwarelizenz als Open- Source-Lizenz durch die OSI erfolgt anhand der Rechte, die die Lizenz dem Nutzer 
einer so lizenzierten Software einräumt. Sie erfolgt nicht anhand der Auflagen, die die OpenSource-Lizenz dem Nutzer auferlegt. 
Für eine Einteilung der Open-Source-Lizenzen anhand der Pflichten haben sich die Kategorien 
»permissive Lizenzen«, Lizenzen mit schwachem »Copyleft« und‚ Lizenzen mit starkem 
»Copyleft« eingebürgert. Einige OSS-Lizenzen verpflichten dazu, bei Weitergabe der Software 
den Quellcode zugänglich zu machen. Damit ist manchmal das sog. Copyleft verbunden: 
Danach darf ein Nutzer eine von ihm veränderte Open-Source-Software nur zu den für die 
ursprüngliche Open-Source-Software geltenden Lizenzbedingungen an Dritte weitergeben. 
Die permissiven Lizenzen stellen solche Anforderungen nicht. Die Einteilung in permissive 
Lizenzen und Lizenzen mit starkem und schwachem Copyleft bietet nur eine erste Orientierung. 
Was konkret zu tun ist, um Open-Source-Software lizenzgemäß zu nutzen, ergibt sich stets erst 
aus der konkreten Open-Source-Lizenz. Die Lizenzbedingungen sind natürlich auch bei einer wirtschaftlichen Nutzung von Open-SourceSoftware zu beachten. Daher werden Unternehmen, die Leistungen im Zusammenhang mit OSS 
anbieten oder OSS in eigenen Produkten nutzen, den Umfang ihrer Nutzungsrechte anhand der 
konkreten Softwarelizenz klären, die mit dem Vertrieb verbundenen Risiken abschätzen und ihre 
Entwicklungs- und Vermarktungsstrategie in technischer und rechtlicher Hinsicht darauf ausrichten müssen. In diesem Kontext ist besonders hervorzuheben, dass Open-Source-Software – 
entgegen landläufigen Vorurteilen – sehr wohl kommerziell eingesetzt und vertrieben werden 
darf. Einzig für die Nutzung der Software selbst dürfen keine Lizenzgebühren verlangt werden. 
Unternehmen können jedoch neue Geschäftsmodelle mit und um Open-Source-Software herum 
aufbauen, die nicht auf das traditionelle Lizenzsoftwaregeschäft zurückgreifen. Andererseits sind Unternehmen bei der Einbettung von fremderstellter Open-Source-Software in 
eigene Produkte vor Herausforderungen gestellt, die über die bloße Lizenzerfüllung hinausgehen. 
So müssen sie etwa überlegen, inwieweit sie eine Mängelhaftung für die fremderstellte OSS 
übernehmen können, wenn eine Fehlerbehebung nicht von den Urhebern der Software oder 
der Open-Source-Community angeboten wird. Zu solchen Herausforderungen gehört auch die 
Analyse, ob und wann ein Unternehmen geschäftskritische Alleinstellungsmerkmale in eine 
adaptierte Open-Source-Software hineinprogrammiert, deren Lizenz die Offenlegung des Codes 
im Falle einer Weitergabe des Programms an Kunden vorsieht. Glücklicherweise sind die 
wenigsten Verbesserungen an einem Open-Source-Code wirklich geschäftskritisch. Und viele 
Lizenzen verlangen auch keine solche Offenlegung. Um solche Herausforderungen in den Griff zu bekommen, ist zunächst die Erfassung der im 
Unternehmen und in seinen Produkten verwendeten Open-Source-Software erforderlich. 
Daraus wird sich eine unternehmenseigene Steuerung und Kontrolle der Open-Source-Software 
entwickeln. Hierzu sind ein technisches Software-Management und ein rechtliches Lizenzmanagement durchaus hilfreich. Mehr noch: Die Einrichtung entsprechender Strukturen gehört 
letztlich zu den unternehmerischen Organisationspflichten. Darüber hinaus kann eine 
differenzierte Vertragsgestaltung bei Beschaffung und Vertrieb von Open-Source-Software 
helfen. Bei der Gestaltung von Verträgen über Leistungen im Zusammenhang mit Open-SourceSoftware (etwa Implementierung, Anpassung, Zusatzprogrammierung etc.) und bei der 
Einrichtung eines Lizenzmanagements ist die Beteiligung fachkundiger unternehmensinterner 
oder externer Rechtsberater empfehlenswert.

Trotz der Freiheiten, die Open Source Software offensichtlich bietet, unterliegen die meisten Projekte rechtlichen Schutzmaßnahmen, einschließlich dem Marken- Patent- und Urheberrecht. Aufgrund dessen können, aus der Verwendung von OSS resultierende Lizenzfragen oder rechtliche Konsequenzen, zu einer maßgeblichen Einschränkung insbesondere im Hinblick auf eine mögliche Verteilung oder Weitergabe an Dritte führen, die von Unternehmen genau analysiert und überprüft werden muss.  

In dieser Thesis werden die grundsätzlichen Vorteile und Risiken, die sich aus den verschiedenen Lizenzvereinbarungen bei der Verwendung von OSS ergeben analysiert und die Migration in den Devops-Emntwicklungsprozeess dargestellt. Verdeutlicht wird die Themenstellung anhand der ausgeführten Analysen anahnd eines Beispiels bei der msg Systems.

In diesem Abschnitt wird beschrieben, welche derzeitige Problemstellung innerhalb der DevOps Produktentwicklung bei der Integration von FOSS vorliegt, die mittels der Thesis gelöst bzw verbessert werden kann. Insbesondere wird dabei auf die derzeitige Ist-Situation eingegangen, welche Herausforderungen momentan vorliegen (bei msg und allgemein) und wie der aktuelle Stand der Wissenschaft in diesem Kontext ist. 
Zudem wird eine Herleitung zu den Ziel und den Forschungsfragen dieser Thesis hergestellt.  

\subsection{Ziel und Fragestellung}
In diesem Abschnitt wird das konkrete Ziel der Thesis festgelegt und die Forschungsfragen, an denen sich die Thesis ausrichtet und zu beantworten. Die Forschungsfragen wurden zunächst folgendermaßen festgelegt: \\


1. Welche relevanten Lizenzinformationen können aus FOSS automatisiert extrahiert und verwendet werden? (Lizenzmodelle)\\


2. Welche Einschränkungen sind bei der Verwendung von FOSS in Hinblick auf die verschiedenen Lizenzinformationen zu berücksichtigen? (Haftungs-/Garantiebedingungen, Autorenhinweise, Copyleft-Bedinungen)\\


3. Welche prozessualen Anpassungen müssen anhand der gewonnenen Erkenntnisse getroffen werden, um auf mögliche Veränderungen in den Lizenzinformationen reaktionsschnell vorbereitet zu sein?\\ 


4. Wie können relevante Lizenzinformationen und haftungsbedingte Anforderungen anhand eines Produktes automatisch aus den verwendeten Lizenzmodellen aus FOSS herausgearbeitet werden? (Programm, Makros ???)\\


5. Wie kann die Transparenz der gesammelten Informationen insgesamt sichergestellt werden? (Wie kommen die Informationen schnell an das Devops-Team)\\ 


\subsection{Aufbau dieser Arbeit}
In diesem Abschnitt wird die Methodik, das Vorgehen und die Gliederung der Thesis beschrieben
\dwi{ich glaube du brauchst den aufbau der arbeit nicht beschreiben. das ergibt sich implizit. falls du irgendwo die konkrete
kapitelstruktur erläutern willst, wäre das auf hauptkapitelebene am ende der einleitung angemessen.
du hast keinen experimentierteil, daher glaube ich nicht das sich ein extra "methodik" abschnitt lohnt - oder überzeug mich?}

\section{Fachlicher Rahmen}
In diesem Abschnitt wird der fachliche Rahmen dieser Arbeit beschrieben. Zunächst wird in Kapitel 2.1 auf die Grundlagen und die Methoden von DevOps eingegangen. In diesem Rahmen werden unter anderem die Vorteile und das wesentliche Toolchain von DevOps näher erläutert. Im nächsten Teil dieses Kapitels wird OSS geschildert. Dabei werden insbesondere die Grundlagen, lizenzrechtliche Gegebenheiten und die juristischen Konsequenzen, bei einer Verletzung der Schutzrechte, ausführlich behandelt.

\subsection{DevOps}
Wie bereits beschrieben, handelt es sich bei dem Begriff DevOps um ein zusammengesetztes Wort aus den Begriffen Development und IT-Operations. 

\subsubsection{Der Begriff DevOps}
Trotz einer ausgeprägten Zielsetzung und definierten Arbeitsweisen gibt es keine allgemein akzeptierte Definition des Begriffs DevOps.



\subsubsection{Grundlagen}
Heutzutage müssen insbesondere Unternehmen in der Lage sein, einerseits auf veränderte Marktbedingungen schnell zu reagieren und anderseits stabile und qualitativ hochwertige Systeme zu integrieren und diese gleichzeitig zuverlässig zu unterhalten. \cite{humble_why_2011} 

Anhand des DevOps-Ansatzes wird eine Kultur oder Umgebung etabliert, durch die das Erstellen, Testen und Freigeben von Software schnell, häufig und zuverlässiger erfolgen kann. \cite[S.xxviii]{sharma_devops_2017}

In diesem Zusammenhang sollen Ineffizienzen in den Softwareentwicklungs-, Release- und Betriebsprozessen vermieden werden, die durch die organisatorische Trennung zwischen den Prozessen \cite{lwakatare_devops_2019} oder durch die Fehlkommunikation zwischen den Teammitgliedern\cite{ebert_devops_2016} verursacht werden. 

Die daraus resultierenden Vorteile, die sich bei der Verwendung von DevOps ergeben, reichen von einer schelleren Produktbereitstellung und Problemlösung, bessere Ressourcenauslastung und Automatisierung bishin zu einer stabileren Betriebsumgebung.    

Insgesamt soll das Ziel verfolgt werden, die Softwarebereitstellung kontinuierlich sicherzustellen um so reaktionschnell auf Veränderungen am Markt oder Kundenanforderungen reagieren zu können.

Gründe für die Einführung von DevOps können vielschichtig sein.

Wie bereits beschrieben, ist es für Teams innerhalb der Entwicklung nicht möglich, neue Softwareversionen freizugeben oder Softwareänderungen schnell vorzunehmen, wenn der Betrieb die jeweiligen Funktionen nur langsam bereitstellen kann. \cite[S. 7,8]{sharma_devops_2017} 

Die Folgen sind eine verspätete Bereitstellung von Releases, Fehler in den Releases oder eine fehlende Dokumentation. \cite[S. 24]{alt_innovationsorientiertes_2017}

Hinzu kommen Probleme wie mangelndes Knowhow von Entwicklen über die Betriebnahme der verwendeten System, mangelndes Vertrauen in den Ops-Bereich bei fehlender Stabilität der Systeme, verzögertes Testen an Funktionalitäten oder Verschwendung durch mangelnde Wiederverwendung von Quellcode. \cite{humble_why_2011}  

Vor diesem Hintergrund stehen die Bereiche der Entwicklung und des Betriebs oft im Zielkonflikt "Agilität vs. Stabilität" und sehen sich mit verschiedenen Hindernissen konfrontiert, darunter unbefriedigende Testumgebungen und schlechter Informationsfluss. \cite{lwakatare_devops_2019} \cite[S. 8]{sharma_devops_2017}, \cite{konig_devopswelcome_2019}

Die Abbildung 1 zeigt den chronischen Konflikt, der oftmals innerhalb der IT-Organsisation herrscht, der durch fehlende Interaktion zwischen diesen Bereichen enstehen, die häufig unterschiedliche Ziele und Prozesse verfolgen. \cite[S. 349 - 350]{kim_devops-handbuch_2017}

\begin{figure}[h]
    \centering
    \includegraphics[scale=0.6]{Bilder/Core Conflict Clouds}
    \caption{Zentraler chronischer Konflikt nach Gene Kim \cite[S. 349]{kim_devops-handbuch_2017}}
\end{figure}

Mittels DevOps sollen die Interessen aller an der Bereitstellung von Software Beteiligten ausglichen werden, mit besonderem Schwerpunkt auf Entwicklern, Testern und Betriebspersonal. \cite{humble_why_2011}

Die querschnittlich aufgestellten Teams, lösen sich aus abgegrenzten Organisationseinheiten und Verantwortungsbreichen, aus den sogenannten 'Silos' und treiben gemeinsame Ergebnisse durch eine effektive Zusammenarbeit voran. \cite[S.5]{halstenberg_devops_2020} \cite{sollner_devops_2017}

Generell lässt sich die Werte von DevOps in dem bekannten Akronym CALMS zusammenfassen: Kultur (Culture), Automatisierung (Automation), Schlankheit (Lean), Messung (Measurement) und Teilen (Sharing).  

Der Aspekt der Kultur beinhaltet zunächst die Ausrichtung nach dem Menschen.

Hierbei spielt die Kollaboration als ein funktionsübergreifendes Team und die Orientierung nach Kundenwünschen eine tragende Rolle für eine DevOps-Organisation. \cite[S.5]{halstenberg_devops_2020} 

Die Automatisierung von Entwicklung, Implementierung und Tests ist der Schlüssel zum Erreichen niedriger Vorlaufzeiten und damit zu schnellem Feedback. \cite{humble_why_2011}

'Lean' steht in diesem Zusammenhang für die Vermeidung von Verschwendung jedlicher Ressourcen, Wertgeneration, Transparenz und ganzheitliche Betrachtung und Optimierung von Prozessen.

Der Faktor der Messung orientiert sich an den messbaren Daten, um den Fortschritt eines Unternehmens zu begutachten. 

Die definierten und erhobenen Kennzahlen reichen dabei von den Verfügbarkeiten und den Zeiten für die Fehlerbehebung oder Codeänderungen bis zu den Zeiten für die Anforderungsänderungen. \cite[S. 7]{halstenberg_devops_2020}  

Die letzte Säule beschreibt das Teilen von Informationen, Wissen, Vorgehensweisen und Praktiken innerhalb eines oder zwischen verschiedenen Teams unterschiedlicher Abteilungen. \cite{halstenberg_devops_2020} 

Dabei soll eine Umgebung geschaffen werden, in der gegenseitige Austausch, die Kommunikation und die gemeinsame Nutzung im Vordergrund stehen.

Bei dem gesamten Vorteilen und Mehrwerten, die durch DevOps erreicht werden können, kann DevOps jedoch als kein Nullsummenspiel oder Selbstläufer gesehen werden. \cite{humble_why_2011} 

Der DevOps-Ansatz benötigt zunächst einen kulturellen Wandel, was eine große Herausforderung für viele Unternehmen darstellen kann. 

Standardisierte Herangehensweisen sind von einer Vielzahl von Unternehmen über die Jahre tiefgreifend verankert worden, wodurch Mitarbeiter ihre gewohnten Arbeitsabläufe anpassen müssten. 
 
Dies reicht von der Erlernung neuer Tools, Technologien und Methoden, Aufbau einer Kommunikation für den gegenseitigen Austausch, vollständige Automatisierung, Verschmelzung etablierter Rollen und Zuständigkeiten, Schwierigkeiten bei der Implementierung eines automatisierten Deployment-Prozesses oder die Übernahme neuer Aufgaben und Verantworlichkeiten. \cite{lwakatare_devops_2019}, \cite[S. 594 - 595]{abrahamsson_product-focused_2016}, \cite[S. 43 - 45]{halstenberg_devops_2020}

Da sich die Bedeutung von DevOps in den letzten Jahren verschoben hat und immer wieder neue Tools für DevOps auftauchen, handelt es sich bei DevOps um eine stetige Weiterentwicklung. \cite[S. 595]{abrahamsson_product-focused_2016} 

Daher gibt es keinen Standard fester Praktiken im Zusammenhang mit DevOps, wodurch sich nicht festlegen lässt, welche Praktiken für DevOps eingesetzt werden sollten. 

%Die Trennung zwischen Projekten und Betrieb ist zu einer schwerwiegenden Einschränkung geworden, einerseits für die Fähigkeit von Unternehmen, neue Funktionen schneller auf den Markt zu bringen, andererseits für die der IT, stabile und qualitativ hochwertige Systeme und Dienste zu warten. \cite{humble_why_2011} 

%Ersetzen von Softwareentwicklungsmethode durch Kultur, Bewegung oder Praxis.
%Hinzufügung des Hinweises auf Automatisierung

%In diesem Abschnitt werden die grundsächlichen Merkmale von Devops beschrieben. Zudem werden wesentliche Vorteile beschrieben, durch die die Integration von DevOps möglich sind. Des weiteren wird auf das CALMS-Modell (Culture, Automation, Lean, Measurement und Sharing) eingegangen.

%DevOps gilt nicht als ein Nullsummenspiel, indem die Bereitstellungen häufig und zuverlässig in einer stabilen Produktumgebung erreicht werden, sondern es ist ein Ansatz zur Behebung der genannten Probleme durch Kultur, Automatisierung, Schlankheit, Messung und gemeinsame Nutzung, welches auch als CALMS bekannt ist. \cite{humble_why_2011}

\subsubsection{DevOps-Kultur}
Durch die intensive Interaktion der Bereiche Entwicklung und Betrieb müssen unterschiedliche organisatorische und kulturelle Veränderungen durchgeführt werden, damit die DevOps-Kultur etabliert werden kann. 

In diesem Abschnitt wird zunächst auf die wesentlichen Bestandteile und die Ziele der Kultur eingegangen, damit die Zusammenarbeit zwischen den beiden Bereichen gelingt.

In diesem Rahmen baut DevOps auf den Hauptprinzipien, als das Model der \textit{'The Three Ways'} auf, die von dem Autor und Erfinder des DevOps-Ansatzes Gene Kim \cite{kim_devops-handbuch_2017} definiert wurden. 
\dwi{ist das glasklar belegbar, das dieser Kim der urheber/namensgeber ist das ist eine starke aussage, die ich
ansonsten weiter oben im kontext "devops geht zurück auf Gene Kim und XYZ"}

Dieses Model der drei Wege bilden die zugrunde liegenden Prinzipien von DevOps ab, indem das Verhalten und die Muster von DevOps näher beschrieben werden. \cite[S. 9 - 44]{kim_devops-handbuch_2017}, \cite{kim_three_2012}  

Zunächst bildet der erste Weg die Grundlage für DevOps ab und betont die Leistung des gesamten Systems, im Gegensatz zur Leistung eines einzelnen Teams, Silos oder Abteilungen. 

Der Fokus liegt hierbei auf einem schnellen Arbeitsfluss des gesamten Systems, der durch die IT ermöglicht wird. 

In diesem Sinne müssen sich Opimierungen auf das gesamte System auswirken und sich nicht auf einen bestimmten Bestandteil beziehen. 

Das Ergebnis einer lokalen Veränderung zeigt meist keine Wirkung und würde im schlechteren Fall eine Verschlimmerung der Gesamtsituation herbeiführen. \cite[S. 252]{tiemeyer_handbuch_2021} 

Zudem sollte eine Veränderung, die ausschließlich stromab eine Verbesserung innerhalb des Wertestroms zeigt und stromab stark erschwert, kann nicht förderlich für das gesamte System sein. \cite[S. 252]{tiemeyer_handbuch_2021}
\dwi{und stromab stark erschwert ?}

Den Anfang stellt der Kunde dar, über die Entwicklung bis hin zu den Operations.

Dabei wird das Produkt, basierend auf den identifizierten Anforderungen des Kundenanforderungen, von der Entwicklung erstellt und in den Betrieb übergeben, wo das Ergebnis dem Kunden ausgeliefert wird.\cite[S. 12]{halstenberg_devops_2020} 

Der erste Schritt greift zudem den Grundgedanken des Lean-Ansatzes auf, indem ein Fehler nicht an nachfolgede Arbeitseinheiten weitergegeben werden darf um damit einen verbesserter Arbeitsfluss aufrecht gehalten werden kann. \cite[S. 252]{tiemeyer_handbuch_2021}  

Daher sind alle Arbeiten während dieses Schrittes sichtbar und in kleinen Aufgaben aufgeteilt, die in bestimmten Intervallen ausgeführt werden. 

Zu den Zielen des Ersten Weges gehört, dass bekannte Fehler an nachfolgende Arbeitsplätze nicht weitergegeben werden, dass eine lokale Optimierung niemals zu einer globalen Verschlechterung führt und dass versucht wird, ein tiefes Verständnis des gesamten Systems zu erlangen. \cite{kim_three_2012}  

\begin{figure}[h]
    \centering
    \includegraphics[scale=0.6]{Bilder/First Way.png}
    \caption{Der erste Weg: Systemdenken \cite{kim_three_2012}}
\end{figure}

Der zweite Weg beschreibt das Gestalten von effizienten Feedbackschleifen, um einerseits Fehler schnell und frühzeitig zu erkennen und zu beheben und andererseits den Prozess durchgängig im Auge zu behalten.

Der zweite Schritt stellt den eigentlichen zentralen Ansatzes von DevOps dar, wobei das Feedback von verschiedener Natur sein kann. \cite[S. 254]{tiemeyer_handbuch_2021} 

Dies sollte sowohl auf Seiten von einzelnen Teams untereinander als auch zwischen Entwicklung und Betrieb eingebettet werden. \cite[S. 94]{ravichandran_devops_2016}

Denn je schneller Probleme oder Auswirkungen kommuniziert werden, desto besser kann eine Vorgehensweise festgelegt werden. 

Damit soll verhindert werden, dass Fehler kein zweites Mal auftreten, sich kontinuierlich aufbauen können und sich möglicherweise auf neue Aufgaben auswirken. 

Durch die erleicherte Kommunikation zwischen den einzelnen Teamitgliedern als Basis von Feedbackschleifen, führt dies zu einer erhöhten Effizienz des gesamten Teams. 

Zudem können neue oder geänderte Kundenanforderungen mittels schneller Feedbackschleifen an die Entwicklung weitergegeben werden, um möglichst zeitnah die geeigneten Maßnahmen treffen zu können.
\dwi{woher kommen die kundenanforderungen? das kann ja fachlich neu oder erkenntnisse aus dem betrieb/nutzung der anwendung sein. letzteres ist ja hier gemeint, das sollte ggf klargestellt werden.}

Entscheidend ist, dass das Hinzufügen weiterer Kontrollschritte und schwerfällige Genehmigungsprozesse innerhalb großer Arbeitssyssteme die Wahrscheinlichkeit für zukünftige Fehler maßgeblich erhöht. \cite[S. 31]{kim_devops-handbuch_2017} 

Daher müssen DevOps-Teams die Freiheit haben, schnell Fehler zu beseitigen und ihre Arbeitsweise ohne organisatorische Hürden durchführen zu können.   

Die Ergebnisse des zweiten Wegs sind einerseits die Sicherstellung einer verbesserten Qualität und die Sicherheit des Arbeitssystems und andererseits die Möglichkeit des Aufbaus neuen Wissens.

\begin{figure}[h]
    \centering
    \includegraphics[scale=0.6]{Bilder/Second Way.png}
    \caption{Der zweite Weg: Feedback-Schleifen verstärken \cite{kim_three_2012}}
\end{figure}

Während sich der erste Weg technische und der zweite Weg organisatorische Aspekte beinhaltet, zielt der dritte Weg auf die Etablierung kultureller Änderungen ab um Innovationen weiter voranzutreiben. 

Dieser beschreibt die Schaffung einer Kultur die einen dynamischen Ansatz zum kontniuierlichen Experimentieren und firmenweites Lernen ermöglicht. 

\begin{figure}[h]
    \centering
    \includegraphics[scale=0.6]{Bilder/Third Way.png}
    \caption{Der dritte Weg: Kultur des kontniuierlichen Experimentierens und Lernens \cite{kim_three_2012}}
\end{figure}

Mittels dieses Ansatzes kann tiefer in die Materie und damit Risiken eingegangen werden, um schwierige oder versteckte Probleme identifizieren zu können. 

Infolgedessen sollen Fehler nicht nur aktiv beseitigt, sondern proaktiv danach gesucht und verbessert werden. \cite[S. 255]{tiemeyer_handbuch_2021}

Für eine erfolgreiche Durchführung des dritten Weges, sollte die etablierte Kultur auf Vertrauen beruhen. 

Je höher der Grad des Vertrauens jedes Mitarbeiters, desto wahrscheinlicher und schneller ist die Weitergabe von gewonnenen Erkenntnissen. \cite[S. 357]{kim_phoenix_2014}

Gemäß der DevOps-Kultur entfällt insbesondere die Bewertung von entdeckten Fehlern, sondern die Möglichkeit von diesen zu lernen und sich zu verbessern, rückt in den Vordergrund. 

Dieser Ansatz steigert die Motivitation jedes Mitgliedes eines DevOps-Teams und fördert das Experimentieren und daraus ableitende Eingehen eines Risikos, wodurch ein globales Lernen ermöglicht wird. 

Zu den Ergebnissen des Dritten Weges gehören die Einplanung von Zeit für die Verbesserung der täglichen Abläufe, die Schaffung von Routinen, die das Team für das Eingehen von Risiken honorieren und die Einbeziehung von Fehlern in das System, um die Belastbarkeit zu erhöhen.


%Durch die Interaktion der beiden Bereiche Development und Operations gehen auch organisatorische Änderungen mit einher, damit eine umfassende DevOps-Kultur geschaffen werden kann. In diesem Abschnitt wird insbesondere auf die Ziele dieser Kultur eingegangen, damit die Zusammenarbeit zwischen den beiden Bereichen gelingt. In diesem Rahmen wird unter anderem auf die wesentlichen Aspekte/Werte wie Kontniuierliches Lernen, Experimentieren, Ingenieurskultur, Kultur der Effektivität, Produktdenken oder die Übernahme von Verantwortung eingegangen. Auch wird der Vergleich und die Nachteile des traditionellen Silodenkens aufgegriffen. 

\subsubsection{DevOps-Lifecycle}
Nachdem die grundsätzliche Idee, Prinzipien und die etablierte Kultur von DevOps beschrieben worden sind, wird in diesem Absatz näher auf die inhaltlichen Aspekte von DevOps eingegangen.

Der DevOps-Lebenszyklus besteht aus einer aus mehreren iterativen und häufig automatisierten Workflows, die innerhalb eines größeren, iterativen und automatisierten Zyklus ausgeführt werden.

Grundsätzlich baut der Entwicklungszyklus innerhalb einer DevOps-Umgebung auf sieben Phasen auf, wie in der unteren Abbildung zu erkennen ist.

Der DevOps-Zyklus verläuft stets in einer Schleife, wodurch sowohl der Prozess als Ganzes als auch die Phasen sich untereinander durchgehend wiederholen, um eine stetige Neuimplementierung, Weiterentwicklung, Zusammenarbeit und Feedback sicherzustellen. 

Aufgrund der Iteration es Zykluses können Fehler in den einzelnen Phasen frühzeitig erkannt und beiseitigt werden und tragen zur Verbesserung der gesamten Phase bei. 

Die Abbildung zeigt die Abgrenzung zwischen den Mitgliedern des Development- und den Operationsteam.

Jede Seite trifft einerseits auf seperate und andererseits auf ständig zusammenarbeitende Teams zum Ziel die Geschwindigkeit, Zuverlässigkeit und die Qualität des Prozesses zu gewährleisten. 

\begin{figure}[h]
    \centering
    \includegraphics[scale=0.6]{Bilder/DevOps Lebenszyklus.png}
    \caption{Entwicklungszyklus innerhalb einer DevOps-Umgebung, angelehnt an \cite[S. 16]{halstenberg_devops_2020}}
\end{figure}

\paragraph{Plan}

In dieser Phase werden zunächst die wesentliche Anforderungen anhand der Bedürfnisse aller Stakeholder oder durch erhaltenes Feedback festgelegt, mit dem Ziel die Entwicklung und die Auslieferung entsprechend zu planen. \cite[s. 16]{halstenberg_devops_2020}  

Hinzu kommen die definierten Problembeschreibungen als auch der Umfang. 

Ziel ist es die entsprechenden Ressourcen und die Entwicklungen während des ganzen Zeitraums zu planen und einzuteilen. 

Darüber hinaus können Entwickler sich einen Überblick über die Verwendung von Systemen, Features, Funktionalitäten Risiken und Einschränkungen verschaffen. \cite{yarlagadda_devops_2021} 

Auch die Durchführbarkeit und Machbarkeit wird über den gesamten Zeitraum besprochen und geplant. 

Innerhalb dieser Phase kommen Methoden der agilen Softwareentwicklung wie Scrum, Kanban oder Extreme Programming zum Einsatz.

So kann mittels eines Kanban-Boards jeder Arbeitsschritt angefangen bei einem definierten Zustand bis zur Erledigung dieser Aufgabe, anhand vertikalen Lanes als mögliche Zustände, als Karte bewegt werden. \cite[S. 88]{huttermann_devops_2012}  

Durch die Planung und Koordination mittels Kanban-Board können nicht reine Entwicklungstätigkeiten, sondern auch Schwerpunkte im Hinblick auf den Betrieb zur Infrastruktur oder Wartung abgebildet werden. \cite{schaefer_devops_2017} 

In diesem Zuge können Schwachstellen und Engpässe schnell aufgedeckt werden, wodruch der gesamte Prozessfluss optimiert wird. 

Zur kurzfristigen Kontrolle können die auf Scrum basierenden Planungsgrundsätze verwendet werden, da dieses auf zeitlich begrenzte Iterationen und die Verteilung von Rollen und Zuständigkeiten setzt.

Im Rahmen von Scrum werden zunächst alle am System zu erledigenden Arbeiten im 'Release Backlog' festgehalten. 

Während der Planung für den Sprint werden Features und Funktionen aus dem Release Backlog ausgewählt und in das 'Sprint-Backlog' oder nach Prioritäten gesetzten Aufgaben aufgenommen, die im nächsten Sprint abgeschlossen werden sollen. \cite{cohen_introduction_2004}

Infoldessen können mögliche Entwicklungsaufwände auf einen zeitlichen Umfang versehen ('Timeboxing'). \cite[s. 17]{halstenberg_devops_2020}

Die Organisation des Teams findet in kleinen Sitzungen und monatlichen Sprints oder Iterationen statt. 

Wichtige Gemeinsamkeit beider Methoden ist die Arbeit nach dem Pull-Prinzip, wobei Tasks eigenständig bearbeitet werden, wenn die entsprechende Kapazität zur Verfügung steht. \cite{concas_agile_2007} 

Der Hauptunterschied zwischen beiden Methoden besteht im wesentlichen in der Zielsetzung und des zeitlichen Horizonts. 

Während Scrum sich auf die iterative Produktentwicklung konzentiert und das Produkt von Teration zu Iteration erweitert und verbessert, arbeitet die Kanban-Methode auf die kontinuierliche Verbesserung der Prozesse, kürzere Vorlaufzeiten und Vermeidung von Verschwendung, hin. \cite{concas_agile_2007}

Zudem legt Scrum den aktuellen Arbeitsfortschritt und die Planung auf die folgenden Sprints fest, während sich die Planung bei Kanban auf mehrere Wochen bezieht. \cite[s. 5]{verona_practical_2016} 

Insbesondere die messbaren und eindeutigen Vorteile von DevOps zeigen sich innerhalb kürzerer Zyklen, die wiederum die längeren Zyklen effizienter machen. 

\paragraph{Code und Build}

Nachdem die wesentlichen Aufgaben zugewiesen worden sind, stellt die hautpsächliche Akitivität innerhalb dieser Phase die Entwicklung dar. 

Störende Schritte wie beispeilsweise das Testen, die Inbetriebnahme oder die Berücksichtigung der Anfoderungen sollten in dieser Phase bereits berücksicht sein. \cite[S. 18]{halstenberg_devops_2020}  

Die Forschritte des entwickeltenden Programmcode werden laufend in Sprint Reviews, Daily Scrum Meetings kommuniziert an das gesamte DevOps-Team kommuniziert. 

Um eine einheitliche Basis zu schaffen, arbeiten alle Entwicklerteams mit gemeinsamen Tools und Plugins und legen einheitliche Vorgaben für die Qualität des Quellcodes fest. 

Nachdem die Aufgabe abgeschlossen ist, wird der neu entwickelte und lauffähige Programmcode in kleinen Bestandteilen an ein Versionsverwaltungstool ('Code Repository') übergeben.\cite[S. 18]{halstenberg_devops_2020}  

Diese Vorgehensweise wird auch als Push bezeichnet, der wiederrum einen Pull-Request auslöst. 

In diesem Zusammenhang erfolgt ein Code-Review, bei dem der Pull-Request bestätigt wird sobald der Quellcode den Anforderungen funktional entspricht.

Analog zum Pull-Request finden automatisierte Testfälle und die (ausführbare) Konfguration der Test- und Produktionsumgebung statt. \cite[S. 18]{halstenberg_devops_2020}  

Ist das Testen erfolgreich durchlaufen, kann der Quellcode übernommen werden. 

Zudem werden alle Veränderungen am Programmcode, neue Entwicklungen oder Fehleranfälligkeiten dokumentiert und für das gesamte DevOps-Team zur Verfügung gestellt.

\paragraph{Test}

Neben der Entwicklungsphase erfolgt innerhalb der DevOps-Zyklen eine intensive Testphase, die in einer eigenen Ungebung durchgeführt wird und sich vor den Phasen der Intergration und der Bereitstellung befindet. 

Während dieser Phase werden intensive Tests der Anwendungen und Funktionalitäten innerhalb einer eigenen Serverumgebung durchlaufen, die auch als 'Staging Environment' bezeichnet werden. \cite[S. 16]{verona_practical_2016} 

Dabei wird eine Umgebung geschaffen, die sich ähnlich der Produktionsvariante verhält, auf die die finale Version bereitgestellt wird und daher ein Testen unter annährend realen Bedingungen erlaubt. \cite[S. 5.2/5.6]{bass_devops_2015}

Durch die Verwendung von realen Daten sollen Funktionaliäten einerseits unter Berücksichtigung der Infrastruktur geprüft und andererseits simuliert werden, wie sich das System unabhängig von der Entwicklungsumgebung verhält. \cite[S. 16]{verona_practical_2016} 

Während die Entwicklerteams Unit- und Integrationstests durchführen, beteiligt sich der Betrieb an Integrations- und Lasttests, um die Betriebsbereitschaft zu beurteilen. \cite[S. 127]{sturm_devops_2017}  

Neben den automatisierten Tests finden darüber hinaus auch manuelle Tests statt, um Schachstellen oder Risiken in Bereichen wie Sicherheit innerhalb der Anwendung zu identifizieren und zu beseitigen.

\paragraph{Release und Deploy}

Die Phase des Release beschreibt den Prozess der Vorbereitung für die Bereitstellung des Builds an entwickelten Funktionalitäten in die produktive Umgebung. \cite[S. 20]{halstenberg_devops_2020} 

Innerhalb dieser Phase entscheidet sich welche Änderung in dem Release enthalten sein soll. 

Abhängig des Reifegrades des Release-Prozesses kann die Übergabe sowohl manuell oder automatisch erfolgen, wobei Entwickler die Möglichkeit haben, Funktionalitäten für den Kunden zu deaktivieren bis diese einsatzbereit sind. \cite{thedev_eight_2019}

In diesem Zuge können neue Versionen innerhalb fester Zeiträume erfolgen oder automatisiert, nach erfolgreicher Übergabe des Quellcodes durch die Testphase. \cite{thedev_eight_2019}

Innerhalb der Phase des Deploys findet das eigentliche Ausrollen des neuen Builds statt.  

Grundsätzlich kann dieser Schritt automatisiert durchgeführt werden, damit es keine Einschränkungen des laufenden Betriebs gibt. \cite{thedev_eight_2019}  

Bei Problemen innerhalb des Deployments, kann der letzte Stand aus der Produktivumgebung wiederhergestellt werden, indem die neue Umgebung parallel zur bestehenden Produktionsumgebung aufgebaut wird. 

Häufig kann die Phase des Deploys mit der Phase des Releases übereinstimmen, obwohl diese Phase im Wesentlichen nur die Auslieferung der getesteten Software in die Produktivumgebung beschreibt. \cite[S. 20]{halstenberg_devops_2020}   

Im Rahmen dieser Phase sind zwei grundlegende Rollen verbreitet. \cite[s. 20]{halstenberg_devops_2020} 

Zum einen trifft der Release Coordinator die Entscheidungen über den Produktiveinsatz eines Releases und überwacht den Entwicklungsfortschritt. 

Zum anderen überwacht der Site Reliability Engineer die IT-Services, die notwendigen Tools und Abhängigkeiten innerhalb der Inbetriebnahme des Releases. 

\paragraph{Operate}

Diese Phase beschäftigt sich hauptsächlich um die Wartung und den Support, nachdem die Änderungen live gegangen sind. 

In diesem Rahmen werden einerseits Lastspitzen oder Tiefpunkte anhand der aktiven Nutzerzahl automasiert und effektiv abgefangen und andererseits benötigte Ressourcen zur Verfügung gestellt, um die Produktivumgebung erfolgreich zu betreiben. \cite{thedev_eight_2019}    

Zudem kann der Kunde Feedback zu geben, welches gesammelt und ausgewertet wird, um das Nutzerverhalten der User besser zu verstehen und die zukünftige Entwicklung zu verbessern. 

Häufig wird als finale Phase, die Monitor-Phase zusätzlich in den DevOps-Lebenszyklus aufgenommen, wobei diese oftmals in die Operate-Phase mitaufgenommen werden kann.

Dabei handelt es sich neben dem Kundenfeedback, um die Erhebung von weiteren Daten wie den auftretenden Fehlern, Leistungsverhalten, Zugriffszahlen oder Kapazitäten. \cite{thedev_eight_2019}    

Die erhobenen Informationen werden an die Produktmanager und die Entwicklungsteams oder iterativ an die Planungsphase weitergeleitet, um das nächste Feature am User abzuleiten. \cite[s. 21]{halstenberg_devops_2020} 

Damit kann sichergestellt werden, dass der aktuelle Zustand einer Anwendung überwacht und verwaltet wird, um sich auf Änderungen vorzubereiten und etwaige Fehler zu beheben. \cite[S. 127]{sturm_devops_2017} \\

















 






















%Nun wurde die Idee hinter dem DevOps-Ansatz wird in den vorherigen Kapiteln beschrieben. Als nächstes soll nun näher auf die inhaltlichen Themen von DevOps eingegangen werden, sowie der DevOps Life Cycle und dessen Phasen. Damit der Life Cycle nicht zu weitgreifend wird, wird lediglich der Life Cycle, der bei der msg systems ag verwendet wird, näher beschrieben. Damit sollte der Leser ein Verständnis für die weiteren Punkte im Proof Of Concept erlangen.\\




%Gemäß Fluri u.a. \cite[S. 259 - 283]{tokarski_strategische_2018} muss ein optimaler DevOps-Prozess drei Voraussetzungen erfüllen, um erfolgreich etabliert zu werden. Erste Voraussetzung bilden zunächst die \textit{automatisierten und definierten Prozesse}. Um Prozesse optimiert und effizient zu etablieren und zu automatisieren, bedarf es zunächst einer umfassenden Definition über die Funktionsweise und den Ablauf dieses Prozesses. Hinzu kommt eine \textit{integrierte Infrastruktur(Toolchain)}. Anhand technischer Komponenten als geeignete Werkzeuge, die dem DevOps-Team zur Verfügung gestellt werden, kann die Automatisierung optimal abgebildet werden. Letzte grundsätzliche Voraussetzung stellt das \textit{Rahmenwerk von Aktivitäten für die Teams} auf. Diese Voraussetzung zielt auf die Etablierung einer DevOps-Kultur, die bereits im vorherigen Kapitel dargestellt wurde. Gemäß Fluri u.a. \cite[S. 259 - 283]{tokarski_strategische_2018} wird der gesamte Entwicklungszyklus in DevOps durch acht Schritte und vier Prozessgebiete abgebildet. 

\subsubsection{DevOps-Methoden/Softwareauslieferungsprozesses}
Während sich agile Praktiken im Hinblick auf die Softwareentwicklung auf eine kontinuierliche Planung, Flexibilität und eine schnelle Reaktion auf sich ändernde Kundenanforderungen fokussieren, können DevOps-Praktiken dazu verwendet werden, den Arbeitsfluss vom Kunden über die Entwicklung, den Betrieb und zurück kontinuierlich auszubauen und damit die Qualität und Belastbarkeit der Software zu steigern \cite{fitzgerald_continuous_2014} \cite[S. 264]{tokarski_strategische_2018}. Die \textit{'Continuous Everything'}-Methoden spiegeln die Idee einer kontinuierlichen Verbesserung und Automatisierung innerhalb eines Devops-Prozesses wieder. Wie in Abbildung 2.7 zu erkennen ist, können sich diese Methoden auf mehrere Entwicklungsphasen konzentrieren und werden in diesem Abschnitt im Einzelnen beschrieben.  

\begin{figure}[h]
    \centering
    \includegraphics[scale=0.5]{Bilder/Continuous Everything.png}
    \caption{Continuous-Methoden innerhalb des Devops-Lebenszykluses, angelehnt an \cite[S. 16]{halstenberg_devops_2020}}
\end{figure}

\subsubsection{2.1.6.1. Agile Entwicklung} $~$

Die agile Entwicklung beschreibt die Verwendung von agilen Methoden innerhalb des Softwareentwicklungsprozesses als eine wesentliche Grundvoraussetzung für den DevOps-Prozess. Oftmals wird dieser Prozess auch als Continuous Planning (dt. kontinuierliches Planen) bezeichnet und reicht von der Phase des Planes bis zur Phase des Builds \cite{fitzgerald_continuous_2014}. Ziel ist es, sicherzustellen, dass die Investitionsentscheidungen während des gesamten Lebenszyklus auf die Bedürfnisse des Kunden abgestimmt worden sind. 

\subsubsection{2.1.6.2. Continuous Integration} $~$

Die Methode des Continous Integration (dt. kontinuierliche Integration, kurz: CI) beschreibt grundsätzlich die Gewährleistung einer sicheren und lückenlosen Integration von Codeänderungen in die vorhandenen Umgebungen \cite[S. 266]{tokarski_strategische_2018}. Ziel ist es, die Qualität der Software sicherzustellen und schnelles Feedback über die Integrierbarkeit vor der Auslieferung zum Kunden zu erhalten \cite[S. 266]{tokarski_strategische_2018}. Kernelement stellt ein Versionsverwaltungssystem (auch: Repository) dar, dessen wesentliche Aufgabe es ist, den DevOps-Teams dabei zu helfen, den Code von mehreren Entwicklern zu organisieren, Änderungen zu verfolgen und automatisierte Tests zu ermöglichen. Zunächst werden neuer oder geänderter Code nach der Entwicklung und Prüfung regelmäßig und in möglichst kurzen Abständen in einem gemeinsamen Repository gemergt (dt. zusammengeführt) \cite[S. 13-16]{sharma_devops_2017}. In diesem Zuge wird der Code automatisiert in einem Build kompiliert. Die neu erstellten Artefakte, werden in eine lauffähige Umgebung integriert und automatisiert getestet um sicherzustellen, ob die neuen Codeänderungen einer Komponente innerhalb der gesamten Anwendung lauffähig sind.\\\\ Dies ist essentziell für den Prozess, da häufig viele Entwickler an der Codebasis mit leicht unterschiedlichen Versionen arbeiten und daher überprüft werden muss, ob die verschiedenen Änderungen richtig zusammenarbeiten \cite[S. 69]{verona_practical_2016}. Aufgrund des regelmäßigen Integrierens der Codeänderungen wird gewährleistet, dass häufige automatisierte Tests durchführt werden, den Entwicklern stets der aktuellste Code zur Verfügung steht und Entwickler nicht darauf warten müssen, einzelne Codeabschnitte am Tag der Veröffentlichung auf einmal zu integrieren \cite{thedev_eight_2019}. Durch die entstehende Flexibilität und Geschwindigkeit können Fehler schneller und leichter behoben werden, da die Programmbestandteile kleiner und weniger komplex sind und das Debugging insgesamt sinkt \cite{thedev_eight_2019}. Wie an der Abbildung zu erkennen ist, umfassen die CI-Schritte die Codekompilierung, die Durchführung von Unit- und Akzeptanztests, die Validierung der Codeabdeckung, die Überprüfung der Einhaltung von Codierungsstandards und die Erstellung von Bereitstellungspaketen \cite{fitzgerald_continuous_2014}. 

\subsubsection{2.1.6.3. Continuous Delivery} $~$

Bei dem Ansatz des Continuous Delivery (dt. kontinuierliches Ausliefern, kurz: CD) handelt es sich um die nächste Stufe des Continuous Integration. Die Methode der Continuous Delivery baut auf einer regelmäßigen und automatisierten Bereitstellung des Builds an den Testbereich, zur anschließenden Bewertung und einer potentiellen Freigabe, auf \cite[S. 16 - 18]{sharma_devops_2017}. Da regelmässig Builds durch die Continuous Integration erzeugt werden, müssen diese zeitnah in andere Umgebungen weitergeleitet werden \cite[S. 16 - 18]{sharma_devops_2017}. Voraussetzung ist der Aufbau einer Continuous-Delivery-Pipeline, mit dem Ziel, das Ausliefern der Software möglichst automatisiert für die Bereitstellung neuer Releases durchzuführen \cite[S. 10]{wolff_continuous_2016}. \\\\ Sobald ein neues Artefakt innerhalb des Repositorys übertragen wurde, wird die Continuous-Delivery-Pipeline ausgelöst, wie in Abbildung 7 zu sehen ist \cite[S. 14]{verona_practical_2016}. In diesem Rahmen werden die fehlerfreien Builds automatisiert innerhalb eines produktionsähnlichen Staging- oder Testbereich bereitgestellt, um nach dem Testen zu bewerten, wie sich die neu entstandene Version des Builds produktionsnah verhält und letztlich in die Produktion verlagert werden kann \cite[S. 16]{sharma_devops_2017}, \cite{thedev_eight_2019}. Falls das Testen in der Pipeline fehlschlägt, werden die Entwickler informiert und haben die Gelegenheit, kurzfristig Anpassungen an dem jeweiligen Build vorzunehmen oder diesen zu verwerfen. Die Methode des Continuous Delivery beinhaltet mehrere Vorteile für das gesamte DevOps-Team. So wird aufgrund des hohen Grades an Automatisierung der Release-Prozess maßgeblich verbessert, indem Risiken und Engpässe durch die häufige Auslieferung von kleinen Features vermieden werden können und damit ein kontinuierlicher Integrationsfluss sichergestellt werden kann \cite[S. 18]{wolff_continuous_2016}.

\subsubsection{2.1.6.4. Continuous Deployment} $~$

Die letzte Phase der Delivery-Pipeline ist das Continuous Deployment (dt. kontinuierliche Bereitstellung, kurz: CD). Kernaufgabe des Continuous Deployments ist die voll automatisierte Überführung des Codes in die Produktivumgebung mithilfe der Delivery-Pipeline \cite[S. 29]{alt_innovationsorientiertes_2017}. Aufgrund der automatisierten Freigabe des Releases, muss sowohl die Qualität als auch die Lauffähigkeit der Pipeline besonders gesichert sein und ist daher abhängig von der Phase des Continuous Delivery \cite[S. 269]{tiemeyer_handbuch_2021}. Continuous Deployment entspricht der höchsten Stufe einer Delivery-Pipeline, was den DevOps-Teams ermöglicht, kleinste Features und Änderungen automatisiert für den Anwender ausliefern zu können. Dies wäre theoretisch das höchste Ziel der voll automatisierten Softwareentwicklung \cite{humble_why_2011}. Durch diesen höchsten Grad können die Vorlaufzeiten niedrig gehalten und folglich schnelles Feedback erhalten werden \cite{humble_why_2011}. Das Ziel ist es, die Zeit bis zur Markteinführung von der Software zu verkürzen, indem jeder Commit in die Produktivumgebung bereitgestellt wird. Viele Entwickler und Unternehmen lehnen die Methode des Continuous Deployments jedoch ab, da es ein Risiko darstellt, wenn eingecheckter Code durch das Testing fehlgeschlagen ist und automatsiert in die Produktivumgebung bereitgestellt wird \cite[S. 269]{tiemeyer_handbuch_2021}. Um dieses Risiko möglichst gering zu halten, müssen einerseits nahezu produktionsreife Codes und robuste Test-Frameworks vorhanden sein und andererseits alle DevOps-Mechanismen zuverlässig arbeiten \cite[S. 269]{tiemeyer_handbuch_2021}.\\\\  Darüber hinaus verlangt der Continuous-Deployment-Ansatz eine starke Architekturaufsicht und Teamdisziplin, damit das Release nicht die Qualität oder den von den Kunden realisierten Nutzen in Mitleidenschaft zieht \cite[S. 119 - 120]{erder_continuous_2016}. Sowohl Continuous Delivery als auch Continuous Deployment werden in der Literatur oftmals synonym verwendet, da beide auf analogen Konzepten basieren. Der Unterschied zwischen beiden Methoden besteht darin, dass innerhalb des Continuous Deployment ein automatisiertes Ausliefern auf die Produktivumgebung erfolgt, während dies im Rahmen des Continuous Delivery manuell entschieden werden kann und wiederrum von den fachlichen Erfordernissen des Kunden abhängt \cite[S. 29 - 30]{alt_innovationsorientiertes_2017}. Allerdings ist, "`\textit{die Fähigkeit zur kontinuierlichen Bereitstellung wichtiger als die tatsächliche kontinuierliche Bereitstellung für die Produktion}"' \cite[S. 19]{sharma_devops_2017}. Damit schlussfolgert Sharma, dass Continuous Delivery ein Muss ist, aber Continuous Deployment als eine Option angesehen werden kann.  

\subsubsection{2.1.6.5. Continuous Monitoring} $~$

Die Vorgehensweise des Continuous Monitoring umfasst die durchgängige Überwachung, der zugrunde liegenden Infrastruktur und des im Betrieb befindlichen Quellcodes \cite{van_hoorn_continuous_2012}. Dabei stellt das Ops-Team sicher, dass die Anwendung in der Produktion funktioniert wie gewünscht und die Umgebung stabil läuft. Hierfür haben die Ops-Teams eigene Tools zur Überwachung ihrer Umgebung und laufenden Systeme, und zwar von der Prozessebene bis hinunter zu Ebenen, die niedriger sind, als es die Systemüberwachungstools erlauben würden \cite[S. 26]{sharma_devops_2017}. Oftmals werden Selbstüberwachungs- und Analysefunktionen direkt in die zu entwickelnden Anwendungen eingebaut, um eine kontinuierliche End-to-End-Überwachung zu gewährleisten \cite[S. 26]{sharma_devops_2017}. Neben der Anwendungs- und Systemleistung muss das Benutzerverhalten der Anwendung und die Benutzerzufriedenheit ebenfalls überwacht werden, um ein detailliertes Feedback zu erhalten \cite[S. 112 - 113]{erder_continuous_2016}. Wie anhand der Abbildung 7 zu sehen ist, kann dieses Feedback in die Phase der Entwicklung zurückfließen, um bessere Entscheidungen bei der Entwicklung der nächsten Änderung treffen zu können, Probleme zu beheben und neue Anforderungen und Funktionen zu berücksichtigen. Im Ergebnis bildet diese Feedbackschleife ein Instrument, zur kontinuierlichen Gestaltung und Orientierung für das Softwareprodukt.   

\subsubsection{2.1.6.6. Infrastructur-as-a-Code} $~$

Automatisierung gilt als eine Grundvoraussetzung innerhalb der Devops- Umgebung und erstreckt sich nicht nur auf die Bereiche der Entwicklung und Bereitstellung, sondern auch auf die zugrundeliegende Infrastruktur \cite[S. 272]{tiemeyer_handbuch_2021}. Insbesondere durch die Verwendung von Continuous Integration, ist die Anzahl der Umgebungen und ihrer Instanzen stark angestiegen, da täglich Builds getestet, validiert und bei Konfigurationsänderungen angepasst werden müssen \cite[S. 19]{sharma_devops_2017}. Obwohl die Vorgehensweise des \textit{Infrastructur-as-a-Code} (kurz: IaC) keine 'Continuous'- Bezeichnung besitzt, ist diese Praktik wesentlicher Bestandteil der gängigen DevOps-Methoden \cite[S. 30]{alt_innovationsorientiertes_2017}. Anstatt manuelle Änderungen durch einen Administrator, welcher schrittweise ein neues System einrichtet oder umkonfiguriert, durchzuführen, werden Netzwerkeinstellungen, Parameter und weitere Konfigurationen als Code in einer Konfigurationsdatei beschrieben \cite{juner_praxisbasierte_2017}, \cite{luber_was_2020}. Diese Datei wird in einem Repository zur Verfügung gestellt und kann bereits beim Aufbau einer Infrastrukturumgebung automatisiert erstellt und in die Entwicklung miteinbezogen werden. Durch die entstehende Versionierung sind alle Änderungen überprüfbar, reproduzierbar und bei Fehlern können Rollbacks auf die frühere Version durchgeführt werden \cite[S. 272]{tiemeyer_handbuch_2021}. \textit{"Dadurch kann jeder produktivähnliche Umgebungen in Minuten erhalten, ohne ein Ticket aufmachen oder gar Wochen warten zu müssen."} \cite[S. 107]{kim_devops-handbuch_2017}. Voraussetzung für IaC ist es, Systemadministratoren frühzeitig innerhalb des Softwareentwicklungsprozesses einzubeziehen und das Verständnis der Entwickler für die auf dem Produkt basierende Infrastruktur zu schärfen \cite[S. 30]{alt_innovationsorientiertes_2017}. Mittels IaC würde die alleinige Verantwortung für alle Phasen wie das Design, Umsetzung, Test, Installation und Betrieb bei einem DevOps-Team liegen \cite{kasteleiner_devops_2019}.







% Continuous Planning gilt als ein ganzheitliches Unterfangen, dass ein engere Intergration zwischen Planung und Ausführung erfordert und an dem sowohl Kunden als auch die Entwickler beteiligt sind. \cite{fitzgerald_continuous_2014} 

% Wie bereits in der Phase des Planens beschrieben, können agile Methoden wie Scrum und Kanban zum Einsatz kommen um die entsprechenden Ressourcen und die Entwicklungen während des ganzen Zeitraums zu planen und einzuteilen.

% Inerhalb dieser Methode werden Features in kleinen Inkrementen geplant und entwickelt, um diese innerhalb eines Sprints umzusetzen und folglich die Durchlaufzeit bis zur Auslieferung kurzzuhalten. \cite[S. 266]{tokarski_strategische_2018} 

% Infolgedessen beinhalten die Ergebnisse dieser Methode ausliefbare und getestete Funktionalitäten nach jedem Sprint. 




% Darüber hinaus werden Änderungen sichtbarer und bilden eine starke Grundlage für zukünftige Änderungen. 




% Aufgrund der steigenden Automatisierung der Pipeline als auch durch IaC werden die Umgebungen immer identischer, was widerrum dem Grundgedanken des Continuous Integration entspricht, da der Quellcode sehr früh auf eine produktionsnahe Umgebung integriert wird und dadurch Probleme schnell sichtbar werden. \cite[S. 111 - 113]{kim_devops-handbuch_2017} 

\subsubsection{DevOps-Toolchain}
Nutzer von DevOps-Verfahren setzen im Rahmen ihrer DevOps-Toolchain oft bestimmte DevOps-freundliche Tools ein. Ziel dieser Tools ist es, die verschiedenen Phasen des Workflows zur Softwarebereitstellung (auch als Pipeline bezeichnet) noch stärker zu straffen, zu verkürzen und zu automatisieren. Viele derartige Tools sind auch an wesentlichen DevOps-Grundsätzen wie Automatisierung, Zusammenarbeit und Integration zwischen Entwicklungs- und Betriebsteam ausgerichtet. Hier werden einige gängige Tools beschrieben, die innerhalb der unterschiedlichen Phasen des DevOps-Lebenszyklus genutzt werden.\\


Planen: In dieser Phase werden der geschäftliche Nutzen und die geschäftlichen Anforderungen festgelegt. Tools wie Jira oder Git helfen dabei, bekannte Probleme nachzuverfolgen, und unterstützen das Projektmanagement.\\


Codieren: In dieser Phase stehen das Softwaredesign und die Erstellung von Softwarecode im Mittelpunkt. Tools dafür sind beispielsweise GitHub, GitLab, Bitbucket oder Stash.\\


Entwickeln. In dieser Phase werden Softwarebuilds und -versionen verwaltet. Automatisierte Tools unterstützen das Kompilieren und Packen von Code für künftige Produktionsfreigaben. Mithilfe von Quellcode-Repositorys oder Paket-Repositorys wird außerdem die zur Produktfreigabe benötigte Infrastruktur „verpackt“. Beispiel-Tools sind Docker, Ansible, Puppet, Chef, Gradle, Maven und JFrog Artifactory.\\


Testen. In dieser Phase wird durch kontinuierliches Testen (manuell oder automatisiert) eine optimale Codequalität gesichert. Beispiel-Tools sind JUnit, Codeception, Selenium, Vagrant, TestNG und BlazeMeter.\\


Implementieren. In dieser Phase können Tools genutzt werden, die das Managen, Koordinieren, zeitbezogene Planen und Automatisieren von Produktversionen für die Produktion unterstützen. Beispiel-Tools sind Puppet, Chef, Ansible, Jenkins, Kubernetes, OpenShift, OpenStack, Docker und Jira.
Betrieb. In dieser Phase geht es um das Management der Software während der Produktion. Beispiel-Tools sind Ansible, Puppet, PowerShell, Chef, Salt und Otter.\\


Überwachen. In dieser Phase werden Informationen über Probleme mit bestimmten Softwareversionen in der Produktion erkannt und erfasst. Beispiel-Tools sind New Relic, Datadog, Grafana, Wireshark, Splunk, Nagios und Slack.

\subsubsection{Automatisierung und IaC}
Unter „Infrastructure as Code“ versteht man im Wesentlichen, dass sämtliche Konfigurationsskripte, Installationsskripte, etc. versioniert in einem Software Repository abgelegt werden. Dadurch wird der gesamte Aufbau einer Infrastrukturumgebung als Code gespeichert und kann automatisiert aufgebaut werden.  Dies ist für den schnellen Aufbau von Testumgebungen ein großer Vorteil und kann darüber hinaus auch in den Entwicklungsprozess integriert werden. Änderungen an der Infrastruktur werden wie auch bei Software-Code mit Hilfe von Versionsverwaltungssystemen registriert.Hierdurch werden die Änderungen nachvollziehbar, was die Wiederholbarkeit steigert und unter anderem hilfreich sein kann für das Beheben von Fehlern. Darüber hinaus wird auch dadurch der Gedanke der agilen Methoden in die Welt der Systemadministratoren getragen. So kann dieser Denkansatz für das Managen der Infrastruktur weitergedacht werden. 


\subsubsection{Herausforderungen/Nachteile}
In diesem Abschnitt werden Befürchtungen oder Nachteile, die bei der Integration von DevOps auftreten können.  

\begin{itemize}

\item Die Verwaltung der IT-Infrastruktur, also ununterbrochene Verfügbarkeit der Produkte/Dienstleistungen sicherzustellen

\item Sicherheit des Betriebs gewährleisten

\item Fähigkeiten und Verständnis für die Integration zwischen verschiedenen Softwaremodulen, Cloud-Systemen und maßgeschneiderten DevOps-Lösungen

\item Langfristige Partnerschaft mit einem DevOps-Dienstleister, für den IT-Betriebs

\end{itemize}

\subsection{FOSS}
Obwohl Mitte des letzten kommerziellen Jahrhunderts die Dominanz innerhalb des Softwaremarktes bei vertriebener Software lag, gewann die Rolle der Open-Source-Software (OSS) zunehmend an Bedeutung, insbesondere durch die offene Bereitsstellung und Entwicklung des freien Betriebssystems 'Linux' \cite[S. 8 - 11]{wichmann_linux-_2005}, \cite[S. 1]{will_open-source-software_2003}. Bereits im Jahre 2019 lag der Einsatz von OSS ab einer Unternehmensgröße von über 2.000 Mitarbeitern bei 86 \%, was bei dieser Unternehmensgröße jedem zweiten Unternehmen entspricht \cite[S. 15]{bitkom_open_2020}. Insgesamt enstand durch wettbewerbsfähige OSS ein grundlegender Umbruch in der Softwarebranche innerhalb der letzten Jahre \cite[S. 185]{bitzer_entwicklung_2007}, \cite{fitzgerald_transformation_2006}. Unter OSS versteht man einen öffentlich zugänglichen Quellcode, den jeder einsehen, verändern und für sich nutzen kann. Durch die Idee der freien Verfügbarkeit des Quellcodes, der entfallenden Lizenzkosten für den Einsatz und der freien Weiterentwicklungsmöglichkeiten, erweist sich OSS als eine einfache und kostengünstige Innovationsquelle für Unternehmen und folglich als eine realistische Alternative zu proprietärer Software \cite[S. 21,22]{allmann_open_2019}. Das Potential und der resultierende Nutzen kann sowohl für Entwickler als auch für Unternehmen sehr vielschichtig sein. Obwohl für ein Unternehmen das Hauptaugenmerk des Einsatzes von OSS auf einen wirtschaftlichen Gewinn in Form von gesenkten Lizenz- und Entwicklungskosten und von Entwicklungszeit abzielt, profitieren Entwickler von dem Wissensaustausch, der einfachen Handhabbarkeit und der Flexibilität, die OSS mit sich bringt \cite{lerner_economic_2005}. Zunächst können Entwickler einerseits Softwarefunktionalitäten nach unternehmensinternen Prozessabläufen abgestimmt entwickeln und andererseits durch den unmittelbaren Zugang zum Quellcode frühzeitig überprüfen, ob ein wiederkehrendes Problem bearbeitet oder nach der jeweligen Problematik individualisiert werden kann. Durch die verfügbaren Standardfunktionalitäten, kann die OSS neben der Möglichkeit der Weiterentwicklung von bereits bestehender Software, dazu verwendet werden, eine Basis für weitere Entwicklungen zu schaffen \cite[S. 37/38]{kesler_anpassung_2013}. Ferner bietet der offene Standard von OSS ein hohes Maß an Anwendungsfeldern, da Entwickler mit wenig Aufwand offenliegende Schnittstellen implementieren können und folglich innerhalb der Hardwareauswahl flexibel und unabhängig sind \cite[S. 2]{kesler_anpassung_2013}, \cite[S. 21,22]{allmann_open_2019}. Durch die Verwendung von OSS sind Entwickler stark unabhängig von Software und von großen Anbietern. Im Gegensatz zu proprietärer Software können Schwachstellen und Sicherheitslücken durch das frühzeitige Testen schneller aufgedeckt und analysiert werden \cite[S. 30/31]{kees_open_2015}.\\\\ Eine weitere wichtige Rolle des Entwicklungsprozesses mittels OSS ist die Wiederverwendung von Quellcode. Anstatt einer zeit- und kostenintensiven Neuentwicklung von Software, bleiben viele allgemeine oder wiederholende Funktionalitäten innerhalb eines neuen Projektes oftmals gleich und müssen demnach nicht verändert werden \cite{henkel_code_2010}. Entwickler versuchen nicht 'das Rad neu zu erfinden', sondern suchen gezielt nach Lösungen von bereits bekannten Problemen. Vor diesem Hintergrund sparen Entwickler viel Zeit und Ressourcen, indem Quellcode, Templates oder Algorithmen wiederverwendet werden, was in der Vergangenheit durch eine maßgeschneiderte Implementierung in kleinem Maßstab erfüllt worden wären \cite{spinellis_how_2004}. Obwohl die Verwendung von OSS auf der einen Seite an Popularität zunimmt, kann auf der anderen Seite zu erheblichen Risiken führen. Insbesondere die Folge der Wiederverwendung kann eine massive Verschachtelung und viele transitive Abhängigkeiten von OSS-Komponenten voraussetzen, durch die die Komplexität erheblich steigt \cite{thelen_beschleunigung_2021}. Jede dieser Komponenten hat eine oder mehrere Lizenzen zum Ziel von rechtlichen Schutzmaßnahmen für den Lizenzgeber, die die Nutzungsbedingungen der jeweiligen Komponente spezifizieren und stark einschränken. Durch die Einbeziehung unterschiedlicher Lizenzmodelle in die OSS-Projekte sollen die Rechte des Urhebers geschützt und die damit uneingeschränkte lizenzfreie Nutzung ohne die jeweilige Genehmigung beschränkt werden \cite{widmer_open-source-lizenzen_2006}. Je nach möglicher Verteilung oder Weitergabe an Dritte der modifizierten Software und Art des Lizenzmodells können unterschiedliche Risiken infolge eines unüberwachten Einsatzes von OSS für ein Unternehmen entstehen. Die daraus resultierenden Lizenzfragen oder rechtlichen Konsequenzen können zu einer maßgeblichen Einschränkung führen, die von Unternehmen bereits im Vorfeld genau analysiert und überprüft werden muss. Mit zunehmender Anzahl von Mitarbeitern und verteilten Standorten wird die Bewältigung dieser Herausforderungen jedoch komplizierter. Die Lösung ist frühzeitige Aufklärung und Schulungen über den technischen und rechtlichen Umgang von OSS, durch die Unternehmen. Diese Unterstützung schafft eine Basis für die verantwortungsvolle Einhaltung von Lizenzbedingungen, angefangen bei den Entwicklern selbst. 

%open-source.2.0 \cite{fitzgerald_transformation_2006}


\subsubsection{Entwicklungen und Unterscheidungen von Software}
\subsubsection{Freiheiten als Merkmale von FOSS}
\subsubsection{Nutzungsrechte an FOSS}
\subsubsection{Lizenzarten und Lizenzmodelle für FOSS}
\subsubsection{Urheberrechtliche Aspekte}
\subsubsection{Vertragrechtliche Aspekte}
\subsubsection{Kritische Betrachtung bei der Verwendung von FOSS}

\subsection{FOSS meets DevOps}
\subsubsection{Hintergründe von der Einbindung von FOSS}
\subsubsection{Einbindung von FOSS in die DevOps-Kultur}
\subsubsection{Gängige Konstellationen auf vertraglicher Basis}
Welche Konstellationen/ welche Vertragsbedinungen müssen angegeben sein (zwischen Entwickler und Auftraggeber), dait FOSS in einer DevOps-Umgebung verwendet werden kann (Haftungsbeschränkungen)
\subsubsection{Vorteile der Integration}
\subsubsection{Wesentliche Hindernisse}
\subsubsection{Aktuelle Beispiele}




\section{Proof of Concept}

Während die manuelle Checkliste dafür verwendet wird, präventiv OSS-Komponenten mit riskanten Lieznzmodellen bereits zu Beginn des Entwicklungsprozesses zu vermeiden, verfolgt die automatisierte Checkliste das Ziel, zusätzlich zu einer weiteren Überprüfung von neuen OSS-Komponenten, bereits implementierte auf ihre Lizenzmodelle zu überprüfen, bevor diese ausgeliefert werden. Diese wird innerhalb des folgenden Kapitels in Form eines Proof of Concept (PoC) näher beschrieben. In dem Rahmen dieser Arbeit stellt der PoC einen Prototypen dar, der die prinzipelle Durchführbarkeit einer automatisierten Checkliste aufzeigt, wobei dieser nicht direkt innerhalb des Build-Prozess integriert sondern unabhängig von dem jetzigen Entwicklungsprozess entwickelt wurde. Die automatisierte Checkliste soll dabei als eine reduzierte Version des Endprodukts umgesetzt werden, die auf ihre Nutzbarkeit und Funktionalität getestet und bewertet werden kann. Hierbei wird der Prototyp nicht alle Merkmale und Funktionen eines marktreifen Produkts aufweisen, allerdings wird einerseits die generelle Nutzbarkeit in Abhängigkeit der bestehenden Systemumgebung und andererseits die Demonstration der entwickelten Funktionalitäten zum Zweck einer generellen Realisierbarkeit aufgezeigt. \cite{dreher_prototyping_2018} Gleichzeitig dient der PoC als eine Basis für weitere Weiterentwicklungsmöglichkeiten. Da die direkte Entwicklung innerhalb eines aktiven Projekts große Herausforderungen und Probleme mit sich bringen kann, wurde der Prototyp mit Kollegen der msg systems ag, ausgearbeitet und innerhalb dieser Arbeit festgehalten. Der PoC wurde mit der Skizzierung der grundsätzlichen Idee begonnen. Wesentliche Elemente waren hierbei die momentane Problembeschreibung als auch die darauf aufbauende Lösung, die umgesetzt werden soll. Die Ausarbeitung als auch der Umfang des zugrundeliegenden Problems wurden zunächst durch die Gespräche mit mehreren Teammitgliedern und durch die Teilnahme an Teammeetings anfänglich skizziert. In diesem Rahmen wurde \textit{das Fehlen eines unterstützenden Prozesses hinsichtlich der Bewertung und Prüfung von eingesetzter OSS} als das wesentliche Problem identifiziert, welches mithilfe einer automatisierten Überprüfung gelöst werden sollte. Demnach sollte der Protoytyp verwendete OSS-Komponenten anhand ihrer Lizenzmodelle analysieren und bei einem, mit Risiko verbundenen Lizenzmodell, den jeweiligen Entwickler darüber zeitnah in Kenntnis setzen. Die weitere Anforderungsdefinition, Gestaltung und Einbeziehung erfolgte im Zuge der Prozessmodellierung und der dazugehörigen Anpassung des Soll-Prozesses und wurde in vier grundlegende Schritte unterteilt.





\newpage
\bibliography{Literatur}
\bibliographystyle{apalike}

\end{document}


%Bilder einfügen
%\begin{figure}[h]
    %\centering
    %\includegraphics[scale=0.7]{Bilder/what-is-foss}
    %\caption{Ein Beispiel}
%\end{figure}

