\documentclass[12pt,titlepage]{article}
\usepackage[ngerman]{babel}
\usepackage[utf8]{inputenc}
\usepackage{color}
\usepackage[a4paper,lmargin={4cm},rmargin={2cm},
tmargin={2.5cm},bmargin = {2.5cm}]{geometry}
\usepackage{amssymb}
\usepackage{amsthm}
\usepackage{graphicx}
\usepackage{lscape}
\usepackage{longtable} 
\usepackage{booktabs}
% Makros für allgemein nützliche Dinge wie Korrekturfeedback :-)
% Dies hier auskommentieren, falls du in der Ansicht keine Kommentare sehen willst. 
%\newcommand{\dwi}[1]{}
% Dies hier mit % kommentieren, falls du in der Ansicht keine Kommentare sehen willst. 
\newcommand{\dwi}[1]{\textcolor{red}{\footnotesize#1}}

% Tests..
%\newcommand{\dwi}[1]{\marginpar{\textcolor{red}{\tiny#1}}}

\begin{document}
\title{DevOps meets OSS: Automatisierte Integration von OSS in die DevOps-Produktentwicklung\\}
\author{Claudia Arnold}

\maketitle

\begin{abstract}
   Unter dem Begriff OSS bekannt, bietet die Open Source Software eine Reihe an Möglichkeiten im Bereich der Softwareentwicklung. Durch die sinkenden Nutzungesbescchränkungen im Vergleich zur konventioneller Software gilt OSS als ein Instrument zur schnellen Integration von Software in die eigene Entwicklung.  
   \dwi{hier kommt noch mehr oder? könnte sich lohnen dies mal im gesamtscope der arbeit zu formulieren um sich den rahmen bewusst zu machen.}
\end{abstract}

\newpage  
\begin{center} 
    \vspace*{\fill}
    \textit{‘Free software’ is a matter of liberty, not price. To understand the concept, you should think of ‘free’ as in ‘free speech,’ not as in ‘free beer’.}
    \newline
    \newline - Richard Matthew Stallman
    \vspace*{\fill}
\end{center}

\newpage
\dwi{also es ist der knaller was du hier an wissen angehäuft hast, respekt. da hast du eine ordentliche recherche gemacht.
die aufgabe ist allerdings jetzt, das wissen auf ein transportierbares maß zu reduzieren und zu fokussieren
- hier gilt qualität vor quantität, und das ist auch ein wesentlicher teil deiner leistung. nicht alles was du 
zum thema weißt muss notwendigerweise als text in deine arbeit, hier und da liest es sich als ob du in einem paper ein
für dich wertvolles konzept gefunden hast du dann einen satz dazu irgendwo einfügst wo er am besten passt.
hier sind gezielte referenzen und knackige sätze noch viel wertvoller. du weißt .. es muss immer einer arbeiten, der autor oder der leser.
und je weniger der leser arbeiten muss desto besser ist deine arbeit. du schreibst ja auch (noch?) keine doktorarbeit :-)}
\section{Einleitung}
\begin{quote}
\textit{In real open source, you have the right to control your own destiny}\newline -Linus Torvalds
\end{quote}

Verändertes Konsumentenverhalten, schnelle Reaktion auf veränderte Kundenwuensche und Time-to-Market sind wesentliche Anforderungen, die für den wirtschaftlichen Erfolg eines Unternehmens insbesondere in Zeiten der Digitalisierung kennzeichnet sind und daher große Herausforderungen mit sich bringen. git staDie daraus resultierenden steigenden Anforderungen an neuer Funktionalität und die damit verbundene stetige Weiterentwicklung erfordern in erster Linie eine neue Dynamik innerhalb der Entwicklug, Konzepte und Tools. In diesem Zusammenhang wird der Softwareentwicklungsprozess nach agilen Methoden unablässig und gehört in den meisten Unternehmen bereits zum Standard einer IT-Organisation. 

In diesem Rahmen kommt der Ansatz des DevOps zum Tragen, nach dessen Grundsätzen der Softwareentwicklungsprozess, angefangen von der Idee, über die schnelle Entwicklung bishin zum Einsatz innerhalb der Produktivumgebung, durchgeführt wird. DevOps ist ein Kofferwort aus den Begriffen Development und IT-Operations und beschreibt im Wesentlichen die technische und organisatorische Verbindung zwischen der Softwareentwicklung und der Systemadministration. Charakterisch sind verkürzte Releasezyklen in einem inkrementellen Prozess und ein hoher Automatisierungsgrad, wodurch eine exakte Planung, Vermeidung von kritischen Fehlern und eine schnelle Reaktion auf Kundenanforderungen ermöglicht wird. Wesentliches Ziel von DevOps ist es, die gesamte Entwicklung transparent, flexibel und agil zu gestalten, wodurch die Produktivität und Effizenz maßgeblich gesteigert und das endgültige Produkt schneller zum Kunden ausgeliefert werden kann. 

Neben der agilen Softwareentwicklung ist ein weiteres wesentliches Kennzeichen des DevOps-Konzeptes die Grundsätze des Lean Manufactoring, nach denen die Minimierung von Verschwendung und die Maximierung der Produktivität im Vordergrund steht. Dabei spielt insbesondere die Wiederverwendung von Software eine wesentliche Rolle. Anstatt einer zeit- und kostenintensiven Neuentwicklung von Software, bleiben viele allgemeine oder wiederholende Funktionalitäten innerhalb eines neuen Projektes oftmals gleich und müssen demnach nicht verändert werden. Vor diesem Hintergrund sparen Entwickler viel Zeit und Ressourcen, indem Quellcode, Templates oder Algorithmen wiederverwendet werden, was wiederrum dem Grundgedanken von DevOps entspricht. 

Ausgehend hiervon ist die Integration von bestehenden Softwarekonzepten ein wesentlicher Vorteil, wodurch die Verwendung von Open Source Software zunehmend an Bedeutung im Unternehmensfeld gewinnt. Als eine realistische Alternative zu herkömlicher Software, erlangte die Open-Source-Bewegung mit der Idee von frei zugänglicher Software in den letzten Jahrzehnten immer mehr an Popularität. Unter Open Source Software versteht man einen öffentlich zugänglichen Quellcode, den jeder einsehen, verändern und für sich nutzen kann. Im Gegensatz zur herkömlicher proprietärer Software, erhalten Nutzer eine Open Source Software meist kostenlos und ohne weitere Kosten für Lizenzvereinbarungen. Durch den unmittelbaren Zugang zum Quellcode, haben Entwickler die Möglichkeit einerseits, am Quellcode mit nur wenigen Ressourcen zu experimentieren und zu überprüfen, ob sich dieser ein wiederkehrendes Problem innerhalb des Projektes beinhaltet oder anderseits nach der jeweligen Problematik zu individualisieren und zu verändern. Die Notwendigkeit einer kosten- und zeitspielige Entwicklung entfällt und die Verwendung von bereits gelösten Paradigmen rückt in den Vordergrund.   

% Häufige Vorgehensweise: Moduale Architektur von Open-Sorce projekten
% Beispiele an OSS aufzeigen: Betriebssysteme, Server-Anwendungen oder Office-Produkte
% Evtl hier nochmal auf Linux, als erstes populäres Produkt im OSS-Bereich eingehen 

Das daraus resultierende Innovationspotenzial kann, basierend auf der freien Gestaltungsfreiheit die OSS im Gegensatz zur kommentieller Lizenzsoftware ermöglicht, genutzt werden, Ideen umzusetzen ohne zu stark an unternehmensinterne Vorgaben gebunden zu sein oder unternehmensinterne Prozesse oder Prozessabläufe in die Entwicklung einzubinden. Vor diesem Hintergrund kann eine umfassende Integration von Open Source Software in vorhandene Unternehmenstrukturen im Bereich der Entwicklung, ausschließlich in einem agilen Umfeld durchgeführt werden, womit DevOps als ideale Plattform geeignet ist.  

Trotz der Freiheiten, die Open Source Software offensichtlich bietet, unterliegen die meisten Projekte rechtlichen Schutzmaßnahmen, einschließlich dem Marken- Patent- und Urheberrecht. Aufgrund dessen können, aus der Verwendung von OSS resultierende Lizenzfragen oder rechtliche Konsequenzen, zu einer maßgeblichen Einschränkung insbesondere im Hinblick auf eine mögliche Verteilung oder Weitergabe an Dritte führen, die von Unternehmen genau analysiert und überprüft werden muss.  

In dieser Thesis werden die grundsätzlichen Vorteile und Risiken, die sich aus den verschiedenen Lizenzvereinbarungen bei der Verwendung von OSS ergeben analysiert und die Migration in den Devops-Emntwicklungsprozeess dargestellt. Verdeutlicht wird die Themenstellung anhand der ausgeführten Analysen anahnd eines Beispiels bei der msg Systems. 




Durch die freie Verfügbarkeit des Quellcodes, die entfallenen Lizenzkosten für den Einsatz und die freie Weiterentwicklungsmöglichkeiten, erweist sich OSS als eine einfache und kostengünstige Innovationsquelle für Unternehmen. 

Dabei reicht das Anwendungsfeld für die Verwendung von OSS in diesem Zusammenhang von der Automobilindustrie bis zur Versicherungsbranche und steht daher im direkten Wettbewerb zu proprietären Softwareangeboten. 

\subsection{Problemstellung}
Laut einer aktuellen Studie des Bundesverband Informationswirtschaft, Telekommunikation und neue Medien e. V. (Bitkom) wird der Einsatz von Open Source Software in Unternehmen, im Jahre 2019 bei einer Unternehmensgröße von über 2.000 Mitarbeitern, bereits auf 86 Prozent geschätzt. Dabei reicht das Anwendungsfeld von der Automobilindustrie bis hin zu Banken und Versicherungen und steht daher im direkten Wettbewerb zu namenhaften proprietaeren Softwareangeboten. Ausgehend hiervon gehört die Etablierung von Open Source Software in die Softwareentwicklung bereits zur gängigen IT-Unternehmenskultur.  

Durch die freie Verfügbarkeit des Quellcodes, kann die entsprechende Software schnell und einfach eingebunden werden. 



Open-Source-Software gilt als zuverlässig und sicher. Für manche ist sie sogar ein Innovationsmotor mit ungeheurem Potenzial. Neben der Vielfalt der Anwendungsszenarien steht die Fülle 
der Anwendungen. Freie Software – ein gängiges anderes Etikett für Open-Source-Software – 
hat echte wirtschaftliche Bedeutung gewonnen. Die Serverinfrastruktur einiger großer Internetfirmen basiert zum größten Teil auf Linux. Google, Facebook oder Amazon sind bekennende 
Nutzer. Die Gründe dafür sind simpel: Unternehmen müssen für den Einsatz von Open-SourceSoftware keine Lizenzkosten einkalkulieren. Das vereinfacht den Aufbau skalierender Umgebungen signifikant. Außerdem können die Unternehmen die genutzte freie Software im Rahmen 
der jeweiligen Lizenzbestimmungen entsprechend eigener Zwecke und Ziele weiterentwickeln. 
Selbst wenn die modifizierte Version danach geschäftskritische Elemente enthält – und der 
wirklich geschäftskritische Anteil eines Softwaresystems ist in der Regel sehr klein –, gibt es 
immer noch gute Einsatzszenarien, die deren Schutz in Verbindung mit Open-Source-Software 
gewährleisten.


so selbstverständlich die Verwendung ist, so verbreitet sind 
oft auch Wissenslücken über Grundanforderungen im Umgang mit OSS, sei es als Basis für einen 
erfolgreichen Einsatz im Alltag oder als Voraussetzung für die Verwendung für eigene Softwareprojekte. Ein gängiges Missverständnis besteht z.B. darin, dass aus der einfachen und unentgeltlichen Verfügbarkeit von OSS auf das Fehlen jeglicher juristischen Einbettung geschlossen wird. 

Grundsätzlich unterliegt Software dem Urheberrecht. Daher 

Durch die freie Verfügbarkeit des Quellcodes, kann die entsprechende Software schnell und einfach eingebunden werden. 


Die implizierte Freiheit, die sich bei der Verwendung von OSS ergibt, beschränkt sich dementsprechend auf die Einsicht, Nutzung, Modifikation und Distribustion des Quellcodes.    

jedoch nicht uneingeschränkt lizenzfrei genutzt werden. 



Mittels einer Lizenzierung der entsprechenden Software soll sichergestellt werden, dass die Rechte zur Nutzung der Open Source Software nur mit Genehmigung des Urhebers geschützt bleiben. Ab dem Zeitpunkt der Weitergabe von veränderter Software, müssen entsprechenden Lizenzmodelle berücksichtigt werden, um vorrangig das geistige Eigentum des Urhebers zu schützen.  

Bereits bei groben Veränderungen im Laufe des Projektes, hat die Veränderungen von bestimmten SChutzmaßnahmen graviernede Auswirkungen auf das DevOps-Team. Viele erhalten nicht die benötigten Informationen, dass eine Veränderung vorliegt, die das Projekt maßgeblich beeinflußen könnte. 

Demgegenüber haben Urherber bei einer Verletzung an ihrer Open Source Software, die sich aus den einzelnen Lizenzmodellen ergibt, die Möglichkeit rechtliche Ansprüche geltend zu machen. Die Folge sind Unterlassungsansprüchen und kostenspieliger Aufwendungsersatz. Die Konsequenz wäre eine sofortige Einstellung der bearbeiteten Software und möglicherweise eine neue Entwicklung, der Funktionalitäten ohne das jeweilige Open Source als Basis./ Nutzer verliehrt das Recht die Software weiterhin zu nutzen.  


Eine
Lizenz (v lat: licere = erlauben) ist eine Erlaubnis oder Genehmigung zur Nutzung
eines Rechts durch den Urheber oder Inhaber dieses Rechts11. Im juristischen
Bereich stellt die Lizenz einen Vertrag dar, durch welchen einfache oder
ausschließliche Rechte eingeräumt werden können. 


Software ist in der Regel urheberrechtlich geschützt. Dies gilt auch für Freie Software oder Open-Source (OSS). Die für proprietäre Software eingeräumten Softwarelizenzen sind dahingehend ausgerichtet, die Freiheit der Nutzung, Verbreitung und Veränderung der Software einzuschränken. Der Softwarehersteller kann so durchsetzen, eine angemessene Vergütung für die Entwicklungsleistungen zu erhalten, Art und Umfang der Werknutzung zu bestimmen und die Software vor ungewollter Veränderung zu schützen. 


1. Freie Weitergabe
Die Lizenz darf niemanden in seinem Recht einschränken, die Software als Teil eines Software-Paketes, das Programme unterschiedlichen Ursprungs enthält, zu verschenken oder zu verkaufen. Die Lizenz darf für den Fall eines solchen Verkaufs keine Lizenz- oder sonstigen Gebühren festschreiben.

2. Quellcode
Das Programm muss den Quellcode beinhalten. Die Weitergabe muss sowohl für den Quellcode, als auch für die kompilierte Form zulässig sein. Wenn das Programm in irgendeiner Form ohne Quellcode weitergegeben wird, so muss es eine allgemein bekannte Möglichkeit geben, den Quellcode zum Selbstkostenpreis zu bekommen, vorzugsweise als gebührenfreien Download aus dem Internet. Der Quellcode soll die Form eines Programms haben, das ein Programmierer vorzugsweise bearbeitet. Ein absichtlich unverständlich geschriebener Quellcode ist daher nicht zulässig. Zwischenformen des Codes, so wie sie etwa ein Präprozessor oder ein Konverter („Translator”) erzeugt, sind unzulässig.

3. Abgeleitete Software
Die Lizenz muss Veränderungen und Derivate zulassen. Außerdem muss sie es zulassen, dass die solcher Art entstandenen Programme unter denselben Lizenzbestimmungen weiter vertrieben werden können wie die Ausgangssoftware.

4. Unversehrtheit des Quellcodes des Autors
Die Lizenz darf die Möglichkeit, den Quellcode in veränderter Form weiterzugeben, nur dann einschränken, wenn sie vorsieht, dass zusammen mit dem Quellcode so genannte „Patch files” weitergegeben werden dürfen, die den Programmcode bei der Kompilierung verändern. Die Lizenz muss die Weitergabe von Software, die aus einem veränderten Quellcode entstanden ist, ausdrücklich erlauben. Die Lizenz kann verlangen, dass die abgeleiteten Programme einen anderen Namen oder eine andere Versionsnummer als die Ausgangssoftware tragen.

5. Keine Diskriminierung von Personen oder Gruppen
Die Lizenz darf niemanden benachteiligen.

6. Keine Einschränkungen bezüglich des Einsatzfeldes
Die Lizenz darf niemanden daran hindern, das Programm in einem bestimmten Bereich einzusetzen. Beispielsweise darf sie den Einsatz des Programms in einem Geschäft oder in der Genforschung nicht ausschließen.

7. Weitergabe der Lizenz
Die Rechte an einem Programm müssen auf alle Personen übergehen, die diese Software erhalten, ohne dass für diese die Notwendigkeit besteht, eine eigene, zusätzliche Lizenz zu erwerben.

8. Die Lizenz darf nicht auf ein bestimmtes Produktpaket beschränkt sein
Die Rechte an dem Programm dürfen nicht davon abhängig sein, ob das Programm Teil eines bestimmten Software-Paketes ist. Wenn das Programm aus dem Paket herausgenommen und im Rahmen der zu diesem Programm gehörenden Lizenz benutzt oder weitergegeben wird, so sollen alle Personen, die dieses Programm dann erhalten, alle Rechte daran haben, die auch in Verbindung mit dem ursprünglichen Software-Paket gewährt wurden.

9. Die Lizenz darf die Weitergabe zusammen mit anderer Software nicht einschränken
Die Lizenz darf keine Einschränkungen enthalten bezüglich anderer Software, die zusammen mit der lizenzierten Software weitergegeben wird. So darf die Lizenz z. B. nicht verlangen, dass alle anderen Programme, die auf dem gleichen Medium weitergegeben werden, auch quelloffen sein müssen.


Um zu gewährleisten, dass EntwicklerInnen, die mit GNU-Software arbeiten, den Wissenspool ebenfalls nähren, verpflichtet die Lizenz ihre NutzerInnen zur Weitergabe der Derivate unter gleichen Bedingungen (Copyleft). Die Idee freier Lizenzen geht also auf die Software-Entwicklung zurück, die durch freien Code kollaborative Arbeiten ermöglichen wollte. 


Alle Open Source Lizenzen haben die Einräumung eines Vervielfältigungs- und Verarbeitungsrechts gemeinsam, welches meist bestimmten Voraussetzungen bzw. Beschränkungen unterliegt. Die einzelnen Lizenzen unterscheiden sich vor allem hinsichtlich der Nutzungsbedingungen und der Verpflichtungen, die dem Lizenznehmer zur Wahrnehmung des Vervielfältigungsrechts und des Verarbeitungsrechts aufgegeben werden.

Wichtigster Gesichtspunkt ist, welche Anforderungen an die Weiterverbreitung von veränderten Versionen der Software bzw. neuer Software gestellt werden, die auf Grundlage von Open Source Software entwickelt wurde.



Von Beginn an haben sich verschiedene Lizenzmodelle für OSS herausgebildet, die sich hinsichtlich der Nutzungsbedingungen teilweise recht deutlich unterscheiden. Zur Sichtung, Sammlung 
und Ordnung der vielfältigen Lizenzmodelle hat sich die Open Source Initiative (OSI) gegründet. 
Die OSI hat Kriterien aufgestellt, nach denen sie eine Lizenz als Open-Source-Lizenz klassifiziert 
und in die offizielle Liste der Open-Source-Lizenzen aufnimmt. Die Klassifikation einer Softwarelizenz als Open- Source-Lizenz durch die OSI erfolgt anhand der Rechte, die die Lizenz dem Nutzer 
einer so lizenzierten Software einräumt. Sie erfolgt nicht anhand der Auflagen, die die OpenSource-Lizenz dem Nutzer auferlegt. 
Für eine Einteilung der Open-Source-Lizenzen anhand der Pflichten haben sich die Kategorien 
»permissive Lizenzen«, Lizenzen mit schwachem »Copyleft« und‚ Lizenzen mit starkem 
»Copyleft« eingebürgert. Einige OSS-Lizenzen verpflichten dazu, bei Weitergabe der Software 
den Quellcode zugänglich zu machen. Damit ist manchmal das sog. Copyleft verbunden: 
Danach darf ein Nutzer eine von ihm veränderte Open-Source-Software nur zu den für die 
ursprüngliche Open-Source-Software geltenden Lizenzbedingungen an Dritte weitergeben. 
Die permissiven Lizenzen stellen solche Anforderungen nicht. Die Einteilung in permissive 
Lizenzen und Lizenzen mit starkem und schwachem Copyleft bietet nur eine erste Orientierung. 
Was konkret zu tun ist, um Open-Source-Software lizenzgemäß zu nutzen, ergibt sich stets erst 
aus der konkreten Open-Source-Lizenz. Die Lizenzbedingungen sind natürlich auch bei einer wirtschaftlichen Nutzung von Open-SourceSoftware zu beachten. Daher werden Unternehmen, die Leistungen im Zusammenhang mit OSS 
anbieten oder OSS in eigenen Produkten nutzen, den Umfang ihrer Nutzungsrechte anhand der 
konkreten Softwarelizenz klären, die mit dem Vertrieb verbundenen Risiken abschätzen und ihre 
Entwicklungs- und Vermarktungsstrategie in technischer und rechtlicher Hinsicht darauf ausrichten müssen. In diesem Kontext ist besonders hervorzuheben, dass Open-Source-Software – 
entgegen landläufigen Vorurteilen – sehr wohl kommerziell eingesetzt und vertrieben werden 
darf. Einzig für die Nutzung der Software selbst dürfen keine Lizenzgebühren verlangt werden. 
Unternehmen können jedoch neue Geschäftsmodelle mit und um Open-Source-Software herum 
aufbauen, die nicht auf das traditionelle Lizenzsoftwaregeschäft zurückgreifen. Andererseits sind Unternehmen bei der Einbettung von fremderstellter Open-Source-Software in 
eigene Produkte vor Herausforderungen gestellt, die über die bloße Lizenzerfüllung hinausgehen. 
So müssen sie etwa überlegen, inwieweit sie eine Mängelhaftung für die fremderstellte OSS 
übernehmen können, wenn eine Fehlerbehebung nicht von den Urhebern der Software oder 
der Open-Source-Community angeboten wird. Zu solchen Herausforderungen gehört auch die 
Analyse, ob und wann ein Unternehmen geschäftskritische Alleinstellungsmerkmale in eine 
adaptierte Open-Source-Software hineinprogrammiert, deren Lizenz die Offenlegung des Codes 
im Falle einer Weitergabe des Programms an Kunden vorsieht. Glücklicherweise sind die 
wenigsten Verbesserungen an einem Open-Source-Code wirklich geschäftskritisch. Und viele 
Lizenzen verlangen auch keine solche Offenlegung. Um solche Herausforderungen in den Griff zu bekommen, ist zunächst die Erfassung der im 
Unternehmen und in seinen Produkten verwendeten Open-Source-Software erforderlich. 
Daraus wird sich eine unternehmenseigene Steuerung und Kontrolle der Open-Source-Software 
entwickeln. Hierzu sind ein technisches Software-Management und ein rechtliches Lizenzmanagement durchaus hilfreich. Mehr noch: Die Einrichtung entsprechender Strukturen gehört 
letztlich zu den unternehmerischen Organisationspflichten. Darüber hinaus kann eine 
differenzierte Vertragsgestaltung bei Beschaffung und Vertrieb von Open-Source-Software 
helfen. Bei der Gestaltung von Verträgen über Leistungen im Zusammenhang mit Open-SourceSoftware (etwa Implementierung, Anpassung, Zusatzprogrammierung etc.) und bei der 
Einrichtung eines Lizenzmanagements ist die Beteiligung fachkundiger unternehmensinterner 
oder externer Rechtsberater empfehlenswert.

Trotz der Freiheiten, die Open Source Software offensichtlich bietet, unterliegen die meisten Projekte rechtlichen Schutzmaßnahmen, einschließlich dem Marken- Patent- und Urheberrecht. Aufgrund dessen können, aus der Verwendung von OSS resultierende Lizenzfragen oder rechtliche Konsequenzen, zu einer maßgeblichen Einschränkung insbesondere im Hinblick auf eine mögliche Verteilung oder Weitergabe an Dritte führen, die von Unternehmen genau analysiert und überprüft werden muss.  

In dieser Thesis werden die grundsätzlichen Vorteile und Risiken, die sich aus den verschiedenen Lizenzvereinbarungen bei der Verwendung von OSS ergeben analysiert und die Migration in den Devops-Emntwicklungsprozeess dargestellt. Verdeutlicht wird die Themenstellung anhand der ausgeführten Analysen anahnd eines Beispiels bei der msg Systems.

In diesem Abschnitt wird beschrieben, welche derzeitige Problemstellung innerhalb der DevOps Produktentwicklung bei der Integration von FOSS vorliegt, die mittels der Thesis gelöst bzw verbessert werden kann. Insbesondere wird dabei auf die derzeitige Ist-Situation eingegangen, welche Herausforderungen momentan vorliegen (bei msg und allgemein) und wie der aktuelle Stand der Wissenschaft in diesem Kontext ist. 
Zudem wird eine Herleitung zu den Ziel und den Forschungsfragen dieser Thesis hergestellt.  

\subsection{Ziel und Fragestellung}
In diesem Abschnitt wird das konkrete Ziel der Thesis festgelegt und die Forschungsfragen, an denen sich die Thesis ausrichtet und zu beantworten. Die Forschungsfragen wurden zunächst folgendermaßen festgelegt: \\


1. Welche relevanten Lizenzinformationen können aus FOSS automatisiert extrahiert und verwendet werden? (Lizenzmodelle)\\


2. Welche Einschränkungen sind bei der Verwendung von FOSS in Hinblick auf die verschiedenen Lizenzinformationen zu berücksichtigen? (Haftungs-/Garantiebedingungen, Autorenhinweise, Copyleft-Bedinungen)\\


3. Welche prozessualen Anpassungen müssen anhand der gewonnenen Erkenntnisse getroffen werden, um auf mögliche Veränderungen in den Lizenzinformationen reaktionsschnell vorbereitet zu sein?\\ 


4. Wie können relevante Lizenzinformationen und haftungsbedingte Anforderungen anhand eines Produktes automatisch aus den verwendeten Lizenzmodellen aus FOSS herausgearbeitet werden? (Programm, Makros ???)\\


5. Wie kann die Transparenz der gesammelten Informationen insgesamt sichergestellt werden? (Wie kommen die Informationen schnell an das Devops-Team)\\ 


\subsection{Aufbau dieser Arbeit}
In diesem Abschnitt wird die Methodik, das Vorgehen und die Gliederung der Thesis beschrieben
\dwi{ich glaube du brauchst den aufbau der arbeit nicht beschreiben. das ergibt sich implizit. falls du irgendwo die konkrete
kapitelstruktur erläutern willst, wäre das auf hauptkapitelebene am ende der einleitung angemessen.
du hast keinen experimentierteil, daher glaube ich nicht das sich ein extra "methodik" abschnitt lohnt - oder überzeug mich?}


\section{Fachlicher Rahmen}
In diesem Abschnitt wird der fachliche Rahmen dieser Arbeit beschrieben. Zunächst wird in Kapitel 2.1 auf die Grundlagen und die Methoden von DevOps eingegangen. In diesem Rahmen werden unter anderem die Vorteile und das wesentliche Toolchain von DevOps näher erläutert. Im nächsten Teil dieses Kapitels wird OSS geschildert. Dabei werden insbesondere die Grundlagen, lizenzrechtliche Gegebenheiten und die juristischen Konsequenzen, bei einer Verletzung der Schutzrechte, ausführlich behandelt.

\subsection{DevOps}
Wie bereits beschrieben, handelt es sich bei dem Begriff DevOps um ein zusammengesetztes Wort aus den Begriffen Development und IT-Operations. 

\subsubsection{Hintergrund}
Während der letzten Jahrzehnte hat sich die Rolle der Informationstechnologie (IT) innerhalb eines Unternehmens gravierend verändert. 

Denn die frühere Rolle bestand darin, Geschäftsprozesse zu unterstützen und skalierbare IT-Dienste bereitszustellen \cite{haffke_transformative_2017}, die auf betriebliche Kosteneffizienz ausgelegt waren und wodurch die gesamte Unternehmenslandschaft statisch aufgebaut war. \cite[S. 16]{ravichandran_devops_2016}

Aufgrund der Globalisierung und des technologischen Fortschrittes sind heutige Geschäftsmodelle als immer komplexer und schnelllebiger, wodurch die größte Herausforderung für Unternehmen darin besteht, kontinuierlich innovativ und schlank zu bleiben. \cite{haffke_transformative_2017}  

Die daraus resultierende Verschiebung zeigt, dass IT eine zunehmend zentrale Rolle für Unternehmen einnimmt und als ein Enabler agiert um weiterhin wettbewerbsfähig zu bleiben. \cite{haffke_transformative_2017} 

Wesentlicher Kernaspekt der IT ist die Softwareentwickung. 

In den Anfangsjahren der Softwareentwicklung lag der Schwerpunkt auf den Bereich der Anforderungen, wodurch die Modelle starken Kontrollen und strukurierten Vorgehen unterlagen. \cite{kneuper_sixty_2017}

Als eines der bekanntesten, traditionellen Softwareentwicklungsmodellen gilt das Wasserfallmodell, welches sich durch eine prozessorientierte Entwicklung kennzeichnet. \cite{bakaji_waterfall_2012} 

Dabei werden bereits vor Projektbeginn alle Anforderungen gesammelt und klassifiziert, um die Sofware dahingehend zu entwickeln und zu testen. 

Infolgedessen werden die Anforderungen zwar klar und in einem festgelegten Zeitraum festgelegt \cite{bakaji_waterfall_2012}, allerdings können sich im Laufe des Projeks Problemdefinitionen und Anforderungen ändern, wenn neue Lösungen in Betracht gezogen werden müssen, so dass über die gesamte Softwareentwicklung hinweg viel mehr Flexibilität erforderlich ist. \cite[S. 17]{ravichandran_devops_2016}

Dies betrifft insbesondere die Nachfrage der Kunden oder die der Stakeholder nach immer besseren und schnelleren Softwareprodukten.

Die daraus entstehende Veränderung verursachte eine Anpassung der gängigen Praktiken und Vorgehensweisen innerhalb des IT-Bereichs. 

Anfang des 21. Jahrhunderts kamen agile Methode auf, die in der Softwareentwicklung etabliert worden sind. 

Gründe waren der wachsende Time-to-Market und der zunehmende Wettbewerb, teilweise bedingt durch das Aufkommen des World Wide Web und die daraus resultierende Notwendigkeit, sich besser an unklare oder ändernde Anforderungen anzupassen. \cite{haffke_transformative_2017} 

Bei der agilen Softwareentwicklung handelt es sich um eine iterative Arbeitsweise, die sich aus  unterschiedlichen Methoden und Techniken zusammensetzt.

Kernaspekte sind dabei die Aufnahme von kontinuierlichen Anpassungen anhand von neuen oder veränderten Anforderungen innerhalb des Gesamtprozesses, Begrenzung der Dokumentation auf das Wesentliche und die Arbeitsweise mit kleinen und regelmäßigen Iterationen in überschaubaren Inkrementen. \cite{cohen_introduction_2004}, \cite{bakaji_waterfall_2012}, \cite[S. 18]{ravichandran_devops_2016}

Durch eine iterative Problemlösung und häufiges Feedback, können komplexe Geschäftsprozesse gelöst und damit die Zufriedenheit der Kunden sichergestellt werden. 

Vorteilhaft ist, dass sich Unternehmen schnell an neue Informationen anpassen, Fehler früher gefunden werden, sofortige Verbesserungen der Software durchgeführt wird, Lernfortschritt erhöht und eine schnelle Entwicklung und Auslieferung möglich ist. \cite[S. 18]{ravichandran_devops_2016}

Insgesamt war die wachsende Bedeutung der Technologie im Zeitalter der Digitalisierung, die sich mittelbar auf das Geschäftsleben oder insbesondere auf den Bereich der Softwareentwickung ausgewirkt hat, letztlich der Grund für die Einführung von DevOps. \cite{haffke_transformative_2017} 

Obwohl der Begriff 'DevOps' das erste Mal von Patrick Debois geprägt worden ist, als er im Jahr 2009 die erste DevOpsDays-Veranstaltung in Gent/Belgien organisierte, liegen die Ursprünge dieses Konzepts weit früher. \cite[S.3]{kim_devops-handbuch_2017}  

Die Wurzeln können zunächst im Bereich der Lean-Bewegung angesehen werden, welches darauf basiert, dass das hergestellte Produkt kontinuierlich verbessert, Verschwendung vermieden, auf schnelles Feedback und kürzeres Time-to-Market setzt. \cite[S.2]{kim_devops-handbuch_2017} \cite{nelson_devops_2016} \cite[S. 11]{ravichandran_devops_2016}

Letztlich wurde dieser Grundgedanke am Anfang des 21. Jahrhunderts aufgegriffen und mit dem Konzept des agilen Manifestes erweitert.

Hintergrund war die notwendige Abkehr von einer starren und dokumentationslastigen Softwareentwickung hin zu einem iterativen Ansatzes. \cite[S.4]{sharma_devops_2017}  

Zu dieser Zeit war ein generelles Problem, dass Entwickler Softwareänderungen und Updates schnell liefern konnten, aber der Betrieb, der die Verfügbarkeit von Anwendungen gewährleistet, diese Funktionen nur langsam bereitgestellen konnte, da dieser starre Änderungsmanagementprozesse hatte.

Dieses Problem wird oftmals als \textit{Wall of Conflict} bezeichnet und beschreibt Probleme im Arbeitsfluss durch fehlende Informationen, wie Abbildung 1 schematisch zeigt. \cite{kawaguchi_wall_2020}, \cite[S. 26]{huttermann_devops_2012} 

\begin{figure}[h]
    \centering
    \includegraphics[scale=0.6]{Bilder/Wall of Conflict.png}
    \caption{Wall of Conflict, angelehnt an \cite{hering_devops_2018}}
\end{figure}

Gründe dafür waren, dass sowohl Entwickler- als auch Betriebsteam unterschiedliche Ziele, Denkweisen und Prozessen verfolgt haben. \cite{wettinger_streamlining_2016} 

Der daraus resultierende Engpass führte zu einem erheblichen Mehraufwand insbesondere im Bereich des Ops-Teams und zu verspäteten Auslieferungen von Softwareprojekten.

Mittels des DevOps-Ansatzes sollen die Silos zwischen der Entwicklung des Betriebs aufgebrochen werden, indem kleine und funktionsübergreifende Teams gebildet werden, die die DevOps-Praktiken und Prinzipien verfolgen. \cite{ebert_devops_2016}, \cite{wiedemann_research_2019}  

Hierbei werden automatisierte Pipelines aufgebaut können und manuelle Arbeiten reduziert um eine kontinuierliche Bereitsstellung sicherzustellen. \cite[S.3,5]{verona_practical_2016}

Demnach sollte einerseits eine dauerhafte Interaktion mit dem Kunden stattfinden und andererseits keine Messung von großen Meilensteinen mehr erfolgen. \cite[S.5]{sharma_devops_2017} 

Aus diesem Grundgedanken heraus, handelte es sich bei dem DevOps-Ansatzes zunächst um eine kulturelle Bewegung, die die Unterschiede zwischen Dev und Ops verändern sollte und mittels Automatisierung, die Bereitstellung schneller effizienter und letztlich kontinuierlicher zu gestalten. \cite[S.5]{sharma_devops_2017} 

Im Laufe der Jahre hat sich das Ziel von DevOps von der reinen Geschwindigkeit hin zu Geschwindigkeit, Zuverlässigkeit und Qualität geändert.\cite[S.xxix]{sharma_devops_2017}  

Aufgrund dessen kann DevOps nicht als ein vollständig neuer Ansatz, sondern vielmehr als eine Weiterentwicklung bereits bekannter Konzepte angesehen werden. \cite[S. 23]{alt_innovationsorientiertes_2017}



\subsubsection{Der Begriff 'DevOps'}
Trotz einer ausgeprägten Zielsetzung und definierten Arbeitsweisen gibt es keine allgemein akzeptierte Definition des Begriffs DevOps.



\subsubsection{Grundlagen}
Heutzutage müssen insbesondere Unternehmen in der Lage sein, einerseits auf veränderte Marktbedingungen schnell zu reagieren und anderseits stabile und qualitativ hochwertige Systeme zu integrieren und diese gleichzeitig zuverlässig zu unterhalten. \cite{humble_why_2011} 

Anhand des DevOps-Ansatzes wird eine Kultur oder Umgebung etabliert, durch die das Erstellen, Testen und Freigeben von Software schnell, häufig und zuverlässiger erfolgen kann. \cite[S.xxviii]{sharma_devops_2017}

In diesem Zusammenhang sollen Ineffizienzen in den Softwareentwicklungs-, Release- und Betriebsprozessen vermieden werden, die durch die organisatorische Trennung zwischen den Prozessen \cite{lwakatare_devops_2019} oder durch die Fehlkommunikation zwischen den Teammitgliedern\cite{ebert_devops_2016} verursacht werden. 

Die daraus resultierenden Vorteile, die sich bei der Verwendung von DevOps ergeben, reichen von einer schelleren Produktbereitstellung und Problemlösung, bessere Ressourcenauslastung und Automatisierung bishin zu einer stabileren Betriebsumgebung.    

Insgesamt soll das Ziel verfolgt werden, die Softwarebereitstellung kontinuierlich sicherzustellen um so reaktionschnell auf Veränderungen am Markt oder Kundenanforderungen reagieren zu können.

Gründe für die Einführung von DevOps können vielschichtig sein.

Wie bereits beschrieben, ist es für Teams innerhalb der Entwicklung nicht möglich, neue Softwareversionen freizugeben oder Softwareänderungen schnell vorzunehmen, wenn der Betrieb die jeweiligen Funktionen nur langsam bereitstellen kann. \cite[S. 7,8]{sharma_devops_2017} 

Die Folgen sind eine verspätete Bereitstellung von Releases, Fehler in den Releases oder eine fehlende Dokumentation. \cite[S. 24]{alt_innovationsorientiertes_2017}

Hinzu kommen Probleme wie mangelndes Knowhow von Entwicklen über die Betriebnahme der verwendeten System, mangelndes Vertrauen in den Ops-Bereich bei fehlender Stabilität der Systeme, verzögertes Testen an Funktionalitäten oder Verschwendung durch mangelnde Wiederverwendung von Quellcode. \cite{humble_why_2011}  

Vor diesem Hintergrund stehen die Bereiche der Entwicklung und des Betriebs oft im Zielkonflikt "Agilität vs. Stabilität" und sehen sich mit verschiedenen Hindernissen konfrontiert, darunter unbefriedigende Testumgebungen und schlechter Informationsfluss. \cite{lwakatare_devops_2019} \cite[S. 8]{sharma_devops_2017}, \cite{konig_devopswelcome_2019}

Die Abbildung 1 zeigt den chronischen Konflikt, der oftmals innerhalb der IT-Organsisation herrscht, der durch fehlende Interaktion zwischen diesen Bereichen enstehen, die häufig unterschiedliche Ziele und Prozesse verfolgen. \cite[S. 349 - 350]{kim_devops-handbuch_2017}

\begin{figure}[h]
    \centering
    \includegraphics[scale=0.6]{Bilder/Core Conflict Clouds}
    \caption{Zentraler chronischer Konflikt nach Gene Kim \cite[S. 349]{kim_devops-handbuch_2017}}
\end{figure}

Mittels DevOps sollen die Interessen aller an der Bereitstellung von Software Beteiligten ausglichen werden, mit besonderem Schwerpunkt auf Entwicklern, Testern und Betriebspersonal. \cite{humble_why_2011}

Die querschnittlich aufgestellten Teams, lösen sich aus abgegrenzten Organisationseinheiten und Verantwortungsbreichen, aus den sogenannten 'Silos' und treiben gemeinsame Ergebnisse durch eine effektive Zusammenarbeit voran. \cite[S.5]{halstenberg_devops_2020} \cite{sollner_devops_2017}

Generell lässt sich die Werte von DevOps in dem bekannten Akronym CALMS zusammenfassen: Kultur (Culture), Automatisierung (Automation), Schlankheit (Lean), Messung (Measurement) und Teilen (Sharing).  

Der Aspekt der Kultur beinhaltet zunächst die Ausrichtung nach dem Menschen.

Hierbei spielt die Kollaboration als ein funktionsübergreifendes Team und die Orientierung nach Kundenwünschen eine tragende Rolle für eine DevOps-Organisation. \cite[S.5]{halstenberg_devops_2020} 

Die Automatisierung von Entwicklung, Implementierung und Tests ist der Schlüssel zum Erreichen niedriger Vorlaufzeiten und damit zu schnellem Feedback. \cite{humble_why_2011}

'Lean' steht in diesem Zusammenhang für die Vermeidung von Verschwendung jedlicher Ressourcen, Wertgeneration, Transparenz und ganzheitliche Betrachtung und Optimierung von Prozessen.

Der Faktor der Messung orientiert sich an den messbaren Daten, um den Fortschritt eines Unternehmens zu begutachten. 

Die definierten und erhobenen Kennzahlen reichen dabei von den Verfügbarkeiten und den Zeiten für die Fehlerbehebung oder Codeänderungen bis zu den Zeiten für die Anforderungsänderungen. \cite[S. 7]{halstenberg_devops_2020}  

Die letzte Säule beschreibt das Teilen von Informationen, Wissen, Vorgehensweisen und Praktiken innerhalb eines oder zwischen verschiedenen Teams unterschiedlicher Abteilungen. \cite{halstenberg_devops_2020} 

Dabei soll eine Umgebung geschaffen werden, in der gegenseitige Austausch, die Kommunikation und die gemeinsame Nutzung im Vordergrund stehen.

Bei dem gesamten Vorteilen und Mehrwerten, die durch DevOps erreicht werden können, kann DevOps jedoch als kein Nullsummenspiel oder Selbstläufer gesehen werden. \cite{humble_why_2011} 

Der DevOps-Ansatz benötigt zunächst einen kulturellen Wandel, was eine große Herausforderung für viele Unternehmen darstellen kann. 

Standardisierte Herangehensweisen sind von einer Vielzahl von Unternehmen über die Jahre tiefgreifend verankert worden, wodurch Mitarbeiter ihre gewohnten Arbeitsabläufe anpassen müssten. 
 
Dies reicht von der Erlernung neuer Tools, Technologien und Methoden, Aufbau einer Kommunikation für den gegenseitigen Austausch, vollständige Automatisierung, Verschmelzung etablierter Rollen und Zuständigkeiten, Schwierigkeiten bei der Implementierung eines automatisierten Deployment-Prozesses oder die Übernahme neuer Aufgaben und Verantworlichkeiten. \cite{lwakatare_devops_2019}, \cite[S. 594 - 595]{abrahamsson_product-focused_2016}, \cite[S. 43 - 45]{halstenberg_devops_2020}

Da sich die Bedeutung von DevOps in den letzten Jahren verschoben hat und immer wieder neue Tools für DevOps auftauchen, handelt es sich bei DevOps um eine stetige Weiterentwicklung. \cite[S. 595]{abrahamsson_product-focused_2016} 

Daher gibt es keinen Standard fester Praktiken im Zusammenhang mit DevOps, wodurch sich nicht festlegen lässt, welche Praktiken für DevOps eingesetzt werden sollten. 

%Die Trennung zwischen Projekten und Betrieb ist zu einer schwerwiegenden Einschränkung geworden, einerseits für die Fähigkeit von Unternehmen, neue Funktionen schneller auf den Markt zu bringen, andererseits für die der IT, stabile und qualitativ hochwertige Systeme und Dienste zu warten. \cite{humble_why_2011} 

%Ersetzen von Softwareentwicklungsmethode durch Kultur, Bewegung oder Praxis.
%Hinzufügung des Hinweises auf Automatisierung

%In diesem Abschnitt werden die grundsächlichen Merkmale von Devops beschrieben. Zudem werden wesentliche Vorteile beschrieben, durch die die Integration von DevOps möglich sind. Des weiteren wird auf das CALMS-Modell (Culture, Automation, Lean, Measurement und Sharing) eingegangen.

%DevOps gilt nicht als ein Nullsummenspiel, indem die Bereitstellungen häufig und zuverlässig in einer stabilen Produktumgebung erreicht werden, sondern es ist ein Ansatz zur Behebung der genannten Probleme durch Kultur, Automatisierung, Schlankheit, Messung und gemeinsame Nutzung, welches auch als CALMS bekannt ist. \cite{humble_why_2011}

\subsubsection{DevOps-Kultur}
Durch die intensive Interaktion der Bereiche Entwicklung und Betrieb müssen unterschiedliche organisatorische und kulturelle Veränderungen durchgeführt werden, damit die DevOps-Kultur etabliert werden kann. 

In diesem Abschnitt wird zunächst auf die wesentlichen Bestandteile und die Ziele der Kultur eingegangen, damit die Zusammenarbeit zwischen den beiden Bereichen gelingt.

In diesem Rahmen baut DevOps auf den Hauptprinzipien, als das Model der \textit{'The Three Ways'} auf, die von dem Autor und Erfinder des DevOps-Ansatzes Gene Kim \cite{kim_devops-handbuch_2017} definiert wurden. 
\dwi{ist das glasklar belegbar, das dieser Kim der urheber/namensgeber ist das ist eine starke aussage, die ich
ansonsten weiter oben im kontext "devops geht zurück auf Gene Kim und XYZ"}

Dieses Model der drei Wege bilden die zugrunde liegenden Prinzipien von DevOps ab, indem das Verhalten und die Muster von DevOps näher beschrieben werden. \cite[S. 9 - 44]{kim_devops-handbuch_2017}, \cite{kim_three_2012}  

Zunächst bildet der erste Weg die Grundlage für DevOps ab und betont die Leistung des gesamten Systems, im Gegensatz zur Leistung eines einzelnen Teams, Silos oder Abteilungen. 

Der Fokus liegt hierbei auf einem schnellen Arbeitsfluss des gesamten Systems, der durch die IT ermöglicht wird. 

In diesem Sinne müssen sich Opimierungen auf das gesamte System auswirken und sich nicht auf einen bestimmten Bestandteil beziehen. 

Das Ergebnis einer lokalen Veränderung zeigt meist keine Wirkung und würde im schlechteren Fall eine Verschlimmerung der Gesamtsituation herbeiführen. \cite[S. 252]{tiemeyer_handbuch_2021} 

Zudem sollte eine Veränderung, die ausschließlich stromab eine Verbesserung innerhalb des Wertestroms zeigt und stromab stark erschwert, kann nicht förderlich für das gesamte System sein. \cite[S. 252]{tiemeyer_handbuch_2021}
\dwi{und stromab stark erschwert ?}

Den Anfang stellt der Kunde dar, über die Entwicklung bis hin zu den Operations.

Dabei wird das Produkt, basierend auf den identifizierten Anforderungen des Kundenanforderungen, von der Entwicklung erstellt und in den Betrieb übergeben, wo das Ergebnis dem Kunden ausgeliefert wird.\cite[S. 12]{halstenberg_devops_2020} 

Der erste Schritt greift zudem den Grundgedanken des Lean-Ansatzes auf, indem ein Fehler nicht an nachfolgede Arbeitseinheiten weitergegeben werden darf um damit einen verbesserter Arbeitsfluss aufrecht gehalten werden kann. \cite[S. 252]{tiemeyer_handbuch_2021}  

Daher sind alle Arbeiten während dieses Schrittes sichtbar und in kleinen Aufgaben aufgeteilt, die in bestimmten Intervallen ausgeführt werden. 

Zu den Zielen des Ersten Weges gehört, dass bekannte Fehler an nachfolgende Arbeitsplätze nicht weitergegeben werden, dass eine lokale Optimierung niemals zu einer globalen Verschlechterung führt und dass versucht wird, ein tiefes Verständnis des gesamten Systems zu erlangen. \cite{kim_three_2012}  

\begin{figure}[h]
    \centering
    \includegraphics[scale=0.6]{Bilder/First Way.png}
    \caption{Der erste Weg: Systemdenken \cite{kim_three_2012}}
\end{figure}

Der zweite Weg beschreibt das Gestalten von effizienten Feedbackschleifen, um einerseits Fehler schnell und frühzeitig zu erkennen und zu beheben und andererseits den Prozess durchgängig im Auge zu behalten.

Der zweite Schritt stellt den eigentlichen zentralen Ansatzes von DevOps dar, wobei das Feedback von verschiedener Natur sein kann. \cite[S. 254]{tiemeyer_handbuch_2021} 

Dies sollte sowohl auf Seiten von einzelnen Teams untereinander als auch zwischen Entwicklung und Betrieb eingebettet werden. \cite[S. 94]{ravichandran_devops_2016}

Denn je schneller Probleme oder Auswirkungen kommuniziert werden, desto besser kann eine Vorgehensweise festgelegt werden. 

Damit soll verhindert werden, dass Fehler kein zweites Mal auftreten, sich kontinuierlich aufbauen können und sich möglicherweise auf neue Aufgaben auswirken. 

Durch die erleicherte Kommunikation zwischen den einzelnen Teamitgliedern als Basis von Feedbackschleifen, führt dies zu einer erhöhten Effizienz des gesamten Teams. 

Zudem können neue oder geänderte Kundenanforderungen mittels schneller Feedbackschleifen an die Entwicklung weitergegeben werden, um möglichst zeitnah die geeigneten Maßnahmen treffen zu können.
\dwi{woher kommen die kundenanforderungen? das kann ja fachlich neu oder erkenntnisse aus dem betrieb/nutzung der anwendung sein. letzteres ist ja hier gemeint, das sollte ggf klargestellt werden.}

Entscheidend ist, dass das Hinzufügen weiterer Kontrollschritte und schwerfällige Genehmigungsprozesse innerhalb großer Arbeitssyssteme die Wahrscheinlichkeit für zukünftige Fehler maßgeblich erhöht. \cite[S. 31]{kim_devops-handbuch_2017} 

Daher müssen DevOps-Teams die Freiheit haben, schnell Fehler zu beseitigen und ihre Arbeitsweise ohne organisatorische Hürden durchführen zu können.   

Die Ergebnisse des zweiten Wegs sind einerseits die Sicherstellung einer verbesserten Qualität und die Sicherheit des Arbeitssystems und andererseits die Möglichkeit des Aufbaus neuen Wissens.

\begin{figure}[h]
    \centering
    \includegraphics[scale=0.6]{Bilder/Second Way.png}
    \caption{Der zweite Weg: Feedback-Schleifen verstärken \cite{kim_three_2012}}
\end{figure}

Während sich der erste Weg technische und der zweite Weg organisatorische Aspekte beinhaltet, zielt der dritte Weg auf die Etablierung kultureller Änderungen ab um Innovationen weiter voranzutreiben. 

Dieser beschreibt die Schaffung einer Kultur die einen dynamischen Ansatz zum kontniuierlichen Experimentieren und firmenweites Lernen ermöglicht. 

\begin{figure}[h]
    \centering
    \includegraphics[scale=0.6]{Bilder/Third Way.png}
    \caption{Der dritte Weg: Kultur des kontniuierlichen Experimentierens und Lernens \cite{kim_three_2012}}
\end{figure}

Mittels dieses Ansatzes kann tiefer in die Materie und damit Risiken eingegangen werden, um schwierige oder versteckte Probleme identifizieren zu können. 

Infolgedessen sollen Fehler nicht nur aktiv beseitigt, sondern proaktiv danach gesucht und verbessert werden. \cite[S. 255]{tiemeyer_handbuch_2021}

Für eine erfolgreiche Durchführung des dritten Weges, sollte die etablierte Kultur auf Vertrauen beruhen. 

Je höher der Grad des Vertrauens jedes Mitarbeiters, desto wahrscheinlicher und schneller ist die Weitergabe von gewonnenen Erkenntnissen. \cite[S. 357]{kim_phoenix_2014}

Gemäß der DevOps-Kultur entfällt insbesondere die Bewertung von entdeckten Fehlern, sondern die Möglichkeit von diesen zu lernen und sich zu verbessern, rückt in den Vordergrund. 

Dieser Ansatz steigert die Motivitation jedes Mitgliedes eines DevOps-Teams und fördert das Experimentieren und daraus ableitende Eingehen eines Risikos, wodurch ein globales Lernen ermöglicht wird. 

Zu den Ergebnissen des Dritten Weges gehören die Einplanung von Zeit für die Verbesserung der täglichen Abläufe, die Schaffung von Routinen, die das Team für das Eingehen von Risiken honorieren und die Einbeziehung von Fehlern in das System, um die Belastbarkeit zu erhöhen.


%Durch die Interaktion der beiden Bereiche Development und Operations gehen auch organisatorische Änderungen mit einher, damit eine umfassende DevOps-Kultur geschaffen werden kann. In diesem Abschnitt wird insbesondere auf die Ziele dieser Kultur eingegangen, damit die Zusammenarbeit zwischen den beiden Bereichen gelingt. In diesem Rahmen wird unter anderem auf die wesentlichen Aspekte/Werte wie Kontniuierliches Lernen, Experimentieren, Ingenieurskultur, Kultur der Effektivität, Produktdenken oder die Übernahme von Verantwortung eingegangen. Auch wird der Vergleich und die Nachteile des traditionellen Silodenkens aufgegriffen. 

\subsubsection{Entwicklungszyklus innerhalb DevOps}
Nachdem die grundsätzliche Idee, Prinzipien und die etablierte Kultur von DevOps beschrieben worden sind, wird in diesem Absatz näher auf die inhaltlichen Aspekte von DevOps eingegangen.

Der DevOps-Lebenszyklus besteht aus einer aus mehreren iterativen und häufig automatisierten Workflows, die innerhalb eines größeren, iterativen und automatisierten Zyklus ausgeführt werden.

Grundsätzlich baut der Entwicklungszyklus innerhalb einer DevOps-Umgebung auf sieben Phasen auf, wie in der unteren Abbildung zu erkennen ist.

Der DevOps-Zyklus verläuft stets in einer Schleife, wodurch sowohl der Prozess als Ganzes als auch die Phasen sich untereinander durchgehend wiederholen, um eine stetige Neuimplementierung, Weiterentwicklung, Zusammenarbeit und Feedback sicherzustellen. 

Aufgrund der Iteration es Zykluses können Fehler in den einzelnen Phasen frühzeitig erkannt und beiseitigt werden und tragen zur Verbesserung der gesamten Phase bei. 

Die Abbildung zeigt die Abgrenzung zwischen den Mitgliedern des Development- und den Operationsteam.

Jede Seite trifft einerseits auf seperate und andererseits auf ständig zusammenarbeitende Teams zum Ziel die Geschwindigkeit, Zuverlässigkeit und die Qualität des Prozesses zu gewährleisten. 

\begin{figure}[h]
    \centering
    \includegraphics[scale=0.6]{Bilder/DevOps Lebenszyklus.png}
    \caption{Entwicklungszyklus innerhalb einer DevOps-Umgebung, angelehnt an \cite[S. 16]{halstenberg_devops_2020}}
\end{figure}

\paragraph{Plan}

In dieser Phase werden zunächst die wesentliche Anforderungen anhand der Bedürfnisse aller Stakeholder oder durch erhaltenes Feedback festgelegt, mit dem Ziel die Entwicklung und die Auslieferung entsprechend zu planen. \cite[s. 16]{halstenberg_devops_2020}  

Hinzu kommen die definierten Problembeschreibungen als auch der Umfang. 

Ziel ist es die entsprechenden Ressourcen und die Entwicklungen während des ganzen Zeitraums zu planen und einzuteilen. 

Darüber hinaus können Entwickler sich einen Überblick über die Verwendung von Systemen, Features, Funktionalitäten Risiken und Einschränkungen verschaffen. \cite{yarlagadda_devops_2021} 

Auch die Durchführbarkeit und Machbarkeit wird über den gesamten Zeitraum besprochen und geplant. 

Innerhalb dieser Phase kommen Methoden der agilen Softwareentwicklung wie Scrum, Kanban oder Extreme Programming zum Einsatz.

So kann mittels eines Kanban-Boards jeder Arbeitsschritt angefangen bei einem definierten Zustand bis zur Erledigung dieser Aufgabe, anhand vertikalen Lanes als mögliche Zustände, als Karte bewegt werden. \cite[S. 88]{huttermann_devops_2012}  

Durch die Planung und Koordination mittels Kanban-Board können nicht reine Entwicklungstätigkeiten, sondern auch Schwerpunkte im Hinblick auf den Betrieb zur Infrastruktur oder Wartung abgebildet werden. \cite{schaefer_devops_2017} 

In diesem Zuge können Schwachstellen und Engpässe schnell aufgedeckt werden, wodruch der gesamte Prozessfluss optimiert wird. 

Zur kurzfristigen Kontrolle können die auf Scrum basierenden Planungsgrundsätze verwendet werden, da dieses auf zeitlich begrenzte Iterationen und die Verteilung von Rollen und Zuständigkeiten setzt.

Im Rahmen von Scrum werden zunächst alle am System zu erledigenden Arbeiten im 'Release Backlog' festgehalten. 

Während der Planung für den Sprint werden Features und Funktionen aus dem Release Backlog ausgewählt und in das 'Sprint-Backlog' oder nach Prioritäten gesetzten Aufgaben aufgenommen, die im nächsten Sprint abgeschlossen werden sollen. \cite{cohen_introduction_2004}

Infoldessen können mögliche Entwicklungsaufwände auf einen zeitlichen Umfang versehen ('Timeboxing'). \cite[s. 17]{halstenberg_devops_2020}

Die Organisation des Teams findet in kleinen Sitzungen und monatlichen Sprints oder Iterationen statt. 

Wichtige Gemeinsamkeit beider Methoden ist die Arbeit nach dem Pull-Prinzip, wobei Tasks eigenständig bearbeitet werden, wenn die entsprechende Kapazität zur Verfügung steht. \cite{concas_agile_2007} 

Der Hauptunterschied zwischen beiden Methoden besteht im wesentlichen in der Zielsetzung und des zeitlichen Horizonts. 

Während Scrum sich auf die iterative Produktentwicklung konzentiert und das Produkt von Teration zu Iteration erweitert und verbessert, arbeitet die Kanban-Methode auf die kontinuierliche Verbesserung der Prozesse, kürzere Vorlaufzeiten und Vermeidung von Verschwendung, hin. \cite{concas_agile_2007}

Zudem legt Scrum den aktuellen Arbeitsfortschritt und die Planung auf die folgenden Sprints fest, während sich die Planung bei Kanban auf mehrere Wochen bezieht. \cite[s. 5]{verona_practical_2016} 

Insbesondere die messbaren und eindeutigen Vorteile von DevOps zeigen sich innerhalb kürzerer Zyklen, die wiederum die längeren Zyklen effizienter machen. 

\paragraph{Code und Build}

Nachdem die wesentlichen Aufgaben zugewiesen worden sind, stellt die hautpsächliche Akitivität innerhalb dieser Phase die Entwicklung dar. 

Störende Schritte wie beispeilsweise das Testen, die Inbetriebnahme oder die Berücksichtigung der Anfoderungen sollten in dieser Phase bereits berücksicht sein. \cite[S. 18]{halstenberg_devops_2020}  

Die Forschritte des entwickeltenden Programmcode werden laufend in Sprint Reviews, Daily Scrum Meetings kommuniziert an das gesamte DevOps-Team kommuniziert. 

Um eine einheitliche Basis zu schaffen, arbeiten alle Entwicklerteams mit gemeinsamen Tools und Plugins und legen einheitliche Vorgaben für die Qualität des Quellcodes fest. 

Nachdem die Aufgabe abgeschlossen ist, wird der neu entwickelte und lauffähige Programmcode in kleinen Bestandteilen an ein Versionsverwaltungstool ('Code Repository') übergeben.\cite[S. 18]{halstenberg_devops_2020}  

Diese Vorgehensweise wird auch als Push bezeichnet, der wiederrum einen Pull-Request auslöst. 

In diesem Zusammenhang erfolgt ein Code-Review, bei dem der Pull-Request bestätigt wird sobald der Quellcode den Anforderungen funktional entspricht.

Analog zum Pull-Request finden automatisierte Testfälle und die (ausführbare) Konfguration der Test- und Produktionsumgebung statt. \cite[S. 18]{halstenberg_devops_2020}  

Ist das Testen erfolgreich durchlaufen, kann der Quellcode übernommen werden. 

Zudem werden alle Veränderungen am Programmcode, neue Entwicklungen oder Fehleranfälligkeiten dokumentiert und für das gesamte DevOps-Team zur Verfügung gestellt.

\paragraph{Test}

Neben der Entwicklungsphase erfolgt innerhalb der DevOps-Zyklen eine intensive Testphase, die in einer eigenen Ungebung durchgeführt wird und sich vor den Phasen der Intergration und der Bereitstellung befindet. 

Während dieser Phase werden intensive Tests der Anwendungen und Funktionalitäten innerhalb einer eigenen Serverumgebung durchlaufen, die auch als 'Staging Environment' bezeichnet werden. \cite[S. 16]{verona_practical_2016} 

Dabei wird eine Umgebung geschaffen, die sich ähnlich der Produktionsvariante verhält, auf die die finale Version bereitgestellt wird und daher ein Testen unter annährend realen Bedingungen erlaubt. \cite[S. 5.2/5.6]{bass_devops_2015}

Durch die Verwendung von realen Daten sollen Funktionaliäten einerseits unter Berücksichtigung der Infrastruktur geprüft und andererseits simuliert werden, wie sich das System unabhängig von der Entwicklungsumgebung verhält. \cite[S. 16]{verona_practical_2016} 

Während die Entwicklerteams Unit- und Integrationstests durchführen, beteiligt sich der Betrieb an Integrations- und Lasttests, um die Betriebsbereitschaft zu beurteilen. \cite[S. 127]{sturm_devops_2017}  

Neben den automatisierten Tests finden darüber hinaus auch manuelle Tests statt, um Schachstellen oder Risiken in Bereichen wie Sicherheit innerhalb der Anwendung zu identifizieren und zu beseitigen.

\paragraph{Release und Deploy}

Die Phase des Release beschreibt den Prozess der Vorbereitung für die Bereitstellung des Builds an entwickelten Funktionalitäten in die produktive Umgebung. \cite[S. 20]{halstenberg_devops_2020} 

Innerhalb dieser Phase entscheidet sich welche Änderung in dem Release enthalten sein soll. 

Abhängig des Reifegrades des Release-Prozesses kann die Übergabe sowohl manuell oder automatisch erfolgen, wobei Entwickler die Möglichkeit haben, Funktionalitäten für den Kunden zu deaktivieren bis diese einsatzbereit sind. \cite{thedev_eight_2019}

In diesem Zuge können neue Versionen innerhalb fester Zeiträume erfolgen oder automatisiert, nach erfolgreicher Übergabe des Quellcodes durch die Testphase. \cite{thedev_eight_2019}

Innerhalb der Phase des Deploys findet das eigentliche Ausrollen des neuen Builds statt.  

Grundsätzlich kann dieser Schritt automatisiert durchgeführt werden, damit es keine Einschränkungen des laufenden Betriebs gibt. \cite{thedev_eight_2019}  

Bei Problemen innerhalb des Deployments, kann der letzte Stand aus der Produktivumgebung wiederhergestellt werden, indem die neue Umgebung parallel zur bestehenden Produktionsumgebung aufgebaut wird. 

Häufig kann die Phase des Deploys mit der Phase des Releases übereinstimmen, obwohl diese Phase im Wesentlichen nur die Auslieferung der getesteten Software in die Produktivumgebung beschreibt. \cite[S. 20]{halstenberg_devops_2020}   

Im Rahmen dieser Phase sind zwei grundlegende Rollen verbreitet. \cite[s. 20]{halstenberg_devops_2020} 

Zum einen trifft der Release Coordinator die Entscheidungen über den Produktiveinsatz eines Releases und überwacht den Entwicklungsfortschritt. 

Zum anderen überwacht der Site Reliability Engineer die IT-Services, die notwendigen Tools und Abhängigkeiten innerhalb der Inbetriebnahme des Releases. 

\paragraph{Operate}

Diese Phase beschäftigt sich hauptsächlich um die Wartung und den Support, nachdem die Änderungen live gegangen sind. 

In diesem Rahmen werden einerseits Lastspitzen oder Tiefpunkte anhand der aktiven Nutzerzahl automasiert und effektiv abgefangen und andererseits benötigte Ressourcen zur Verfügung gestellt, um die Produktivumgebung erfolgreich zu betreiben. \cite{thedev_eight_2019}    

Zudem kann der Kunde Feedback zu geben, welches gesammelt und ausgewertet wird, um das Nutzerverhalten der User besser zu verstehen und die zukünftige Entwicklung zu verbessern. 

Häufig wird als finale Phase, die Monitor-Phase zusätzlich in den DevOps-Lebenszyklus aufgenommen, wobei diese oftmals in die Operate-Phase mitaufgenommen werden kann.

Dabei handelt es sich neben dem Kundenfeedback, um die Erhebung von weiteren Daten wie den auftretenden Fehlern, Leistungsverhalten, Zugriffszahlen oder Kapazitäten. \cite{thedev_eight_2019}    

Die erhobenen Informationen werden an die Produktmanager und die Entwicklungsteams oder iterativ an die Planungsphase weitergeleitet, um das nächste Feature am User abzuleiten. \cite[s. 21]{halstenberg_devops_2020} 

Damit kann sichergestellt werden, dass der aktuelle Zustand einer Anwendung überwacht und verwaltet wird, um sich auf Änderungen vorzubereiten und etwaige Fehler zu beheben. \cite[S. 127]{sturm_devops_2017} \\

















 






















%Nun wurde die Idee hinter dem DevOps-Ansatz wird in den vorherigen Kapiteln beschrieben. Als nächstes soll nun näher auf die inhaltlichen Themen von DevOps eingegangen werden, sowie der DevOps Life Cycle und dessen Phasen. Damit der Life Cycle nicht zu weitgreifend wird, wird lediglich der Life Cycle, der bei der msg systems ag verwendet wird, näher beschrieben. Damit sollte der Leser ein Verständnis für die weiteren Punkte im Proof Of Concept erlangen.\\




%Gemäß Fluri u.a. \cite[S. 259 - 283]{tokarski_strategische_2018} muss ein optimaler DevOps-Prozess drei Voraussetzungen erfüllen, um erfolgreich etabliert zu werden. Erste Voraussetzung bilden zunächst die \textit{automatisierten und definierten Prozesse}. Um Prozesse optimiert und effizient zu etablieren und zu automatisieren, bedarf es zunächst einer umfassenden Definition über die Funktionsweise und den Ablauf dieses Prozesses. Hinzu kommt eine \textit{integrierte Infrastruktur(Toolchain)}. Anhand technischer Komponenten als geeignete Werkzeuge, die dem DevOps-Team zur Verfügung gestellt werden, kann die Automatisierung optimal abgebildet werden. Letzte grundsätzliche Voraussetzung stellt das \textit{Rahmenwerk von Aktivitäten für die Teams} auf. Diese Voraussetzung zielt auf die Etablierung einer DevOps-Kultur, die bereits im vorherigen Kapitel dargestellt wurde. Gemäß Fluri u.a. \cite[S. 259 - 283]{tokarski_strategische_2018} wird der gesamte Entwicklungszyklus in DevOps durch acht Schritte und vier Prozessgebiete abgebildet. 

\subsubsection{Continuous Everything als DevOps-Methoden}
Während sich agile Praktiken im Hinblick auf die Softwareentwicklung auf eine kontinuierliche Planung, Flexibilität und eine schnelle Reaktion auf sich ändernde Kundenanforderungen fokussieren, können DevOps-Praktiken dazu verwendet werden, den Arbeitsfluss vom Kunden über die Entwicklung, den Betrieb und zurück kontinuierlich auszubauen und damit die Qualität und Belastbarkeit der Software zu steigern \cite{fitzgerald_continuous_2014} \cite[S. 264]{tokarski_strategische_2018}. Die \textit{'Continuous Everything'}-Methoden spiegeln die Idee einer kontinuierlichen Verbesserung und Automatisierung innerhalb eines Devops-Prozesses wieder. Wie in Abbildung 2.7 zu erkennen ist, können sich diese Methoden auf mehrere Entwicklungsphasen konzentrieren und werden in diesem Abschnitt im Einzelnen beschrieben.  

\begin{figure}[h]
    \centering
    \includegraphics[scale=0.5]{Bilder/Continuous Everything.png}
    \caption{Continuous-Methoden innerhalb des Devops-Lebenszykluses, angelehnt an \cite[S. 16]{halstenberg_devops_2020}}
\end{figure}

\subsubsection{2.1.6.1. Agile Entwicklung} $~$

Die agile Entwicklung beschreibt die Verwendung von agilen Methoden innerhalb des Softwareentwicklungsprozesses als eine wesentliche Grundvoraussetzung für den DevOps-Prozess. Oftmals wird dieser Prozess auch als Continuous Planning (dt. kontinuierliches Planen) bezeichnet und reicht von der Phase des Planes bis zur Phase des Builds \cite{fitzgerald_continuous_2014}. Ziel ist es, sicherzustellen, dass die Investitionsentscheidungen während des gesamten Lebenszyklus auf die Bedürfnisse des Kunden abgestimmt worden sind. 

\subsubsection{2.1.6.2. Continuous Integration} $~$

Die Methode des Continous Integration (dt. kontinuierliche Integration, kurz: CI) beschreibt grundsätzlich die Gewährleistung einer sicheren und lückenlosen Integration von Codeänderungen in die vorhandenen Umgebungen \cite[S. 266]{tokarski_strategische_2018}. Ziel ist es, die Qualität der Software sicherzustellen und schnelles Feedback über die Integrierbarkeit vor der Auslieferung zum Kunden zu erhalten \cite[S. 266]{tokarski_strategische_2018}. Kernelement stellt ein Versionsverwaltungssystem (auch: Repository) dar, dessen wesentliche Aufgabe es ist, den DevOps-Teams dabei zu helfen, den Code von mehreren Entwicklern zu organisieren, Änderungen zu verfolgen und automatisierte Tests zu ermöglichen. Zunächst werden neuer oder geänderter Code nach der Entwicklung und Prüfung regelmäßig und in möglichst kurzen Abständen in einem gemeinsamen Repository gemergt (dt. zusammengeführt) \cite[S. 13-16]{sharma_devops_2017}. In diesem Zuge wird der Code automatisiert in einem Build kompiliert. Die neu erstellten Artefakte, werden in eine lauffähige Umgebung integriert und automatisiert getestet um sicherzustellen, ob die neuen Codeänderungen einer Komponente innerhalb der gesamten Anwendung lauffähig sind.\\\\ Dies ist essentziell für den Prozess, da häufig viele Entwickler an der Codebasis mit leicht unterschiedlichen Versionen arbeiten und daher überprüft werden muss, ob die verschiedenen Änderungen richtig zusammenarbeiten \cite[S. 69]{verona_practical_2016}. Aufgrund des regelmäßigen Integrierens der Codeänderungen wird gewährleistet, dass häufige automatisierte Tests durchführt werden, den Entwicklern stets der aktuellste Code zur Verfügung steht und Entwickler nicht darauf warten müssen, einzelne Codeabschnitte am Tag der Veröffentlichung auf einmal zu integrieren \cite{thedev_eight_2019}. Durch die entstehende Flexibilität und Geschwindigkeit können Fehler schneller und leichter behoben werden, da die Programmbestandteile kleiner und weniger komplex sind und das Debugging insgesamt sinkt \cite{thedev_eight_2019}. Wie an der Abbildung zu erkennen ist, umfassen die CI-Schritte die Codekompilierung, die Durchführung von Unit- und Akzeptanztests, die Validierung der Codeabdeckung, die Überprüfung der Einhaltung von Codierungsstandards und die Erstellung von Bereitstellungspaketen \cite{fitzgerald_continuous_2014}. 

\subsubsection{2.1.6.3. Continuous Delivery} $~$

Bei dem Ansatz des Continuous Delivery (dt. kontinuierliches Ausliefern, kurz: CD) handelt es sich um die nächste Stufe des Continuous Integration. Die Methode der Continuous Delivery baut auf einer regelmäßigen und automatisierten Bereitstellung des Builds an den Testbereich, zur anschließenden Bewertung und einer potentiellen Freigabe, auf \cite[S. 16 - 18]{sharma_devops_2017}. Da regelmässig Builds durch die Continuous Integration erzeugt werden, müssen diese zeitnah in andere Umgebungen weitergeleitet werden \cite[S. 16 - 18]{sharma_devops_2017}. Voraussetzung ist der Aufbau einer Continuous-Delivery-Pipeline, mit dem Ziel, das Ausliefern der Software möglichst automatisiert für die Bereitstellung neuer Releases durchzuführen \cite[S. 10]{wolff_continuous_2016}. \\\\ Sobald ein neues Artefakt innerhalb des Repositorys übertragen wurde, wird die Continuous-Delivery-Pipeline ausgelöst, wie in Abbildung 7 zu sehen ist \cite[S. 14]{verona_practical_2016}. In diesem Rahmen werden die fehlerfreien Builds automatisiert innerhalb eines produktionsähnlichen Staging- oder Testbereich bereitgestellt, um nach dem Testen zu bewerten, wie sich die neu entstandene Version des Builds produktionsnah verhält und letztlich in die Produktion verlagert werden kann \cite[S. 16]{sharma_devops_2017}, \cite{thedev_eight_2019}. Falls das Testen in der Pipeline fehlschlägt, werden die Entwickler informiert und haben die Gelegenheit, kurzfristig Anpassungen an dem jeweiligen Build vorzunehmen oder diesen zu verwerfen. Die Methode des Continuous Delivery beinhaltet mehrere Vorteile für das gesamte DevOps-Team. So wird aufgrund des hohen Grades an Automatisierung der Release-Prozess maßgeblich verbessert, indem Risiken und Engpässe durch die häufige Auslieferung von kleinen Features vermieden werden können und damit ein kontinuierlicher Integrationsfluss sichergestellt werden kann \cite[S. 18]{wolff_continuous_2016}.

\subsubsection{2.1.6.4. Continuous Deployment} $~$

Die letzte Phase der Delivery-Pipeline ist das Continuous Deployment (dt. kontinuierliche Bereitstellung, kurz: CD). Kernaufgabe des Continuous Deployments ist die voll automatisierte Überführung des Codes in die Produktivumgebung mithilfe der Delivery-Pipeline \cite[S. 29]{alt_innovationsorientiertes_2017}. Aufgrund der automatisierten Freigabe des Releases, muss sowohl die Qualität als auch die Lauffähigkeit der Pipeline besonders gesichert sein und ist daher abhängig von der Phase des Continuous Delivery \cite[S. 269]{tiemeyer_handbuch_2021}. Continuous Deployment entspricht der höchsten Stufe einer Delivery-Pipeline, was den DevOps-Teams ermöglicht, kleinste Features und Änderungen automatisiert für den Anwender ausliefern zu können. Dies wäre theoretisch das höchste Ziel der voll automatisierten Softwareentwicklung \cite{humble_why_2011}. Durch diesen höchsten Grad können die Vorlaufzeiten niedrig gehalten und folglich schnelles Feedback erhalten werden \cite{humble_why_2011}. Das Ziel ist es, die Zeit bis zur Markteinführung von der Software zu verkürzen, indem jeder Commit in die Produktivumgebung bereitgestellt wird. Viele Entwickler und Unternehmen lehnen die Methode des Continuous Deployments jedoch ab, da es ein Risiko darstellt, wenn eingecheckter Code durch das Testing fehlgeschlagen ist und automatsiert in die Produktivumgebung bereitgestellt wird \cite[S. 269]{tiemeyer_handbuch_2021}. Um dieses Risiko möglichst gering zu halten, müssen einerseits nahezu produktionsreife Codes und robuste Test-Frameworks vorhanden sein und andererseits alle DevOps-Mechanismen zuverlässig arbeiten \cite[S. 269]{tiemeyer_handbuch_2021}.\\\\  Darüber hinaus verlangt der Continuous-Deployment-Ansatz eine starke Architekturaufsicht und Teamdisziplin, damit das Release nicht die Qualität oder den von den Kunden realisierten Nutzen in Mitleidenschaft zieht \cite[S. 119 - 120]{erder_continuous_2016}. Sowohl Continuous Delivery als auch Continuous Deployment werden in der Literatur oftmals synonym verwendet, da beide auf analogen Konzepten basieren. Der Unterschied zwischen beiden Methoden besteht darin, dass innerhalb des Continuous Deployment ein automatisiertes Ausliefern auf die Produktivumgebung erfolgt, während dies im Rahmen des Continuous Delivery manuell entschieden werden kann und wiederrum von den fachlichen Erfordernissen des Kunden abhängt \cite[S. 29 - 30]{alt_innovationsorientiertes_2017}. Allerdings ist, "`\textit{die Fähigkeit zur kontinuierlichen Bereitstellung wichtiger als die tatsächliche kontinuierliche Bereitstellung für die Produktion}"' \cite[S. 19]{sharma_devops_2017}. Damit schlussfolgert Sharma, dass Continuous Delivery ein Muss ist, aber Continuous Deployment als eine Option angesehen werden kann.  

\subsubsection{2.1.6.5. Continuous Monitoring} $~$

Die Vorgehensweise des Continuous Monitoring umfasst die durchgängige Überwachung, der zugrunde liegenden Infrastruktur und des im Betrieb befindlichen Quellcodes \cite{van_hoorn_continuous_2012}. Dabei stellt das Ops-Team sicher, dass die Anwendung in der Produktion funktioniert wie gewünscht und die Umgebung stabil läuft. Hierfür haben die Ops-Teams eigene Tools zur Überwachung ihrer Umgebung und laufenden Systeme, und zwar von der Prozessebene bis hinunter zu Ebenen, die niedriger sind, als es die Systemüberwachungstools erlauben würden \cite[S. 26]{sharma_devops_2017}. Oftmals werden Selbstüberwachungs- und Analysefunktionen direkt in die zu entwickelnden Anwendungen eingebaut, um eine kontinuierliche End-to-End-Überwachung zu gewährleisten \cite[S. 26]{sharma_devops_2017}. Neben der Anwendungs- und Systemleistung muss das Benutzerverhalten der Anwendung und die Benutzerzufriedenheit ebenfalls überwacht werden, um ein detailliertes Feedback zu erhalten \cite[S. 112 - 113]{erder_continuous_2016}. Wie anhand der Abbildung 7 zu sehen ist, kann dieses Feedback in die Phase der Entwicklung zurückfließen, um bessere Entscheidungen bei der Entwicklung der nächsten Änderung treffen zu können, Probleme zu beheben und neue Anforderungen und Funktionen zu berücksichtigen. Im Ergebnis bildet diese Feedbackschleife ein Instrument, zur kontinuierlichen Gestaltung und Orientierung für das Softwareprodukt.   

\subsubsection{2.1.6.6. Infrastructur-as-a-Code} $~$

Automatisierung gilt als eine Grundvoraussetzung innerhalb der Devops- Umgebung und erstreckt sich nicht nur auf die Bereiche der Entwicklung und Bereitstellung, sondern auch auf die zugrundeliegende Infrastruktur \cite[S. 272]{tiemeyer_handbuch_2021}. Insbesondere durch die Verwendung von Continuous Integration, ist die Anzahl der Umgebungen und ihrer Instanzen stark angestiegen, da täglich Builds getestet, validiert und bei Konfigurationsänderungen angepasst werden müssen \cite[S. 19]{sharma_devops_2017}. Obwohl die Vorgehensweise des \textit{Infrastructur-as-a-Code} (kurz: IaC) keine 'Continuous'- Bezeichnung besitzt, ist diese Praktik wesentlicher Bestandteil der gängigen DevOps-Methoden \cite[S. 30]{alt_innovationsorientiertes_2017}. Anstatt manuelle Änderungen durch einen Administrator, welcher schrittweise ein neues System einrichtet oder umkonfiguriert, durchzuführen, werden Netzwerkeinstellungen, Parameter und weitere Konfigurationen als Code in einer Konfigurationsdatei beschrieben \cite{juner_praxisbasierte_2017}, \cite{luber_was_2020}. Diese Datei wird in einem Repository zur Verfügung gestellt und kann bereits beim Aufbau einer Infrastrukturumgebung automatisiert erstellt und in die Entwicklung miteinbezogen werden. Durch die entstehende Versionierung sind alle Änderungen überprüfbar, reproduzierbar und bei Fehlern können Rollbacks auf die frühere Version durchgeführt werden \cite[S. 272]{tiemeyer_handbuch_2021}. \textit{"Dadurch kann jeder produktivähnliche Umgebungen in Minuten erhalten, ohne ein Ticket aufmachen oder gar Wochen warten zu müssen."} \cite[S. 107]{kim_devops-handbuch_2017}. Voraussetzung für IaC ist es, Systemadministratoren frühzeitig innerhalb des Softwareentwicklungsprozesses einzubeziehen und das Verständnis der Entwickler für die auf dem Produkt basierende Infrastruktur zu schärfen \cite[S. 30]{alt_innovationsorientiertes_2017}. Mittels IaC würde die alleinige Verantwortung für alle Phasen wie das Design, Umsetzung, Test, Installation und Betrieb bei einem DevOps-Team liegen \cite{kasteleiner_devops_2019}.







% Continuous Planning gilt als ein ganzheitliches Unterfangen, dass ein engere Intergration zwischen Planung und Ausführung erfordert und an dem sowohl Kunden als auch die Entwickler beteiligt sind. \cite{fitzgerald_continuous_2014} 

% Wie bereits in der Phase des Planens beschrieben, können agile Methoden wie Scrum und Kanban zum Einsatz kommen um die entsprechenden Ressourcen und die Entwicklungen während des ganzen Zeitraums zu planen und einzuteilen.

% Inerhalb dieser Methode werden Features in kleinen Inkrementen geplant und entwickelt, um diese innerhalb eines Sprints umzusetzen und folglich die Durchlaufzeit bis zur Auslieferung kurzzuhalten. \cite[S. 266]{tokarski_strategische_2018} 

% Infolgedessen beinhalten die Ergebnisse dieser Methode ausliefbare und getestete Funktionalitäten nach jedem Sprint. 




% Darüber hinaus werden Änderungen sichtbarer und bilden eine starke Grundlage für zukünftige Änderungen. 




% Aufgrund der steigenden Automatisierung der Pipeline als auch durch IaC werden die Umgebungen immer identischer, was widerrum dem Grundgedanken des Continuous Integration entspricht, da der Quellcode sehr früh auf eine produktionsnahe Umgebung integriert wird und dadurch Probleme schnell sichtbar werden. \cite[S. 111 - 113]{kim_devops-handbuch_2017} 

\subsubsection{Integrierte Infrastruktur (Toolchain)}
Nutzer von DevOps-Verfahren setzen im Rahmen ihrer DevOps-Toolchain oft bestimmte DevOps-freundliche Tools ein. Ziel dieser Tools ist es, die verschiedenen Phasen des Workflows zur Softwarebereitstellung (auch als Pipeline bezeichnet) noch stärker zu straffen, zu verkürzen und zu automatisieren. Viele derartige Tools sind auch an wesentlichen DevOps-Grundsätzen wie Automatisierung, Zusammenarbeit und Integration zwischen Entwicklungs- und Betriebsteam ausgerichtet. Hier werden einige gängige Tools beschrieben, die innerhalb der unterschiedlichen Phasen des DevOps-Lebenszyklus genutzt werden.\\


Planen: In dieser Phase werden der geschäftliche Nutzen und die geschäftlichen Anforderungen festgelegt. Tools wie Jira oder Git helfen dabei, bekannte Probleme nachzuverfolgen, und unterstützen das Projektmanagement.\\


Codieren: In dieser Phase stehen das Softwaredesign und die Erstellung von Softwarecode im Mittelpunkt. Tools dafür sind beispielsweise GitHub, GitLab, Bitbucket oder Stash.\\


Entwickeln. In dieser Phase werden Softwarebuilds und -versionen verwaltet. Automatisierte Tools unterstützen das Kompilieren und Packen von Code für künftige Produktionsfreigaben. Mithilfe von Quellcode-Repositorys oder Paket-Repositorys wird außerdem die zur Produktfreigabe benötigte Infrastruktur „verpackt“. Beispiel-Tools sind Docker, Ansible, Puppet, Chef, Gradle, Maven und JFrog Artifactory.\\


Testen. In dieser Phase wird durch kontinuierliches Testen (manuell oder automatisiert) eine optimale Codequalität gesichert. Beispiel-Tools sind JUnit, Codeception, Selenium, Vagrant, TestNG und BlazeMeter.\\


Implementieren. In dieser Phase können Tools genutzt werden, die das Managen, Koordinieren, zeitbezogene Planen und Automatisieren von Produktversionen für die Produktion unterstützen. Beispiel-Tools sind Puppet, Chef, Ansible, Jenkins, Kubernetes, OpenShift, OpenStack, Docker und Jira.
Betrieb. In dieser Phase geht es um das Management der Software während der Produktion. Beispiel-Tools sind Ansible, Puppet, PowerShell, Chef, Salt und Otter.\\


Überwachen. In dieser Phase werden Informationen über Probleme mit bestimmten Softwareversionen in der Produktion erkannt und erfasst. Beispiel-Tools sind New Relic, Datadog, Grafana, Wireshark, Splunk, Nagios und Slack.


%Working: Last step, tbd
\subsection{OSS}
Obwohl Mitte des letzten kommerziellen Jahrhunderts die Dominanz innerhalb des Softwaremarktes bei vertriebener Software lag, gewann die Rolle der Open-Source-Software (OSS) zunehmend an Bedeutung, insbesondere durch die offene Bereitsstellung und Entwicklung des freien Betriebssystems 'Linux' \cite[S. 8 - 11]{wichmann_linux-_2005}, \cite[S. 1]{will_open-source-software_2003}. Bereits im Jahre 2019 lag der Einsatz von OSS ab einer Unternehmensgröße von über 2.000 Mitarbeitern bei 86 \%, was bei dieser Unternehmensgröße jedem zweiten Unternehmen entspricht \cite[S. 15]{bitkom_open_2020}. Insgesamt enstand durch wettbewerbsfähige OSS ein grundlegender Umbruch in der Softwarebranche innerhalb der letzten Jahre \cite[S. 185]{bitzer_entwicklung_2007}, \cite{fitzgerald_transformation_2006}. Unter OSS versteht man einen öffentlich zugänglichen Quellcode, den jeder einsehen, verändern und für sich nutzen kann. Durch die Idee der freien Verfügbarkeit des Quellcodes, der entfallenden Lizenzkosten für den Einsatz und der freien Weiterentwicklungsmöglichkeiten, erweist sich OSS als eine einfache und kostengünstige Innovationsquelle für Unternehmen und folglich als eine realistische Alternative zu proprietärer Software \cite[S. 21,22]{allmann_open_2019}. Das Potential und der resultierende Nutzen kann sowohl für Entwickler als auch für Unternehmen sehr vielschichtig sein. Obwohl für ein Unternehmen das Hauptaugenmerk des Einsatzes von OSS auf einen wirtschaftlichen Gewinn in Form von gesenkten Lizenz- und Entwicklungskosten und von Entwicklungszeit abzielt, profitieren Entwickler von dem Wissensaustausch, der einfachen Handhabbarkeit und der Flexibilität, die OSS mit sich bringt \cite{lerner_economic_2005}. Zunächst können Entwickler einerseits Softwarefunktionalitäten nach unternehmensinternen Prozessabläufen abgestimmt entwickeln und andererseits durch den unmittelbaren Zugang zum Quellcode frühzeitig überprüfen, ob ein wiederkehrendes Problem bearbeitet oder nach der jeweligen Problematik individualisiert werden kann. Durch die verfügbaren Standardfunktionalitäten, kann die OSS neben der Möglichkeit der Weiterentwicklung von bereits bestehender Software, dazu verwendet werden, eine Basis für weitere Entwicklungen zu schaffen \cite[S. 37/38]{kesler_anpassung_2013}. Ferner bietet der offene Standard von OSS ein hohes Maß an Anwendungsfeldern, da Entwickler mit wenig Aufwand offenliegende Schnittstellen implementieren können und folglich innerhalb der Hardwareauswahl flexibel und unabhängig sind \cite[S. 2]{kesler_anpassung_2013}, \cite[S. 21,22]{allmann_open_2019}. Durch die Verwendung von OSS sind Entwickler stark unabhängig von Software und von großen Anbietern. Im Gegensatz zu proprietärer Software können Schwachstellen und Sicherheitslücken durch das frühzeitige Testen schneller aufgedeckt und analysiert werden \cite[S. 30/31]{kees_open_2015}.\\\\ Eine weitere wichtige Rolle des Entwicklungsprozesses mittels OSS ist die Wiederverwendung von Quellcode. Anstatt einer zeit- und kostenintensiven Neuentwicklung von Software, bleiben viele allgemeine oder wiederholende Funktionalitäten innerhalb eines neuen Projektes oftmals gleich und müssen demnach nicht verändert werden \cite{henkel_code_2010}. Entwickler versuchen nicht 'das Rad neu zu erfinden', sondern suchen gezielt nach Lösungen von bereits bekannten Problemen. Vor diesem Hintergrund sparen Entwickler viel Zeit und Ressourcen, indem Quellcode, Templates oder Algorithmen wiederverwendet werden, was in der Vergangenheit durch eine maßgeschneiderte Implementierung in kleinem Maßstab erfüllt worden wären \cite{spinellis_how_2004}. Obwohl die Verwendung von OSS auf der einen Seite an Popularität zunimmt, kann auf der anderen Seite zu erheblichen Risiken führen. Insbesondere die Folge der Wiederverwendung kann eine massive Verschachtelung und viele transitive Abhängigkeiten von OSS-Komponenten voraussetzen, durch die die Komplexität erheblich steigt \cite{thelen_beschleunigung_2021}. Jede dieser Komponenten hat eine oder mehrere Lizenzen zum Ziel von rechtlichen Schutzmaßnahmen für den Lizenzgeber, die die Nutzungsbedingungen der jeweiligen Komponente spezifizieren und stark einschränken. Durch die Einbeziehung unterschiedlicher Lizenzmodelle in die OSS-Projekte sollen die Rechte des Urhebers geschützt und die damit uneingeschränkte lizenzfreie Nutzung ohne die jeweilige Genehmigung beschränkt werden \cite{widmer_open-source-lizenzen_2006}. Je nach möglicher Verteilung oder Weitergabe an Dritte der modifizierten Software und Art des Lizenzmodells können unterschiedliche Risiken infolge eines unüberwachten Einsatzes von OSS für ein Unternehmen entstehen. Die daraus resultierenden Lizenzfragen oder rechtlichen Konsequenzen können zu einer maßgeblichen Einschränkung führen, die von Unternehmen bereits im Vorfeld genau analysiert und überprüft werden muss. Mit zunehmender Anzahl von Mitarbeitern und verteilten Standorten wird die Bewältigung dieser Herausforderungen jedoch komplizierter. Die Lösung ist frühzeitige Aufklärung und Schulungen über den technischen und rechtlichen Umgang von OSS, durch die Unternehmen. Diese Unterstützung schafft eine Basis für die verantwortungsvolle Einhaltung von Lizenzbedingungen, angefangen bei den Entwicklern selbst. 

%open-source.2.0 \cite{fitzgerald_transformation_2006}


\subsubsection{Entwicklungen und Unterscheidungen von Software}
\subsubsection{Freiheiten als Merkmale von FOSS}
\subsubsection{Nutzungsrechte an FOSS}
\subsubsection{Lizenzarten und Lizenzmodelle für FOSS}
\subsubsection{Urheberrechtliche Aspekte}
\subsubsection{Vertragrechtliche Aspekte}
\subsubsection{Kritische Betrachtung bei der Verwendung von FOSS}



\section{OSS meets DevOps}
Wie bereits beschrieben, ist ein wesentliches Merkmal von Software die Wiederverwendung von Komponenten. 

Die Folge der massiven Wiederverwendung insbesondere durch OSS-Kompo

Die Verschachtelung hat in den letzten Jahren durch die massive Wiederverwendung von OSS-Komponenten zugenommen. 

Die Entwicklung eines Softwareprodukts ohne den Einsatz von OSS-Software ist heutzutage fast unmöglich. 

Wir befinden uns in einer neuen Software-Ära, in der es ohne den Einsatz von OSS-Komponenten nicht mehr möglich ist, die geforderte Time-to-Market zu erreichen und somit nicht mehr wettbewerbsfähig zu sein. 

Die starke Verschachtelung von OSS-Komponenten durch die vielen transitiven Abhängigkeiten erhöht die Komplexität erheblich. 

Jede dieser Komponenten hat eine oder mehrere Lizenzen, die die Nutzungsbedingungen der jeweiligen Komponente spezifizieren. Somit kann die Arbeit auch dem Bereich des Lizenzrechts zugeordnet werden.

In kleinen Unternehmen (50-100 Mitarbeiter, ein Standort) lassen sich diese Herausforderungen noch durch persönliche Kommunikation bewältigen. 

Mit zunehmender Anzahl von Mitarbeitern und verteilten Standorten wird die Bewältigung dieser Herausforderungen jedoch immer komplizierter. 

Daher ist eine zentrale Unternehmensinformationsbasis ratsam, um die Lösung der genannten Herausforderungen in der Praxis zu unterstützen.

Die Verantwortung für die Einhaltung der Lizenzbedingungen liegt bei den Nutzern selbst.



Nachdem der theoretische Rahmen und die Funktionsweise beider Bereiche beschrieben worden ist, wird in diesem Kapitel näher auf die Integration der OSS innerhalb des Prozesses der DevOps-Produktentwicklung, eingegangen.

\subsection{Integration von OSS in die DevOps-Produktentwicklung}
In diesem Rahmen wurde zunächst der derzeitige IST-Zustand mittels BPMN \dwi{ist doch wurscht mit was, oder?} modelliert und der künftige SOLL-Zustand durch die Verwendung der OSS angepasst.

\subsubsection{Betrachtung des IST-Zustandes des Softwareentwicklungsprozesses}
Eine große Herausforderung bei der Modellierung des Softwareentwicklungsprozesses war eine genaue Momentaufnahme eines kontinierlichen arbeitenden Prozesses. 
\dwi{ich sehe das das für dich eine große herausforderung ist,
aber würde dem jeder leser zustimmen? warum schreibst du das, bzw was möchtest du damit transportieren?}

\begin{figure}[p]
    \centering
    \includegraphics[angle=90, scale=0.6]{Bilder/IST-Prozess_neu.png}
    \caption{Softwareentwicklungsprozess im IST-Zustand basierend auf dem internen msg-Projekt}
\end{figure}
\dwi{die bildqualität ist noch sehr grob - ich nehme an du exportierst das am ende nochmal in 600+dpi?}

Im Rahmen dieser Arbeit bildet die Grundlage des vorgestellten Prozesses, \dwi{den?} Softwareentwicklungsprozess basierend auf dem internen Projekt \dwi{"dem" klingt wie es gibt nur eins. sprich doch einfach von einem beispielprojekt} innerhalb der msg systems ag, ab.
\dwi{irgendwie ist der satz etwas holprig}

Dieser Prozess beschreibt die Vorgehensweise und die darauf aufbauenden Aufgaben, die im Laufe der Entwicklung zu erbringen sind und an dem konkreten Projekt festgelegt wurden.
\dwi{"erbringt" man eine aufgabe? und du beschreibst ja sicher nicht die aufgaben die in der entwicklung zu erledigen sind (z.b. user stories), sondern beschreibst die abläufe entlang derer diese aufgaben erledigt werden. vielleicht einfach den absatz weglassen.}

Da das entsprechende Team mit agilen Methoden wie Scrum und Kanban und mittels DevOps-Methoden arbeitet, erwies sich die Modellierung des IST-Prozesses basierend auf einem Sprint am geeignesten.
\dwi{Vorschlag für die beiden letzen absätze in "kompakt":
Der Beispielprozess beschreibt kontextrelevante Aspekte der projektspezifischen DevOps-Ausprägung und wird entlang eines Entwicklungssprints erklärt.}

Da die Rollen von DevOps, sowohl den Tätigkeitsbereich als auch den entsprechenden Personenrahmen abbilden, werden die Aufgaben innerhalb DevOps als eine zusammenhängende und eigenständige Rolle behandelt und trägt daher den Namen 'Entwicklungsteam'.

Somit entsprechen innerhalb dieses konkreten Projektes Dev und Ops, Entwicklern und nehmen infolgedessen beide Tätigkeitsbereiche, also die Entwicklung als auch die Verwaltung, gemeinsam wahr. \dwi{weglassen. du hast devops schon vorher ausführlich beschrieben..}

Ferner fungiert der Product Owner, als die nächste Rolle, als die Schnittstelle zwischen verschiedenen Stakeholdern oder Kunden und dem Entwicklungsteam. 

Er vertritt demnach die Interessen des Kunden in den Entwicklungsprozess, ohne aktiv in die Softwareentwicklung einzugreifen.

Ferner \dwi{"Ferner" wiederholt. hier vielleicht die vorigen 3 sätze zusammenführen.} erstellt der Product Owner den Product Backlog und legt die Reihenfolge der Items, die bearbeitet werden müssen, fest. 

Zum besseren Verständnis \dwi{du meinst wohl aus gründen der übersichtlichkeit? du schneidest ja vertikal um die elemente auf der seite unterzubringen} wurde der Gesamtprozess in drei Teile (Abbildung 10 - 12) unterteilt und diese einzeln erläutert.\\

\begin{figure}[h]
    \centering
    \includegraphics[scale=0.5]{Bilder/IST-Prozess_first Part.png}
    \caption{Erster Teil des Softwareentwicklungsprozess im IST-Zustand basierend auf dem internen msg-Projekt}
\end{figure}

Der erste Startpunkt des Prozesses stellt die Erstellung des Product Backlogs durch den Product Owner dar, welches durchgängig priorisiert und innerhalb der Abbildung 10 dargestellt wird. 

Die Priorisierung findet sequentiell statt, da mehrere Entwicklungen und Änderungen eines Items des Product Backlogs parallel und gleichzeitig und nicht als eine Aufgabenschleife stattfinden. \dwi{den satz kapier ich nicht. es ist sequentiell weils parallel ist?}

Das Ergebnis ist das priorisierte Product Backlog, welches eine Liste der Anforderungen für alle Produktänderungen darstellt, die anhand einer Schätzung in eine abzuarbeitende Reihenfolge gebracht werden.
\dwi{.."anhand einer schätzung".. versteh ich nicht? die schätzung legt die komplexität einer story fest ("wie groß ist der stein"). die storypoints werden nicht in direktem sinne zur priorisierung genutzt ("welche steine will ich zuerst weg haben")}

Ist die Priorisierung erfolgt, startet die Iteration mit dem Sprint Planning Meeting. 

Hierbei wird abhängig von den verfügbaren Kapazitäten und den Ergebnissen des letzten Sprints entschieden, welche User Stories zuerst bearbeitet werden. 
\dwi{die priorisierung entscheidet, welche stories als nächstes bearbeitet werden. im sprint planning wird eigentlich nur entschieden, wie viel man sich entlang der schon priorisierten liste vornimmt ("wie viele steine bekommen wir in den nächsten zwei wochen mit den leuten hier weggeschafft, wenn wir wissen welche zuerst und wie groß sie sind")}

Sowohl die erstellten Incidents, die als sehr dringend gelten und sofort beseitigt werden müssen, als auch die ausgewählten User Stories des Product Backlogs bilden den Sprint Backlog, welcher somit die geplanten Umsetzungen an Entwicklungen innerhalb des Releases darstellt.

Parallel hierzu beginnt der Change-/Incidentprozess, bei dem zunächst die täglichen Informationen verarbeitet werden. 

Diese Unterprozesse sind essentiell für den Gesamtprozess, da die entsprechenden Ergebnisse aller Unterprozesse zu einer möglichen Codeänderung und infolgedessen zu einer Verwendung von OSS führen können. 

Die Auswirkungen erstrecken sich daher auf den Product Backlog aus, indem festgelegt wird, welche geplanten oder ungeplanten Änderungen zu den bestehenden Anfoderungen am Produkt entwickelt werden müssen und bestimmen daher einen wesentlichen Teil des Inhalts des Product Backlogs.  

Wesentlicher Unterschied beider Prozesse ist, dass der Change eine geplante Veränderung voraussetzt, während der Incident eine ungeplante Entwicklung beansprucht.  

Der Changeprozess bescheibt den Prozess für eine Änderung, die sich auf eine geplante Modifikation der Definition eines bestehenden oder neuen Produkts bezieht und für die das DevOps-Teams verantwortlich ist. 

Der Prozess umfasst die Erfassung jedes Requests, die Dokumentation, Genehmigung und Kontrolle und stellt sicher, dass Change Requests definiert, geplant, effizient, wirtschaftlich und mit vertretbarem Risiko abgewickelt werden. 

Ein Ziel des Change Managements ist es, die Nachvollziehbarkeit von Anforderungen und Änderungen zu gewährleisten.

Währenddessen umfasst das Incident Management den gesamten organisatorischen und technischen Prozess für Reaktionen und Maßnahmen auf Störungen des IT-Betriebs.

Das Spektrum möglicher Incidents reicht von Fehlermeldungen durch Anwender bis hin zu automatisierten Alarmen und Warnungen durch das Monitoring.

Das Ziel des Incident Managements ist die Wiederherstellung eines zugesagten Dienstes in einem definierten Prozess und in optimaler Zeit. 

Dies kann auch Umgehungslösungen beinhalten. 

Incident Management deckt keine (neuen) Anforderungen, Wünsche, Anfragen oder ähnliches ab.  

Obwohl der Change- und Incidentprozess vom Product Owner begonnen wird, ist dieser ausschließlich für den Changeprozess verantwortlich. 

Der Incidentprozess wird anhand des täglichen Monitorings vom Entwicklerteam durchgeführt. 

Nach der Verarbeitung der täglichen Informationen, werden die auftretenden Bugs zunächst in Incident und Changeprozesses unterschieden, da beide Prozesse unterschiedlich behandelt werden.

Innerhalb des Changeprozesses werden die Änderungen definiert und verfeinert und die Anforderungen festgelegt, die entwickelt werden müssen. 

Das Ergebnis fließt in den Product Backlog ein, bei dem die Items wiederrum priorisiert werden. 

Sollte der Bug als Incident bestätigt werden, so wird dieser zunächst erstellt. 

Sollte es sich bei dem erstellten Incident um ein schwerwiegendes Problem handeln, der sofort beseitigt werden muss wird dieser zunächst in das Sprint Backlog übertragen.

Dies ist wichtig, da durch dieses Vorgehen Auskunft während des nächsten Sprint darüber gegeben werden kann, wie hoch die Anzahl der Fehlermeldungen oder Störungen aufgrund ausgefallener Systeme ist und demnach eine Anpassung erforderlich macht. 

Sollte ein Problem auftreten, welches eine Codeänderung benötigt aber nicht als dringlich gilt, wird nach der Erstellung des Incidents, dieser als ein Item in der überarbeitete Product Backlog eingetragen. 

Falls die vermeitliche Incident nicht bestätigt wird, wird dieser vermerkt und geschlossen, wodurch der Incidentprozess ab diesem Zeitpunkt als geschlossen gilt. \\ 

\begin{figure}[h]
    \centering
    \includegraphics[scale=0.6]{Bilder/IST-Prozess_second Part.png}
    \caption{Zweiter Teil des Softwareentwicklungsprozess im IST-Zustand basierend auf dem internene msg-Projekt}
\end{figure}

Wie in Abbildung 11 zu erkennen ist, startet die Iteration nach dem durchgeführten Sprint Planning Meeting, wodurch der Implmentierungsvorgang für die jeweiligen Sprint Backlog Items beginnen kann. 

In diesem Rahmen sind Codeänderungen notwendig, die sich auf vorhandenen Code oder eine Neuimplementierung beziehen können. 

Je nach Backlog Item entscheidet der Entwickler, welche Maßnahmen durchgeführt werden müssen und dokummentiert etwaige Fehler, die sich während der Entwicklung ergeben können.  

Nachdem der Pull-Request für das Commit durchgeführt wurde, werden mehrere Aufgaben parallel erledigt. 

Dazu gehört die Durchführung der Unit-Test und Codeanalyse und die Bestätigung des Reviewer.

Diese Vorgehensweise ist ebenfalls essentiell für den weiteren Ablauf des Softwareentwicklungsprozesses und beruht auf dem Vier-Augen-Prinzip, um die Qualität und Wartbarkeit sicherzustellen und zu erhöhen.  

Dabei wird ein vorher festgelegter Teil des Quellcodes von Entwicklern, die nicht direkt an der Entwicklung des jeweiligen Codes beteiligt waren, begutachtet und nach Fehlern untersucht. 

Umso später Fehler im Softwareentwicklungsprozesses gefunden werden, desto aufwändiger sind diese zu beheben. 

Darüber hinaus ermöglichen regelmäßige Reviews einen Wissensaustausch, eine Dokumentation über etwaige Schwachstellen und eine kontinierliche Verbesserung der eigenen Kenntnisse, was wiederrum die Kultur von DevOps wiederspiegelt. 

Sollten Fehler gefunden werden oder die Test fehlschlagen, so wird der entsprechende Entwickler informiert. 

Sobald alle Änderungen erfolgreich implementiert worden sind, werden diese auf das Master Branch gemergt. 

\begin{figure}[h]
    \centering
    \includegraphics[scale=0.6]{Bilder/IST-Prozess_third Part.png}
    \caption{Dritter Teil des Softwareentwicklungsprozess im IST-Zustand basierend auf dem internen msg-Projekt}
\end{figure}

Nachdem das Mergen erfolgreich durchgeführt wurde, wird das Release auf seine Vollständigkeit und Richtigkeit überprüft und im Review vorgestellt. 

Die Vorstellung erfolgt ebenfalls durch den entsprechenden Entwickler, der die jeweilige Entwicklung den jeweligen Stakeholdern präsentiert. 

Ab diesem Zeitpunkt endet die Iteration. 

Werden weitere Anpassungen benötigt, müssen diese aufgearbeitet werden und fließen in das priorisierte Backlog zurück.

Sind keine Änderungen notwendig, wird das Release bereitgestellt. 

Das Projekt endet sobald das Product Backlog leer ist.

% \paragraph{Agility Master}

% Der Agility Master verantwortet einen effektiven und effizienten Prozess und stellt sicher, dass die Regeln des agil und Lean Management eingehalten werden. 

% Ferner fungiert der Agility Master als ein Moderator für Team-Ereignisse und versucht Probleme und Hindernisse zeitnah zu lösen. 

% Der Agility Master ist mit dem Scrum Master identisch. \dwi{warum führst du dann den namen agility master ein? es hilft dem leser (der nicht aus HAF kommt) nicht, hier einen begriff zu lesen der dann am ende mit einem konzept verglichen wird, das er schon kennt.}

% \paragraph{Dev}

% Diese Rolle entspricht der Rolle des Softwareentwickler oder Development-Teams. 

% Je nach Projektanforderungen kann sich die Rolle des Entwicklers beispielsweise auf das Frontend- oder Backendentwicklung richten.

% \paragraph{Product Owner und Product Designer}
% \dwi{Product Designer ist eine kreation in unserem projekt, und würde in der scrum-welt eher als proxy-po bezeichnet werden. in deinem kontext hilft es aber beides nicht um deine inhalte zu transportieren - nimm nur PO und gut.}

% Zunächst fungiert der Product Owner als die Schnittstelle zwischen verschiedenen Stakeholdern oder Kunden und dem Entwicklungsteam. 

% Er vertritt demnach die Interessen des Kunden in den Entwicklungsprozess, ohne aktiv in die Softwareentwicklung einzugreifen.

% Ferner erstellt der Product Owner den Product Backlog und legt die Reihenfolge der Items, die bearbeitet werden müssen, fest. 

% Wie der Name schon sagt, beschreibt die Rolle des Product Designers die Gestaltung oder den Entwurf des Produkts. 

% Hierzu gehören die Definitionen von UX-Anforderungen und Spezifikationen oder die Schaffung von Schnittstellen, im Hinblick auf die Fuktionalität und das erwartende Verhalten des Produkts. 

% Innerhalb des HAF-Teams überschneiden sich die Rollen des Product Owner und des Product Designers, weshalb diese beiden Positionen in einer Swimlane dargestellt werden.  

% \paragraph{Ops}
% \dwi{du hast jetzt zum einen konkrete rollen die an personen hängen als swimlanes, und dann zwei lanes die an "dev" und "ops" hängen - das sind logische tätigkeitsbereiche. das ist "ungenau". entscheid dich für eine konsistente unterteilung.. und "dev" und "ops" zu trennen ist eigentlich nochmal eine ganz eigene diskussion. vllt kannst die beiden lanes auch zusammenführen unter "developer", sofern du beim rollenbasierten schnitt bleibst.}

% Die Rolle des Administratoren oder IT-Betrieb entspricht dem anderen Teil des DevOps-Teams. 

% In erster Linie sorgt diese Rolle für die Stabilität der IT-Infrastruktur und ermöglichen einen durchgängigen, laufenden Betrieb.  




% Der erste Unterprozess stellt zunächst die Fehlermeldungen dar. 

% Diese können bei der Überprüfung des Monitorings auftreten oder mittels des Kunden an das Entwicklerteam übermittelt worden sein. 

% Sobald Codeänderungen durchgeführt werden müssen, fließen die entsprechenden Anforderungen in den Product Backlog ein. 

% Sollten lediglich Parameter angepasst werden, die eine Konfigurationsänderung mit sich bringen, wird der Prozess der Fehleranalyse nach der entsprechenden Anpassung geschlossen.

















\newpage
\subsubsection{Anpassungen des IST-Zustandes durch die Verwendung von OSS}
Obwohl die Nutzung von OSS Folgen für das gesamte DevOps-Team hat, wird die Entscheidung über die Nutzung von der Rolle des Entwicklers getroffen, wodurch die Prozessanpassungen vorrangig im Bereich der Entwicklung vorgenommen werden müssen. 

Daher wurde für ein besseres Verständnis, ausschließlich die Rolle des Entwicklerteams, mit einem entsprechenden Start und Endpunkt modelliert und die restlichen Rollen nicht weiter behandelt.

\begin{figure}[p]
    \centering
    \includegraphics[angle=90, scale=0.5]{Bilder/SOLL-Prozess.png}
    \caption{Einbindung von OSS in den Softwareentwicklungsprozess basierend auf dem HAF-Projekt}s
\end{figure}

Da zu dem Zeitpunkt der Nutzung von OSS die Projektstruktur bereits besteht und das Product Backlog ebenfalls erstellt wurde, bildet den Anfang des DevOps-Prozesses nun die Anpassung des priorisierten Product Backlogs.

Nachdem das Sprint Planning Meeting durchgeführt worden ist, kann die Iteration gestartet werden. 

\begin{figure}[h]
    \centering
    \includegraphics[scale=0.8]{Bilder/SOLL-Prozess_first Part.png}
    \caption{Erster Teil des Softwareentwicklungsprozess im SOLL-Zustand basierend auf dem HAF-Projekt}
\end{figure}

Wie auch bei dem IST-Zustand muss sich zunächst die Frage gestellt werden, ob Codeänderungen durchgeführt werden müssen, unabhängig ob es auf Änderungen eines vorhandenen Codes oder auf eine Neuimplentierung bezieht. 

Sollten Codeänderungen vorgenommen werden, wird der Implementierungsvorgang für das Backlog Item gestartet. 

Ab diesem Zeitpunkt setzt die Nutzung von OSS die Prozessänderungen in Gang.

\begin{figure}[h]
    \centering
    \includegraphics[scale=0.8]{Bilder/SOLL-Prozess_second Part.png}
    \caption{Zweiter Teil des Softwareentwicklungsprozess im SOLL-Zustand basierend auf dem HAF-Projekt}
\end{figure}

Um präventiv maßgebliche Konsequenzen durch die Nutzung von OSS zu vermeiden, sollten sich die betreffenden Entwickler die Frage stellen, ob tatsächliche eine neue Software oder Softwarekomponete in das Projekt eingebunden werden muss. 

Falls dies der Fall ist, muss zunächst ein initaler OSS-Check durchgeführt werden. 

Dieser dient dazu, das Lizenzmodell der verwendeten OSS-Komponete zu überprüfen und diese mit den 'ungefährlichen' Lizenzmodellen zu vergleichen. 

Dies hat zum Vorteil, dass der Entwickler bereits vor einer langen Implementierung eine Aussage darüber treffen kann, ob sich aus Sicht der Lizenzmodelle eine Verwendung eignet oder ob ein Lizenzmodell vorliegt, welches zumindest ein beschränktes Copyleft aufweist. 

Sollte die verwendete OSS-Komponete ein derartiges Lizenzmodell aufweisen, hat der Entwickler ausreichend Zeit, nach eine 'ungefährlichen' Alternative zu suchen.  

Wird die OSS-Kompoente letztlich verwendet, wird ein Pull-Request für ein Commit durchgeführt. 

An diesem Punkt befindet sich der Entwickler ebenfalls, wenn keine neue OSS genutzt wird. 

Anschließend hierzu werden wie im gängigen Prozess, Unit-Tests und Codeanalysen durchgeführt und diese  vom Reviewer bestätigen lassen.

Gleichzeitig zu diesen Schritten muss eine neue Anpassung des Prozesses im Hinblick auf die Verwendung von OSS als ein automatisierten OSS-Check durchgeführt werden.

Während bei einer Verwendung von neuer OSS die Lizenzmodelle zeitnah überprüft werden können, kann es Fälle geben, in denen Versionen einer bereits implementierten OSS-Komponete geändert werden und daraus eine Verschiebung der betreffenden Lizenzmodelle stattfinden kann. 

Dies kann dazu führen, dass OSS-Komponeten ein starkes Copyleft aufweisen, obwohl diese mit einem schwachen oder keinen Copyleft implementiert worden sind. 

Ferner würde bei einer gängigen Entwicklung diese Situation nicht auffallen und bietet daher ein großes Risiko. 

Durch einen automatisierten OSS-Check können die Lizenzmodelle der verwendeten OSS-Komponeten anhand ihrer Versionen dargestellt werden. 

Damit erhalten alle Entwickler einen genauen Überblick über die verwendeten OSS-Komponeten und deren akutellen Lizenzstatus. 

Zu beachten ist dabei, dass der automatisierte OSS-Check nach jedem Commit ausgeführt werden muss, damit die Entwickler genug Zeit haben, nach einer Lösung für eine mögliche Lizenzproblematik zu suchen.

Durch die zeitnahe Übermittlung der notwendigen Informationen wird die Transparenz zwischen den DevOps-Mitgliedern sichergestellt. 

Die Automatisierung des OSS-Checks entspricht sowohl der Kultur als auch den Methoden nach den DevOps-Prinzipien.

Sollte die OSS-Komponete kein schwieriges Copyleft aufweisen, wird dieser Zustand zunächst vom Reviewer bestätigt und kann dann als Backlog Item erfolgreich implementiert und auf das Master Branch gemergt werden. 

Falls dies nicht der Fall ist, muss die OSS-Thematik zunächst geklärt werden.

Sollte die entstandene Problematik nicht behoben werden, so muss der Projektleiter informiert werden und die Dev-Beteiligung endet zunächst an dieser Stelle. 

Kann die OSS-Thematik gelöst werden, muss diese zunächst überarbeitet werden. 

Dies reicht von einem Rückgang auf eine alte Version, wobei an dieser Stelle die gesamte Funktionalität gewährleistet werden muss, bis zur Verwerfung der entsprechenden Kompoente durch den Entwickler. 

Dieser Aufgabenbereich kann mehrere Unteraufgaben aufweisen und die Lösung ist demnach spezifisch auf die Komponete und deren Entwicklung ausgelegt. 

Nach der Überarbeitung erfolgt eine erneute Bestätigung durch einen Reviewer, zum Ziel, die Kompoente erfolgreich auf den Master Branch zu mergen. 

\begin{figure}[h]
    \centering
    \includegraphics[scale=0.8]{Bilder/SOLL-Prozess_third Part.png}
    \caption{Dritter Teil des Softwareentwicklungsprozess im SOLL-Zustand basierend auf dem HAF-Projekt}
\end{figure}

Sobald das Release auf seine Vollständigkeit und Richtigkeit überprüft worden ist, wird zur erneuten Kontrolle der Lizenzmodelle ein Report erstellt. 

Dieser dient als eine Nachdokumentation, einerseits falls sich unbeabsichtigt Änderungen ergeben haben, die bisher nicht gesehen wurden und andererseits zur Vorlage für den Stakeholder, um den Status der verwendeten OSS-Komponeten transparent zu präsentieren. 

Die Iteration endet mit der Vorstellung der Ergebnisse im Review. 

Basierend auf der Nachdokumentation muss anschließend analysiert werden, ob Auffälligkeiten innerhalb der Lizenzthematik festgestellt wurden.  

Ergeben sich tatsächlich Auffälligkeiten, müssen diese zunächst definiert werden und werden in den Product Backlog übermittelt. 

Müssen keine Veränderungen vorgenommen werden, wird das Release bereitgestellt. 

Genau wie bei dem gängigen Softwareentwicklungsprozess im IST-Zustand, endet dieser Prozess ebenfalls mit einem leeren Product Backlog.  



















\subsection{Umfang der Integration anhand festgelegter Szenarien}
Um das Themenfeld der Verwendung von OSS in den DevOps-Prozess in einem festen Rahmen erfassen zu können, musste zunächst der Umfang dieser Problemstellung reduziert werden.  

In diesem Rahmen wurde der Umgang mit der Nutzung von OSS, insbesondere im Hinblick auf die Lizenzmodelle auf drei Szenarien beschränkt. 

Diese stellen zunächst drei Ausgangssituationen dar, unter welchen Bedinungen OSS verwendet werden darf und welche Einschränkungen dabei getroffen werden müssen.   

\paragraph{Szenario 1: Verwendung von OSS zur Testzwecken}

Bei dem ersten Szenario beschränkt sich die Nutzung von OSS ausschließlich auf Unterstützung- oder Testzwecke wie bespielsweise Treiber. 

Dementsprechend soll dieses Szenario verdeutlichen, dass die Verwendung von OSS ausschlißlich als Werkzeug fungiert und keine Abhängigkeit zwischen der Funktionalität und der verwendeten OSS-Komponente stattfindet. 

Würde die OSS wegfallen, wäre die Funktionalität des Projekts weiterhin gewährleistet. 

Ferner werden keine Kopien des Original-Quellcodes innerhalb eines Repositorys hochgeladen, sowohl manuell als auch mittels Skript. 

Infolge der ausschließlichen Unterstützungsfunktion erfolgt innerhalb dieses Szenarios keine Auslieferung der Komponete an den Kunden. 

\paragraph{Szenario 2: Verwendung von OSS-Bibliotheken als Referenz}

Im Rahmen des zweiten Szenarios werden OSS-Bibliotheken über API-Aufrufe innerhalb des entwickelnden Quellcodes referenziert und blieben daher unverändert. 

Ferner wird der Quellcode der OSS-Bibliothek während des Builds und Deployments nur vorübergehend heruntergeladen, um die ausführbare Datei zu erstellen. 

Da die Bibliothek ausschließlich als Referenz verwendet wird, werden wie bei Szenario eins keine Kopien des originalen Quellcodes der Bibliothek innerhalb eines Repositorys hochgeladen.

Da viele Systembibliotheken unter ein starkes Lizenzmodell fallen, kann das bestehende Projekt mit der entsprechenden Bibliothek durch die Referenzierung zu einem Programm zusammengeführt werden und fällt unter die Verbindung \textit{statisches Verlinken}. \cite[S. 211]{jorg_it-vertrage_2007}

Es entsteht demnach eine direkte Verbindung zu einer lizensierten OSS. 

Aus diesem Grund sollten innerhalb dieses Szenarios weitgreifend auf OSS-Komponenten mit einem starken Copyleft vermieden werden. 

\paragraph{Szenario 3: Verwendung von OSS als modifizierter Quellcode}

Innerhalb des letzten Szenarios findet eine tatsächliche Integration der OSS innerhalb der Entwicklung des Projekts statt.

In diesem Rahmen wird der Quellcode heruntergeladen, entsprechend den Anforderungen angepasst und letztlich in das Repository hochgeladen. 

Dies kann sowohl auf eine reine Modifikation des Quellcodes oder die Erstellung neuer Dateien basieren.  

Ab diesem Zeitpunkt ist die Berücksichtigung der Lizenzmodelle essentiell, wodurch bestimmte Bedinungen erfüllen werden müssen. 

Um weitgreifende Folgen zu vermeiden, sollten in diesem Szenario auschließlich OSS verwendet werden, die kein Copyleft oder teilweise ein beschränktes Copyleft aufweisen.

Jedes File, welches modifiziert oder neu erstellt wurde, muss zur Kennzeichnung einer Modifikation mit einem Header-Kommentar versehen werden.

%Frage an Daniel: Szenarios grafisch darstellen?????

\subsection{Manuelle Überprüfung basierend auf einer Checkliste}
Wie bereits im Kapitel 3.1.2 beschrieben, muss durch die Verwendung von OSS der konkrete Prozess innerhalb des HAF-Projektes angepasst werden. 

Die erste notwendige Anpassung erfolgt mittels einer Überprüfung der verwendeten Lizenzmodelle basierend auf einer manuellen Checkliste durch die entsprechenden Entwickler.  

Durch das strukturierte und mehrmalige Arbeiten mit einer manuellen Checkliste entsteht eine Routine für das Entwicklerteam innerhalb des gesamten Softwareentwicklungsprozesses.
\dwi{haha ... theoretisch klappt die praxis immer!}

Ferner sind Softwarekomponenten oftmals stark miteinander verschachtelt und enthalten zudem viele transitive Abhängigkeiten, wodurch das Ziel verfolgt wird, mit der manuellen Checkliste eine gemeinsame Basis zu schaffen.

Der Einsatz schafft Effizenz und eine Zeitersparnis, da nur das Lizenzmodell betrachetet wird, was in diesem Moment verwendet werden möchte. \dwi{möchte -> soll} 

Dieser Schritt entspricht zwar nicht der gängigen DevOps-Kultur einen möglichst hohen Grad an Automatisierung zu erlangen, jedoch kann bereits durch einen kurzen manuellen Check 'gefährliche' und 'ungefährliche' Lizenzmodelle voneinander unterschieden werden bevor eine langwierige Entwicklung stattfindet. 

Ferner ist an diesem anfänglichen Entwicklungszeitpunkt die Überprüfung mittels einer automatisierten Checkliste bedenklich, da mit der Software zunächst vorrangig expermentiert wird und eine vollständige Integration in die bestehende Anwendung somit nicht feststeht.

Andernfalls müsste jede heruntergeladene OSS direkt in das Softwareentwicklungsprozess integriert werden, um diese anschließend automatisiert überprüfen zu lassen.

Die Folgen wären ein Verlust von wichtigen Ressourcen, Effizenz und Zeit.

Ziel des Einsatzes der manuellen Checkliste ist es, sowohl die technische als auch juristische Faktoren bei der Verwendung von OSS zu berücksichtigen und ein gemeinsames Verständnis zu erreichen.

Innerhalb der kommerziellen Softwareentwicklung stellt dies eine große Herausforderung dar.

Softwarearchitekten haben oftmals eine starke funktionale und strukurelle Sichtweise und wenig Affinität zu Lizenztexten und benötigen daher Hilfe bei der Auswahl von OSS-Komponenten. 

Juristen hingegen haben ein Verständnis für Lizenzrecht, allerdings fällt es ihnen schwer, die tatsächliche technische Ausprägung einer Softwarekomponente rechtlich zu intepretieren.

Benötigt wird eine aussagekräftige Entscheidungsgrundlage, die aufzeigt, ob der Einsatz einer bestimmten Komponente rechtlich zulässig ist.

Darüber hinaus sollte bei der Bewertung der Lizenzmodelle auch die jewelige Nutzung der Komponenten berücksichtigt werden, die einen erheblichen Einfluss auf die zu erfüllenden Lizenzbedingungen haben kann. 

So ist beispielsweise die interne Nutzung oft unbedenklich und bedarf keiner Bedingungen, während eine Weitergabe viele Einschränkungen mit sich bringt. 

Die unten stehende manuelle Checkliste und die zugrunde liegenden Elemente wurden bereits innerhalb der msg systems ag als OSS-Guide erstellt worden und im Rahmen dieser Arbeit auf das Problemfeld erweitert und angepasst.
%Hier Fußzeile zu dem OSS-Guide von Ralf und Navina  
%Thesis von Navina nochmal lesen und paragraphen abändern

\paragraph{Use Types}

Zunächst werden die 'Use Types' also die unterschiedlichen Nutzungarten erläutert. 

Diese geben Auskunft darüber, welche Bedinungen zu den jeweiligen Nutzungsarten erfüllt sein müssen. 

So enthalten einige OSS-Lizenzen beispielsweise die Klausel, dass Modifikationen des Quellcodes nur gestattet sind, wenn diese 'unter derselben OSS-Lizenz wieder allen zur Verfügung gestellt werden.'

Die vorgestellten Szenarien dienen als jeweilige Ausgangssituation, während die Nutzungsarten die jeweilige Verwendung detailliert beschreiben. 

\subparagraph{Auflistung der Nutzungsarten}

Insgesamt gibt es 14 zu beachtende Nutzungsarten.

Diese wurden in sieben verschiedene Kategorien unterteilt: Format (format), Abhängigkeit (dependency), Auslieferung (delivery), Einsatz (usage), Kommunikation (communication), Bündelung (bundling) und Artifakte (artifact). 
\dwi{referenzen zu den tabellen im text!!}

%Frage Daniel: kann ich die hauptsächlichen Überschriften auf englisch lassen,oder soll ich alles auf deutsch übersetzten ?????

%Zweizeilig
% \newcommand\T{\rule{0pt}{4ex}}
% \newcommand\B{\rule[-3ex]{0pt}{0pt}}
%Vierzeilig
\newcommand\A{\rule{0pt}{7ex}}
\newcommand\C{\rule[-6ex]{0pt}{0pt}}
%Dreizeilig
\newcommand\D{\rule{0pt}{5ex}}
\newcommand\E{\rule[-4ex]{0pt}{0pt}}
%Sechszeilig
\newcommand\F{\rule{0pt}{9ex}}
\newcommand\G{\rule[-8ex]{0pt}{0pt}}

\begin{landscape}
\begin{longtable}[h]{|l|c|c||c|c|c|}
    \toprule
    \textbf{Use Types} & \textbf{Erklärung} & \textbf{Beispiel} & \textit{SZ 1} & \textit{SZ 2} & \textit{SZ 3} \\
    \midrule
    \hline
    \T format: source & \parbox{7cm}{Komponente wird im unverfälschten Quellformat geliefert} & WEB-INF/jquery.js & - & - & \checkmark \B \\
    \hline
    \T format: compiled & \parbox{7cm}{Komponente wird in kompiliertem Format bereitgestellt} & com/example/foo.class & - & \checkmark & \checkmark \B \\
    \hline
    \A dependency: optional  & \parbox{7cm}{Komponente wird bei Bedarf geladen, und das Produkt würde vernünftigerweise ohne sie funktionieren} & JDBC driver & \checkmark & - & - \C \\
    \hline
    \D dependency: mandatory & \parbox{7cm}{Komponente wird dynamisch/statisch geladen/verlinkt und das Produkt funktioniert nicht ohne sie} & Hibernate ORM & - & \checkmark & \checkmark \E \\
    \hline
    \A delivery: internal & \parbox{7cm}{Komponente wird intern verwendet, ohne dass sie an andere Rechtssubjekte weitergegeben wird (z. B. zeitlich begrenzte Komponenten)} & Gradle/Ant/Maven & \checkmark & - & - \C \\
    \hline
    \D delivery: distributed & \parbox{7cm}{Komponente wird an andere Rechtssubjekte verteilt (z.B. Laufzeitkomponenten)} & lib/example-1.2.3.jar & - & \checkmark & \checkmark \E \\
    \hline
    \T usage: local-call & \parbox{7cm}{Komponente (über Produkt) wird lokal zur Ausführung aufgerufen} & C:\textbackslash  Example\textbackslash example.jar & - & \checkmark & \checkmark \B \\
    \hline
    \A communication: process & \parbox{7cm}{Komponente wird aufgerufen von Produkt über direkten prozessinternen Mechanismus (Funktionsaufruf, Dispatch-Tabelle, usw.)} & component\_function() & \checkmark & \checkmark & \checkmark \C \\
    \hline
    \D bundling: standalone & \parbox{7cm}{Komponenten-Artefakte, die noch völlig eigenständig und als solche erkennbar sind} & example-1.2.3.jar & \checkmark & \checkmark & \checkmark \E \\
    \hline
    \A artifact: pristine & \parbox{7cm}{Alle Komponenten-Artefakte sind unverändert, d.h. genau so, wie sie ursprünglich vom vorgelagerten Hersteller erhalten wurden} & example-1.2.3.jar!com/example/foo.class & \checkmark & \checkmark & - \C \\
    \hline
    \T artifact: modified & \parbox{7cm}{Komponenten-Artefakte wurden hinzugefügt/ersetzt/entfernt} & example-1.2.3.jar!com
    /example/addon.class & - & - & \checkmark \B \\
    \hline
    \bottomrule
\end{longtable}
\end{landscape}

\paragraph{Umfang und Ausprägung der Verpflichtungen}

Je nach Nutzungsart müssen verschiedene und festgelegte Verpflichtungen innerhalb eines vorgeschriebenen Rahmens erfüllt werden oder können unter bestimmten Bedingungen ausgeschlossen werden. \cite{tldr_legal_software_2012}

Aufgrund dessen kann der Umfang der Verplichtung sich um eine Last, die der Nutzer erfüllen muss handeln oder um eine bestimmte Nutzungsart beziehen, indem einzelne Verpflichtungen erfüllt werden müssen oder die gesamte Nutzung ausgeschlossen wird. 

\subparagraph{Umfang der Verpflichtungsarten} $~$
\\

\begin{tabular}[h]{|r|c|l|}
    \hline\hline
    Verpflichtungsumfang & Zweck & Erklärung \\
    \hline\hline
    \A \parbox{4cm}{OBLIGATION (OBL)} & \parbox{5cm}{Eine Bedingung muss erfüllt werden, um die Lizenzbedingungen zu erfüllen} & \parbox{5cm}{Eine Lizenzvereinbarung enthält mehrere Bedingungen, die erfüllt werden müssen} \C \\
    \hline
    \F \parbox{4cm}{NOT OBLIGATION SINGLE (NOS)} & \parbox{5cm}{Eine Bedingung ist aufgrund der Nutzung ausgeschlossen} & \parbox{5cm}{Wenn eine bestimmte Nutzung nicht eingeschränkt ist, muss die Verpflichtung nicht erfüllt werden, um die Lizenzbedingungen einzuhalten} \G \\
    \hline
    \F \parbox{4cm}{NOT OBLIGATION GLOBAL (NOG)} & \parbox{5cm}{Alle Lizenzbedingungen werden aufgrund einer bestimmten Nutzungsart ausgeschlossen} & \parbox{5cm}{Einige Lizenzvereinbarungen enthalten die Aussage, dass die Lizenzbedingungen nicht gelten, wenn die Komponente auf eine bestimmte Weise verwendet wird} \G \\

    \hline
\end{tabular}
\dwi{referenzen zu den tabellen im text!!}

\subparagraph{Ausprägung einer zu erfüllenden Verpflichtung}

Im Hinblick auf die Verpflichtungen, die der Nutzer erfüllen muss, lassen sich mehrere Elemente definieren. 

Diese geben an welche Bedingungen, innerhalb einer Verpflichtung erfüllt werden müssen. 

Ziel ist es, mittels der unterschiedlichen Ausprägungen der Verpflichtung im Zusammenhang mit den jeweiligen Use Types, eine auf das Lizenzmodell abgestimmte Aufgabenliste zu erstellen, die die jeweiligen Entwickler berücksichtigen müssen. \\

\begin{tabular}[h]{|r|c|l|}
    \hline\hline
    Verpflichtungselemente & Kurzform & Erklärung \\
    \hline\hline
    \D \parbox{4cm}{No Liability} & NO-LIABILITY & \parbox{6cm}{Der Urheber der Komponente kann nicht für Schäden haftbar gemacht werden, die er verursacht} \E \\
    \hline
    \D \parbox{4cm}{Keep Copyright Information} & KEEP-COPYRIGHT & \parbox{6cm}{Die Copyright-Informationen des Autors der Komponente müssen beibehalten werden} \E\\
    \hline
    \D \parbox{4cm}{Provide License Text} & PROVIDE-LICENSE & \parbox{6cm}{Der Lizenztext der Komponete muss vollständig angegeben werden} \E \\
    \hline
    \D \parbox{4cm}{Provide Source Code} & PROVIDE-SOURCE & \parbox{6cm}{Der Quellcode der Komponete muss vollständig angegeben werden} \E \\
    \hline
    \A \parbox{4cm}{Advertizement Clause} & ADV-CLAUSE & \parbox{6cm}{Die Dokumentation und/oder Anwendung muss einen Hinweis auf die Komponente (und ihren Autor) enthalten} \C \\ 
    \hline
    \A \parbox{4cm}{Name Change Required} & RENAME & \parbox{6cm}{Der Name der Komponente muss geändert werden (im Falle von Änderungen und Weiterverbreitung)} \C \\
    \hline
    \D \parbox{4cm}{No Relicensing Allowed} & NO-RELICENSE & \parbox{6cm}{Die Komponente kann nicht unter einer anderen benutzerdefinierten Lizenz erneut lizenziert werden} \E \\
    \hline
    \D \parbox{4cm}{Non-Military Use Only} & CTX-NON-MIL & \parbox{6cm}{Die Komponente darf nicht in militärischen oder nuklearen Kontexten verwendet werden} \E \\ 
    \hline
    \D \parbox{4cm}{Non-Commercial Use Only} & CTX-NON-COM & \parbox{6cm}{Die Komponente darf nicht in kommerziellen Kontexten verwendet werden} \E \\
    \hline
    \D \parbox{4cm}{Weak Copyleft Effect} & COPYLEFT-STRONG & \parbox{6cm}{Die Lizenz hat einen schwachen/eingeschränkten Copyleft-Effekt} \E \\
    \hline
    \T \parbox{4cm}{Strong Copyleft Effect} & COPYLEFT-WEAK & \parbox{6cm}{Die Lizenz hat eine starke/ vollständigen Copyleft-Effekt} \B \\
    \hline
    \A \parbox{4cm}{Non OSS Definition Compliant} & NON-OSS-DEF & \parbox{6cm}{Die Lizenz enthält Bedingungen, die nicht mit der Definition von Open Source Software übereinstimmen} \C \\
    \hline 
    \A \parbox{4cm}{Other Obligations} & OTHER & \parbox{6cm}{Die Lizenz enthält beliebige andere wichtige Bedingungen, die von uns nicht modelliert/abgedeckt werden (Fallback)} \C \\
    \hline 

\end{tabular}

\paragraph{Überprüfung von Apache 2.0 mittels erweiterer manueller Checkliste}

Die folgende Tabelle \dwi{öhm welche denn ;-)? referenzen zu den tabellen im text!!} beinhaltet die beschriebenen Informationen bezüglich der Nutzungstypen und der jeweiligen Verplfichtungselementen und zeigt diese innerhalb einer Matrix an und stellt die ausführliche manuelle Checkliste dar.

Zunächst wurde die Checkliste anhand eines Lizenzmodells, in diesem Fall \textit{Apache License 2.0}, exemplarisch dargestellt, da dieses vermehrt innerhalb des internen msg-Projektes verwendet wird.  

Leere Felder zeigen an, dass die Lizenz keine Aussage über die jeweilige Verpflichtung und die entsprechende Nutzungsart macht. 

Gefüllte Felder enthalten die Kennzeichnung welche jeweilige Verpflichtung für die jewelige Nutzungsart  durchgeführt werden muss. 

Die benötigten Informationen können implizit aus dem angegebenen Lizenzauszug des Lizenzmodells hervorgehen oder wurden explizit angegeben.

Die verbleibenden Lizenzmodelle, die neben der Apache Lizenz verwendet wurde, wurden im Anhang aufgelistet.   

\begin{figure}[p]
    \centering
    \includegraphics[angle=90, scale=1.0]{Bilder/Manuelle Checkliste.png}
\end{figure}

\newpage
\subparagraph{Komprimierung der manuellen Checkliste}

Um eine bessere Übersicht über die wesentlichen Elemente zu gewinnen, wurde in diesem Schritt die manuelle Checkliste komprimiert. 
\dwi{nach welchen kriterien?}

Damit wird das Ziel verfolgt, eine möglichst leicht verständliche Checkliste zu generieren, indem nur die wesentlichen Elemente eines Lizenzmodells angezeigt werden. 

\begin{figure}[h]
    \centering
    \includegraphics[scale=0.6]{Bilder/Manuelle Checkliste_komprimiert.png}
\end{figure}





% Diese Checkliste wurde mit einigen Kollegen der msg systems ag zusammengestellt und entwickelt. dh Ralf und Navina erwähnen 

\section{Proof of Concept}
Während die manuelle Checkliste dafür verwendet wird, präventiv OSS-Komponenten mit riskanten Lieznzmodellen bereits zu Beginn des Entwicklungsprozesses zu vermeiden, verfolgt die automatisierte Checkliste das Ziel, zusätzlich zu einer weiteren Überprüfung von neuen OSS-Komponenten, bereits implementierte auf ihre Lizenzmodelle zu überprüfen, bevor diese ausgeliefert werden. Diese wird innerhalb des folgenden Kapitels in Form eines Proof of Concept (PoC) näher beschrieben. In dem Rahmen dieser Arbeit stellt der PoC einen Prototypen dar, der die prinzipelle Durchführbarkeit einer automatisierten Checkliste aufzeigt, wobei dieser nicht direkt innerhalb des Build-Prozess integriert sondern unabhängig von dem jetzigen Entwicklungsprozess entwickelt wurde. Die automatisierte Checkliste soll dabei als eine reduzierte Version des Endprodukts umgesetzt werden, die auf ihre Nutzbarkeit und Funktionalität getestet und bewertet werden kann. Hierbei wird der Prototyp nicht alle Merkmale und Funktionen eines marktreifen Produkts aufweisen, allerdings wird einerseits die generelle Nutzbarkeit in Abhängigkeit der bestehenden Systemumgebung und andererseits die Demonstration der entwickelten Funktionalitäten zum Zweck einer generellen Realisierbarkeit aufgezeigt. \cite{dreher_prototyping_2018} Gleichzeitig dient der PoC als eine Basis für weitere Weiterentwicklungsmöglichkeiten. Da die direkte Entwicklung innerhalb eines aktiven Projekts große Herausforderungen und Probleme mit sich bringen kann, wurde der Prototyp mit Kollegen der msg systems ag, ausgearbeitet und innerhalb dieser Arbeit festgehalten. Der PoC wurde mit der Skizzierung der grundsätzlichen Idee begonnen. Wesentliche Elemente waren hierbei die momentane Problembeschreibung als auch die darauf aufbauende Lösung, die umgesetzt werden soll. Die Ausarbeitung als auch der Umfang des zugrundeliegenden Problems wurden zunächst durch die Gespräche mit mehreren Teammitgliedern und durch die Teilnahme an Teammeetings anfänglich skizziert. In diesem Rahmen wurde \textit{das Fehlen eines unterstützenden Prozesses hinsichtlich der Bewertung und Prüfung von eingesetzter OSS} als das wesentliche Problem identifiziert, welches mithilfe einer automatisierten Überprüfung gelöst werden sollte. Demnach sollte der Protoytyp verwendete OSS-Komponenten anhand ihrer Lizenzmodelle analysieren und bei einem, mit Risiko verbundenen Lizenzmodell, den jeweiligen Entwickler darüber zeitnah in Kenntnis setzen. Die weitere Anforderungsdefinition, Gestaltung und Einbeziehung erfolgte im Zuge der Prozessmodellierung und der dazugehörigen Anpassung des Soll-Prozesses und wurde in vier grundlegende Schritte unterteilt.





\subsection{Anforderungsdefinition}
Als ersten Schritt wurde anhand der Problembeschreibung die darauf aufbauenden Anforderungen definiert. 

In diesem Rahmen wurden die technischen Spezifikationen, die die automatisierte Checkliste aufweisen muss, festgelegt. 

Im Laufe des PoC wurden die Anforderungen weiter präzisiert.
%Hier das Plugin von Michael Haas erklären und warum es nicht geworden ist. 
% hier der Vergleich von verschiedenen Open-Source-Scannern 
\subsection{Konzeption}
Um ein besseres Verständnis zu ermöglichen, wird zunächst auf die generelle Funktionsweise von Maven näher eingegangen.

\subsubsection{Workflow Maven}
Innerhalb Maven sind Strukuren oder Abläufe, die zur Kompilierung, Testen, Paketierung und Veröffentlichung von Projekten benötigt werden, bereits vorgegeben und müssen nicht mehr definiert werden. \cite[S. 27]{spiller_maven_2011} Dies erleichtert insbesondere die Verwaltung von Projektabhängigkeiten, da Bibliotheken einfach ausgetauscht werden können und die entsprechenden Abhängigkeiten sowie die transitiven Abhängigkeiten aufgelöst und entfernt werden. In Abbildung 23 wird die generelle Funktionsweise von Maven dargestellt und auf besondere Eigenschaften näher eingegangen. 

\begin{figure}[h]
    \centering
    \includegraphics[scale=0.5]{Bilder/Workflow_Maven.png}
    \caption{Funktionsweise von Maven, angelehnt an \cite{guntur_understanding_2020}}
\end{figure}

\paragraph{4.2.1.1. POM} $~$
 
Die Funktionsweise und die Verwaltung des Projektes baut grundsätzlich auf dem 'Project Object Model' (kurz: POM) auf und bildet daher den Kern eines jeden Maven-Projektes ab. Ferner bietet Maven die Möglichkeit Projektabhängigkeiten, Projektumgebungen und Projektbeziehungen in einer seperaten, externen pom.xml-Datei, die während des Build-Prozesses verwendet werden sollen zu deklarieren. \cite[S. 3]{varanasi_introducing_2019}\cite{the_apache_software_foundation_maven_2002} Zudem erzeugt Maven Artefakte, die einzeln ausgeliefert und in die Repositorys übertragen werden. \cite[S. 29]{spiller_maven_2011}

\paragraph{4.2.1.2.Repository} $~$

Ferner verwendet Maven Repositorys. Innerhalb des Local Repositorys werden alle Abhängigkeiten abgelegt, die für die Erstellung des Builds benötigt werden. Zunächst überprüft Maven, ob sich benötigten Abhängigkeiten bereits innerhalb des Local Repositorys befinden. \cite[S. 45 - 47]{loukides_maven_2008} Ist dies der Fall, wird die Datei, ohne die Erstellung einer Kopie im lokalen Verzeichnis, innerhalb des Local Repositorys verwendet, ansonsten versucht Maven die Datei auf das Local Repository herunterzuladen und zu kopieren, um diese aussschließlich lokal verwendet zu können. \cite[S. 115]{spiller_maven_2011}   

\paragraph{4.2.1.3. Build-Lifecycle} $~$

Prinzipiell sind Lifecycles abstrahierte Arbeitsschritte innerhalb festen Phasen, die in einer bestimmten Reihenfolge durchlaufen und für den Build-Prozess essentiell sind. \cite[S. 57]{varanasi_introducing_2019} 
Maven definiert dabei drei Lifecycles\cite[S. 72 - 76]{spiller_maven_2011}: 

\begin{itemize}
    \item \textit{clean}: Löschung, was keine Notwendigkeit mehr hat 

    \item \textit{site}: Erzeugen von Projektdokumentationen
    
    \item \textit{build/default}: Standardisierter Ablauf zum Erzeugen einer Anwendung 

\end{itemize}

Sobald Maven die einzelnen Phasen eines Lifecycles durchläuft, wie in der Abbildung 24 zu erkennen ist, werden die Ziele jedes Plugins ausgeführt, die mit jeder einzelnen Phase verbunden sind. \cite[S. 39]{loukides_maven_2008} 

\begin{figure}[h]
    \centering
    \includegraphics[scale=0.4]{Bilder/lifecycle_maven.png}
    \caption{Zusammenspiel Plugin-Goals und Lifecycle, \cite[S. 59]{varanasi_introducing_2019}}
\end{figure}

\paragraph{4.2.1.4. Plugins in Maven} $~$

Maven ist ein Framework zum Ausführen von Plugins, indem benötigte Funktionen als Plugins anhand des Goals in den Softwareentwicklungsprozess ausgeführt werden. Infolgedessen kann jede Funktionalität durch ein Plugin ersetzt werden, womit Plugins eine zentrale Rolle innerhalb des Softwareentwicklungsprozesses zufällt. Mittels der Auslagerung unterschiedlicher Aufgaben, ist sowohl der Einsatz als auch die Wartung einfacher wie bei großen Abhängigkeiten, da die Aktualisierung durch Maven durchgeführt werden.  

\subsubsection{Entwurf des PoC basierend auf dem Ayoy - Maven Licence Vertify Plugin}

Kernelement des Ayoy-Plugins ist die Prüfung und der darauf aufbauende Vergleich von Lizenzmodellen und deren Abhängigkeiten innerhalb eines Projektes.

\begin{figure}[h]
    \centering
    \includegraphics[scale=0.4]{Bilder/Ayoy-Plugin.png}
    \caption{Funktionsweise des Ayoy-Plugin}
\end{figure}

Anhand der Abbildung 25 wird zunächst die generelle Funktionsweise schemenhaft dargestellt, um ein besseres Verständnis des Workflows dieses Plugins zu erhalten. Alle Komponeten des ausführbaren Programms beinhalten zunächst jeweils eine pom.xml-Datei und die darin enthaltenen Bibliotheken als zu prüfende Abhängigkeiten. Eine pom.xml enthält neben der groupId, artifactId und version zur Identifikation, die Lizenz des Projektes, verschiedene Abhängigkeiten und die entsprechenden Plugins, die zur Ausführung benötigt werden. Die für den Vergleich essentielle licence.xml, befindet sich src-Verzeichnis eines jeden Programmes und muss dementsprechend, in jedem Programm hinterlegt werden. Die darin enthaltenen Lizenzmodelle werden jeweils mit Namen und der dazu gehörigen URL versehen. Die Angabe der URL muss zwingend eingehalten werden, da der Vergleich auf diesem Wert basiert, während der Name aussschließlich für die Ausgabe verwendet wird. Ist keine URL einer Lizenz vorhanden, wird ein Fehler angezeigt. Die licence-Datei wird als ein xml-Format gespeichert und enthält alle erlaubten ('valid') und nicht genehmigten ('forbidden') Lizenzmodelle, wie in der Abbildung 26 dargestellt. 

\begin{figure}[h]
    \centering
    \includegraphics[scale=0.45]{Bilder/licencesxml.png}
    \caption{Genehmigte und nicht genehmigte Lizenzmodelle innerhalb der licence.xml}
\end{figure}

Nachdem der Build-Prozess angestoßen wird, vergleicht das Plugin alle Lizenzen der unterschiedlichen pom.xml-Dateien und deren Abhängigkeiten des aktuellen Projekts mit der entsprechenden licence.xml. Wenn die Anforderungen erfüllt sind, wird der Build-Prozess erfolgreich ausgeführt, ansonsten wird dieser abgebrochen, wie anhand der Abbildung 25 zu sehen ist.\\ Ferner kann der Build-Prozess unterschiedlich gestartet werden. Sollte das Plugin, gemäß der Anforderungsdefinition, automatisiert und daher regelmäßig durchgeführt werden, muss der entsprechende Aufruf innerhalb der pom.xml konfiguriert werden, wie Abbildung 27 zeigt. 

\begin{figure}[h]
    \centering
    \includegraphics[scale=0.45]{Bilder/PluginConfigurationzumAufruf.png}
    \caption{Eingebunder Aufruf innerhalb der pom.xml}
\end{figure}

Bei unregelmäßiger Verwendung kann der Aufruf, anhand Abbildung 28 ersichtlich, über die Kommandozeile mittels folgenden Befehl eingegeben werden:  

\begin{figure}[h]
    \centering
    \includegraphics[scale=0.5]{Bilder/mvn-Aufruf.png}
    \caption{Aufruf der pom.xml innerhalb der Kommandozeile}
\end{figure}

Der Aufruf spiegelt den Pfad des Goals des Plugins innerhalb des Repositorys dar: 

\begin{itemize}
    \item \textbf{mvn}: Syntax zum Ausführen des Maven-Befehls
    \item \textbf{se.ayoy.maven-plugins}: Aufruf der Bibliothek innerhalb des Repositorys
    \item \textbf{ayoy-license-verifier-maven-plugin}: Aufruf des Plugins auf dem Repositorys
    \item \textbf{verify}: Goal des Plugins
\end{itemize}

Die Prüfung der Abhängigkeiten ist neben der Prüfung der direkten Lizenzen von Komponenten essentiell, da Bibliotheken ebenfalls Lizenzmodelle besitzen, die ein starkes Copyleft aufweisen können und aufgrund dessen von dem Plugin erkannt werden. Ferner prüft das Plugin zudem eingebundene Abhängigkeiten, die weitere Abhängigkeiten aufweisen und diese widerrum ein risikoreiches Copyleft beinhalten können. In diesem Rahmen besteht die Möglichkeit, dass sich die Lizenzen der Abhängigkeiten aufgrund einer Versionsänderung auf ein Lizenmodel mit beschränkter oder starker Copyleft umgestellen. Sowohl Versionsumstellungen als auch transitive Abhängigkeiten sind oftmals für den Entwickler nicht sofort erkennbar und bieten ein hohes Potential 'gefährliche' Lizenzmodelle zu übersehen. Während transitive Abhängigkeiten zwingend betrachtet werden müssen, sollten Abhängigkeiten die aussschließlich zum Testen verwendet werden, nicht innerhalb des Plugins einbezogen werden, sowie innerhalb des Kapitels 3.2.1. in Szenario eins beschrieben wurde. Da diese Abhängigkeiten aussschließlich für Testzwecke verwendet und folglich nicht ausgeliefert werden, ist eine Beachtung der Lizenzmodelle nicht zwingend notwendig. 

\begin{figure}[h]
    \centering
    \includegraphics[scale=0.5]{Bilder/allowedmiisingLicence.png}
    \caption{Zu Testzwecken eingebettete Abhängigkeit innerhalb der allowedMissingLicence.xml}
\end{figure}

Zu diesem Zweck müssen die entsprechenden Lizenzen zunächst innerhalb der allowedMissingLicence.xml eingebettet werden, wie in Abbildung 29 dargestellt. Die Lizenzmodelle der darin eingegebenen Abhängigkeiten werden vom Plugin ignoriert, ohne das Build-Prozess fehlschlägt, falls eine risikoreiche Lizenzen gefunden wird. Als Ergebnis des Build-Prozesses wird die SNAPSHOT.jar als eine ausführbare Datei erstellt. 








% Die POM muss drei wesentliche Tags enthalten \cite[S. 77 - 78]{spiller_maven_2011}: 

% \begin{itemize}
%     \item \textit{groupId}: Enthält eine eindeutige, identifizierbare Bezeichnung für ein Projekt
%     \item \textit{artifactId}: Enthält eine eindeutige, identifizierbare Bezeichnung für ein Artefakt bzw. Projekt pro groupId
%     \item \textit{version}: Enthält die aktuelle Version des Artefakts
% \end{itemize}









%Maven Vorstellung und deren funktionsweise (UML)
%UML-diagramm zur besseren Darstellung
\subsection{Implementierung}
Um ein besseres Verständnis zu erlangen, wird das Ayoy-Plugin im Rahmen der Implementierung verschiedenen Tests unterworfen, die jeweils einen erwartenden Zustand aufweisen und basierend auf den SOLL-Prozess die jeweilige Reaktion des Entwicklers beschreiben. An dieser Stelle erfolgt das Testen des Ayoy-Plugins mittels eines einfachen Maven-Programms, welches aussschließlich einen String mit 'Hello World' ausgibt. Um das Plugin automatisiert innerhalb dieses Programmes einzubinden, wurde der entsprechende Aufruf innerhalb der pom.xml eingebettet.Obwohl das Programm ein einfaches Beispiel darstellt, werden die Stärken und die Schwachstellen des Ayoy-Plugins verdeutlicht und erleichtert eine mögliche Fehlersuche. 

\subsubsection{Fall 1: Keine Auffälligkeiten innerhalb der Lizenzen}$~$

Dieser Fall beschreibt die Prüfung, bei der alle Lizenzen und deren Abhängigkeiten kein Copyleft aufweisen und daher innerhalb des Projektes verwendet und an die Stakeholder ausgeliefert werden können. Innerhalb des Programmes wurden Lizenzen mit starker Copyleft aus der pom.xml entfernt, fehlende Lizenzen innerhalb der licence.xml hinzugefügt und Abhängigkeiten, die aussschließlich zu Testzwecken verwendet wurden, innerhalb der allowedMissingLicence.xml eingebettet.  

\begin{itemize}
    \item \textbf{Situation}: Alle Lizenzen und deren Abhängigkeiten sind \textit{erlaubt}
    
    \begin{figure}[h]
        \centering
        \includegraphics[scale=0.5]{Bilder/Fall1Situation.png}
        \caption{Erlaubte Lizenz des Programms}
    \end{figure}

    \item \textbf{Erwarteter Zustand}: Build-Prozess wird \textit{erfolgreich} durchgeführt 
    
    \begin{figure}[h]
        \centering
        \includegraphics[scale=0.5]{Bilder/Fall1Zustand.png}
        \caption{Erfolgreicher Build-Prozess aufgrund einer 'valid'-Lizenz}
    \end{figure}

    \item \textbf{Reaktion anhand des Soll-Prozesses}: Entwickler kann innerhalb des Softwareentwicklungsprozesses weiter verfahren. 
\end{itemize}

\subsubsection{Fall 2: Verwendung von Lizenzen mit starker Copyleft} $~$

Dieser Fall beschreibt die Prüfung, bei der Lizenzen erkannt wurden, die ein starkes Copyleft aufweisen und daher nicht in den Softwareentwicklungsprozesses eingebunden werden sollten. Innerhalb des Programmes wurde beabsichtigt die GNU, also ein Lizenzmodell mit starker Copyleft der pom.xml hinzugefügt, fehlende Lizenzen innerhalb der licence.xml hinzugefügt und Abhängigkeiten, die aussschließlich zu Testzwecken verwendet wurden, innerhalb der allowedMissingLicence.xml eingebettet. 

\begin{itemize}
    \item \textbf{Situation}: Einige verwendete Lizenzen weisen \textit{unerlaubte} Lizenmodelle auf
    
    \begin{figure}[h]
        \centering
        \includegraphics[scale=0.4]{Bilder/Fall2Situation.png}
        \caption{Verbotene Lizenz des Programms anhand der licence.xml}
    \end{figure}

    \newpage
    \item \textbf{Erwarteter Zustand}: Build-Prozess schlägt \textit{fehl} 
    
    \begin{figure}[h]
        \centering
        \includegraphics[scale=0.4]{Bilder/Fall2Zustand.png}
        \caption{Fehlgeschlagener Build-Prozess aufgrund einer 'forbidden'-Lizenz}
    \end{figure}

    \item \textbf{Reaktion anhand des Soll-Prozesses}: Entwickler muss die OSS-Thematik umgehend klären, indem entschieden wird, ob die Komponente entfernt wird oder der Copyleft-Effekt aufgrund einer Versionsänderung zustande kam. Kann die Komponente nicht ausgetauscht aber dringend benötigt werden, muss der Projektleiter entsprechend informiert werden.  
\end{itemize}

\subsubsection{Fall 3: Verwendung von unbekannten Lizenzen} $~$

Dieser Fall beschreibt die Prüfung, bei der Lizenzen innerhalb der pom.xml erkannt wurden, die bisher nicht in der licence.xml vorhanden sind und daher für das Plugin als unbekannt gelten. Innerhalb des Programmes wurden unbekannte Lizenzen der pom.xml hinzugefügt, die innerhalb der licence.xml noch nicht vorhanden sind und Abhängigkeiten, die aussschließlich zu Testzwecken verwendet wurden, innerhalb der allowedMissingLicence.xml eingebettet. 

\newpage
\begin{itemize}
    \item \textbf{Situation}: Einige verwendete Lizenzen weisen \textit{unbekannte} Lizenmodelle auf
    
    \begin{figure}[h]
        \centering
        \includegraphics[scale=0.5]{Bilder/Fall3Situation.png}
        \caption{Unbekannte Lizenz des Programms}
    \end{figure}

    \item \textbf{Erwarteter Zustand}: Build-Prozess schlägt \textit{fehl} 
    
    \begin{figure}[h]
        \centering
        \includegraphics[scale=0.4]{Bilder/Fall3Zustand.png}
        \caption{Fehlgeschlagener Build-Prozess aufgrund einer 'Unknown'-Lizenz}
    \end{figure}

    \item \textbf{Reaktion anhand des Soll-Prozesses}: Entwickler muss zunächst anhand des Lizenzmodells, den Status des Copylefts recherchieren und anschließend die Lizenzen innerhalb der licence.xml an der entsprechenden Stelle hinzufügen. 
\end{itemize}

\subsubsection{Fall 4: Verwendung von unbekannten Abhängigkeiten} $~$

Dieser Fall beschreibt die Prüfung, bei der Lizenzen innerhalb Abhängigkeiten erkannt wurden, die bisher nicht in der licence.xml vorhanden sind. Im Rahmen dieses Falles werden zudem transitive Abhängigkeiten deutlich, also Abhängigkeiten die ihrerseits weitere Abhängigkeiten besitzen. Innerhalb des Programmes wurden unbekannte Abhängigkeiten der pom.xml hinzugefügt, dessen Lizenzen innerhalb der licence.xml noch nicht vorhanden sind und Abhängigkeiten, die aussschließlich zu Testzwecken verwendet wurden, nicht innerhalb der allowedMissingLicence.xml eingebettet. 

\begin{itemize}
    \item \textbf{Situation}: Einige verwendete Abhängigkeiten weisen \textit{unbekannte} Lizenmodelle auf
    
    \newpage
    \begin{figure}[h]
        \centering
        \includegraphics[scale=0.5]{Bilder/Fall4Situation.png}
        \caption{Unbekannte Abhängigkeit des Programms}
    \end{figure}

    \item \textbf{Erwarteter Zustand}: Build-Prozess schlägt \textit{fehl} 
    
    \begin{figure}[h]
        \centering
        \includegraphics[scale=0.4]{Bilder/Fall4Zustand.png}
        \caption{Fehlgeschlagener Build-Prozess aufgrund einer 'Unknown'-Abhängigkeit}
    \end{figure}

    \item \textbf{Reaktion anhand des Soll-Prozesses}: Entwickler muss zunächst anhand des Lizenzmodells der Abhängigkeit, den Status des Copylefts über mehrere Abhängigkeiten recherchieren und anschließend die Lizenzen innerhalb der licence.xml an der entsprechenden Stelle hinzufügen. Sollte es sich hierbei um ein Lizenzmodell handeln, welches aussschließlich zu Testzwecken verwendet wird, muss die Abhängigkeit der allowedMissingLicence.xml zusätzlich hinzugefügt werden. 
\end{itemize}

\subsubsection{Fall 5: Verwendung von Abhängigkeiten zu Testzwecken} $~$

Dieser Fall beschreibt die Prüfung, bei der Abhängigkeiten erkannt wurden, die ihrerseits ein Lizenzmodell mit starker Copyleft aufweisen und aussschließlich für Testzwecke verwendet werden. Innerhalb des Programmes wurden Abhängigkeiten der pom.xml hinzugefügt, dessen Lizenzen innerhalb der licence.xml unter 'forbidden' stehen und nicht innerhalb der allowedMissingLicence.xml eingebettet sind. 

\begin{itemize}
    \item \textbf{Situation}: Einige verwendete Abhängigkeiten weisen \textit{unerlaubte} Lizenmodelle auf, die zum testen verwendet werden
    
    \begin{figure}[h]
        \centering
        \includegraphics[scale=0.37]{Bilder/Fall5Situation.png}
        \caption{Abhängigkeit zu Testzwecken anhand der licence und allowedMissingLicence.xml}
    \end{figure}

    \item \textbf{Erwarteter Zustand}: Build-Prozess schlägt \textit{fehl} 

    \begin{figure}[h]
        \centering
        \includegraphics[scale=0.4]{Bilder/Fall5Zustand.png}
        \caption{Fehlgeschlagener Build-Prozess aufgrund einer 'Unknown' und 'Forbidden'-Abhängigkeit}
    \end{figure}

    \item \textbf{Reaktion anhand des Soll-Prozesses}: Entwickler muss zunächst sicherstellen, dass die verwendete Abhängigkeit aussschließlich für Testzwecke verwendet wird und dieses der allowedMissingLicence.xml hinzufügen. 
    
    \begin{figure}[h]
        \centering
        \includegraphics[scale=0.37]{Bilder/Fall5REaktion.png}
        \caption{Hinzugefügte Abhängigkeit für Testzwecke innerhalb der licence und allowedMissingLicence.xml}
    \end{figure}

\end{itemize}

% Entwicklung starten und die Funktionalitäten beschreiben 
\subsection{Evaluation}
Zunächst konnte mittels der Implentierung aufgezeigt werden, dass sich das Plugin ideal für den benötigten automatisierten OSS-Check eignet. 

Durch die Fehlermeldungen und die Abbrüche des Build-Prozesses wird der Entwickler innerhalb des Softwareentwicklungsprozess in Kenntnis gesetzt, ob eine Lizenz mit einem beschränkten und/oder starken Copyleft vorliegt. 

Zunächst muss berücksichtigt werden, dass Verweise auf die licence.xml und die allowedMissingLicence.xml innerhalb des Aufrufs des Plugins innerhalb der pom.xml befindet. 

Daher muss die Struktur und der entsprechende Pfad der licence.xml und allowedMissingLicence.xml zwingend eingehalten werden.

Zudem müssen beide Dateien das Grundgerüst für den Vergleich bereits beinhalten, also nicht leer sein. 

Sollten diese Strukturen nicht eingehalten werden, wird ebenfalls eine Fehlermeldung angezeigt. 

Darüber hinaus bietet das Plugin nicht den höchsten Grad der Automatisierung, da Entwickler bei unbekannten und transitiven Abhängigkeiten und Lizenzen zunächst recherchieren müssen, ob diese einen Copyleft-Effekt besitzen oder aussschließlich zu Testzwecken verwendet werden.  

Daher kann eine vollständige Abkehr von manuellen Tätigkeiten mittels des Ayoy-Plugins nicht vorrausgesetzt werden. 

Allerdings ist zu bedenken, dass im Verhältnis zum jetztigen IST-Zustand ein erheblicher Zeitaufwand reduziert werden kann. 

An dieser Stelle muss darüber hinaus festgehalten werden, dass keine Evaluierung des Ayoy-Plugins innerhalb des jetztigen Projektes vorliegt. 

Der PoC wurde aufgrund von einer zeitintensiven Integration in den bestehenden Projektablauf nicht durchgeführt, womit die vollständige Bewertung insbesondere durch die Entwickler ausfällt. 

Allerdings zeigt der PoC die grundsätzliche Funtkions- und Vorgehensweise auf, wie das Plugin zu handhaben ist und welches Potential und Vorteile es als integriertes Tool bietet.
 

%Automatisierter OSS-Check
%Möglichkeiten, die bestehen, zur automatisierten Einbindung 
%--> Tool M.Haas(Schlecht bis sehr schlecht)
%--> Fertiges Tool implementieren(Wo genau?Anforderungen?, Unterstützung von wem)
%--> Eigene Entwicklung :'( )
%--> Letzte aber unschöne Lösung: Theoretische Implementierung 
%--------> Verbindung zwischen M.Haas Tools und gängigen Tools nicht ganz klar => 1. Excel-Format, 2. Verschiedenartig
%Tool-Vergleich(Konzeption), Analyse des Tools, Validierung/Evaluation 
%einbindung in den Prozess(Soll)
%Sicherstellung der Transparenz des DevOps-Teams


\section{Schlussfolgerung und Ausblick}
In dieser Masterthesis konnte aufgezeigt werden, dass alle Ziele dieser Arbeit erreicht wurden und die anfangs beschriebenen Forschungsfragen beantwortet werden konnten. Zunächst konnten wesentliche Informationen aufgearbeitet werden, um einen umfassenden Überblick über die Themengebiete von DevOps und OSS zu erhalten. So wurden innerhalb des Bereichs von DevOps konkrete Themen wie Kultur, Methoden und Funktionsweisen aufgezeigt. Die dabei gewonnenen Informationen waren insbesondere für das Verständnis des jetzigen Softwareentwicklungsprozesses bei der msg systems ag und der darauf aufbauenden Modellierung des Ist- und Soll-Prozesses, essenziell. So musste beispielsweise der hohe Grad an Automatisierung als ein wesentlicher Kernaspekt von DevOps bei der Modellierung des Soll-Prozesses als Kriterium berücksichtigt werden. Innerhalb des Themengebietes von OSS wurden vorrangig praxisrelevante Informationen herangezogen. An dieser Stelle wurde ein Überblick über unterschiedliche, aber teilweise bereits eingesetzte Lizenzarten und deren Lizenzvereinbarungen geschaffen. So wie das Zitat am Anfang dieser Arbeit beschreibt, ist OSS zwar frei zugänglich und kostenlos erhältlich, sollte aber nicht mit freiem Umgang verwechselt werden. An dieser Stelle wurden die entsprechenden Einschränkungen, Rechte und Pflichten und juristische Konsequenzen aufgezeigt, um das Verständnis des Einsatzes von OSS zu verbessern. Mit der Modellierung der Abläufe des Ist-Prozesses wurde zunächst eine Basis erstellt, um den bisherigen Prozess, mit dem Ziel einer umfassenden Bewertung und Überprüfung bei der Verwendung von OSS, anzupassen. Der Soll-Prozess hingegen enthält neue Abläufe und Aufgaben, die auf den Einsatz von OSS und der Verwendung von Copyleft-Klauseln abgestimmt sind. Ausgehend davon war es innerhalb des Soll-Prozesses wichtig, eine Möglichkeit für das gesamte DevOps-Team zu schaffen, um den Einsatz von neu hinzugefügten, bereits in den Softwareentwicklungsprozess eingebundenen OSS-Komponenten und deren Abhängigkeiten zu überprüfen. In dieser Hinsicht dient die manuelle Checkliste dazu, die gesammelten Informationen an das DevOps-Team weiterzugeben und als eine Möglichkeit, präventive Kontrollen zu dem Einsatz von OSS herzustellen. Die automatisierte Checkliste hingegen kann dazu verwendet werden, neue Strukturen und Maßnahmen innerhalb des Entwicklungsprozesses anhand der eingesetzten OSS zu etablieren und diente dem Proof of Concept als Konzept. Im Rahmen des PoC wurde mittels das Ayoy-Plugin aufgezeigt, dass bestehende OSS-Komponenten nach ihren Lizenzen und Abhängigkeiten überprüft und beurteilt werden können. Durch den Abbruch oder den erfolgreichen Builds konnte sichergestellt werden, dass das DevOps-Team der msg systems ag transparent über den Zustand der jeweiligen OSS-Komponente in Kenntnis gesetzt werden. Insbesondere OSS-Komponenten mit starken Copyleft-Klauseln können so reaktionsschell bemerkt werden. Durch die aufgezeigten Testfälle sollen alltägliche Situationen darstellen, die sich im Zusammenhang des Einsatzes von OSS ergeben. Ausgehend von dem praktischen Teil dieser Arbeit, hat sich gezeigt, dass der Softwareentwicklungsprozesses des Beispielprojektes der msg systems ag geändert werden kann und anhand des Pluging als eine technische Umsetzung in den Entwicklungsprozess möglich ist. 

Als Empfehlung an die msg systems ag kann an dieser Stelle festgehalten werden, dass die Weiterentwicklung des Plugins ein mögliches Handlungsfeld darstellt.So müsste dieser in den Softwareentwicklungsprozess des Beispielprojektes produktiv integriert werden, um einerseits Schwachstellen oder Stärken herausfinden zu können und anderseits das Plugin, nach dem entsprechenden Softwareentwicklungsprozesses weiteerzuentwickeln. Hinzu kommt die Möglichkeit bestehende Dokumente über verwendete Lizenzen in ein xml.Format zu parsen und dieses ebenfalls in das Plugin zu integrieren. Darüber hinaus besteht die Möglichkeit, eine Nachdokumentation anzufertigen, um einen Überblick der verwendeten Lizenzen jedes DevOps-Mitgliedes zu erhalten. Ferner kann zu einem späteren Zeitpunkt, die manuelle Checkliste als eine webbasierte Lösung zu entwickeln. Hierbei könnten die Verpflichtungen, Lizenzen und Nutzungstypen direkt eingegeben werden, um die Liste direkt zu erhalten und nicht mehr manuell abzulesen brauchen. 









\newpage
\bibliography{Literatur}
\bibliographystyle{apalike}

\end{document}


