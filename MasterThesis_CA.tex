\documentclass[12pt,titlepage]{article}
\usepackage[ngerman]{babel}
\usepackage[utf8]{inputenc}
\usepackage{color}
\usepackage[a4paper,lmargin={4cm},rmargin={2cm},
tmargin={2.5cm},bmargin = {2.5cm}]{geometry}
\usepackage{amssymb}
\usepackage{amsthm}
\usepackage{graphicx}

\begin{document}

\title{DevOps meets FOSS: Automatisierte Integration von FOSS in die DevOps-Produktentwicklung\\}

\author{Claudia Arnold}

\maketitle

\begin{abstract}
   Unter dem Begriff FOSS bekannt, bietet die Free Open Source Software eine Reihe an Möglichkeiten im Bereich der Softwareentwicklung. Durch die sinkenden Nutzungesbescchränkungen im Vergleich zur konventioneller Software gilt FOSS als ein Instrument zur schnellen Integration von Software in die eigene Entwicklung.  
    \end{abstract}

\section{Einleitung}

\begin{quote}
\textit{In real open source, you have the right to control your own destiny}\newline -Linus Torvalds
\end{quote}

Verändertes Konsumentenverhalten, schnelle Reaktion auf veränderte Kundenwuensche und Time-to-Market sind wesentliche Anforderungen, die für den wirtschaftlichen Erfolg eines Unternehmens insbesondere in Zeiten der Digitalisierung kennzeichnet sind und daher große Herausforderungen mit sich bringen. git staDie daraus resultierenden steigenden Anforderungen an neuer Funktionalität und die damit verbundene stetige Weiterentwicklung erfordern in erster Linie eine neue Dynamik innerhalb der Entwicklug, Konzepte und Tools. In diesem Zusammenhang wird der Softwareentwicklungsprozess nach agilen Methoden unablässig und gehört in den meisten Unternehmen bereits zum Standard einer IT-Organisation. 

In diesem Rahmen kommt der Ansatz des DevOps zum Tragen, nach dessen Grundsätzen der Softwareentwicklungsprozess, angefangen von der Idee, über die schnelle Entwicklung bishin zum Einsatz innerhalb der Produktivumgebung, durchgeführt wird. DevOps ist ein Kofferwort aus den Begriffen Development und IT-Operations und beschreibt im Wesentlichen die technische und organisatorische Verbindung zwischen der Softwareentwicklung und der Systemadministration. Charakterisch sind verkürzte Releasezyklen in einem inkrementellen Prozess und ein hoher Automatisierungsgrad, wodurch eine exakte Planung, Vermeidung von kritischen Fehlern und eine schnelle Reaktion auf Kundenanforderungen ermöglicht wird. Wesentliches Ziel von DevOps ist es, die gesamte Entwicklung transparent, flexibel und agil zu gestalten, wodurch die Produktivität und Effizenz maßgeblich gesteigert und das endgültige Produkt schneller zum Kunden ausgeliefert werden kann. 

Neben der agilen Softwareentwicklung ist ein weiteres wesentliches Kennzeichen des DevOps-Konzeptes die Grundsätze des Lean Manufactoring, nach denen die Minimierung von Verschwendung und die Maximierung der Produktivität im Vordergrund steht. Dabei spielt insbesondere die Wiederverwendung von Software eine wesentliche Rolle. Anstatt einer zeit- und kostenintensiven Neuentwicklung von Software, bleiben viele allgemeine oder wiederholende Funktionalitäten innerhalb eines neuen Projektes oftmals gleich und müssen demnach nicht verändert werden. Vor diesem Hintergrund sparen Entwickler viel Zeit und Ressourcen, indem Quellcode, Templates oder Algorithmen wiederverwendet werden, was wiederrum dem Grundgedanken von DevOps entspricht. 

Ausgehend hiervon ist die Integration von bestehenden Softwarekonzepten ein wesentlicher Vorteil, wodurch die Verwendung von Open Source Software zunehmend an Bedeutung im Unternehmensfeld gewinnt. Als eine realistische Alternative zu herkömlicher Software, erlangte die Open-Source-Bewegung mit der Idee von frei zugänglicher Software in den letzten Jahrzehnten immer mehr an Popularität. Unter Open Source Software versteht man einen öffentlich zugänglichen Quellcode, den jeder einsehen, verändern und für sich nutzen kann. Im Gegensatz zur herkömlicher proprietärer Software, erhalten Nutzer eine Open Source Software meist kostenlos und ohne weitere Kosten für Lizenzvereinbarungen. Durch den unmittelbaren Zugang zum Quellcode, haben Entwickler die Möglichkeit einerseits, am Quellcode mit nur wenigen Ressourcen zu experimentieren und zu überprüfen, ob sich dieser ein wiederkehrendes Problem innerhalb des Projektes beinhaltet oder anderseits nach der jeweligen Problematik zu individualisieren und zu verändern. Die Notwendigkeit einer kosten- und zeitspielige Entwicklung entfällt und die Verwendung von bereits gelösten Paradigmen rückt in den Vordergrund.   

% Häufige Vorgehensweise: Moduale Architektur von Open-Sorce projekten
% Beispiele an OSS aufzeigen: Betriebssysteme, Server-Anwendungen oder Office-Produkte
% Evtl hier nochmal auf Linux, als erstes populäres Produkt im OSS-Bereich eingehen 

Das daraus resultierende Innovationspotenzial kann, basierend auf der freien Gestaltungsfreiheit die OSS im Gegensatz zur kommentieller Lizenzsoftware ermöglicht, genutzt werden, Ideen umzusetzen ohne zu stark an unternehmensinterne Vorgaben gebunden zu sein oder unternehmensinterne Prozesse oder Prozessabläufe in die Entwicklung einzubinden. Vor diesem Hintergrund kann eine umfassende Integration von Open Source Software in vorhandene Unternehmenstrukturen im Bereich der Entwicklung, ausschließlich in einem agilen Umfeld durchgeführt werden, womit DevOps als ideale Plattform geeignet ist.  

Trotz der Freiheiten, die Open Source Software offensichtlich bietet, unterliegen die meisten Projekte rechtlichen Schutzmaßnahmen, einschließlich dem Marken- Patent- und Urheberrecht. Aufgrund dessen können, aus der Verwendung von OSS resultierende Lizenzfragen oder rechtliche Konsequenzen, zu einer maßgeblichen Einschränkung insbesondere im Hinblick auf eine mögliche Verteilung oder Weitergabe an Dritte führen, die von Unternehmen genau analysiert und überprüft werden muss.  

In dieser Thesis werden die grundsätzlichen Vorteile und Risiken, die sich aus den verschiedenen Lizenzvereinbarungen bei der Verwendung von OSS ergeben analysiert und die Migration in den Devops-Emntwicklungsprozeess dargestellt. Verdeutlicht wird die Themenstellung anhand der ausgeführten Analysen anahnd eines Beispiels bei der msg Systems. 




Durch die freie Verfügbarkeit des Quellcodes, die entfallenen Lizenzkosten für den Einsatz und die freie Weiterentwicklungsmöglichkeiten, erweist sich OSS als eine einfache und kostengünstige Innovationsquelle für Unternehmen. 

Dabei reicht das Anwendungsfeld für die Verwendung von OSS in diesem Zusammenhang von der Automobilindustrie bis zur Versicherungsbranche und steht daher im direkten Wettbewerb zu proprietären Softwareangeboten. 

\subsection{Problemstellung}

Laut einer aktuellen Studie des Bundesverband Informationswirtschaft, Telekommunikation und neue Medien e. V. (Bitkom) wird der Einsatz von Open Source Software in Unternehmen, im Jahre 2019 bei einer Unternehmensgröße von über 2.000 Mitarbeitern, bereits auf 86 Prozent geschätzt. Dabei reicht das Anwendungsfeld von der Automobilindustrie bis hin zu Banken und Versicherungen und steht daher im direkten Wettbewerb zu namenhaften proprietaeren Softwareangeboten. Ausgehend hiervon gehört die Etablierung von Open Source Software in die Softwareentwicklung bereits zur gängigen IT-Unternehmenskultur.  

Durch die freie Verfügbarkeit des Quellcodes, kann die entsprechende Software schnell und einfach eingebunden werden. 



Open-Source-Software gilt als zuverlässig und sicher. Für manche ist sie sogar ein Innovationsmotor mit ungeheurem Potenzial. Neben der Vielfalt der Anwendungsszenarien steht die Fülle 
der Anwendungen. Freie Software – ein gängiges anderes Etikett für Open-Source-Software – 
hat echte wirtschaftliche Bedeutung gewonnen. Die Serverinfrastruktur einiger großer Internetfirmen basiert zum größten Teil auf Linux. Google, Facebook oder Amazon sind bekennende 
Nutzer. Die Gründe dafür sind simpel: Unternehmen müssen für den Einsatz von Open-SourceSoftware keine Lizenzkosten einkalkulieren. Das vereinfacht den Aufbau skalierender Umgebungen signifikant. Außerdem können die Unternehmen die genutzte freie Software im Rahmen 
der jeweiligen Lizenzbestimmungen entsprechend eigener Zwecke und Ziele weiterentwickeln. 
Selbst wenn die modifizierte Version danach geschäftskritische Elemente enthält – und der 
wirklich geschäftskritische Anteil eines Softwaresystems ist in der Regel sehr klein –, gibt es 
immer noch gute Einsatzszenarien, die deren Schutz in Verbindung mit Open-Source-Software 
gewährleisten.


so selbstverständlich die Verwendung ist, so verbreitet sind 
oft auch Wissenslücken über Grundanforderungen im Umgang mit OSS, sei es als Basis für einen 
erfolgreichen Einsatz im Alltag oder als Voraussetzung für die Verwendung für eigene Softwareprojekte. Ein gängiges Missverständnis besteht z.B. darin, dass aus der einfachen und unentgeltlichen Verfügbarkeit von OSS auf das Fehlen jeglicher juristischen Einbettung geschlossen wird. 

Grundsätzlich unterliegt Software dem Urheberrecht. Daher 

Durch die freie Verfügbarkeit des Quellcodes, kann die entsprechende Software schnell und einfach eingebunden werden. 


Die implizierte Freiheit, die sich bei der Verwendung von OSS ergibt, beschränkt sich dementsprechend auf die Einsicht, Nutzung, Modifikation und Distribustion des Quellcodes.    

jedoch nicht uneingeschränkt lizenzfrei genutzt werden. 



Mittels einer Lizenzierung der entsprechenden Software soll sichergestellt werden, dass die Rechte zur Nutzung der Open Source Software nur mit Genehmigung des Urhebers geschützt bleiben. Ab dem Zeitpunkt der Weitergabe von veränderter Software, müssen entsprechenden Lizenzmodelle berücksichtigt werden, um vorrangig das geistige Eigentum des Urhebers zu schützen.  

Bereits bei groben Veränderungen im Laufe des Projektes, hat die Veränderungen von bestimmten SChutzmaßnahmen graviernede Auswirkungen auf das DevOps-Team. Viele erhalten nicht die benötigten Informationen, dass eine Veränderung vorliegt, die das Projekt maßgeblich beeinflußen könnte. 

Demgegenüber haben Urherber bei einer Verletzung an ihrer Open Source Software, die sich aus den einzelnen Lizenzmodellen ergibt, die Möglichkeit rechtliche Ansprüche geltend zu machen. Die Folge sind Unterlassungsansprüchen und kostenspieliger Aufwendungsersatz. Die Konsequenz wäre eine sofortige Einstellung der bearbeiteten Software und möglicherweise eine neue Entwicklung, der Funktionalitäten ohne das jeweilige Open Source als Basis./ Nutzer verliehrt das Recht die Software weiterhin zu nutzen.  


Eine
Lizenz (v lat: licere = erlauben) ist eine Erlaubnis oder Genehmigung zur Nutzung
eines Rechts durch den Urheber oder Inhaber dieses Rechts11. Im juristischen
Bereich stellt die Lizenz einen Vertrag dar, durch welchen einfache oder
ausschließliche Rechte eingeräumt werden können. 


Software ist in der Regel urheberrechtlich geschützt. Dies gilt auch für Freie Software oder Open-Source (OSS). Die für proprietäre Software eingeräumten Softwarelizenzen sind dahingehend ausgerichtet, die Freiheit der Nutzung, Verbreitung und Veränderung der Software einzuschränken. Der Softwarehersteller kann so durchsetzen, eine angemessene Vergütung für die Entwicklungsleistungen zu erhalten, Art und Umfang der Werknutzung zu bestimmen und die Software vor ungewollter Veränderung zu schützen. 


1. Freie Weitergabe
Die Lizenz darf niemanden in seinem Recht einschränken, die Software als Teil eines Software-Paketes, das Programme unterschiedlichen Ursprungs enthält, zu verschenken oder zu verkaufen. Die Lizenz darf für den Fall eines solchen Verkaufs keine Lizenz- oder sonstigen Gebühren festschreiben.

2. Quellcode
Das Programm muss den Quellcode beinhalten. Die Weitergabe muss sowohl für den Quellcode, als auch für die kompilierte Form zulässig sein. Wenn das Programm in irgendeiner Form ohne Quellcode weitergegeben wird, so muss es eine allgemein bekannte Möglichkeit geben, den Quellcode zum Selbstkostenpreis zu bekommen, vorzugsweise als gebührenfreien Download aus dem Internet. Der Quellcode soll die Form eines Programms haben, das ein Programmierer vorzugsweise bearbeitet. Ein absichtlich unverständlich geschriebener Quellcode ist daher nicht zulässig. Zwischenformen des Codes, so wie sie etwa ein Präprozessor oder ein Konverter („Translator”) erzeugt, sind unzulässig.

3. Abgeleitete Software
Die Lizenz muss Veränderungen und Derivate zulassen. Außerdem muss sie es zulassen, dass die solcher Art entstandenen Programme unter denselben Lizenzbestimmungen weiter vertrieben werden können wie die Ausgangssoftware.

4. Unversehrtheit des Quellcodes des Autors
Die Lizenz darf die Möglichkeit, den Quellcode in veränderter Form weiterzugeben, nur dann einschränken, wenn sie vorsieht, dass zusammen mit dem Quellcode so genannte „Patch files” weitergegeben werden dürfen, die den Programmcode bei der Kompilierung verändern. Die Lizenz muss die Weitergabe von Software, die aus einem veränderten Quellcode entstanden ist, ausdrücklich erlauben. Die Lizenz kann verlangen, dass die abgeleiteten Programme einen anderen Namen oder eine andere Versionsnummer als die Ausgangssoftware tragen.

5. Keine Diskriminierung von Personen oder Gruppen
Die Lizenz darf niemanden benachteiligen.

6. Keine Einschränkungen bezüglich des Einsatzfeldes
Die Lizenz darf niemanden daran hindern, das Programm in einem bestimmten Bereich einzusetzen. Beispielsweise darf sie den Einsatz des Programms in einem Geschäft oder in der Genforschung nicht ausschließen.

7. Weitergabe der Lizenz
Die Rechte an einem Programm müssen auf alle Personen übergehen, die diese Software erhalten, ohne dass für diese die Notwendigkeit besteht, eine eigene, zusätzliche Lizenz zu erwerben.

8. Die Lizenz darf nicht auf ein bestimmtes Produktpaket beschränkt sein
Die Rechte an dem Programm dürfen nicht davon abhängig sein, ob das Programm Teil eines bestimmten Software-Paketes ist. Wenn das Programm aus dem Paket herausgenommen und im Rahmen der zu diesem Programm gehörenden Lizenz benutzt oder weitergegeben wird, so sollen alle Personen, die dieses Programm dann erhalten, alle Rechte daran haben, die auch in Verbindung mit dem ursprünglichen Software-Paket gewährt wurden.

9. Die Lizenz darf die Weitergabe zusammen mit anderer Software nicht einschränken
Die Lizenz darf keine Einschränkungen enthalten bezüglich anderer Software, die zusammen mit der lizenzierten Software weitergegeben wird. So darf die Lizenz z. B. nicht verlangen, dass alle anderen Programme, die auf dem gleichen Medium weitergegeben werden, auch quelloffen sein müssen.


Um zu gewährleisten, dass EntwicklerInnen, die mit GNU-Software arbeiten, den Wissenspool ebenfalls nähren, verpflichtet die Lizenz ihre NutzerInnen zur Weitergabe der Derivate unter gleichen Bedingungen (Copyleft). Die Idee freier Lizenzen geht also auf die Software-Entwicklung zurück, die durch freien Code kollaborative Arbeiten ermöglichen wollte. 


Alle Open Source Lizenzen haben die Einräumung eines Vervielfältigungs- und Verarbeitungsrechts gemeinsam, welches meist bestimmten Voraussetzungen bzw. Beschränkungen unterliegt. Die einzelnen Lizenzen unterscheiden sich vor allem hinsichtlich der Nutzungsbedingungen und der Verpflichtungen, die dem Lizenznehmer zur Wahrnehmung des Vervielfältigungsrechts und des Verarbeitungsrechts aufgegeben werden.

Wichtigster Gesichtspunkt ist, welche Anforderungen an die Weiterverbreitung von veränderten Versionen der Software bzw. neuer Software gestellt werden, die auf Grundlage von Open Source Software entwickelt wurde.



Von Beginn an haben sich verschiedene Lizenzmodelle für OSS herausgebildet, die sich hinsichtlich der Nutzungsbedingungen teilweise recht deutlich unterscheiden. Zur Sichtung, Sammlung 
und Ordnung der vielfältigen Lizenzmodelle hat sich die Open Source Initiative (OSI) gegründet. 
Die OSI hat Kriterien aufgestellt, nach denen sie eine Lizenz als Open-Source-Lizenz klassifiziert 
und in die offizielle Liste der Open-Source-Lizenzen aufnimmt. Die Klassifikation einer Softwarelizenz als Open- Source-Lizenz durch die OSI erfolgt anhand der Rechte, die die Lizenz dem Nutzer 
einer so lizenzierten Software einräumt. Sie erfolgt nicht anhand der Auflagen, die die OpenSource-Lizenz dem Nutzer auferlegt. 
Für eine Einteilung der Open-Source-Lizenzen anhand der Pflichten haben sich die Kategorien 
»permissive Lizenzen«, Lizenzen mit schwachem »Copyleft« und‚ Lizenzen mit starkem 
»Copyleft« eingebürgert. Einige OSS-Lizenzen verpflichten dazu, bei Weitergabe der Software 
den Quellcode zugänglich zu machen. Damit ist manchmal das sog. Copyleft verbunden: 
Danach darf ein Nutzer eine von ihm veränderte Open-Source-Software nur zu den für die 
ursprüngliche Open-Source-Software geltenden Lizenzbedingungen an Dritte weitergeben. 
Die permissiven Lizenzen stellen solche Anforderungen nicht. Die Einteilung in permissive 
Lizenzen und Lizenzen mit starkem und schwachem Copyleft bietet nur eine erste Orientierung. 
Was konkret zu tun ist, um Open-Source-Software lizenzgemäß zu nutzen, ergibt sich stets erst 
aus der konkreten Open-Source-Lizenz. Die Lizenzbedingungen sind natürlich auch bei einer wirtschaftlichen Nutzung von Open-SourceSoftware zu beachten. Daher werden Unternehmen, die Leistungen im Zusammenhang mit OSS 
anbieten oder OSS in eigenen Produkten nutzen, den Umfang ihrer Nutzungsrechte anhand der 
konkreten Softwarelizenz klären, die mit dem Vertrieb verbundenen Risiken abschätzen und ihre 
Entwicklungs- und Vermarktungsstrategie in technischer und rechtlicher Hinsicht darauf ausrichten müssen. In diesem Kontext ist besonders hervorzuheben, dass Open-Source-Software – 
entgegen landläufigen Vorurteilen – sehr wohl kommerziell eingesetzt und vertrieben werden 
darf. Einzig für die Nutzung der Software selbst dürfen keine Lizenzgebühren verlangt werden. 
Unternehmen können jedoch neue Geschäftsmodelle mit und um Open-Source-Software herum 
aufbauen, die nicht auf das traditionelle Lizenzsoftwaregeschäft zurückgreifen. Andererseits sind Unternehmen bei der Einbettung von fremderstellter Open-Source-Software in 
eigene Produkte vor Herausforderungen gestellt, die über die bloße Lizenzerfüllung hinausgehen. 
So müssen sie etwa überlegen, inwieweit sie eine Mängelhaftung für die fremderstellte OSS 
übernehmen können, wenn eine Fehlerbehebung nicht von den Urhebern der Software oder 
der Open-Source-Community angeboten wird. Zu solchen Herausforderungen gehört auch die 
Analyse, ob und wann ein Unternehmen geschäftskritische Alleinstellungsmerkmale in eine 
adaptierte Open-Source-Software hineinprogrammiert, deren Lizenz die Offenlegung des Codes 
im Falle einer Weitergabe des Programms an Kunden vorsieht. Glücklicherweise sind die 
wenigsten Verbesserungen an einem Open-Source-Code wirklich geschäftskritisch. Und viele 
Lizenzen verlangen auch keine solche Offenlegung. Um solche Herausforderungen in den Griff zu bekommen, ist zunächst die Erfassung der im 
Unternehmen und in seinen Produkten verwendeten Open-Source-Software erforderlich. 
Daraus wird sich eine unternehmenseigene Steuerung und Kontrolle der Open-Source-Software 
entwickeln. Hierzu sind ein technisches Software-Management und ein rechtliches Lizenzmanagement durchaus hilfreich. Mehr noch: Die Einrichtung entsprechender Strukturen gehört 
letztlich zu den unternehmerischen Organisationspflichten. Darüber hinaus kann eine 
differenzierte Vertragsgestaltung bei Beschaffung und Vertrieb von Open-Source-Software 
helfen. Bei der Gestaltung von Verträgen über Leistungen im Zusammenhang mit Open-SourceSoftware (etwa Implementierung, Anpassung, Zusatzprogrammierung etc.) und bei der 
Einrichtung eines Lizenzmanagements ist die Beteiligung fachkundiger unternehmensinterner 
oder externer Rechtsberater empfehlenswert.

Trotz der Freiheiten, die Open Source Software offensichtlich bietet, unterliegen die meisten Projekte rechtlichen Schutzmaßnahmen, einschließlich dem Marken- Patent- und Urheberrecht. Aufgrund dessen können, aus der Verwendung von OSS resultierende Lizenzfragen oder rechtliche Konsequenzen, zu einer maßgeblichen Einschränkung insbesondere im Hinblick auf eine mögliche Verteilung oder Weitergabe an Dritte führen, die von Unternehmen genau analysiert und überprüft werden muss.  

In dieser Thesis werden die grundsätzlichen Vorteile und Risiken, die sich aus den verschiedenen Lizenzvereinbarungen bei der Verwendung von OSS ergeben analysiert und die Migration in den Devops-Emntwicklungsprozeess dargestellt. Verdeutlicht wird die Themenstellung anhand der ausgeführten Analysen anahnd eines Beispiels bei der msg Systems.

In diesem Abschnitt wird beschrieben, welche derzeitige Problemstellung innerhalb der DevOps Produktentwicklung bei der Integration von FOSS vorliegt, die mittels der Thesis gelöst bzw verbessert werden kann. Insbesondere wird dabei auf die derzeitige Ist-Situation eingegangen, welche Herausforderungen momentan vorliegen (bei msg und allgemein) und wie der aktuelle Stand der Wissenschaft in diesem Kontext ist. 
Zudem wird eine Herleitung zu den Ziel und den Forschungsfragen dieser Thesis hergestellt.  

\subsection{Ziel und Fragestellung}

In diesem Abschnitt wird das konkrete Ziel der Thesis festgelegt und die Forschungsfragen, an denen sich die Thesis ausrichtet und zu beantworten. Die Forschungsfragen wurden zunächst folgendermaßen festgelegt: \\


1. Welche relevanten Lizenzinformationen können aus FOSS automatisiert extrahiert und verwendet werden? (Lizenzmodelle)\\


2. Welche Einschränkungen sind bei der Verwendung von FOSS in Hinblick auf die verschiedenen Lizenzinformationen zu berücksichtigen? (Haftungs-/Garantiebedingungen, Autorenhinweise, Copyleft-Bedinungen)\\


3. Welche prozessualen Anpassungen müssen anhand der gewonnenen Erkenntnisse getroffen werden, um auf mögliche Veränderungen in den Lizenzinformationen reaktionsschnell vorbereitet zu sein?\\ 


4. Wie können relevante Lizenzinformationen und haftungsbedingte Anforderungen anhand eines Produktes automatisch aus den verwendeten Lizenzmodellen aus FOSS herausgearbeitet werden? (Programm, Makros ???)\\


5. Wie kann die Transparenz der gesammelten Informationen insgesamt sichergestellt werden? (Wie kommen die Informationen schnell an das Devops-Team)\\ 


\subsection{Aufbau dieser Arbeit}

In diesem Abschnitt wird die Methodik, das Vorgehen und die Gliederung der Thesis beschrieben
\dwi{ich glaube du brauchst den aufbau der arbeit nicht beschreiben. das ergibt sich implizit. falls du irgendwo die konkrete
kapitelstruktur erläutern willst, wäre das auf hauptkapitelebene am ende der einleitung angemessen.
du hast keinen experimentierteil, daher glaube ich nicht das sich ein extra "methodik" abschnitt lohnt - oder überzeug mich?}

\section{Fachlicher Rahmen}

In diesem Abschnitt werden grundlegende Prinzipien zu DevOps und FOSS dargetellt, um damit das Verständnis der Leser zu stärken.\\


Heutzutage müssen insbesondere Unternehmen in der Lage sein, einerseits auf veränderte Marktbedingungen schnell zu reagieren und anderseits stabile und qualitativ hochwertige Systeme zu integrieren und diese gleichzeitig zuverlässig zu unterhalten. \cite{humble_why_2011} 

Anhand des DevOps-Ansatzes wird eine Kultur oder Umgebung etabliert, durch die das Erstellen, Testen und Freigeben von Software schnell, häufig und zuverlässiger erfolgen kann. \cite[S.xxviii]{sharma_devops_2017}

In diesem Zusammenhang sollen Ineffizienzen in den Softwareentwicklungs-, Release- und Betriebsprozessen vermieden werden, die durch die organisatorische Trennung zwischen den Prozessen \cite{lwakatare_devops_2019} oder durch die Fehlkommunikation zwischen den Teammitgliedern\cite{ebert_devops_2016} verursacht werden. 

Die daraus resultierenden Vorteile, die sich bei der Verwendung von DevOps ergeben, reichen von einer schelleren Produktbereitstellung und Problemlösung, bessere Ressourcenauslastung und Automatisierung bishin zu einer stabileren Betriebsumgebung.    

Insgesamt soll das Ziel verfolgt werden, die Softwarebereitstellung kontinuierlich sicherzustellen um so reaktionschnell auf Veränderungen am Markt oder Kundenanforderungen reagieren zu können.

Gründe für die Einführung von DevOps können vielschichtig sein.

Wie bereits beschrieben, ist es für Teams innerhalb der Entwicklung nicht möglich, neue Softwareversionen freizugeben oder Softwareänderungen schnell vorzunehmen, wenn der Betrieb die jeweiligen Funktionen nur langsam bereitstellen kann. \cite[S. 7,8]{sharma_devops_2017} 

Die Folgen sind eine verspätete Bereitstellung von Releases, Fehler in den Releases oder eine fehlende Dokumentation. \cite[S. 24]{alt_innovationsorientiertes_2017}

Hinzu kommen Probleme wie mangelndes Knowhow von Entwicklen über die Betriebnahme der verwendeten System, mangelndes Vertrauen in den Ops-Bereich bei fehlender Stabilität der Systeme, verzögertes Testen an Funktionalitäten oder Verschwendung durch mangelnde Wiederverwendung von Quellcode. \cite{humble_why_2011}  

Vor diesem Hintergrund stehen die Bereiche der Entwicklung und des Betriebs oft im Zielkonflikt "Agilität vs. Stabilität" und sehen sich mit verschiedenen Hindernissen konfrontiert, darunter unbefriedigende Testumgebungen und schlechter Informationsfluss. \cite{lwakatare_devops_2019} \cite[S. 8]{sharma_devops_2017}, \cite{konig_devopswelcome_2019}

Die Abbildung 1 zeigt den chronischen Konflikt, der oftmals innerhalb der IT-Organsisation herrscht, der durch fehlende Interaktion zwischen diesen Bereichen enstehen, die häufig unterschiedliche Ziele und Prozesse verfolgen. \cite[S. 349 - 350]{kim_devops-handbuch_2017}

\begin{figure}[h]
    \centering
    \includegraphics[scale=0.6]{Bilder/Core Conflict Clouds}
    \caption{Zentraler chronischer Konflikt nach Gene Kim \cite[S. 349]{kim_devops-handbuch_2017}}
\end{figure}

Mittels DevOps sollen die Interessen aller an der Bereitstellung von Software Beteiligten ausglichen werden, mit besonderem Schwerpunkt auf Entwicklern, Testern und Betriebspersonal. \cite{humble_why_2011}

Die querschnittlich aufgestellten Teams, lösen sich aus abgegrenzten Organisationseinheiten und Verantwortungsbreichen, aus den sogenannten 'Silos' und treiben gemeinsame Ergebnisse durch eine effektive Zusammenarbeit voran. \cite[S.5]{halstenberg_devops_2020} \cite{sollner_devops_2017}

Generell lässt sich die Werte von DevOps in dem bekannten Akronym CALMS zusammenfassen: Kultur (Culture), Automatisierung (Automation), Schlankheit (Lean), Messung (Measurement) und Teilen (Sharing).  

Der Aspekt der Kultur beinhaltet zunächst die Ausrichtung nach dem Menschen.

Hierbei spielt die Kollaboration als ein funktionsübergreifendes Team und die Orientierung nach Kundenwünschen eine tragende Rolle für eine DevOps-Organisation. \cite[S.5]{halstenberg_devops_2020} 

Die Automatisierung von Entwicklung, Implementierung und Tests ist der Schlüssel zum Erreichen niedriger Vorlaufzeiten und damit zu schnellem Feedback. \cite{humble_why_2011}

'Lean' steht in diesem Zusammenhang für die Vermeidung von Verschwendung jedlicher Ressourcen, Wertgeneration, Transparenz und ganzheitliche Betrachtung und Optimierung von Prozessen.

Der Faktor der Messung orientiert sich an den messbaren Daten, um den Fortschritt eines Unternehmens zu begutachten. 

Die definierten und erhobenen Kennzahlen reichen dabei von den Verfügbarkeiten und den Zeiten für die Fehlerbehebung oder Codeänderungen bis zu den Zeiten für die Anforderungsänderungen. \cite[S. 7]{halstenberg_devops_2020}  

Die letzte Säule beschreibt das Teilen von Informationen, Wissen, Vorgehensweisen und Praktiken innerhalb eines oder zwischen verschiedenen Teams unterschiedlicher Abteilungen. \cite{halstenberg_devops_2020} 

Dabei soll eine Umgebung geschaffen werden, in der gegenseitige Austausch, die Kommunikation und die gemeinsame Nutzung im Vordergrund stehen.

Bei dem gesamten Vorteilen und Mehrwerten, die durch DevOps erreicht werden können, kann DevOps jedoch als kein Nullsummenspiel oder Selbstläufer gesehen werden. \cite{humble_why_2011} 

Der DevOps-Ansatz benötigt zunächst einen kulturellen Wandel, was eine große Herausforderung für viele Unternehmen darstellen kann. 

Standardisierte Herangehensweisen sind von einer Vielzahl von Unternehmen über die Jahre tiefgreifend verankert worden, wodurch Mitarbeiter ihre gewohnten Arbeitsabläufe anpassen müssten. 
 
Dies reicht von der Erlernung neuer Tools, Technologien und Methoden, Aufbau einer Kommunikation für den gegenseitigen Austausch, vollständige Automatisierung, Verschmelzung etablierter Rollen und Zuständigkeiten, Schwierigkeiten bei der Implementierung eines automatisierten Deployment-Prozesses oder die Übernahme neuer Aufgaben und Verantworlichkeiten. \cite{lwakatare_devops_2019}, \cite[S. 594 - 595]{abrahamsson_product-focused_2016}, \cite[S. 43 - 45]{halstenberg_devops_2020}

Da sich die Bedeutung von DevOps in den letzten Jahren verschoben hat und immer wieder neue Tools für DevOps auftauchen, handelt es sich bei DevOps um eine stetige Weiterentwicklung. \cite[S. 595]{abrahamsson_product-focused_2016} 

Daher gibt es keinen Standard fester Praktiken im Zusammenhang mit DevOps, wodurch sich nicht festlegen lässt, welche Praktiken für DevOps eingesetzt werden sollten. 

%Die Trennung zwischen Projekten und Betrieb ist zu einer schwerwiegenden Einschränkung geworden, einerseits für die Fähigkeit von Unternehmen, neue Funktionen schneller auf den Markt zu bringen, andererseits für die der IT, stabile und qualitativ hochwertige Systeme und Dienste zu warten. \cite{humble_why_2011} 

%Ersetzen von Softwareentwicklungsmethode durch Kultur, Bewegung oder Praxis.
%Hinzufügung des Hinweises auf Automatisierung

%In diesem Abschnitt werden die grundsächlichen Merkmale von Devops beschrieben. Zudem werden wesentliche Vorteile beschrieben, durch die die Integration von DevOps möglich sind. Des weiteren wird auf das CALMS-Modell (Culture, Automation, Lean, Measurement und Sharing) eingegangen.

%DevOps gilt nicht als ein Nullsummenspiel, indem die Bereitstellungen häufig und zuverlässig in einer stabilen Produktumgebung erreicht werden, sondern es ist ein Ansatz zur Behebung der genannten Probleme durch Kultur, Automatisierung, Schlankheit, Messung und gemeinsame Nutzung, welches auch als CALMS bekannt ist. \cite{humble_why_2011}

Obwohl Mitte des letzten kommerziellen Jahrhunderts die Dominanz innerhalb des Softwaremarktes bei vertriebener Software lag, gewann die Rolle der Open-Source-Software (OSS) zunehmend an Bedeutung, insbesondere durch die offene Bereitsstellung und Entwicklung des freien Betriebssystems 'Linux' \cite[S. 8 - 11]{wichmann_linux-_2005}, \cite[S. 1]{will_open-source-software_2003}. Bereits im Jahre 2019 lag der Einsatz von OSS ab einer Unternehmensgröße von über 2.000 Mitarbeitern bei 86 \%, was bei dieser Unternehmensgröße jedem zweiten Unternehmen entspricht \cite[S. 15]{bitkom_open_2020}. Insgesamt enstand durch wettbewerbsfähige OSS ein grundlegender Umbruch in der Softwarebranche innerhalb der letzten Jahre \cite[S. 185]{bitzer_entwicklung_2007}, \cite{fitzgerald_transformation_2006}. Unter OSS versteht man einen öffentlich zugänglichen Quellcode, den jeder einsehen, verändern und für sich nutzen kann. Durch die Idee der freien Verfügbarkeit des Quellcodes, der entfallenden Lizenzkosten für den Einsatz und der freien Weiterentwicklungsmöglichkeiten, erweist sich OSS als eine einfache und kostengünstige Innovationsquelle für Unternehmen und folglich als eine realistische Alternative zu proprietärer Software \cite[S. 21,22]{allmann_open_2019}. Das Potential und der resultierende Nutzen kann sowohl für Entwickler als auch für Unternehmen sehr vielschichtig sein. Obwohl für ein Unternehmen das Hauptaugenmerk des Einsatzes von OSS auf einen wirtschaftlichen Gewinn in Form von gesenkten Lizenz- und Entwicklungskosten und von Entwicklungszeit abzielt, profitieren Entwickler von dem Wissensaustausch, der einfachen Handhabbarkeit und der Flexibilität, die OSS mit sich bringt \cite{lerner_economic_2005}. Zunächst können Entwickler einerseits Softwarefunktionalitäten nach unternehmensinternen Prozessabläufen abgestimmt entwickeln und andererseits durch den unmittelbaren Zugang zum Quellcode frühzeitig überprüfen, ob ein wiederkehrendes Problem bearbeitet oder nach der jeweligen Problematik individualisiert werden kann. Durch die verfügbaren Standardfunktionalitäten, kann die OSS neben der Möglichkeit der Weiterentwicklung von bereits bestehender Software, dazu verwendet werden, eine Basis für weitere Entwicklungen zu schaffen \cite[S. 37/38]{kesler_anpassung_2013}. Ferner bietet der offene Standard von OSS ein hohes Maß an Anwendungsfeldern, da Entwickler mit wenig Aufwand offenliegende Schnittstellen implementieren können und folglich innerhalb der Hardwareauswahl flexibel und unabhängig sind \cite[S. 2]{kesler_anpassung_2013}, \cite[S. 21,22]{allmann_open_2019}. Durch die Verwendung von OSS sind Entwickler stark unabhängig von Software und von großen Anbietern. Im Gegensatz zu proprietärer Software können Schwachstellen und Sicherheitslücken durch das frühzeitige Testen schneller aufgedeckt und analysiert werden \cite[S. 30/31]{kees_open_2015}.\\\\ Eine weitere wichtige Rolle des Entwicklungsprozesses mittels OSS ist die Wiederverwendung von Quellcode. Anstatt einer zeit- und kostenintensiven Neuentwicklung von Software, bleiben viele allgemeine oder wiederholende Funktionalitäten innerhalb eines neuen Projektes oftmals gleich und müssen demnach nicht verändert werden \cite{henkel_code_2010}. Entwickler versuchen nicht 'das Rad neu zu erfinden', sondern suchen gezielt nach Lösungen von bereits bekannten Problemen. Vor diesem Hintergrund sparen Entwickler viel Zeit und Ressourcen, indem Quellcode, Templates oder Algorithmen wiederverwendet werden, was in der Vergangenheit durch eine maßgeschneiderte Implementierung in kleinem Maßstab erfüllt worden wären \cite{spinellis_how_2004}. Obwohl die Verwendung von OSS auf der einen Seite an Popularität zunimmt, kann auf der anderen Seite zu erheblichen Risiken führen. Insbesondere die Folge der Wiederverwendung kann eine massive Verschachtelung und viele transitive Abhängigkeiten von OSS-Komponenten voraussetzen, durch die die Komplexität erheblich steigt \cite{thelen_beschleunigung_2021}. Jede dieser Komponenten hat eine oder mehrere Lizenzen zum Ziel von rechtlichen Schutzmaßnahmen für den Lizenzgeber, die die Nutzungsbedingungen der jeweiligen Komponente spezifizieren und stark einschränken. Durch die Einbeziehung unterschiedlicher Lizenzmodelle in die OSS-Projekte sollen die Rechte des Urhebers geschützt und die damit uneingeschränkte lizenzfreie Nutzung ohne die jeweilige Genehmigung beschränkt werden \cite{widmer_open-source-lizenzen_2006}. Je nach möglicher Verteilung oder Weitergabe an Dritte der modifizierten Software und Art des Lizenzmodells können unterschiedliche Risiken infolge eines unüberwachten Einsatzes von OSS für ein Unternehmen entstehen. Die daraus resultierenden Lizenzfragen oder rechtlichen Konsequenzen können zu einer maßgeblichen Einschränkung führen, die von Unternehmen bereits im Vorfeld genau analysiert und überprüft werden muss. Mit zunehmender Anzahl von Mitarbeitern und verteilten Standorten wird die Bewältigung dieser Herausforderungen jedoch komplizierter. Die Lösung ist frühzeitige Aufklärung und Schulungen über den technischen und rechtlichen Umgang von OSS, durch die Unternehmen. Diese Unterstützung schafft eine Basis für die verantwortungsvolle Einhaltung von Lizenzbedingungen, angefangen bei den Entwicklern selbst. 

%open-source.2.0 \cite{fitzgerald_transformation_2006}


\section{Proof of Concept}

Während die manuelle Checkliste dafür verwendet wird, präventiv OSS-Komponenten mit riskanten Lieznzmodellen bereits zu Beginn des Entwicklungsprozesses zu vermeiden, verfolgt die automatisierte Checkliste das Ziel, zusätzlich zu einer weiteren Überprüfung von neuen OSS-Komponenten, bereits implementierte auf ihre Lizenzmodelle zu überprüfen, bevor diese ausgeliefert werden. Diese wird innerhalb des folgenden Kapitels in Form eines Proof of Concept (PoC) näher beschrieben. In dem Rahmen dieser Arbeit stellt der PoC einen Prototypen dar, der die prinzipelle Durchführbarkeit einer automatisierten Checkliste aufzeigt, wobei dieser nicht direkt innerhalb des Build-Prozess integriert sondern unabhängig von dem jetzigen Entwicklungsprozess entwickelt wurde. Die automatisierte Checkliste soll dabei als eine reduzierte Version des Endprodukts umgesetzt werden, die auf ihre Nutzbarkeit und Funktionalität getestet und bewertet werden kann. Hierbei wird der Prototyp nicht alle Merkmale und Funktionen eines marktreifen Produkts aufweisen, allerdings wird einerseits die generelle Nutzbarkeit in Abhängigkeit der bestehenden Systemumgebung und andererseits die Demonstration der entwickelten Funktionalitäten zum Zweck einer generellen Realisierbarkeit aufgezeigt. \cite{dreher_prototyping_2018} Gleichzeitig dient der PoC als eine Basis für weitere Weiterentwicklungsmöglichkeiten. Da die direkte Entwicklung innerhalb eines aktiven Projekts große Herausforderungen und Probleme mit sich bringen kann, wurde der Prototyp mit Kollegen der msg systems ag, ausgearbeitet und innerhalb dieser Arbeit festgehalten. Der PoC wurde mit der Skizzierung der grundsätzlichen Idee begonnen. Wesentliche Elemente waren hierbei die momentane Problembeschreibung als auch die darauf aufbauende Lösung, die umgesetzt werden soll. Die Ausarbeitung als auch der Umfang des zugrundeliegenden Problems wurden zunächst durch die Gespräche mit mehreren Teammitgliedern und durch die Teilnahme an Teammeetings anfänglich skizziert. In diesem Rahmen wurde \textit{das Fehlen eines unterstützenden Prozesses hinsichtlich der Bewertung und Prüfung von eingesetzter OSS} als das wesentliche Problem identifiziert, welches mithilfe einer automatisierten Überprüfung gelöst werden sollte. Demnach sollte der Protoytyp verwendete OSS-Komponenten anhand ihrer Lizenzmodelle analysieren und bei einem, mit Risiko verbundenen Lizenzmodell, den jeweiligen Entwickler darüber zeitnah in Kenntnis setzen. Die weitere Anforderungsdefinition, Gestaltung und Einbeziehung erfolgte im Zuge der Prozessmodellierung und der dazugehörigen Anpassung des Soll-Prozesses und wurde in vier grundlegende Schritte unterteilt.





\end{document}

