\dwi{genereller hinweis: gerade in der einleitung motivierst du ja das themenfeld, und da sollten die
kausalitäten sehr klar sein. beim lesen ist mir hier und da unklar wenn du "daher" schreibst, auf was konkret es
sich bezieht, und wo/wie du das argument aufgebaut hast.}

Verändertes Konsumentenverhalten, schnelle Reaktion auf vielseitige Kundenwünsche und Time-to-Market sind wesentliche Anforderungen, die für den wirtschaftlichen Erfolg eines derzeitigen Unternehmens maßgeblich sind.

Die daraus resultierenden wachsenden Herausforderungen an neuer Funktionsvielfalt von Softwareanwendungen und die damit verbundene stetige Weiterentwicklung erfordern in erster Linie eine neue Dynamik innerhalb der Entwicklung, Konzepte und Tools. 

In diesem Zusammenhang wird der Softwareentwicklungsprozess nach agilen Methoden unablässig und gehört in den meisten Unternehmen bereits zum Standard einer IT-Organisation. 

In diesem Zusammenhang kommt die Herangehensweise des DevOps-Ansatzes zum Tragen.

DevOps ist ein zusammengesetztes Wort aus den Begriffen Development und IT-Operations und beschreibt im Wesentlichen die technische und organisatorische Verbindung zwischen der Softwareentwicklung und der Systemadministration.\cite[S.23]{alt_innovationsorientiertes_2017}

Charakterisch dafür, sind verkürzte Releasezyklen in einem inkrementellen Prozess und ein hoher Automatisierungsgrad. 
\dwi{Charakteristisch}

Hierbei soll eine exakte Planung, Vermeidung von kritischen Fehlern und eine schnelle Reaktion auf Kundenanforderungen ermöglicht werden. \dwi{du solltest "exakte planung" hier ausführen (oder weglassen?), oder es wird leute geben die das komisch kommentieren - die unmöglichkeit exakter planung führt zum agilen vorgehen und der integration von ops, um die "fehlerbehebungsschleifen" kürzer zu machen}

Wesentliches Ziel von DevOps ist es, die gesamte Entwicklung transparent, flexibel und agil zu gestalten, wodurch die Produktivität und Effizenz maßgeblich gesteigert und das endgültige Produkt schneller zum Kunden ausgeliefert werden kann.\cite{hemon_agile_2020}  

Neben der agilen Softwareentwicklung ist ein weiteres wesentliches Kennzeichen des DevOps-Konzeptes die Grundsätze des Lean Manufactoring, nach denen die Minimierung von Verschwendung und die Maximierung der Produktivität im Vordergrund steht.\cite{samulat_raus_2017} 

Dabei spielt insbesondere die Wiederverwendung von Software eine wesentliche Rolle.
\dwi{steht benutzung von OSS irgendwo im "lean manifacturing" paper? ich denke die nicht das entwickler das nur "deswegen" machen - sie sind generell faul und versuchen das rad nicht zweimal zu erfinden. falls du dies paper als deinen anflug zum thema "warum oss in devops" verwendest, vielleicht einen direkteren bezug herstellen oder dies ergänzend aufführen.}

Anstatt einer zeit- und kostenintensiven Neuentwicklung von Software, bleiben viele allgemeine oder wiederholende Funktionalitäten innerhalb eines neuen Projektes oftmals gleich und müssen demnach nicht verändert werden.

Vor diesem Hintergrund sparen Entwickler viel Zeit und Ressourcen, indem Quellcode, Templates oder Algorithmen wiederverwendet werden, was wiederrum dem Grundgedanken von DevOps entspricht. 

Ausgehend hiervon wäre die Integration von bestehenden Softwarefunktionalitäten ein wesentlicher Vorteil gegenüber kommentieller Software und damit ein wirtschaftlicher Gewinn für Unternehmen. 

Als eine realistische Alternative zu proprietärer Software, erlangte Open Source Software mit der Idee von frei zugänglicher Software in den letzten Jahrzehnten immer mehr an Popularität.\cite[S. 21,22]{allmann_open_2019} 

Unter Open Source Software versteht man einen öffentlich zugänglichen Quellcode, den jeder einsehen, verändern und für sich nutzen kann.

Aufgrund der freien Verfügbarkeit des Quellcodes, der entfallenen Lizenzkosten für den Einsatz und der freien Weiterentwicklungsmöglichkeiten, erweist sich OSS als eine einfache und kostengünstige Innovationsquelle für Unternehmen.  

Dabei reicht das Anwendungsfeld für die Verwendung von OSS von der Automobilindustrie bis zur Versicherungsbranche und steht daher im direkten Wettbewerb zu proprietären Softwareangeboten.\cite{bitkom_ev_open_nodate}
\dwi{OSS ist weltweit überall im einsatz - ich denke es macht dich unnötig angreifbar das hier an konkreten industrien festzumachen. vielleicht einfach weglassen.}

Das resultierende Potential kann genutzt werden, um Softwarefunktionalitäten nach unternehmensinternen Prozessabläufen genau abzustimmen und zu entwickeln.

Zudem können Entwickler durch den unmittelbaren Zugang zum Quellcode, frühzeitig überprüfen, ob ein wiederkehrendes Problem bearbeitet oder nach der jeweligen Problematik individualisiert werden kann.

Auch erhebliche Sicherheitsprobleme oder -lücken können bereits während dem Entwicklungsprozess gefunden und schnell beseitigt werden.
\dwi{wie geht das vonstatten? hier ist mir unklar woher dieses potential kommt oder wie es vor dem hintergrund der vorigen ausführungen verfügbar wird.}

Damit kann die Notwendigkeit einer kosten- und zeitspieligen Entwicklung entfallen. 

Um auf ungeplante Aufgaben oder Anforderungen flexibel reagieren zu können, bietet sich die Integration von OSS in den DevOps-Entwicklungsprozess an.    

Vor diesem Hintergrund können Unternehmen sicherstellen, dass die Effizenz und die zeitliche Auslieferung von Softwareprodukten gewährleistet ist. 

Trotz der offensichtlichen Freiheiten und Möglichkeiten, die OSS bietet, unterliegen die meisten Projekte rechtlichen Schutzmaßnahmen, einschließlich dem Marken- Patent- und Urheberrecht.

Durch die Einbeziehung unterschiedlicher Lizenzmodelle in die OSS-Projekte sollen die Rechte des Urhebers geschützt und die damit uneingeschränkte lizenzfreie Nutzung ohne die jeweilige Genehmigung beschränkt werden. 

Je nach möglicher Verteilung oder Weitergabe an Dritte von der modifizierten Software und Art des Lizenzmodells können unterschiedliche Risiken infolge eines unkontrollierten Einsatzes von OSS für ein Unternehmen entstehen.

Die daraus resultierenden Lizenzfragen oder rechtlichen Konsequenzen können zu einer maßgeblichen Einschränkung, die von Unternehmen bereits im Vorfeld genau analysiert und überprüft werden muss.  

In dieser Thesis werden die grundsätzlichen Vorteile und Risiken, die sich aus den verschiedenen Lizenzvereinbarungen bei der Verwendung von OSS ergeben analysiert und die Migration in den Devops-Entwicklungsprozess dargestellt.
\dwi{"möglichkeiten zur integration" statt migration trifft es denk ich besser.}

Verdeutlicht wird die Themenstellung anhand der ausgeführten Analysen anahnd eines Beispiels bei der msg Systems.
\dwi{msg systems ag :-)}

% Häufige Vorgehensweise: Moduale Architektur von Open-Sorce projekten
% Beispiele an OSS aufzeigen: Betriebssysteme, Server-Anwendungen oder Office-Produkte
% Evtl hier nochmal auf Linux, als erstes populäres Produkt im OSS-Bereich eingehen 
