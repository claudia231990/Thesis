Verändertes Konsumentenverhalten, schnelle Reaktion auf vielseitige Kundenwünsche und Time-to-Market sind Anforderungen, die für den wirtschaftlichen Erfolg eines Unternehmens in der heutigen Zeit maßgeblich sind. Die daraus resultierenden wachsenden Herausforderungen an neuer Funktionsvielfalt von Softwareanwendungen und die damit verbundene stetige Weiterentwicklung erfordern in erster Linie eine neue Dynamik innerhalb der Entwicklung, Konzepte und Tools. In diesem Zusammenhang wird der Softwareentwicklungsprozess nach agilen Methoden unabdingbar und gehört in den meisten Unternehmen bereits zum Standard einer IT-Organisation. \\\\ An dieser Stelle kommt die Herangehensweise des DevOps-Ansatzes zum Tragen. DevOps ist ein zusammengesetztes Wort aus den Begriffen Development und IT-Operations. Es beschreibt im Wesentlichen die technische und organisatorische Verbindung zwischen der Softwareentwicklung und der Systemadministration \cite[S. 23]{alt_innovationsorientiertes_2017}. Vorteil dabei sind verkürzte Releasezyklen in einem inkrementellen Prozess und ein hoher Automatisierungsgrad. Wesentliches Ziel von DevOps ist es, die gesamte Entwicklung transparent, flexibel und agil zu gestalten. Hierdurch werden die Produktivität und Effizenz maßgeblich gesteigert und das endgültige Produkt kann schneller zum Kunden ausgeliefert werden \cite{hemon_agile_2020}. Neben der agilen Softwareentwicklung ist ein weiteres wesentliches Kennzeichen des DevOps-Konzeptes die Einbeziehung der Grundsätze des Lean Manufactoring, nach denen die Minimierung von Verschwendung und die Maximierung der Produktivität im Vordergrund steht \cite{samulat_raus_2017}. Dabei spielt insbesondere die Wiederverwendung von Software eine wesentliche Rolle \cite[S. 140 - 141]{poppendieck_lean_2010}, \cite[S. 38]{ravichandran_devops_2016}. Anstatt einer zeit- und kostenintensiven Neuentwicklung von Software, bleiben viele allgemeine oder wiederholende Funktionalitäten innerhalb eines neuen Projektes oftmals gleich und müssen demnach nicht verändert werden. Vor diesem Hintergrund sparen Entwickler viel Zeit und Ressourcen, indem Quellcode, Templates oder Algorithmen wiederverwendet werden, was wiederrum dem Grundgedanken von DevOps entspricht. Ausgehend hiervon wäre die Integration von bestehenden Softwarefunktionalitäten ein wesentlicher Vorteil gegenüber kommerzieller Software und damit ein wirtschaftlicher Gewinn für Unternehmen. \\\\ Als eine realistische Alternative zu proprietärer Software, erlangte Open Source Software (kurz: OSS) mit der Idee von frei zugänglicher Software in den letzten Jahrzehnten immer mehr an Popularität \cite[S. 21,22]{allmann_open_2019}. Unter OSS versteht man einen öffentlich zugänglichen Quellcode, den jeder einsehen, verändern und für sich nutzen kann. Aufgrund der freien Verfügbarkeit des Quellcodes, der entfallenen Lizenzkosten für den Einsatz und der freien Weiterentwicklungsmöglichkeiten, erweist sich OSS als eine einfache und kostengünstige Innovationsquelle für Unternehmen. Das resultierende Potential kann genutzt werden, um Softwarefunktionalitäten nach unternehmensinternen Prozessabläufen genau abzustimmen und zu entwickeln. Zudem können Entwickler durch den unmittelbaren Zugang zum Quellcode frühzeitig überprüfen, ob ein wiederkehrendes Problem bearbeitet oder abhängig von der jeweiligen Problematik individualisiert werden kann. Damit kann die Notwendigkeit einer kosten- und zeitintensiven Entwicklung entfallen.\\\\ Um auf ungeplante Aufgaben oder Anforderungen flexibel reagieren zu können, bietet sich die Integration von OSS in den DevOps-Entwicklungsprozess an. Vor diesem Hintergrund können Unternehmen sicherstellen, dass die Effizenz und die zeitliche Auslieferung von Softwareprodukten gewährleistet ist. Trotz der offensichtlichen Freiheiten und Möglichkeiten, die OSS bietet, unterliegen die meisten Projekte rechtlichen Schutzmaßnahmen, einschließlich dem Marken- Patent- und Urheberrecht. Durch die Einbeziehung unterschiedlicher Lizenzmodelle in OSS-Projekte sollen die Rechte des Urhebers geschützt und die damit uneingeschränkte lizenzfreie Nutzung ohne die jeweilige Genehmigung beschränkt werden. Je nach möglicher Verteilung oder Weitergabe an Dritte von der modifizierten Software und Art des Lizenzmodells können unterschiedliche Risiken infolge eines unkontrollierten Einsatzes von OSS für ein Unternehmen entstehen. Die daraus resultierenden Lizenzfragen oder rechtlichen Konsequenzen können zu einer maßgeblichen Einschränkung führen, die von Unternehmen bereits im Vorfeld genau analysiert und überprüft werden müssen. In dieser Thesis werden die grundsätzlichen Vorteile und Risiken, die sich aus den verschiedenen Lizenzvereinbarungen bei der Verwendung von OSS ergeben, analysiert und die Möglichkeiten zur Integration in den Devops-Entwicklungsprozess dargestellt. Verdeutlicht wird die Themenstellung anhand der ausgeführten Analysen am Beispiel der msg systems ag.


% Häufige Vorgehensweise: Moduale Architektur von Open-Sorce projekten
% Beispiele an OSS aufzeigen: Betriebssysteme, Server-Anwendungen oder Office-Produkte
% Evtl hier nochmal auf Linux, als erstes populäres Produkt im OSS-Bereich eingehen 
