\begin{quote}
\textit{In real open source, you have the right to control your own destiny}\newline -Linus Torvalds
\end{quote}

Verändertes Konsumentenverhalten, schnelle Reaktion auf veränderte Kundenwuensche und Time-to-Market sind wesentliche Anforderungen, die für den wirtschaftlichen Erfolg eines Unternehmens insbesondere in Zeiten der Digitalisierung kennzeichnet sind und daher große Herausforderungen mit sich bringen. git staDie daraus resultierenden steigenden Anforderungen an neuer Funktionalität und die damit verbundene stetige Weiterentwicklung erfordern in erster Linie eine neue Dynamik innerhalb der Entwicklug, Konzepte und Tools. In diesem Zusammenhang wird der Softwareentwicklungsprozess nach agilen Methoden unablässig und gehört in den meisten Unternehmen bereits zum Standard einer IT-Organisation. 

In diesem Rahmen kommt der Ansatz des DevOps zum Tragen, nach dessen Grundsätzen der Softwareentwicklungsprozess, angefangen von der Idee, über die schnelle Entwicklung bishin zum Einsatz innerhalb der Produktivumgebung, durchgeführt wird. DevOps ist ein Kofferwort aus den Begriffen Development und IT-Operations und beschreibt im Wesentlichen die technische und organisatorische Verbindung zwischen der Softwareentwicklung und der Systemadministration. Charakterisch sind verkürzte Releasezyklen in einem inkrementellen Prozess und ein hoher Automatisierungsgrad, wodurch eine exakte Planung, Vermeidung von kritischen Fehlern und eine schnelle Reaktion auf Kundenanforderungen ermöglicht wird. Wesentliches Ziel von DevOps ist es, die gesamte Entwicklung transparent, flexibel und agil zu gestalten, wodurch die Produktivität und Effizenz maßgeblich gesteigert und das endgültige Produkt schneller zum Kunden ausgeliefert werden kann. 

Neben der agilen Softwareentwicklung ist ein weiteres wesentliches Kennzeichen des DevOps-Konzeptes die Grundsätze des Lean Manufactoring, nach denen die Minimierung von Verschwendung und die Maximierung der Produktivität im Vordergrund steht. Dabei spielt insbesondere die Wiederverwendung von Software eine wesentliche Rolle. Anstatt einer zeit- und kostenintensiven Neuentwicklung von Software, bleiben viele allgemeine oder wiederholende Funktionalitäten innerhalb eines neuen Projektes oftmals gleich und müssen demnach nicht verändert werden. Vor diesem Hintergrund sparen Entwickler viel Zeit und Ressourcen, indem Quellcode, Templates oder Algorithmen wiederverwendet werden, was wiederrum dem Grundgedanken von DevOps entspricht. 

Ausgehend hiervon ist die Integration von bestehenden Softwarekonzepten ein wesentlicher Vorteil, wodurch die Verwendung von Open Source Software zunehmend an Bedeutung im Unternehmensfeld gewinnt. Als eine realistische Alternative zu herkömlicher Software, erlangte die Open-Source-Bewegung mit der Idee von frei zugänglicher Software in den letzten Jahrzehnten immer mehr an Popularität. Unter Open Source Software versteht man einen öffentlich zugänglichen Quellcode, den jeder einsehen, verändern und für sich nutzen kann. Im Gegensatz zur herkömlicher proprietärer Software, erhalten Nutzer eine Open Source Software meist kostenlos und ohne weitere Kosten für Lizenzvereinbarungen. Durch den unmittelbaren Zugang zum Quellcode, haben Entwickler die Möglichkeit einerseits, am Quellcode mit nur wenigen Ressourcen zu experimentieren und zu überprüfen, ob sich dieser ein wiederkehrendes Problem innerhalb des Projektes beinhaltet oder anderseits nach der jeweligen Problematik zu individualisieren und zu verändern. Die Notwendigkeit einer kosten- und zeitspielige Entwicklung entfällt und die Verwendung von bereits gelösten Paradigmen rückt in den Vordergrund.   

% Häufige Vorgehensweise: Moduale Architektur von Open-Sorce projekten
% Beispiele an OSS aufzeigen: Betriebssysteme, Server-Anwendungen oder Office-Produkte
% Evtl hier nochmal auf Linux, als erstes populäres Produkt im OSS-Bereich eingehen 

Das daraus resultierende Innovationspotenzial kann, basierend auf der freien Gestaltungsfreiheit die OSS im Gegensatz zur kommentieller Lizenzsoftware ermöglicht, genutzt werden, Ideen umzusetzen ohne zu stark an unternehmensinterne Vorgaben gebunden zu sein oder unternehmensinterne Prozesse oder Prozessabläufe in die Entwicklung einzubinden. Vor diesem Hintergrund kann eine umfassende Integration von Open Source Software in vorhandene Unternehmenstrukturen im Bereich der Entwicklung, ausschließlich in einem agilen Umfeld durchgeführt werden, womit DevOps als ideale Plattform geeignet ist.  

Trotz der Freiheiten, die Open Source Software offensichtlich bietet, unterliegen die meisten Projekte rechtlichen Schutzmaßnahmen, einschließlich dem Marken- Patent- und Urheberrecht. Aufgrund dessen können, aus der Verwendung von OSS resultierende Lizenzfragen oder rechtliche Konsequenzen, zu einer maßgeblichen Einschränkung insbesondere im Hinblick auf eine mögliche Verteilung oder Weitergabe an Dritte führen, die von Unternehmen genau analysiert und überprüft werden muss.  

In dieser Thesis werden die grundsätzlichen Vorteile und Risiken, die sich aus den verschiedenen Lizenzvereinbarungen bei der Verwendung von OSS ergeben analysiert und die Migration in den Devops-Emntwicklungsprozeess dargestellt. Verdeutlicht wird die Themenstellung anhand der ausgeführten Analysen anahnd eines Beispiels bei der msg Systems. 




Durch die freie Verfügbarkeit des Quellcodes, die entfallenen Lizenzkosten für den Einsatz und die freie Weiterentwicklungsmöglichkeiten, erweist sich OSS als eine einfache und kostengünstige Innovationsquelle für Unternehmen. 

Dabei reicht das Anwendungsfeld für die Verwendung von OSS in diesem Zusammenhang von der Automobilindustrie bis zur Versicherungsbranche und steht daher im direkten Wettbewerb zu proprietären Softwareangeboten. 