\begin{quote}
\textit{In real open source, you have the right to control your own destiny}\newline -Linus Torvalds
\dwi{Nette Idee. Aber dann sollte man das auch im ersten Satz aufgreifen.
Oder du machst das a) vor die Einleitung, b) auf eine eigene seite oder c) zum Abstract.}
\end{quote}

\dwi{Insgesamt: 
\begin{enumerate}
    \item Du schreibst sehr lange Sätze. Das ist sehr schwer zu folgen.. du kannst bestimmt die Hälfte nochmal aufteilen.
    \item Bitte alle Abkürzungen (FOSS, OSS etc) beim ersten Erwähnen ausschreiben und dann in Klammern die Abkürzung einführen und ab dann (konsequent!) verwenden. DevOps gehört da als Eigenname nicht dazu (es hat je auch keine Langform)
    \item an verschiedenen stellen bieten sich referenzen auf externe quellen oder arbeiten an. dies solltest du gleich beim arbeiten schon einbauen, weil du dann gedanklich schon drin bist. bei einem späteren durchgang verpasst du sonst ggf was
    oder verlierst einen wichtigen kontext. andersrum hilft dir eine eingebaute referenz manchmal auch, dich an den sinn/scope/kontext des aktuellen abschnitts zu erinnern.
\end{enumerate}
}

Verändertes Konsumentenverhalten, schnelle Reaktion auf veränderte Kundenwuensche und Time-to-Market sind wesentliche Anforderungen, die für den wirtschaftlichen Erfolg eines Unternehmens insbesondere in Zeiten der Digitalisierung kennzeichnet sind und daher große Herausforderungen mit sich bringen.
\dwi{Anforderungen,  die  [...] kennzeichnet sind: anforderungen sind nicht gekennzeichnet? Den ersten Satz baust nochmal um. Hier verwebst du zwei logische Fäden.}
Die daraus resultierenden steigenden Anforderungen an neuer Funktionalität \dwi{steigenden  Anforderungen  an  neuer  Funktionalität: liest sich komisch} und die damit verbundene stetige Weiterentwicklung erfordern in erster Linie eine neue Dynamik innerhalb der Entwicklung, Konzepte und Tools.
In diesem Zusammenhang wird der Softwareentwicklungsprozess nach agilen Methoden unablässig und gehört in den meisten Unternehmen bereits zum Standard einer IT-Organisation. 

In diesem Rahmen kommt der Ansatz des DevOps \dwi{komische formulierung?} zum Tragen, nach dessen Grundsätzen der Softwareentwicklungsprozess, angefangen von der Idee, über die schnelle Entwicklung \dwi{die entwicklung selbst ist nicht schnell. lass das hier einfach weg.} bishin zum Einsatz innerhalb der Produktivumgebung, durchgeführt wird.
DevOps ist ein Kofferwort \dwi{komisches wort} aus den Begriffen Development und IT-Operations und beschreibt im Wesentlichen die technische und organisatorische Verbindung zwischen der Softwareentwicklung und der Systemadministration.
Charakterisch sind verkürzte Releasezyklen in einem inkrementellen Prozess und ein hoher Automatisierungsgrad, wodurch eine exakte Planung, Vermeidung von kritischen Fehlern und eine schnelle Reaktion auf Kundenanforderungen ermöglicht wird.
Wesentliches Ziel von DevOps ist es, die gesamte Entwicklung transparent, flexibel und agil zu gestalten, wodurch die Produktivität und Effizenz maßgeblich gesteigert und das endgültige Produkt schneller zum Kunden ausgeliefert werden kann. 

Neben der agilen Softwareentwicklung ist ein weiteres wesentliches Kennzeichen des DevOps-Konzeptes die Grundsätze des Lean Manufactoring, \dwi{das solltest du hier mit einer referenz belegen.
oder behauptest du das? wenn ja, musst das begründen (warum Lean manu wesentlicher aspekt von devops ist)} nach denen die Minimierung von Verschwendung und die Maximierung der Produktivität im Vordergrund steht.
Dabei spielt insbesondere die Wiederverwendung von Software eine wesentliche Rolle.
Anstatt einer zeit- und kostenintensiven Neuentwicklung von Software, bleiben viele allgemeine oder wiederholende Funktionalitäten innerhalb eines neuen Projektes oftmals gleich und müssen demnach nicht verändert werden.
Vor diesem Hintergrund sparen Entwickler viel Zeit und Ressourcen, indem Quellcode, Templates oder Algorithmen wiederverwendet werden, was wiederrum dem Grundgedanken von DevOps entspricht.

Ausgehend hiervon ist die Integration von bestehenden Softwarekonzepten \dwi{es wird ja software, und nicht deren konzepte verwendet? unklar} ein wesentlicher Vorteil, wodurch die Verwendung von Open Source Software zunehmend an Bedeutung im Unternehmensfeld gewinnt.
Als eine realistische Alternative zu herkömlicher Software \dwi{du musst "herkömmlich" (neben dem typo) hier konkretisieren. du meinst ja "herkömmlich=kommerziell/closed source", oder?}, erlangte die Open-Source-Bewegung \dwi{DIE Bewegung? gibt es da genau eine? referenzen!
hier ist eine gute gelegenheit, auf die verschiedenen organisationen und webseiten zu verweisen, die diese bewegung ausmachen} mit der Idee von frei zugänglicher Software in den letzten Jahrzehnten immer mehr an Popularität.
Unter Open Source Software versteht man einen öffentlich zugänglichen Quellcode, den jeder einsehen, verändern und für sich nutzen kann.
Im Gegensatz zur herkömlicher proprietärer Software, erhalten Nutzer eine Open Source Software meist kostenlos und ohne weitere Kosten \dwi{doppelt umsonst kostet! :-) } für Lizenzvereinbarungen.
Durch den unmittelbaren Zugang zum Quellcode, haben Entwickler einerseits die Möglichkeit, am Quellcode mit nur wenigen Ressourcen zu experimentieren und zu überprüfen, ob sich dieser ein wiederkehrendes Problem innerhalb des Projektes beinhaltet oder anderseits nach der jeweligen Problematik zu individualisieren und zu verändern.
Die Notwendigkeit einer kosten- und zeitspielige Entwicklung entfällt und die Verwendung von bereits gelösten Paradigmen \dwi{was ist ein gelöstes Paradigma?} rückt in den Vordergrund.   

% Häufige Vorgehensweise: Moduale Architektur von Open-Sorce projekten
% Beispiele an OSS aufzeigen: Betriebssysteme, Server-Anwendungen oder Office-Produkte
% Evtl hier nochmal auf Linux, als erstes populäres Produkt im OSS-Bereich eingehen 

Das daraus resultierende Innovationspotenzial kann, basierend auf der freien Gestaltungsfreiheit die OSS \dwi{Den Begriff hast du noch nicht 'korrekt' eingeführt.} im Gegensatz zur kommentieller Lizenzsoftware \dwi{du benutzt häufiger den vergleich OSS zu 'non-OSS', aber mit inkonsistener benennung. einige dich auf ein konzept und verwende das immer} ermöglicht, genutzt werden, Ideen umzusetzen ohne zu stark an unternehmensinterne Vorgaben gebunden zu sein oder unternehmensinterne Prozesse oder Prozessabläufe in die Entwicklung einzubinden. \dwi{das check ich nicht. inwiefern befreit einen die verwendung von OSS von internen vorgaben? es kann sogar anders herum sein - interne vorgaben verbieten die verwendung von OSS oder bestimmten Lizenzmodellen.. bitte klarer.}
Vor diesem Hintergrund kann eine umfassende Integration von Open Source Software in vorhandene Unternehmenstrukturen im Bereich der Entwicklung, ausschließlich \dwi{warum ausschließlich? agile oder nicht hat doch mit OSS nichts zu tun?} in einem agilen Umfeld durchgeführt werden, womit DevOps als ideale Plattform geeignet ist. \dwi{hier solltest noch mal schleifen. mir
erschließt sich die argumentation noch nicht so ganz.}

Trotz der Freiheiten, die Open Source Software offensichtlich bietet, unterliegen die meisten Projekte rechtlichen Schutzmaßnahmen, einschließlich dem Marken- Patent- und Urheberrecht.
Aufgrund dessen können, aus der Verwendung von OSS resultierende Lizenzfragen oder rechtliche Konsequenzen, zu einer maßgeblichen Einschränkung insbesondere im Hinblick auf eine mögliche Verteilung oder Weitergabe an Dritte führen, die von Unternehmen genau analysiert und überprüft werden muss.  

In dieser Thesis werden die grundsätzlichen Vorteile und Risiken, die sich aus den verschiedenen Lizenzvereinbarungen bei der Verwendung von OSS ergeben analysiert und die Migration in den Devops-Emntwicklungsprozeess dargestellt.
Verdeutlicht wird die Themenstellung anhand der ausgeführten Analysen anahnd eines Beispiels bei der msg Systems.

\dwi{Ich lese die Absätze sinngemäß so: \begin{enumerate}
       \item schneller markt braucht schnelle entwicklung
       \item dazu ist devops geeignet
       \item lean manufacturing (modularisierung, wiederverwendung)
       \item definition open source
       \item innovationspotential
       \item aber: lizenzen
       \item scope thesis
   \end{enumerate} 
   das ist so erstmal gut. der aspekt lizenzen gegenüber Devops kommt hier allerdings etwas zu kurz, vor allem wenn du schreibst 'in dieser Thesis werden .. Vorteile und Risiken, die sich aus .. Lizenzvereinbarungen .. ergeben'.
   hier wäre vielleicht noch ein absatz 6.b gut, in dem man etwas mehr zu OSS und lizenzen/modellen sagt. die 'breite' und vielzahl der lizenzen mit ihren auswirkungen ist hier noch nicht sehr greifbar. denke an die zielgruppe: DevOps-Entwickler ohne juristischer hintergrund..}
