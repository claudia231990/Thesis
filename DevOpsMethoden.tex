Um ein klareres Bild über die Methoden, die bei DevOps relevant sind, zu schaffen, werden nun die Begriffe wie Continuous Integration, Continuous Delivery, Continuous Deployment und die Deployment Pipeline erklärt. \\

Unter Continuous Integration versteht man im Wesentlichen den Build-Prozess und die darauffolgenden automatisierten Tests, die durch das Einchecken von neuen Code Changes in das Software Repository ausgelöst werden. Dies sorgt dafür, dass schnell klar ist, ob neue Codeänderungen einer Komponente im Gesamtkontext der Anwendung lauffähig sind. Durch das häufige Testen ist außerdem die Zuordnung des Fehlers zur verursachenden Änderung leichter herzustellen.\\

Der Continuous Integration-Prozess ist nach den Änderungen am Source Code und der Ausführung der Tests abgeschlossen und beginnt daraufhin wieder von vorn. Continuous Delivery setzt an dieser Stelle an und erweitert den Feedback-Zyklus bis in die Produktion. Erst wenn die Anwendung in die Produktion deployt wurde und dem Kunden zur Verfügung steht, ist die "Definition of Done" erfüllt. Vor diesem Hintergrund wird Continuous Delivery auch als finale Stage oder "letzte Meile" von Continuous Integration bezeichnet.\\

Continuous Deployment (das andere „CD“) kann sich auf die automatische Freigabe von Entwickleränderungen vom Repository zur Produktivphase beziehen, wo sie direkt vom Kunden genutzt werden können. Dieser Vorgang soll der Überlastung von Operations-Teams bei manuellen Prozessen entgegenwirken, die die Anwendungsbereitstellung verlangsamen. Continuous Development baut die Vorteile der Continuous Delivery aus, indem auch noch die nächste Phase der Pipeline automatisiert wird.\\

Unter Deployment Pipeline versteht man einen Prozess, der die Schritte Build, Test und Deployment des Softwareauslieferungsprozesses über mehrere Stages hinweg automatisiert durchläuft.\\


Nachfolgend werden verschiedene Testarten und Begriffe erläutert: Staging-Umgebung, Unit Test, Acceptance Test, Integration Test, Smoke Test: