Zunächst konnte mittels der Implentierung aufgezeigt werden, dass sich das Plugin ideal für den benötigten automatisierten OSS-Check eignet. 

Durch die Fehlermeldungen und die Abbrüche des Build-Prozesses wird der Entwickler innerhalb des Softwareentwicklungsprozess in Kenntnis gesetzt, ob eine Lizenz mit einem beschränkten und/oder starken Copyleft vorliegt. 

Zunächst muss berücksichtigt werden, dass Verweise auf die licence.xml und die allowedMissingLicence.xml innerhalb des Aufrufs des Plugins innerhalb der pom.xml befindet. 

Daher muss die Struktur und der entsprechende Pfad der licence.xml und allowedMissingLicence.xml zwingend eingehalten werden.

Zudem müssen beide Dateien das Grundgerüst für den Vergleich bereits beinhalten, also nicht leer sein. 

Sollten diese Strukturen nicht eingehalten werden, wird ebenfalls eine Fehlermeldung angezeigt. 

Darüber hinaus bietet das Plugin nicht den höchsten Grad der Automatisierung, da Entwickler bei unbekannten und transitiven Abhängigkeiten und Lizenzen zunächst recherchieren müssen, ob diese einen Copyleft-Effekt besitzen oder aussschließlich zu Testzwecken verwendet werden.  

Daher kann eine vollständige Abkehr von manuellen Tätigkeiten mittels des Ayoy-Plugins nicht vorrausgesetzt werden. 

Allerdings ist zu bedenken, dass im Verhältnis zum jetztigen IST-Zustand ein erheblicher Zeitaufwand reduziert werden kann. 

An dieser Stelle muss darüber hinaus festgehalten werden, dass keine Evaluierung des Ayoy-Plugins innerhalb des jetztigen Projektes vorliegt. 

Der PoC wurde aufgrund von einer zeitintensiven Integration in den bestehenden Projektablauf nicht durchgeführt, womit die vollständige Bewertung insbesondere durch die Entwickler ausfällt. 

Allerdings zeigt der PoC die grundsätzliche Funtkions- und Vorgehensweise auf, wie das Plugin zu handhaben ist und welches Potential und Vorteile es als integriertes Tool bietet.
 