Die vorliegende Arbeit hat das Ziel, die Verwendung von OSS in die DevOps-Produktentwicklung zu integrieren. Die Erhebung der Daten erfolgte zunächst durch eine umfassende Literaturrecherche. An dieser Stelle wurde zunächst eine geeignte Fachliteratur hinzugezogen, um einen Überblick über den fachlichen Rahmen der Themenfelder OSS und DevOps zu erhalten. Die Fachliteratur beschränkte sich dabei auf Zeitschriftenartikel, Bücher und interne Informationen der msg systems ag. Letzlich wurden wissenschaftliche Artikel über Neuerungen und Bücher über Grundlagen einbezogen wurden. Um den Softwareentwicklungsprozess basierend auf dem Einsatz von OSS anzupassen, wurde zunächst der derzeitige Prozess modelliert und anhand der benötigten Veränderungen und der bisher aufgearbeiteten theoretischen Grundlagen überarbeitet. Der vorliegende Prozess wurde dabei mehreren Korrekturen unterzogen, um alle Gesichtspunkte des Prozesses, der Rollen und der Tätigkeitsfelder zu erfassen. Die dabei hinzugezogenen Korrektoren sind als DevOps-Entwickler direkt in den Prozess miteinbezogen und liefern daher aussagekräftige Informationen über die Problemstellung und Anregungen hinsichtlich des Themenfeldes. Darüber hinaus konnte mit der Analyse des derzeitigen Prozesses wesentliche Forschungsfragen der vorliegenden Arbeit beantwortet werden. Die Anpassungen des Prozesses wurden anhand der manuellen Checkliste und dem Proof of Concept umgesetzt und prototypisch getestet. 
