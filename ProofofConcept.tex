Während die manuelle Checkliste dafür verwendet wird, präventiv OSS-Komponenten mit riskanten Lieznzmodellen bereits zu Beginn des Entwicklungsprozesses zu vermeiden, soll die automatisierte Checkliste das Ziel verfolgen, bereits implementierte OSS-Komponenten auf ihre Lizenzmodelle zu prüfen, bevor diese ausgeliefert werden. 

Diese wird innerhalb des folgenden Kapitels als der Proof of Concept (PoC) näher beschrieben. 

Grundsätzlich bestimmt der PoC die Machtbarkeit einer Idee, also ob die Idee wie geplant funktioniert und in die Realität umgesetzt werden kann, bevor für diese Ressourcen auf Produktionsebene verschwendet werden.  

In dem Rahmen dieser Arbeit stellt der PoC einen Prototypen dar, der die prinzipelle Durchführbarkeit einer automatisierten Checkliste aufzeigt, wobei dieser nicht direkt innerhalb des Build-Prozess integriert sondern unabhängig von dem jetzigen Entwicklungsprozess entwickelt wurde.  

Obwohl der PoC nicht immer mit einem Prototypen gleichzusetzten ist, werden in diesem Fall beide Begriffe gleichermaßen behandelt. 

Die automatisierte Checkliste soll dabei als eine reduzierte Version des Endprodukts umgesetzt werden, die auf ihre Nutzbarkeit und Funktionalität getestet und bewertet werden kann. 

Hierbei wird der Prototyp nicht alle Merkmale und Funktionen eines marktreifen Produkts aufweisen, allerdings wird einerseits die generelle Nutzbarkeit in Abhängigkeit der bestehenden Systemumgebung und andererseits die Demonstration der entwickelten Funktionalitäten zum Zweck einer generellen Realisierbarkeit aufgezeigt. 

Gleichzeitig dient der PoC als eine Basis für weitere Weiterentwicklungsmöglichkeiten. 

Da die direkte Entwicklung innerhalb eines aktiven Projekts große Herausforderungen und Probleme mit sich bringen kann, wurde der Prototyp mit den Kollegen des Projektes bei der msg systems ag, ausgearbeitet und innerhalb dieser Arbeit festgehalten. 

Dies hatte den Vorteil, dass unvorhersehbare Problemfelder berücksichtigt wurden und der Prototyp direkt in die Systemlandschaft eingebunden werden konnte, ohne den gesamten Projektablauf in Mitleidenschaft zu ziehen.

Der PoC wurde mit der Skizzierung der grundsätzlichen Idee begonnen. 

Wesentliche Elemente waren hierbei die momentane Problembeschreibung als auch die darauf aufbauende Lösung, die umgesetzt werden soll. 

Die Ausarbeitung als auch der Umfang des zugrundeliegenden Problems wurden zunächst durch die Gespräche mit mehreren Teammitgliedern und durch die Teilnahme an Teammeetings anfänglich skizziert.

In diesem Rahmen wurde \textit{die fehlende Bewertung und Prüfung bei der Verwendung von OSS} als das wesentliche Problem identifiziert, welches mithilfe einer automatisierten Überprüfung gelöst werden sollte.  

Demnach sollte der Protoytyp verwendete OSS-Komponenten anhand ihrer Lizenzmodelle analysieren und bei einem, mit Risiko verbundenen Lizenzmodell, den jeweiligen Entwickler darüber zeitnah in Kenntnis setzen.  

Die weitere Anforderungsdefinition, Gestaltung und Einbeziehung erfolgte im Zuge der Prozessmodellierung und der dazugehörigen Anpassung des Soll-Prozesses und wurde in vier grundlegende Schritte unterteilt.



