Zusammenfassend kann festgestellt werden, dass alle Ziele dieser Arbeit erreicht wurden und die anfangs beschriebenen Forschungsfragen beantwortet werden konnten.Zunächst konnte durch die gesammelten Informationen ein umfassender Überblick über die Themengebiete von DevOps als auch OSS gewonnenen werden. Im Bereich des DevOps wurde die Kultur, Methoden und Funktionsweisen aufgezeigt, die für den praktischen Teil dieser Arbeit benötigt wurden. So wurde beispielsweise der hohe Grad an Automatisierung als ein wesentlicher Kernaspekt von DevOps bei der Modellierung des Soll-Prozesses als ein Kriterium festgelegt. Innerhalb des Themengebietes von OSS konnte ein Überblick über die Lizenarten und deren Lizenzvereinbarungen geschaffen werden. An dieser Stelle konnten die entsprechenden Einschränkungen, Rechte und Pflichten und juristische Konsequenzen aufgezeigt werden, um das Verständnis des Einsatzes von OSS zu verbessern. Die gewonnenen Informationen lieferte die Basis, um den bisherigen Prozess, mit dem Ziel einer umfassenden Bewertung und Überprüfung bei der Verwendung von OSS anzupassen. Hierzu diente der modellierte Ist-Prozess die wesentliche Grundlage für Veränderungen an den jeweiligen Aufgaben. Bei der Anpassung des Soll-Prozesses war es wichtig, eine Möglichkeit für das DevOps-Team zu gestalten, um den Einsatz von neu hinzugefügten und bereits in den Softwareentwicklungsprozess eingebundenen OSS-Komponenten zu überprüfen.
Ferner konnte im Prozess Aufgaben für eine automatisierte Überprüfung eingebunden werden, um einerseits das Konzept von DevOps und anderseits eine schnelle Möglichkeit zur Überprüfung von bisherigen Lizenzinformtionen der Komponenten als auch deren Abhängigkeiten in den Softwareentwicklungsprozess zu berücksichtigen. Im Rahmen des PoC wurde mittels das Ayoy-Plugin aufgezeigt, dass bestehende OSS-Komponenten nach ihren Lizenzen und Abhängigkeiten überprüft und beurteilt werden können. Durch den Abbruch oder den erfolgreichen Builds konnte sichergestellt werden, dass das DevOps-Team transparent über den Zustand der jeweiligen OSS-Komponente in Kenntnis gesetzt werden. Insbesondere OSS-Komponenten mit starken Copyleft-Klauseln können so reaktionsschell bemerkt werden. Anhand des Soll-Prozess können dementsprechend utnerschiedliche Maßnahmen des Entwicklers erfolgen.


Ein möglicher Ausblick stellt die mögliche Weiterentwicklung des Ayoy-Plugins dar. So müsste dieser in den Softwareentwicklungsprozess des Beispielprojektes der msg systems ag integriert werden, um durch die produktive Verwendung, Schwachstellen oder Stärken herausfinden zu können. Hinzu kommt die Möglichkeit bestehende Dokumente über verwendete Lizenzen in ein xml.Format zu parsen und dieses ebenfalls in das Plugin zu integrieren. 