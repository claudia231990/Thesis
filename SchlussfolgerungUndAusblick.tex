In dieser Masterthesis konnte aufgezeigt werden, dass alle Ziele dieser Arbeit erreicht wurden und die anfangs beschriebenen Forschungsfragen beantwortet werden konnten. Zunächst konnten wesentliche Informationen aufgearbeitet werden, um einen umfassenden Überblick über die Themengebiete von DevOps und OSS zu erhalten. So wurden innerhalb des Bereichs von DevOps konkrete Themen wie Kultur, Methoden und Funktionsweisen aufgezeigt. Die dabei gewonnenen Informationen waren insbesondere für das Verständnis des jetzigen Softwareentwicklungsprozesses bei der msg systems ag und der darauf aufbauenden Modellierung des Ist- und Soll-Prozesses, essenziell. So musste beispielsweise der hohe Grad an Automatisierung als ein wesentlicher Kernaspekt von DevOps bei der Modellierung des Soll-Prozesses als Kriterium berücksichtigt werden. Innerhalb des Themengebietes von OSS wurden vorrangig praxisrelevante Informationen herangezogen. An dieser Stelle wurde ein Überblick über unterschiedliche, aber teilweise bereits eingesetzte Lizenzarten und deren Lizenzvereinbarungen geschaffen. So wie das Zitat am Anfang dieser Arbeit beschreibt, ist OSS zwar frei zugänglich und kostenlos erhältlich, sollte aber nicht mit freiem Umgang verwechselt werden. An dieser Stelle wurden die entsprechenden Einschränkungen, Rechte und Pflichten und juristische Konsequenzen aufgezeigt, um das Verständnis des Einsatzes von OSS zu verbessern. Mit der Modellierung der Abläufe des Ist-Prozesses wurde zunächst eine Basis erstellt, um den bisherigen Prozess, mit dem Ziel einer umfassenden Bewertung und Überprüfung bei der Verwendung von OSS, anzupassen. Der Soll-Prozess hingegen enthält neue Abläufe und Aufgaben, die auf den Einsatz von OSS und der Verwendung von Copyleft-Klauseln abgestimmt sind. Ausgehend davon war es innerhalb des Soll-Prozesses wichtig, eine Möglichkeit für das gesamte DevOps-Team zu schaffen, um den Einsatz von neu hinzugefügten, bereits in den Softwareentwicklungsprozess eingebundenen OSS-Komponenten und deren Abhängigkeiten zu überprüfen. In dieser Hinsicht dient die manuelle Checkliste dazu, die gesammelten Informationen an das DevOps-Team weiterzugeben und als eine Möglichkeit, präventive Kontrollen zu dem Einsatz von OSS herzustellen. Die automatisierte Checkliste hingegen kann dazu verwendet werden, neue Strukturen und Maßnahmen innerhalb des Entwicklungsprozesses anhand der eingesetzten OSS zu etablieren und diente dem Proof of Concept als Konzept. Im Rahmen des PoC wurde mittels das Ayoy-Plugin aufgezeigt, dass bestehende OSS-Komponenten nach ihren Lizenzen und Abhängigkeiten überprüft und beurteilt werden können. Durch den Abbruch oder den erfolgreichen Builds konnte sichergestellt werden, dass das DevOps-Team der msg systems ag transparent über den Zustand der jeweiligen OSS-Komponente in Kenntnis gesetzt werden. Insbesondere OSS-Komponenten mit starken Copyleft-Klauseln können so reaktionsschell bemerkt werden. Durch die aufgezeigten Testfälle sollen alltägliche Situationen darstellen, die sich im Zusammenhang des Einsatzes von OSS ergeben. Ausgehend von dem praktischen Teil dieser Arbeit, hat sich gezeigt, dass der Softwareentwicklungsprozesses des Beispielprojektes der msg systems ag geändert werden kann und anhand des Pluging als eine technische Umsetzung in den Entwicklungsprozess möglich ist. 

Als Empfehlung an die msg systems ag kann an dieser Stelle festgehalten werden, dass die Weiterentwicklung des Plugins ein mögliches Handlungsfeld darstellt.So müsste dieser in den Softwareentwicklungsprozess des Beispielprojektes produktiv integriert werden, um einerseits Schwachstellen oder Stärken herausfinden zu können und anderseits das Plugin, nach dem entsprechenden Softwareentwicklungsprozesses weiteerzuentwickeln. Hinzu kommt die Möglichkeit bestehende Dokumente über verwendete Lizenzen in ein xml.Format zu parsen und dieses ebenfalls in das Plugin zu integrieren. Darüber hinaus besteht die Möglichkeit, eine Nachdokumentation anzufertigen, um einen Überblick der verwendeten Lizenzen jedes DevOps-Mitgliedes zu erhalten. Ferner kann zu einem späteren Zeitpunkt, die manuelle Checkliste als eine webbasierte Lösung zu entwickeln. Hierbei könnten die Verpflichtungen, Lizenzen und Nutzungstypen direkt eingegeben werden, um die Liste direkt zu erhalten und nicht mehr manuell abzulesen brauchen. 



