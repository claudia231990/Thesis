Neben Fragen über die Rechte und Pflichten des Lizenznehmers insbesondere im Hinblick auf die Weitervergabe, spielt der Einsatz von OSS eine erhebliche vertragsrechtliche Rolle in Unternehmen. Ferner müssen Unterschiede durchgeführt werden, in welchem Rahmen das Vertragsverhältnis besteht, welche Auswirkungen eine Vertragsverletzung hat und in welchem Umfang sich die Weitergabe rechtlich auswirkt. Die Darstellung von rechtlichen Implikationen ist innerhalb dieser Arbeit ausschließlich auf die Haftung und den Haftungs- und Gewährleistungsausschluss beschränkt. 

\paragraph{2.2.5.1. Begrifflichkeiten innerhalb der rechlichen Untersuchung} $~$
In diesem Abschnitt werden zunächst die wesentlichen Grundlagen der rechtlichen Untersuchung beschrieben, um ein besseres Verständnis für die weiteren Abschnitte sicherzustellen. 

\subparagraph{Urheber und Urheberrecht}$~$

Gemäß § 7 des UrhG gilt der Urheber als eine Person, die der "Schöpfer eines Werkes" darstellt. Das Urheberrecht zielt darauf ab, den Urheber einer Software eindeutig zuzuordnen, um seine persönlichen Rechte an dem Werk schützen zu können. (Quelle) Daher sollte in den OSS-Projekten mit einem Lizenztext der Name des Autors, als Urheber vermerkt sein. Der Urheber gilt als der Rechtsinhaber des Werkes und kann folglich nicht das Urheberrecht an sich sondern nur das darin befindliche Recht gemäß § 31 UrhG auf Nutzung einräumen. 

\subparagraph{Patent}$~$

Ein Patent stellt gemäß § 16 PatG ein Schutzrecht für eine Erfindung dar. Durch die Dauer des Patentes auf 20 Jahre, ist in dieser Zeit die gewerbliche Nutzung der Erfindung für jedermann untersagt. So kann Software patentiert werden, wenn es sich hierbei um computerimplementierte Erfindung über das regelmäßige Zusammenwirken mit einem Computer hinaus einen technischen Inhalt aufgewiesen wird. (Quelle)  

\subparagraph{Haftung}$~$

Die Haftung ist gesetzlich nicht direkt festgelegt, kann sich aber gemäß §§ 276 und 459 BGB ergeben. Insgesamt lässt sich die Haftung als eine Verpflichtung für einen Ersatz durch einen entstandenen Schaden bestimmen. Das Verschulden des Schuldern kann auf Fahrlässigkeit oder Vorsatz beziehen. 

\subparagraph{Gewährleistung}$~$

Gewährleistung beschreibt gemäß § 437 BGB und § 634 BGB die Verpflichtung, Produkte oder Dienstleistungen in einem mangelfreien Zustand zu übergeben. Im Hinblick auf eine Software, würde die Gewährleistung einer vollständigen Funktionalität ohne Fehler entsprechen. 

\paragraph{2.2.5.2. Folgen einer Lizenzverletzung}$~$

Sollte eine Software lizenzwidrig verwendet werden, hat der Urheber das Recht Ansprüche gegen den Rechtsverletzer geltend zu machen. Dies kann sich auf die Zahlung eines Schadensersatzes und/oder auf das Vertriebsverbot der bereits modifizierten Software beziehen.(Quelle) 

\subparagraph{Rechtsfolgen aus Lizenzvertrag}$~$

Aus vertragsrechtlicher Sicht erfolgen Rechtsfolgen basierend auf den Copylfet-Klauseln. Sollten Lizenzen keine Angaben zur Regelung von Verstößen aufweisen, wird von einer gesetzlichen Vertragsverletzung ausgegangen. Sollten Regularien bezüglich eines Lizenzverstoßes vorliegen, führt das meist automatisch zu einem Wegfall des Nutzungsrechts durch die erloschene Zustimmung des Urhebers auf weitere Nutzung. Dies wird als "auflösende Bedingung" bezeichnet.(Quelle) Allerdings erstreckt sich der Wegfall der Nutzungsrechte ausschließlich den Rechtsverletzer, nicht den 'unschuldigen' Dritten. (Quelle)

\subparagraph{Rechtsfolgen aus dem Urheberrecht}$~$

Werden Pflichten des Nutzers innerhalb der Lizenzbestimmungen verletzt, wird neben einer Vertragsverletzung auch von einer Urheberrechtsverletzung ausgegangen. (Quelle) Die Ansprüche des Urhebers sind in diesem Fall gemäß §§ 97 ff. UrhG geregelt und können sofort nach der Lizenzverletzung geltend gemacht werden. 

\begin{itemize}
    \item Unterlassung (§ 97 Abs. 1 UrhG)
    \item Schadensersatz (§ 97 Abs. 2 UrhG)
    \item Vernichtung, Rückruf und Überlassung (§§ 67 ff. iVm 98 UrhG)
    \item Auskunft und Einsicht (§§ 101, 101a UrhG)
\end{itemize}

Gleichzeitig macht sich der Rechtsverletzer wegen unerlaubter Verwertung (§ 106 UrhG) strafbar. Insbesondere bei dem Anspruch auf Schadensersatz können erhebliche Konsquenzen auf ein Unternehmen zukommen. Der Urheber kann in dem Fall des verschuldensabhängigen Schadensersatzanspruchs, die Schadenshöhe nach dem Verletzergewinn, dem konkreten Schaden und nach einer angemessenen Lizenzgebür berechnen lassen. (Quelle) Der Anspruch auf Vernichtung, Rückruf und Überlassung basierend in erster Linie auf Vervielfältigungen, also alle Kopien und auf den verwendeten Vertriebswegen. Häufige Schwierigkeiten findet sich insbesondere im Anspruch auf Auskunft und Einsicht, insbesondere wenn der Quellcode in eine propritäre Software integriert worden ist.(Quelle) Ob dieser Anspruch durchgesetzt werden kann, hängt sehr stark von dem Interessenkonflikt zwischen der Einsicht durch den Lizenzgeber als ein sicheres Mittel der Beweislast und dem geistigen Schutz des Quellcodes des Nutzers, ab.(Quelle)

\paragraph{2.2.5.3. Haftungs- und Gewährleistungsausschluss}$~$

Ein weiteres rechtliches Risiko durch den Einsatz von OSS ergibt sich unter Geltendmachung von ausländischen Rechts. Trotz der Unwirksamkeit nach deutschem Recht, rücken in diesem Fall die Wirksamkeit von Gewährleistungs- und Haftungsausschlüssen in den Vordergrund und müssen beachtet werden. 

\subparagraph{AGB-Recht}$~$

Grundsätzlich lassen sich die OSS-Lizenzen in den Allgemeinen Geschäftsbedingungen (AGBs) gemäß §§ 305 ff. BGB festlegen. Gemäß § 305 BGB stellen AGBs vorformulierte Vertragsbedingungen dar, die ein Vertragspartner einer anderen Partei zum Abschluss eines Vertrags stellt. Daher sind in AGBs durch die Wirksamkeitskontrolle wirksam, sobald diese keine unangemessenen Benachteiligung § 307 Abs. 2 BGB aufweisen. Generell muss der Vertragspartner die Möglichkeit einer zumutbaren Kenntnisnahme erhalten und die Annahme des Angebots durch die Bestätigung der AGBs kenntlich machen. (Quelle) Sollte der Nutzer einer OS-Lizenz mit den vertragsbedingen Lizenzbestimmungen einverstanden sein, nachdem dieser in zumutbarer Weise davon Kenntnis erhalten hat, kann der Lizenzvertrag als geschlossen betrachtet werden. (Quelle) Fraglich wird dieser Umstand bei dem Download eines OSS-Projektes. (Quelle) Das Lesen von einer erheblichen Vielzahl von Lizenzbestimmungen kann als 'unzumutbar' betrachtet werden, selbst dann wenn der Nutzer den AGBs widerwillig zustimmt.

\subparagraph{Wirksamkeit von Gewährleistungsrechten}$~$

Fraglich ist die Wirksamkeit der Gewährleistungsausschlüsse, sobald OSS-Lizenzen als AGBs eingestuft sind. Gemäß §§ 305 ff. BGB können OSS-Lizenzen, die als AGBs eingeordnet werden, nicht von der Gewährleistung ausgeschlossen werden. Angaben diesbezüglich sind nach deutschem Recht daher als unwirksam zu betrachten. Ein Entfernen der Gewährleistungsausschlüsse sollte ebenfalls nicht erfolgen, da diese abhängig von den Bearbeitungs- und Verbreitungsrechten sind. (Quelle) Sollte also der Vermerk des Gewährleistungsausschlusses vom Lizenzgeber entfernt werden, kann der Nutzer weder die OSS verändern oder gar weiterbreiten, was jedoch den Kern der OS-Lizensierung entspricht. Bei Integration von OSS in kommzielle Software gelten die Gewährleistungsverpflichtungen des Anbieters.(Quelle) 

\subparagraph{Wirksamkeit von Haftungsregelungen}$~$

Ähnlich zu den Gewährleistungsausschlüssen enthalten die OS-Lizenzen einen Vermerk über einen Haftungsauschluss. Gemäß §§ 305 ff. BGB entspricht dieser Ausschluss nicht der akutellen Gesetzeslag und ist daher unwirksam, da grundsätzlich keine Haftung, basierend auf Vorsatz oder Fahrlässigkeit, ausgeschlossen werden darf. Nach deutschem Recht kann die Haftung gemäß § 521 BGB, aufgrund einer Schenkung, beschränkt aber nicht ausgeschlossen werden, was bei der Verwendung von OSS-Projektes am nächsten entspricht. (Quelle) \\\\

Insgesamt sind Haftungs- und Gewährleistungsausschlüsse insbesondere im amerikanischen Recht eine beliebte Praxis, die Schuld für etwaige Schäden auf die Verbraucher abzuwälzen. Ein Verbraucher muss nach deutschem Recht immer in der Lage sein, bei vorsätzlichen und fahrlässigen Mängeln einen entsprechenden Sach- oder Rechtsanspruch zu erlangen. Allerdings muss berücksicht werden, dass Ansprüche aus dem Gewährleistungs- und Haftungsauschluss durchaus aus dem Ausland geltend gemacht werden können. (Quelle) Hintergrund ist, dass Anbieter von OSS-Projekten in Ländern wohnhaft sind, die solche Ausschlüsse als legitim erachten. Deutsche Lizenzgeber können Ansprüche in einem geringen Umfang, aus dem Schenkungsvertrag geltend machen. 

