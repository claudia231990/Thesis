Obwohl Mitte des letzten Jahrhunderts die Dominanz innerhalb des Softwaremarktes bei kommentiell vertriebener Software lag, gewann die Rolle der Open-Source-Software (OSS) zunehmend an Bedeutung, insbesondere durch die offene Bereitsstellung und Entwicklung des freien Betriebssystems 'Linux'. \cite[S. 1]{will_open-source-software_2003} (Quelle) Bereits im Jahre 2019 lag der Einsatz von OSS ab einer Unternehmensgröße von über 2.000 Mitarbeitern bei 86 \%, was bei dieser Unternehmensgröße fast jedes zweite Unternehmen entspricht. \cite[S. 15]{bitkom_ev_open_2016} Insgesamt enstand durch wettbewerbsfähige OSS ein grundlegender Umbruch in der Softwarebranche innerhalb der letzten Jahren. \cite[S. 185]{bitzer_entwicklung_2007} \cite{fitzgerald_transformation_2006} Unter OSS versteht man einen öffentlich zugänglichen Quellcode, den jeder einsehen, verändern und für sich nutzen kann. Durch die Idee der freien Verfügbarkeit des Quellcodes, der entfallenen Lizenzkosten für den Einsatz und der freien Weiterentwicklungsmöglichkeiten, erweist sich OSS als eine einfache und kostengünstige Innovationsquelle für Unternehmen und folglich als eine realistische Alternative zu proprietärer Software. \cite[S. 21,22]{allmann_open_2019} Das Potential und der resultierende Nutzen kann sowohl für Entwickler als auch für Unternehmen sehr vielschichitg sein. Obwohl für ein Unternehmen das Hauptaugenmerk des Einsatzes von OSS auf einen wirtschaftlichen Gewinn in Form von gesenkten Lizenz- und Entwicklungskosten und von Entwicklungszeit abzielt, profitieren Entwickler von dem Wissensaustausch, der einfachen Handhabbarkeit und der Flexibilität, die OSS mit sich bringt.(Quelle) Zunächst können Entwickler einerseits Softwarefunktionalitäten nach unternehmensinternen Prozessanläufen abgestimmt entwickeln und andererseits durch den unmittelbaren Zugang zum Quellcode, frühzeitig überprüfen, ob ein wiederkehrendes Problem bearbeitet oder nach der jeweligen Problematik individualisiert werden kann. Durch die verfügbaren Standardfunktionalitäten, kann die OSS neben der Möglichkeit der Weiterentwicklung von bereits bestehender Software, dazu verwendet werden, eine Basis für weitere Entwicklungen zu schaffen. (Quelle) Ferner bietet der offene Standard von OSS ein hohes Maß an Anwendungsfeldern, da Entwickler mit wenig Aufwand offenliegende Schnittstellen implementieren können und folglich innerhalb der Hardwareauswahl flexibel und unabhängig sind. (Quelle) Durch die Verwendung von OSS sind Entwickler stark unabhängig von kommentieller Software und von großen Anbietern. Im Gegensatz zu proprietärer Software können Schwachstellen und Sicherheitslücken durch das frühzeitige Testen schneller aufgedeckt und analysiert werden. (Quelle)\\\\ Eine weitere wichtige Rolle des Entwicklungsprozesses mittels OSS ist die Wiederverwendung von Quellcode. Anstatt einer zeit- und kostenintensiven Neuentwicklung von Software, bleiben viele allgemeine oder wiederholende Funktionalitäten innerhalb eines neuen Projektes oftmals gleich und müssen demnach nicht verändert werden. (Quelle) Entwickler versuchen nicht 'das Rad neu zu erfinden', sondern suchen gezielt nach Lösungen von bereits bekannten Problemen. Vor diesem Hintergrund sparen Entwickler viel Zeit und Ressourcen, indem Quellcode, Templates oder Algorithmen wiederverwendet werden, was in der Vergangenheit durch eine maßgeschneiderte Implementierung in kleinem Maßstab erfüllt worden wären. \cite{spinellis_how_2004} Obwohl die Verwendung von OSS auf der einen Seite an Popularität zunimmt, kann auf der anderen Seite ein Einsatz von OSS zu erheblichen Risiken führen. Insbesondere die Folge der Wiederverwendung kann eine massive Verschachtelung und viele transitive Abhängigkeiten von OSS-Komponenten voraussetzen, durch die die Komplexität erheblich steigt. (Quelle) Jede dieser Komponenten hat eine oder mehrere Lizenzen zum Ziel von rechtlichen Schutzmaßnahmen für den Lizenzgeber, die die Nutzungsbedingungen der jeweiligen Komponente spezifizieren und stark einschränken. Durch die Einbeziehung unterschiedlicher Lizenzmodelle in die OSS-Projekte sollen die Rechte des Urhebers geschützt und die damit uneingeschränkte lizenzfreie Nutzung ohne die jeweilige Genehmigung beschränkt werden. (Quelle) Je nach möglicher Verteilung oder Weitergabe an Dritte der modifizierten Software und Art des Lizenzmodells können unterschiedliche Risiken infolge eines unüberwachten Einsatzes von OSS für ein Unternehmen entstehen. Die daraus resultierenden Lizenzfragen oder rechtlichen Konsequenzen können zu einer maßgeblichen Einschränkung führen, die von Unternehmen bereits im Vorfeld genau analysiert und überprüft werden muss. Mit zunehmender Anzahl von Mitarbeitern und verteilten Standorten wird die Bewältigung dieser Herausforderungen jedoch komplizierter. Die Lösung ist frühzeitige Aufklärung und Schulungen über den technischen und rechtlichen Umgang von OSS, durch die Unternehmen. Diese Unterstützung schafft eine Basis für die verantwortungsvolle Einhaltung von Lizenzbedingungen, angefangen bei den Entwicklern selbst. 

%open-source.2.0 \cite{fitzgerald_transformation_2006}
