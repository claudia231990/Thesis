Trotz detaillierter Prinzipien, die als Basis von DevOps gelten, findet sich keine einheitliche Standarddefinition zu diesem Begriff \cite{sollner_devops_2017}, \cite{smeds_devops_2015}. Je nach Zielsetzung, Unternehmenskultur oder Spezifikation werden in der Literatur viele Definitionen aufgelistet, in denen versucht wird, das Konzept des DevOps zu analysieren und zu beschreiben. Diese Tatsache kann zuweilen daran liegen, dass ein großer Teil der verfügbaren Informationen über DevOps aus Blogs und anderen informellen Veröffentlichungskanälen stammen, die oftmals nicht glaubwürdig oder wissenschaftlich belegt sind \cite{smeds_devops_2015}, \cite{roche_roche_2011}. So führen Hüttermann \cite[S. 3,4]{huttermann_devops_2012} als auch Willis \cite{willis_what_2010} aus, dass DevOps ein Muster für die Zusammenarbeit, Prozesse und Tools ist, wobei Aspekte und Aktivitäten wie Kultur, Automatisierung, Messung und gemeinsame Nutzung umfasst werden. Diese Kernelemente werden ebenfalls von Humble und Molesky \cite{humble_why_2011} als wesentliche Bestandteile einer DevOps-Umgebung beschrieben. Walls \cite[S.1]{walls_building_2013} hingegen ist der Meinung, dass DevOps eine kulturelle Bewegung ist, die mit einer Reihe von Softwareentwicklungspraktiken vermischt wird, die eine schnelle Entwicklung ermöglichen. Jez Humble ein bekannter Autor, Mitbegründer der DevOps Research and Assessment LLC und Vordenker der DevOps-Bewegung, beschreibt DevOps als einen \textit{"movement of people who care about developing and operating reliable, secure, high performance systems at scale, has always — intentionally — lacked a definition or manifesto"} \cite{humble_state_2014}. Humbles Ansicht nach ist diese Tatsache jedoch nicht problematisch da die Auswirkungen von DevOps, der Beweis dafür sind, in wieweit die DevOps-Praktiken und die DevOps-Kultur die IT- als auch die Unternehmensleistung, eine Organisation beeinflussen können \cite{humble_state_2014}. Diese Auswirkungen wurden bis heute in dem jährlichen und 'DevOps Report' \cite{puppet_inc_2020_2021} veröffentlicht und beschreiben wie durch DevOps-gestützte Praktiken eine hohe organisatorische Leistung erzielt und im Ergebniss einen signifikanter Wettbewerbsvorteil erlangt wurde. In dieser Hinsicht bedeutet DevOps für Humbles nicht, dass sich ein Team für die Erstellung und Bereitstellung von Systemen, das Deploment oder den Betrieb dieser Systeme verantwortlich ist \cite{humble_theres_2012}. Daher kann es keine Rollen für einen DevOps-Spezialisten in den Teams geben. Vielmehr erkennt DevOps Problemzustände und strebt ein strategisches Ziel, mit menschlichen, technischen und organisatorischen Lösungsansätzen, an \cite{konig_devopswelcome_2019}, \cite{dyck_towards_2015}. Aufgrund der Vielschichtigkeit und des unterschiedlichen Knowhows in dem DevOps-Umfeld wird in dieser Arbeit ebenfalls die Ansicht von Hüttermann und Humble vertreten, dass DevOps kein Berufsbild im klassischen Sinne darstellt. Vielmehr wird die Ansicht von Penners und Dyck \cite{dyck_towards_2015} vertreten, in der DevOps als ein "\textit{organizational approach that stresses empathy and cross-functional collaboration within and between teams – especially development and IT operations – in software development organizations, in order to operate resilient systems and accelerate delivery of changes."} beschrieben wird.

%Vielmehr baut ein DevOps-Team eine Plattform zur Nutzung und zur Bereitstellung für die gesamte Organisation auf. 

%Diese Plattform kann als ein Produkt angesehen werden, welches das Team als solches weiterentwickelt und die Personen die es nutzen, als Kunden angesehen werden können.



%Roche \cite{roche_roche_2011} geht davon aus, dass es zwei Standpunkte hinsichtlich einer Standarddefinition für DevOps gibt. 

% Die erste Seite identifiziert ein spezifisches Berufsbild für DevOps, wobei Entwickler hauptsächlich am Code arbeiten, Systemadministratoren hauptsächlich mit Systemen arbeiten und DevOps eine Mischung aus diesen beiden Fähigkeiten ist. \cite{muller_whats_2010}

% Die andere Partei erweitert die erste Auffassung und stellt allerdings klar, dass DevOps kein Job im herkömlichen Sinne ist. 

% Mitglieder eines DevOps-Teams sind Entwickler mit etwas mehr Erfahrung und Wissen als ein Systemadministrator oder möglicherweise ein Systemadministrator mit etwas Erfahrung und Wissen als ein Programmierer. \cite{jones_how_2012}

% Daher stellt DevOps eine Verschmelzung von Softwareentwicklung und Support für ein besseres Toolset zur Erkennung und Messung von Problemem in vernetzten Systemen dar und gilt nicht als ein Job der dazu beiträgt, einen neuen Standard für alle in der Softwarebranche zu definieren. \cite{roche_roche_2011} 


% So führt beispielweise Hüttermann\cite[S. 3,4]{huttermann_devops_2012} aus, dass es keinen einheitlichen Begriff für DevOps gibt, der alle Aspekte von DevOps umfasst. Seiner Ansicht nach, werden durch die weite Verbreitung viele verschiedene Inhalte aus unterschiedlichen Perspektiven mit DevOps in Verbindung gebracht. Gemäß Hüttermann beschreibt DevOps eine Mischung aus bekannten, forschrittlichen Praktiken und neuen, innovatioven Ansätzen für allgemeine Herausforderungen im Projektleben der Softwarebereitstellung und des Betriebs. Aufgrund der Vielschichtigkeit und der unterschiedlichen Fähigkeiten und Prioritäten, die sowohl Entwickler als auch Systemadministratoren besitzen, kann DevOps keine Berufsbezeichnung oder eine Abteilung innerhalb einer Organisationsstruktur sein. \cite[S. 9]{huttermann_devops_2012}.



% Darüber hinaus sind kulturelle Aspekte von zentraler Bedeutung innerhalb von einer DevOps-Umgebung, allerdings können diese nicht als Basis angenommen werden. \cite{smeds_devops_2015}