Je nach Umfang der Lizenz werden dem Nutzer unterschiedliche Nutzungsrechte eingeräumt, wobei die Lizenzgeber nicht auf ihr Urheberrecht verzichten und weitere Voraussetzungen für die Vervielfältigung, Modifikation und Weiterverbreitung auferlegen. 

OSS-Lizensierung kann in verschiedenen Formen auftreten, wobei sich die Auswirkungen des Einsatzes im konkreten Fall erheblich voneinander unterscheiden können.   

Ob die Software nach der Modifikation der Öffentlichkeit zur Verfügung gestellt muss und welche weiteren Voraussetzungen dabei getroffen werden müssen, ist in den jeweiligen Rahmenbedinungen innerhalb der Lizenzen verankert. 

Mittels dieses Kapitels werden verschiedene Begrifflichkeiten und Grundlagen im Bereich der Lizensierung für ein besseres Verständnis der weiterführenden Kapitel näher erläutert. 

\paragraph{Umfang im Bereich der Lizensierung}
In welchem Umfang dieses Recht dem Nutzer gewährt wird, ist in den einzelnen Lizenzbedingungen zu finden. 

\subparagraph{Copyright}
Das Model des Copyright stammt urprünglich aus dem amerikanischen Raum und bildet das Gegenstück zu dem deutschen oder europäpischen Urheberrecht. 

Grundsätzlich schützt das Copyright, ähnlich dem deutschen Urheberrecht, das geistige Eigentum an Werken. 

Allerdings, ist es im amerikanischen Raum möglich, das Copyright an einem Werk pauschal und jederzeit abzutreten. 

Dies ist gemäß §29 des deutschen Urherberrechtsgesetzes (UrhG) nicht möglich. 

Das deutsche Urheberrecht ist als ein Persönlichkeitsrecht ausgestattet und besteht im Wesentlichen aus einem persönlichkeits- und einem vermögensrechtlichen Teil, welches nicht getrennt werden darf. 

Der Urheber des Copyrights erhält nur Rechte zur Vervielfältigung oder Verbreitung, weshalb ein Abtreten dieser Rechte nach dem amerikanischen Gesetz unproblematisch ist. 

Das deutsche Urheberrecht sieht allerdings vor, dass im Gegensatz zum Copyright einfache oder ausschließbare Nutzungsrechte eingeräumt werden, da diese Rechte gemäß §31 UrhG als übertragbar gelten. 

Der Umfang der Übertragung werden widerrum im Lizenzvertrag geregelt.  

Diese Nutzungsrechte können insoweit ausgedehnt werden, dass der Urheber selbst von der Nutzung seines eigenen Werkes ausgeschlossen werden kann. 

Grundsätzlich schützt das Urheberrecht die geistige Werk an der Software. 

Allerdings steht lediglich der Quellcode im Vordergrund, weder die Funkitonalität noch die Idee dahinter wird urheberrechtlich gschützt. 

Aufgrund dessen können viele Entwickler die gleichen Funkitonalität entwickeln, obgleich der dazugehörige Quellcode unterschiedlich ist. 

Eine große Herausforderung der OSS stellt die Nachvollziehbarkeit der Rechtsinhaberschaft dar. 

Da eine OSS oftmals aus mehreren Teilen mit unterschiedlichen Autoren besteht, sind alle Entwickler gemäß § 8 UrhG, Schöpfer und damit Miturheber eines Werkes und können demnach nur gemeinsam entscheiden, auf welche Weise die OSS verwertet wird. 

\subparagraph{Copyleft}
Das Prinzip des Copylefts bildet das grundlegende Untscheidungsmerkmal zwischen den einzelnen Lizenzmodellen und kann je nach OSS-Lizenz unterschiedlich weit ausgeprägt sein. 

Die Klauseln des Copylefts beinhalten die Regelungen im Hiblick auf die Vervielfältigung, Modifikation, Weiterverbreitung und letzlich den Zugang zum Quellcode.

Der Begriff und die Methode des Copylefts baut auf Richard Stallman auf. 

Dieser wollte, dass die Rechte eines Nutzers an einer Software nicht genommen werden können und das Programm frei sein sollte und verlangt, \textit{"dass alle modifizierten und erweiterten Programmversionen ebenfalls frei sind."}

So haben Nutzer einer OSS-Lizenz, die unter ein starkes Copyleft fällt, die Pflicht alle Änderungen, Ergänzungen und Ableitungen auf die gleiche Weise für die Allgemeinheit zugänglich zu machen, damit jeder diese Software ebenfalls frei nutzen kann. 

Folglich müssen modifizierte OSS und deren Quellcode, die ein starkes Copyleft besitzen, unter derselben Lizenz veröffentlicht werden, wie das urspüngliche OSS-Projekt.

Ferner existieren, neben Lizenzen mit einem starken Copyleft, Lizenzen mit einem beschränkten Copyleft-Klauseln.

In diesem Fall haben Nutzer die Möglichkeit, veränderten Quellcode oder hinzugefügte Dateien innehalb des Programms unter einer anderen Lizenz zu veröffentlichen als die ursprüngliche Lizenz hatte. 

Lizenzgeber möchten folglich, nur das übernommene Werk als auch die damit verbundenen Modifikationen sicherstellen, nicht jedoch die dazugehörige Lizenz. 

Die abgeschwächteste Form des Copylefts, stellen die Lizenzen ohne Copyleft-Bestimmungen dar und werden auch als permissive Lizenzen bezeichnet.

In diesem Rahmen müssen die Nutzer Modifikationen an der Software nicht unter dieselbe Lizenz stellen wie die Ausgangssoftware oder überhaupt als freie Software weitergeben, wodurch de Nutzer in keiner Weise in seinen Rechten eingeschränkt wird. 

Im Ergebnis lässt sich festhalten, dass Lizenzen mit einem starken Copyleft es Nutzern unmöglich machen, die Software innerhalb einer propritären Software zu integrieren oder zu kombinieren, während Lizenzen mit keinem Copyleft, es Nutzern sehr leicht machen, das Ergebnis seiner Arbeit zu einer Closed-Software zu machen.  

\subparagraph{Nutzungsrecht}
Obwohl der Begriff der Lizenz im Rahmen der Lizensierung oftmals verwendet wird, handelt es sich dabei eigentlich um das Nutzungsrecht, welches dem Nutzer gemäß dem Urherberrechtsgesetz unterschiedliche Rechte einräumt. 

Innerhalb der Lizensierung legt der Urherber der Software fest, in welchem Umfand dem Nutzer Rechte eingeräumt werden und welche weiteren Bedingungen oder Pflichten dieser zu erfüllen hat. 

Sobald die Software heruntergeladen wird, stimmt der Lizennehmer den Lizenzbedingungen des Lizenzgebers zu, wodurch ein Lizenzvertrag zustande kommt. \cite{wilmer_rechtliche_2021} 

Darüber hinaus kann der Nutzer, basierend auf dem Urheberrecht kein Eigentum an der OSS erlangen, sondern ist nur berechtigt diese im Rahmen des Lizenzvertrages zu nutzen. 

Eine Abtretung kann sich daher nur auf das Nutzungsrecht, jedoch nicht auf das Urheberrecht beziehen. 

Im Ergebnis ist eine Softwarelizenz lediglich mit einem Nutzungsrecht gleichzusetzen.\cite[S. 29]{kees_open_2015}

\paragraph{Grundlagen der Lizensierung}
Die Lizensierung soll zwei wesentliche Kernaspekte verfolgen. 

Einerseits soll sichergestellt werden, dass Software flexibel vermarktet werden kann und andererseits Maßnahmen garantieren um die Software vor unerlaubter Nutzung zu schützen. 

Folglich verzichten die Lizenzgeber nicht auf ihr Urheberrecht, sondern stellen diese auf Grundlage von Lizenzverträgen der Allgemeinheit zur Verfügung. 

Im Folgenden werden einige Arten der Lizensierung erläutert, die berücksichtigt werden müssen.  

\subparagraph{Lizenzvertrag}
Ein Lizenzvertrag kommt zustande, wenn der Nutzer als Lizenznehmer mit den Lizenzbedinungen des Urhebers als Lizenzgeber zustimmt. 

Inhaltlich regelt der Lizenzvertrag, welchen Umfang des Nutzungsrechts dem Lizenznehmer eingeräumt wird und welche Verpflichtungen dieser zu erfüllen hat.

Eine Nichteinhaltung der Lizenzbestimmungen kann Vertragsstrafen mitsichziehen. 

Gemäß §§ 145 ff. BGB kommt ein Vertrag zustande, wenn zwei Willenserklärungen, Angebot und Annahme, übereinstimmen. 

Sobald der Lizenzgeber die OSS veröffentlicht, stellt dies ein Angebot für jeden dar, welches mit dem Herunterladen der OSS und dem Zustimmen der Lizenz als eine entsprechende Annahme erfolgt. 

\subparagraph{Duale Lizensierung}
Besitzt eine Software zwei oder mehrere unterschiedliche Lizenzen, wird auch von einer dualen Lizensierung gesprochen.

Duale Lizensierung erlaubt es, Software unter verschiedenen Lizenzen weiterzugeben. 

An dieser Stelle kann der Nutzer, Software die ein starkes Copyleft aufweisen, mit einer propritären Lizenz gegen Entgeld zu erwerben. 

Dies hätte zur Folge, das Modifikationen nicht mehr unter einem starken Copyleft veröffentlicht werden müssen. 

Dieses Vorgehen funktioniert allerdings nicht bei einer bereits modifizierten OSS, die unter starkem Copyleft steht, da eine Überführung von Teilen des Quellcodes, die von Dritten verändert worden sind, in den propritären Vertrieb nicht möglich ist. 

Unternehmen sind dadurch in der Lage, spezielle Features die selten nachgefragt werden, den Kunden direkt in Rechnung stellen. 

Bei einer Software mit keinem Copyleft kann immer eine duale Lizensierung durchgeführt werden, da der Nutzer alle Rechte in Hinblick auf Weiterverbreitung erhalten hat. 






