Eine große Herausforderung bei der Modellierung des Softwareentwicklungsprozesses war eine genaue Momentaufnahme eines kontinierlichen arbeitenden Prozesses. 

\begin{figure}[p]
    \centering
    \includegraphics[angle=90, scale=0.7]{Bilder/IST-Prozess.png}
    \caption{Softwareentwicklungsprozess im IST-Zustand basierend auf dem HAF-Projekt}
\end{figure}

Da das DevOps-Team mit agilen Methoden wie Scrum und Kanban arbeitet, erwies sich die Modellierung des IST-Prozesses basierend auf einem Sprint am geeignesten. 

Der Prozess wurde zunächst in vier essentielle Rollen unterteilt, die bei der Softwareentwicklung als DevOps-Team beteiligt sind: 

\paragraph{Agility Master}

Der Agility Master verantwortet einen effektiven und effizienten Prozess und stellt sicher, dass die Regeln des agil und Lean Management eingehalten werden. 

Ferner fungiert der Agility Master als ein Moderator für Team-Ereignisse und versucht Probleme und Hindernisse zeitnah zu lösen. 

Der Agility Master ist mit dem Scrum Master identisch. 

\paragraph{Dev}

Diese Rolle entspricht der Rolle des Softwareentwickler oder Development-Teams. 

Je nach Projektanforderungen kann sich die Rolle des Entwicklers beispielsweise auf das Frontend- oder Backendentwicklung richten.

\paragraph{Product Owner und Product Designer}

Zunächst fungiert der Product Owner als die Schnittstelle zwischen verschiedenen Stakeholdern oder Kunden und dem Entwicklungsteam. 

Er vertritt demnach die Interessen des Kunden in den Entwicklungsprozess, ohne aktiv in die Softwareentwicklung einzugreifen.

Ferner erstellt der Product Owner den Product Backlog und legt die Reihenfolge der Items, die bearbeitet werden müssen, fest. 

Wie der Name schon sagt, beschreibt die Rolle des Product Designers die Gestaltung oder den Entwurf des Produkts. 

Hierzu gehören die Definitionen von UX-Anforderungen und Spezifikationen oder die Schaffung von Schnittstellen, im Hinblick auf die Fuktionalität und das erwartende Verhalten des Produkts. 

Innerhalb des HAF-Teams überschneiden sich die Rollen des Product Owner und des Product Designers, weshalb diese beiden Positionen in einer Swimlane dargestellt werden.  

\paragraph{Ops}

Die Rolle des Administratoren oder IT-Betrieb entspricht dem anderen Teil des DevOps-Teams. 

In erster Linie sorgt diese Rolle für die Stabilität der IT-Infrastruktur und ermöglichen einen durchgängigen, laufenden Betrieb.  

\begin{figure}[h]
    \centering
    \includegraphics[scale=0.7]{Bilder/IST-Prozess_first Part.png}
    \caption{Erster Teil des Softwareentwicklungsprozess im IST-Zustand basierend auf dem HAF-Projekt}
\end{figure}

\newpage Startpunkt des Prozesses liegt beim Agility Master, indem die Projektorganisation zunächst durchgeführt und die Projektstruktur festgelegt wird.

Zu beachten ist, dass diese zwei Aufgaben ausschließlich am Anfang eines jeden Projekts und nicht in laufenden Projekten stattfinden. 

Nachdem das Product Backlog erstellt wurde muss dieses vom Product Owner priorisiert werden. 

Die Priorisierung findet sequentiell statt, da mehrere Entwicklungen und Änderungen eines Items des Product Backlogs parallel und gleichzeitig und nicht als eine Aufgabenschleife stattfinden. 

Diese Änderungen fließen in den priorisierten Backlog ein. 

Ist die Priorisierung erfolgt, startet die Iteration mit dem Sprint Planning Meeting. 

Abhängig von den verfügbaren Kapazitäten und den Ergebnissen des letzten Sprints wird entschieden, welche User Stories zuerst bearbeitet werden. 

Die ausgewählten User Stories bilden den Sprint Backlog. 

Parallel zu den täglichen Entwicklungen beginnt der Change-/Incidentprozess, bei dem zunächst die täglichen Informationen verarbeitet werden. 

Der Changeprozess bescheibt den prozess für eine Änderung, die sich auf eine geplante Modifikation der Definition eines bestehenden oder neuen Produkts bezieht und für die das DevOps-Teams verantwortlich ist. 

Der Prozess umfasst die Erfassung jedes Requests, die Dokumentation, Genehmigung und Kontrolle und stellt sicher, dass Change Requests definiert, geplant, effizient, wirtschaftlich und mit vertretbarem Risiko abgewickelt werden. 

Ein Ziel des Change Managements ist es, die Nachvollziehbarkeit von Anforderungen und Änderungen zu gewährleisten.

Incident Management umfasst den gesamten organisatorischen und technischen Prozess für Reaktionen und Maßnahmen auf Störungen des IT-Betriebs.

Das Spektrum möglicher Incidents reicht von Fehlermeldungen durch Anwender bis hin zu automatisierten Alarmen und Warnungen durch das Monitoring.

Das Ziel des Incident Managements ist die Wiederherstellung eines zugesagten Dienstes in einem definierten Prozess und in optimaler Zeit. 

Dies kann auch Umgehungslösungen beinhalten. 

Incident Management deckt keine (neuen) Anforderungen, Wünsche, Anfragen oder ähnliches ab.  

Obwohl der Change- und Incidentprozess vom Product Owner bzw. Designer begonnen wird, ist dieser ausschließlich für den Changeprozess verantwortlich. 

Der Incidentprozess wird anhand des täglichen Monitorings vom Ops-Verantwortlichen durchgeführt. 

Nach der Verarbeitung der täglichen Informationen, werden die auftretenden Bugs zunächst in Incident und Changeprozesses unterschieden, da beide Prozesse werden unterschiedlich behandelt werden.

Innerhalb des Changeprozesses werden die Änderungen definiert und verfeinert. 

Nachdem die Änderung geprüft wurde, muss entschieden werden, ob die Änderung durchführbar also entwickelt werden kann. 

Ist dies der Fall wird die Änderung dokumentiert und fließt in den überarbeiteten Product Backlog ein, bei dem die Items wiederrum innerhalb des Product Backlogs priorisiert werden. 

Sollte der Change nicht durchzuführen sein, müssen die Definitionen und Verfeinerungen angepasst werden.

Innerhalb des Incidentprozesses wird der Incident zunächst erstellt und analysiert.

Sollte der Bug als Incident bestätigt werden, so wird dieser dokumentiert und fließt in das überarbeitete Product Backlog und wiederrum in das priorisierte Product Backlog ein. 

Falls dies nicht der Fall ist, wird der Incident angemerkt und geschlossen, wodurch der Incidentprozess ab diesem Zeitpunkt geschlossen wird. 

Sobald die Iteration startet, muss zunächst überprüft werden, ob Codeänderungen notwendig sind. 

Dies kann sich auf Änderungen eines vorhandenen Codes oder auf eine Neuimplementierung beziehen. 

Sobald keine Änderungen notwendig sind, bleibt das entsprechende Item im priorisierten Product Backlog und wird im Release auf seine Vollständigkeit und Richtigkeit überprüft.

Nachdem der Pull-Request für das Commit durchgeführt wurde, werden mehrere Aufgaben parallel erledigt. 

Dazu gehört die Durchführung der Unit-Test und Codeanalyse und die Bestätigung des Reviewer.

Sobald alle Änderungen erfolgreich implementiert worden sind, werden diese auf das Master Branch gemergt. 

Sobald das Release auf seine Vollständigkeit und Richitgkeit überprüft worden ist, wird dieses im Review vorgestellt. 

Ab diesem Zeitpunkt endet die Iteration. 

Werden weitere Anpassungen benötigt, muss das neue Release bearbeitet und definiert werden, wobei diese in das priorisierte Backlog zurückgehen.

Sind keine Änderungen notwendig, wird das Release bereitgestellt. 

Das Projekt endet sobald das product Backlog leer ist.















