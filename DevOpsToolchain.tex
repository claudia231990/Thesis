Nutzer von DevOps-Verfahren setzen im Rahmen ihrer DevOps-Toolchain oft bestimmte DevOps-freundliche Tools ein. Ziel dieser Tools ist es, die verschiedenen Phasen des Workflows zur Softwarebereitstellung (auch als Pipeline bezeichnet) noch stärker zu straffen, zu verkürzen und zu automatisieren. Viele derartige Tools sind auch an wesentlichen DevOps-Grundsätzen wie Automatisierung, Zusammenarbeit und Integration zwischen Entwicklungs- und Betriebsteam ausgerichtet. Hier werden einige gängige Tools beschrieben, die innerhalb der unterschiedlichen Phasen des DevOps-Lebenszyklus genutzt werden.\\


Planen: In dieser Phase werden der geschäftliche Nutzen und die geschäftlichen Anforderungen festgelegt. Tools wie Jira oder Git helfen dabei, bekannte Probleme nachzuverfolgen, und unterstützen das Projektmanagement.\\


Codieren: In dieser Phase stehen das Softwaredesign und die Erstellung von Softwarecode im Mittelpunkt. Tools dafür sind beispielsweise GitHub, GitLab, Bitbucket oder Stash.\\


Entwickeln. In dieser Phase werden Softwarebuilds und -versionen verwaltet. Automatisierte Tools unterstützen das Kompilieren und Packen von Code für künftige Produktionsfreigaben. Mithilfe von Quellcode-Repositorys oder Paket-Repositorys wird außerdem die zur Produktfreigabe benötigte Infrastruktur „verpackt“. Beispiel-Tools sind Docker, Ansible, Puppet, Chef, Gradle, Maven und JFrog Artifactory.\\


Testen. In dieser Phase wird durch kontinuierliches Testen (manuell oder automatisiert) eine optimale Codequalität gesichert. Beispiel-Tools sind JUnit, Codeception, Selenium, Vagrant, TestNG und BlazeMeter.\\


Implementieren. In dieser Phase können Tools genutzt werden, die das Managen, Koordinieren, zeitbezogene Planen und Automatisieren von Produktversionen für die Produktion unterstützen. Beispiel-Tools sind Puppet, Chef, Ansible, Jenkins, Kubernetes, OpenShift, OpenStack, Docker und Jira.
Betrieb. In dieser Phase geht es um das Management der Software während der Produktion. Beispiel-Tools sind Ansible, Puppet, PowerShell, Chef, Salt und Otter.\\


Überwachen. In dieser Phase werden Informationen über Probleme mit bestimmten Softwareversionen in der Produktion erkannt und erfasst. Beispiel-Tools sind New Relic, Datadog, Grafana, Wireshark, Splunk, Nagios und Slack.