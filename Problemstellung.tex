Laut einer aktuellen Studie des Bundesverband Informationswirtschaft, Telekommunikation und neue Medien e. V. (Bitkom) wird der Einsatz von Open Source Software in Unternehmen, im Jahre 2019 bei einer Unternehmensgröße von über 2.000 Mitarbeitern, bereits auf 86 Prozent geschätzt. Dabei reicht das Anwendungsfeld von der Automobilindustrie bis hin zu Banken und Versicherungen und steht daher im direkten Wettbewerb zu namenhaften proprietaeren Softwareangeboten. Ausgehend hiervon gehört die Etablierung von Open Source Software in die Softwareentwicklung bereits zur gängigen IT-Unternehmenskultur.  

Durch die freie Verfügbarkeit des Quellcodes, kann die entsprechende Software schnell und einfach eingebunden werden. 



Open-Source-Software gilt als zuverlässig und sicher. Für manche ist sie sogar ein Innovationsmotor mit ungeheurem Potenzial. Neben der Vielfalt der Anwendungsszenarien steht die Fülle 
der Anwendungen. Freie Software – ein gängiges anderes Etikett für Open-Source-Software – 
hat echte wirtschaftliche Bedeutung gewonnen. Die Serverinfrastruktur einiger großer Internetfirmen basiert zum größten Teil auf Linux. Google, Facebook oder Amazon sind bekennende 
Nutzer. Die Gründe dafür sind simpel: Unternehmen müssen für den Einsatz von Open-SourceSoftware keine Lizenzkosten einkalkulieren. Das vereinfacht den Aufbau skalierender Umgebungen signifikant. Außerdem können die Unternehmen die genutzte freie Software im Rahmen 
der jeweiligen Lizenzbestimmungen entsprechend eigener Zwecke und Ziele weiterentwickeln. 
Selbst wenn die modifizierte Version danach geschäftskritische Elemente enthält – und der 
wirklich geschäftskritische Anteil eines Softwaresystems ist in der Regel sehr klein –, gibt es 
immer noch gute Einsatzszenarien, die deren Schutz in Verbindung mit Open-Source-Software 
gewährleisten.


so selbstverständlich die Verwendung ist, so verbreitet sind 
oft auch Wissenslücken über Grundanforderungen im Umgang mit OSS, sei es als Basis für einen 
erfolgreichen Einsatz im Alltag oder als Voraussetzung für die Verwendung für eigene Softwareprojekte. Ein gängiges Missverständnis besteht z.B. darin, dass aus der einfachen und unentgeltlichen Verfügbarkeit von OSS auf das Fehlen jeglicher juristischen Einbettung geschlossen wird. 

Grundsätzlich unterliegt Software dem Urheberrecht. Daher 

Durch die freie Verfügbarkeit des Quellcodes, kann die entsprechende Software schnell und einfach eingebunden werden. 


Die implizierte Freiheit, die sich bei der Verwendung von OSS ergibt, beschränkt sich dementsprechend auf die Einsicht, Nutzung, Modifikation und Distribustion des Quellcodes.    

jedoch nicht uneingeschränkt lizenzfrei genutzt werden. 



Mittels einer Lizenzierung der entsprechenden Software soll sichergestellt werden, dass die Rechte zur Nutzung der Open Source Software nur mit Genehmigung des Urhebers geschützt bleiben. Ab dem Zeitpunkt der Weitergabe von veränderter Software, müssen entsprechenden Lizenzmodelle berücksichtigt werden, um vorrangig das geistige Eigentum des Urhebers zu schützen.  

Bereits bei groben Veränderungen im Laufe des Projektes, hat die Veränderungen von bestimmten SChutzmaßnahmen graviernede Auswirkungen auf das DevOps-Team. Viele erhalten nicht die benötigten Informationen, dass eine Veränderung vorliegt, die das Projekt maßgeblich beeinflußen könnte. 

Demgegenüber haben Urherber bei einer Verletzung an ihrer Open Source Software, die sich aus den einzelnen Lizenzmodellen ergibt, die Möglichkeit rechtliche Ansprüche geltend zu machen. Die Folge sind Unterlassungsansprüchen und kostenspieliger Aufwendungsersatz. Die Konsequenz wäre eine sofortige Einstellung der bearbeiteten Software und möglicherweise eine neue Entwicklung, der Funktionalitäten ohne das jeweilige Open Source als Basis./ Nutzer verliehrt das Recht die Software weiterhin zu nutzen.  


Eine
Lizenz (v lat: licere = erlauben) ist eine Erlaubnis oder Genehmigung zur Nutzung
eines Rechts durch den Urheber oder Inhaber dieses Rechts11. Im juristischen
Bereich stellt die Lizenz einen Vertrag dar, durch welchen einfache oder
ausschließliche Rechte eingeräumt werden können. 


Software ist in der Regel urheberrechtlich geschützt. Dies gilt auch für Freie Software oder Open-Source (OSS). Die für proprietäre Software eingeräumten Softwarelizenzen sind dahingehend ausgerichtet, die Freiheit der Nutzung, Verbreitung und Veränderung der Software einzuschränken. Der Softwarehersteller kann so durchsetzen, eine angemessene Vergütung für die Entwicklungsleistungen zu erhalten, Art und Umfang der Werknutzung zu bestimmen und die Software vor ungewollter Veränderung zu schützen. 


1. Freie Weitergabe
Die Lizenz darf niemanden in seinem Recht einschränken, die Software als Teil eines Software-Paketes, das Programme unterschiedlichen Ursprungs enthält, zu verschenken oder zu verkaufen. Die Lizenz darf für den Fall eines solchen Verkaufs keine Lizenz- oder sonstigen Gebühren festschreiben.

2. Quellcode
Das Programm muss den Quellcode beinhalten. Die Weitergabe muss sowohl für den Quellcode, als auch für die kompilierte Form zulässig sein. Wenn das Programm in irgendeiner Form ohne Quellcode weitergegeben wird, so muss es eine allgemein bekannte Möglichkeit geben, den Quellcode zum Selbstkostenpreis zu bekommen, vorzugsweise als gebührenfreien Download aus dem Internet. Der Quellcode soll die Form eines Programms haben, das ein Programmierer vorzugsweise bearbeitet. Ein absichtlich unverständlich geschriebener Quellcode ist daher nicht zulässig. Zwischenformen des Codes, so wie sie etwa ein Präprozessor oder ein Konverter („Translator”) erzeugt, sind unzulässig.

3. Abgeleitete Software
Die Lizenz muss Veränderungen und Derivate zulassen. Außerdem muss sie es zulassen, dass die solcher Art entstandenen Programme unter denselben Lizenzbestimmungen weiter vertrieben werden können wie die Ausgangssoftware.

4. Unversehrtheit des Quellcodes des Autors
Die Lizenz darf die Möglichkeit, den Quellcode in veränderter Form weiterzugeben, nur dann einschränken, wenn sie vorsieht, dass zusammen mit dem Quellcode so genannte „Patch files” weitergegeben werden dürfen, die den Programmcode bei der Kompilierung verändern. Die Lizenz muss die Weitergabe von Software, die aus einem veränderten Quellcode entstanden ist, ausdrücklich erlauben. Die Lizenz kann verlangen, dass die abgeleiteten Programme einen anderen Namen oder eine andere Versionsnummer als die Ausgangssoftware tragen.

5. Keine Diskriminierung von Personen oder Gruppen
Die Lizenz darf niemanden benachteiligen.

6. Keine Einschränkungen bezüglich des Einsatzfeldes
Die Lizenz darf niemanden daran hindern, das Programm in einem bestimmten Bereich einzusetzen. Beispielsweise darf sie den Einsatz des Programms in einem Geschäft oder in der Genforschung nicht ausschließen.

7. Weitergabe der Lizenz
Die Rechte an einem Programm müssen auf alle Personen übergehen, die diese Software erhalten, ohne dass für diese die Notwendigkeit besteht, eine eigene, zusätzliche Lizenz zu erwerben.

8. Die Lizenz darf nicht auf ein bestimmtes Produktpaket beschränkt sein
Die Rechte an dem Programm dürfen nicht davon abhängig sein, ob das Programm Teil eines bestimmten Software-Paketes ist. Wenn das Programm aus dem Paket herausgenommen und im Rahmen der zu diesem Programm gehörenden Lizenz benutzt oder weitergegeben wird, so sollen alle Personen, die dieses Programm dann erhalten, alle Rechte daran haben, die auch in Verbindung mit dem ursprünglichen Software-Paket gewährt wurden.

9. Die Lizenz darf die Weitergabe zusammen mit anderer Software nicht einschränken
Die Lizenz darf keine Einschränkungen enthalten bezüglich anderer Software, die zusammen mit der lizenzierten Software weitergegeben wird. So darf die Lizenz z. B. nicht verlangen, dass alle anderen Programme, die auf dem gleichen Medium weitergegeben werden, auch quelloffen sein müssen.


Um zu gewährleisten, dass EntwicklerInnen, die mit GNU-Software arbeiten, den Wissenspool ebenfalls nähren, verpflichtet die Lizenz ihre NutzerInnen zur Weitergabe der Derivate unter gleichen Bedingungen (Copyleft). Die Idee freier Lizenzen geht also auf die Software-Entwicklung zurück, die durch freien Code kollaborative Arbeiten ermöglichen wollte. 


Alle Open Source Lizenzen haben die Einräumung eines Vervielfältigungs- und Verarbeitungsrechts gemeinsam, welches meist bestimmten Voraussetzungen bzw. Beschränkungen unterliegt. Die einzelnen Lizenzen unterscheiden sich vor allem hinsichtlich der Nutzungsbedingungen und der Verpflichtungen, die dem Lizenznehmer zur Wahrnehmung des Vervielfältigungsrechts und des Verarbeitungsrechts aufgegeben werden.

Wichtigster Gesichtspunkt ist, welche Anforderungen an die Weiterverbreitung von veränderten Versionen der Software bzw. neuer Software gestellt werden, die auf Grundlage von Open Source Software entwickelt wurde.



Von Beginn an haben sich verschiedene Lizenzmodelle für OSS herausgebildet, die sich hinsichtlich der Nutzungsbedingungen teilweise recht deutlich unterscheiden. Zur Sichtung, Sammlung 
und Ordnung der vielfältigen Lizenzmodelle hat sich die Open Source Initiative (OSI) gegründet. 
Die OSI hat Kriterien aufgestellt, nach denen sie eine Lizenz als Open-Source-Lizenz klassifiziert 
und in die offizielle Liste der Open-Source-Lizenzen aufnimmt. Die Klassifikation einer Softwarelizenz als Open- Source-Lizenz durch die OSI erfolgt anhand der Rechte, die die Lizenz dem Nutzer 
einer so lizenzierten Software einräumt. Sie erfolgt nicht anhand der Auflagen, die die OpenSource-Lizenz dem Nutzer auferlegt. 
Für eine Einteilung der Open-Source-Lizenzen anhand der Pflichten haben sich die Kategorien 
»permissive Lizenzen«, Lizenzen mit schwachem »Copyleft« und‚ Lizenzen mit starkem 
»Copyleft« eingebürgert. Einige OSS-Lizenzen verpflichten dazu, bei Weitergabe der Software 
den Quellcode zugänglich zu machen. Damit ist manchmal das sog. Copyleft verbunden: 
Danach darf ein Nutzer eine von ihm veränderte Open-Source-Software nur zu den für die 
ursprüngliche Open-Source-Software geltenden Lizenzbedingungen an Dritte weitergeben. 
Die permissiven Lizenzen stellen solche Anforderungen nicht. Die Einteilung in permissive 
Lizenzen und Lizenzen mit starkem und schwachem Copyleft bietet nur eine erste Orientierung. 
Was konkret zu tun ist, um Open-Source-Software lizenzgemäß zu nutzen, ergibt sich stets erst 
aus der konkreten Open-Source-Lizenz. Die Lizenzbedingungen sind natürlich auch bei einer wirtschaftlichen Nutzung von Open-SourceSoftware zu beachten. Daher werden Unternehmen, die Leistungen im Zusammenhang mit OSS 
anbieten oder OSS in eigenen Produkten nutzen, den Umfang ihrer Nutzungsrechte anhand der 
konkreten Softwarelizenz klären, die mit dem Vertrieb verbundenen Risiken abschätzen und ihre 
Entwicklungs- und Vermarktungsstrategie in technischer und rechtlicher Hinsicht darauf ausrichten müssen. In diesem Kontext ist besonders hervorzuheben, dass Open-Source-Software – 
entgegen landläufigen Vorurteilen – sehr wohl kommerziell eingesetzt und vertrieben werden 
darf. Einzig für die Nutzung der Software selbst dürfen keine Lizenzgebühren verlangt werden. 
Unternehmen können jedoch neue Geschäftsmodelle mit und um Open-Source-Software herum 
aufbauen, die nicht auf das traditionelle Lizenzsoftwaregeschäft zurückgreifen. Andererseits sind Unternehmen bei der Einbettung von fremderstellter Open-Source-Software in 
eigene Produkte vor Herausforderungen gestellt, die über die bloße Lizenzerfüllung hinausgehen. 
So müssen sie etwa überlegen, inwieweit sie eine Mängelhaftung für die fremderstellte OSS 
übernehmen können, wenn eine Fehlerbehebung nicht von den Urhebern der Software oder 
der Open-Source-Community angeboten wird. Zu solchen Herausforderungen gehört auch die 
Analyse, ob und wann ein Unternehmen geschäftskritische Alleinstellungsmerkmale in eine 
adaptierte Open-Source-Software hineinprogrammiert, deren Lizenz die Offenlegung des Codes 
im Falle einer Weitergabe des Programms an Kunden vorsieht. Glücklicherweise sind die 
wenigsten Verbesserungen an einem Open-Source-Code wirklich geschäftskritisch. Und viele 
Lizenzen verlangen auch keine solche Offenlegung. Um solche Herausforderungen in den Griff zu bekommen, ist zunächst die Erfassung der im 
Unternehmen und in seinen Produkten verwendeten Open-Source-Software erforderlich. 
Daraus wird sich eine unternehmenseigene Steuerung und Kontrolle der Open-Source-Software 
entwickeln. Hierzu sind ein technisches Software-Management und ein rechtliches Lizenzmanagement durchaus hilfreich. Mehr noch: Die Einrichtung entsprechender Strukturen gehört 
letztlich zu den unternehmerischen Organisationspflichten. Darüber hinaus kann eine 
differenzierte Vertragsgestaltung bei Beschaffung und Vertrieb von Open-Source-Software 
helfen. Bei der Gestaltung von Verträgen über Leistungen im Zusammenhang mit Open-SourceSoftware (etwa Implementierung, Anpassung, Zusatzprogrammierung etc.) und bei der 
Einrichtung eines Lizenzmanagements ist die Beteiligung fachkundiger unternehmensinterner 
oder externer Rechtsberater empfehlenswert.

Trotz der Freiheiten, die Open Source Software offensichtlich bietet, unterliegen die meisten Projekte rechtlichen Schutzmaßnahmen, einschließlich dem Marken- Patent- und Urheberrecht. Aufgrund dessen können, aus der Verwendung von OSS resultierende Lizenzfragen oder rechtliche Konsequenzen, zu einer maßgeblichen Einschränkung insbesondere im Hinblick auf eine mögliche Verteilung oder Weitergabe an Dritte führen, die von Unternehmen genau analysiert und überprüft werden muss.  

In dieser Thesis werden die grundsätzlichen Vorteile und Risiken, die sich aus den verschiedenen Lizenzvereinbarungen bei der Verwendung von OSS ergeben analysiert und die Migration in den Devops-Emntwicklungsprozeess dargestellt. Verdeutlicht wird die Themenstellung anhand der ausgeführten Analysen anahnd eines Beispiels bei der msg Systems.

In diesem Abschnitt wird beschrieben, welche derzeitige Problemstellung innerhalb der DevOps Produktentwicklung bei der Integration von FOSS vorliegt, die mittels der Thesis gelöst bzw verbessert werden kann. Insbesondere wird dabei auf die derzeitige Ist-Situation eingegangen, welche Herausforderungen momentan vorliegen (bei msg und allgemein) und wie der aktuelle Stand der Wissenschaft in diesem Kontext ist. 
Zudem wird eine Herleitung zu den Ziel und den Forschungsfragen dieser Thesis hergestellt.  