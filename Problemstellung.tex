Laut einer aktuellen Studie des Bundesverband Informationswirtschaft, Telekommunikation und neue Medien e. V. (Bitkom) wird der Einsatz von OSS in Unternehmen ab einer Größe von 2.000 Mitarbeitern, im Jahre 2019, bereits auf 86 Prozent geschätzt \cite{bitkom_open_2020}. Im Gegensatz zu proprietärer Software, die auf einen bestimmten Copyrightholder lizensiert sind und damit die freie Nutzung, Verbreitung und Veränderung stark beschränken, zeichnet sich OSS durch eine freie Einsicht, Nutzung, Modifikation und Distribution des Quellcodes aus. In diesem Zusammenhang stellt eine Modifikation die Neuentwicklung, die Fortentwicklung oder die Kombination mit einer weiteren Software, dar. Aufgrund der gewonnenen wirtschaftlichen Bedeutung gehört die Etablierung von OSS in die Softwareentwicklung bereits zur gängigen IT-Unternehmenskultur \cite[S. 21,22]{allmann_open_2019}.\\\\ Trotz der vielen offensichtlichen Vorteile unterliegt OSS in der Regel dem Vertrags-, Patent- und Urheberrecht \cite[S. 28 ff.]{kesler_anpassung_2013}. Durch die entsprechenden Schutzmaßnahmen entfallen die freie und unentgeltliche Nutzbarkeit der Software, da alle Rechte, die sich mit und aus der Software ergeben, zunächst nur dem Urheber zustehen. Dieses Prinzip entspricht dem gängigen Begriffs des Copyright, welches die Rechte am eigenen Werk schützt und als der maßgebliche Unterschied zwischen kommerzieller Software und OSS darstellt. Die rechtlichen Schutzmaßnahmen werden in Form einer Lizenzsierung ausgeübt, indem die Genehmigung oder die Erlaubnis zur Nutzung der OSS nur durch den Urheber eingeräumt wird \cite{wilmer_rechtliche_2021}. Damit soll sichergestellt werden, dass die grundsätzlichen Rechte insbesondere das geistige Eigentum des Urhebers zur Nutzung, Weitergabe oder Vervielfältigung vorrangig geschützt werden.\\\\ In diesem Rahmen haben sich verschiedene Lizenzmodelle für OSS herauskristallisiert, die sich im Bereich der Nutzungsbedingungen maßgeblich unterscheiden und wodurch eine uneingeschränkte lizenzfreie Nutzung nur bedingt möglich ist \cite{wilmer_rechtliche_2021}. Insbesondere die Folge der Wiederverwendung kann eine massive Verschachtelung und viele transitive Abhängigkeiten von OSS-Komponenten voraussetzen, durch die die Komplexität erheblich steigt. Jede dieser Komponenten hat eine oder mehrere Lizenzen zum Ziel von rechtlichen Schutzmaßnahmen für den Lizenzgeber, die die Nutzungsbedingungen der jeweiligen Komponente spezifizieren und stark einschränken. Insbesondere ab dem Zeitpunkt der Weitergabe oder Verteilung von veränderter Software an Dritte, setzen die Beschränkungen im Rahmen eines Vervielfältigungs- und Verarbeitsrechts für den Lizenznehmer ein. In der Praxis werden diese Beschränkungen anhand des Copyleft-Effekts bemerkbar, bei dem die Weitergabe durch den Nutzer einer OSS nur zu den ursprünglichen Lizenzbedingungen möglich ist \cite[S. 184]{sujecki_vertrags-_2005}. Abhängig von der Stärke des Copylefts, ist der Lizenznehmer nach den jeweiligen Vorgaben dazu verpflichtet, alle Modifikation, Erweiterungen und Veränderungen der ursprünglichen Software demnach frei zugänglich zu machen. Grundsätzlich besteht für die Urheber die Möglichkeit, bei Verletzung der Rechte innerhalb der Vertrags-, Haftungs- Patent- und Urheberrechts an ihrer OSS juristische Ansprüche geltend zu machen. Basierend auf der Grundlage der entsprechenden Lizenzmodelle können Unterlassungsansprüche und ein kostenintensiver Schadensersatz auf die Betreffenden zukommen \cite{helmreich_geschaftsrisiken_2012}.\\\\ Die Folge könnte ein Verlust zur Nutzung der entsprechenden Software oder die unverzügliche Einstellung der bearbeiteten Software ohne das entsprechende OSS als Basis, sein. Die Konsequenz ist ein kosten- und zeitintensiver Entwicklungsaufwand für das Unternehmen, da die fehlenden Funktionalitäten durch neue Entwicklungen kompensiert werden müssen. Demnach sollten Unternehmen die Möglichkeit ergreifen, ihre Nutzungsberechtigungen anhand des Lizenzmodelles von angebotenen OSS zu überprüfen und die daraus resultierenden Risiken bei der Verwendung in die eigenen Produkte zu berücksichtigen. Insbesondere im Falle einer Weitergabe des modifizierten Programms an potentzielle Kunden, muss eine mögliche Offenlegung in den Softwareentwicklungsprozess rechtlich und technisch einbezogen werden. Dies gestaltet sich oftmals als schwierig, da sich bei einem Update der Software, Versionen im Laufe eines Entwicklungsprozess geändert haben können, die nicht mehr der ursprünglichen verwendeten Lizenz entspricht. Neben der Erfassung der in den Produkten verwendeten OSS, sind Unternehmen demnach dazu verpflichtet, die entsprechenden OSS Projekte kontinuierlich zu überprüfen und zu beobachten, um die Einhaltung der Schutzmaßnahmen zu gewährleisten. In diesem Zusammenhang können Informationen während eines Softwareentwicklungsprozesses in vielen Fällen nur zeitversetzt an das gesamte DevOps-Team kommuniziert werden. Um dem entgegenzuwirken, ist eine Umstrukturierung oder Anpassung des gängigen DevOps-Entwicklungsprozesses nötig, um auf veränderte Lizenzbedingungen reaktionsschnell vorbereitet zu sein. Die Folge ist ein erneuter kosten- und zeitintensiver Aufwand für das DevOps-Team und das ganze Unternehmen, um alle OSS-Projekte und deren Anpassungen zu steuern und zu kontrollieren.\\\\Daraus ableitend, beschäftigt sich diese Thesis einerseits mit den Grundlagen und juristischen Herausforderungen basierend auf den Lizenzmodellen von OSS, den möglichen rechtlichen Konsequenzen und andererseits mit den prozessualen Veränderungen im Produktentwicklungsprozesses, die bei einem Einsatz von OSS, benötigt werden. Gleichzeitig wird untersucht, wie die gewonnenen Informationen automatisiert an das DevOps-Team gelangen, um diese im Softwareentwicklungsprozess reaktionsschnell zu berücksichtigen und eine größtmögliche Transparenz zu gewährleisten.  

% In diesem Abschnitt wird beschrieben, welche derzeitige Problemstellung innerhalb der DevOps Produktentwicklung bei der Integration von FOSS vorliegt, die mittels der Thesis gelöst bzw verbessert werden kann. Insbesondere wird dabei auf die derzeitige Ist-Situation eingegangen, welche Herausforderungen momentan vorliegen (bei msg und allgemein) und wie der aktuelle Stand der Wissenschaft in diesem Kontext ist. 

