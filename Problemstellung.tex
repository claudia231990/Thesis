Laut einer aktuellen Studie des Bundesverband Informationswirtschaft, Telekommunikation und neue Medien e. V. (Bitkom) wird der Einsatz von OSS in Unternehmen ab einer Größe von 2.000 Mitarbeitern, im Jahre 2019, bereits auf 86 Prozent geschätzt.(Quelle) 

Im Gegensatz zu proprietärer Software, die auf einen bestimmten Copyrightholder lizensiert sind und damit die freie Nutzung, Verbreitung und Veränderung stark beschränken, zeichnet sich OSS durch eine freie Einsicht, Nutzung, Modifikation und Distribustion des Quellcodes aus.   

In diesem Zusammenhang stellt eine Modifikation die Neuentwicklung, die Fortentwicklung oder die Kombination mit einer weiteren Software, dar. 

Aufgrund der gewonnenen wirtschaftlichen Bedeutung gehört die Etablierung von OSS in die Softwareentwicklung bereits zur gängigen IT-Unternehmenskultur.(Quelle) 

Trotz der vielen offensichtlichen Vorteilen unterliegt OSS in der Regel dem Marken- Patent- und Urheberrecht. 

Durch die entsprechenden Schutzmaßnahmen entfallen die freie und unentgeltliche Nutzbarkeit der Software, da alle Rechte, die sich mit und aus der Software ergeben zunächst nur dem Urheber zustehen.  

Dieses Prizip entspricht dem gängigen Begriffs des Copyrights, welches die Rechte am eigenen Werk schützt und als der maßgebliche Unterschied zwischen kommerzieller Software und OSS darstellt. 

Die rechtlichen Schutzmaßnahmen werden in Form einer Lizenzsierung ausgeübt, indem die Genehmigung oder die Erlaubnis zur Nutzung der OSS nur durch den Urheber eingeräumt wird.(Quelle) 

Damit soll sichergestellt werden, dass die grundsätzlichen Rechte insbesondere das geistige Eigentum des Urhebers zur Nutzung, Weitergabe oder Vervielfältigung vorrangig geschützt wird.

In diesem Rahmen haben sich verschiedene Lizenzmodelle für OSS herauskristallisiert, die sich im Bereich der Nutzungsbedingungen maßgeblich unterscheiden und wodurch eine uneingeschränkte lizenzfreie Nutzung nur bedingt möglich ist.(Quelle) 

Insbesondere ab dem Zeitpunkt der Weitergabe oder Verteilung von veränderter Software an Dritte, setzen die Beschränkungen im Rahmen eines Vervielfältigungs- und Verarbeitsrechts für den Lizenznehmer ein.

In der Praxis werden diese Beschränkungen anhand des Copyleft-Effekts bemerkbar, bei dem die Weitergabe durch den Nutzer einer OSS nur zu den ursprünglichen Lizenzbedingungen möglich ist.(Quelle)  

Abhängig von der Stärke des Copylefts, ist der Lizenznehmer nach den jeweiligen Vorgaben dazu verpflichtet, alle Modifikation, Erweiterungen und Veränderungen der ursprünglichen Software demnach frei zugänglich zu machen.(Quelle) 

Ergänzend zu dem Urheberrechtsschutz umfasst der Patentschutz, OSS als eine computerimplementierte Erfindung und darf ohne die Zustimmung des Patentinhabers nicht verwendet werden.

Bei bestehender Patentierung muss der Patentinhaber auch eine patentrechtliche Nutzungserlaubnis erteilen, die meist in einer Patentklausel innerhalb der Lizenz verankert ist. 

Demnach kann neben der Einhaltung des Lizenzmodells zum Schutz des Urhebers, auch die Überprüfung patententierter Schutzrechte ein wichtiger Gesichtspunkt sein.   

Grundsätzlich besteht für die Urheber die Möglichkeit, bei Verletzung aller Rechte innerhalb der Marken- Patent- und Urheberrecht an ihrer OSS, juristische Ansprüche geltend zu machen. 

Basierend auf der Grundlage der entsprechenden Lizenzmodelle können Unterlassungsansprüche und ein kostenintensiver Schadensersatz auf die Betreffenden zukommen.

Die Folge könnte ein Verlust zur Nutzung der entsprechenden Software oder die unverzügliche Einstellung der bearbeiteten Software ohne das entsprechende OSS als Basis, sein. 

Die Konsequenz ist ein kosten- und zeitintensiver Entwicklungsaufwand für das Unternehmen, da die fehlenden Funktionalitäten durch neue Entwicklungen kompensiert werden müssen. 

Demnach sind Unternehmen dazu gezwungen, ihre Nutzungsbrechtigungen anhand des Lizenzmodelles von angebotenen OSS zu überprüfen und die daraus resultierenden Risiken bei der Verwendung in die eigenen Produkte zu berücksichtigen.  

Insbesondere im Falle einer Weitergabe des modifizierten Programms an potentielle Kunden, muss eine mögliche Offenlegung in den Softwareentwicklungsprozess rechtlich und technisch einbezogen werden.  

Dies gestaltet sich oftmals als schwierig, da sich bei einem Update der Software Versionen im Laufe eines Entwicklungsprozeess geändert haben können, die nicht mehr der ursprünglichen verwendeten Lizenz entspricht.

Neben der Erfassung der in den Produkten verwendeten OSS, sind Unternehmen demnach dazu verpflichtet, die entsprechenden OSS Projekte kontinuierlich zu überprüfen und zu beobachten, um die Einhaltung der Schutzmaßnahmen zu gewährleisten.

In diesem Zusammenhang können Informationen in vielen Fällen nur zeitversetzt an das gesamte DevOps-Team, während eines Softwareentwicklungsprozesses, kommuniziert werden.

Um dem entgegenzuwirken, ist eine Umstrukturierung oder Anpassung des gängigen DevOps-Entwicklungsprozesses nötig, um auf veränderte Lizenzbedingungen reaktionsschnell vorbereitet zu sein.

Die Folge ist ein erneuter kosten- und zeitintensiver Aufwand für das DevOps-Team und das ganze Unternehmen, um alle OSS-Projekte und deren Anpassungen zu steuern und zu kontrollieren. 

Daraus ableitend, beschäftigt sich diese Thesis einerseits mit den juristischen Herausforderungen, Problemfeldern und rechtlichen Konsequenzen, die sich aus der Verwendung von OSS ergeben und andererseits mit den prozessualen Veränderungen im Produktentwicklungsprozesses, die bei einer Anpassung von Lizenzbedingungen, benötigt werden. 

Gleichzeitig wird untersucht, wie die gewonnenen Informationen automatisiert an das DevOps-Team gelangen, um diese im Softwareenwicklungsprozess reaktionsschnell zu berücksichtigen und eine größtmögliche Transparenz zu gewährleisten.  

% In diesem Abschnitt wird beschrieben, welche derzeitige Problemstellung innerhalb der DevOps Produktentwicklung bei der Integration von FOSS vorliegt, die mittels der Thesis gelöst bzw verbessert werden kann. Insbesondere wird dabei auf die derzeitige Ist-Situation eingegangen, welche Herausforderungen momentan vorliegen (bei msg und allgemein) und wie der aktuelle Stand der Wissenschaft in diesem Kontext ist. 
