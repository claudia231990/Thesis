Unter „Infrastructure as Code“ versteht man im Wesentlichen, dass sämtliche Konfigurationsskripte, Installationsskripte, etc. versioniert in einem Software Repository abgelegt werden. Dadurch wird der gesamte Aufbau einer Infrastrukturumgebung als Code gespeichert und kann automatisiert aufgebaut werden.  Dies ist für den schnellen Aufbau von Testumgebungen ein großer Vorteil und kann darüber hinaus auch in den Entwicklungsprozess integriert werden. Änderungen an der Infrastruktur werden wie auch bei Software-Code mit Hilfe von Versionsverwaltungssystemen registriert.Hierdurch werden die Änderungen nachvollziehbar, was die Wiederholbarkeit steigert und unter anderem hilfreich sein kann für das Beheben von Fehlern. Darüber hinaus wird auch dadurch der Gedanke der agilen Methoden in die Welt der Systemadministratoren getragen. So kann dieser Denkansatz für das Managen der Infrastruktur weitergedacht werden. 
