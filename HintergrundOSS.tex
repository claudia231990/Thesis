Obwohl der Begriff der 'Open-Source-Software' heutzutage eine steigende Präsenz in vielen Bereichen und Verwendungszwecken erlangt hat, ist dieser historisch gesehen, sehr jung. (Quelle) Zwischen den Jahren 1960 und 1970, war der geschriebene Quellcode, den Entwickler programmierten maschinell geschrieben, da die Bedienbarkeit der Geräte im Vordergrund stand und auf eine stetige Austauschbarkeit ausgerichtet war. (Quelle) Nachdem die Softwareindustrie zunehmend an Bedeutung gewann, erlangten Softwareentwicklungen die dazugehörige Anerkennung. Unternehmen versuchten die eigene Software zu schützen und gingen dazu über, auschließlich kompilierte Software zu vertreiben, um den Quellcode letztlich unzugänglich zu machen. Die Folge war der teure Kauf von Lizenzen durch den Kunden, um die Software entsprechend nutzen zu können. Da eine Rückübersetzung des Maschinencodes in einen lesbaren Programmtext nach der Kompilierung, selbst heutzutage nur mit großen Aufwand zu meistern ist, (Quelle) waren Entwickler nicht mehr in der Lage, die gekaufte Software nach ihren Vorstellungen umzuschreiben und zu individualisieren.\\\\ Das Resultat war wachsende Unzufriedenheit und ein großer Widerstand gegen dieses Vorgehen. Vorreiter dieses Widerstandes war Richard Stallman, der 1983 die 'Free Software Foundation'
(FSF) gründete. (Quelle) Das Ziel der FSF war zunächst die Förderung und Entwicklung von freier Software. In diesem Rahmen fand die Entwicklung des GNU-Projektes statt, als ein uneingeschränktes, freies und nutzbares UNIX-System. (Quelle) Für Stallman und sein Team war das Teilen von Verbesserungen am System eine Selbstverständlichkeit, da Entwickler auf diese Art vom Know-how anderer Entwickler profitieren können. Anfang der neunziger Jahre erhielt die Open-Source-Bewegung eine wirtschafliche Bedeutsamkeit, indem Linus Torvalds, den Quellcode für das Betriebssystem 'Linux' über das Internet veröffentlichte.(Quelle) Torvalds verfolgte damit das Ziel, andere Programmier zu motivieren, Linux zu verbessern, Fehler zu beheben und die modifizierte Version ebenfalls zu veröffentlichen, damit möglichst viele Programmier an der Entwicklung beteiligt sind und einen positiven Nutzen daraus ziehen können.(Quelle) Das Ergebnis war eine zunehmende Popularität und eine steigende Verbreitung von OSS.Heutzutage steht Linux in einem direkten Wettbewerb, zu anderen namenhaften Server-Betriebssystemen von namenhaften Unternehmen. (Quelle) Neben den frei verfügbaren Systemkomponeten, entstand im Rahmen des GNU-Projektes die 'GNU General Public Licence' (kurz: GPL).\\\\ Diese Softwarelizenz hatte erhebliche Auswirkungen auf die zukünftige Open-Source-Bewegung. Auch Torvalds unterstellte Linux unter der GPL-Lizensierung. Innerhalb der GPL wurde festgelegt, das jeder Nutzer freien Zugang zu dem Quellcode erhält, das Recht auf Kopieren und Weitergabe besitzt und die Freiheit, das Programm nach seinen Vorstellungen zu verändern. Allerdings kann die Verbreitung des modifizierten Programms nur unter denselben Bedingungen erfolgen, wodurch eine Änderung der Eigentumsrechte entsteht. Nach und nach kamen weitere Open-Source-Lizenzen hinzu, die sich im Umgang der Eigentumsrechte stark unterscheiden können. Der Begriff des OSS wurde schließlich im Jahre 1998 mit der Gründung der 'Open Source Initiative' (kurz: OSI) geprägt und in diesem Zuge den Begriff der 'Free Software' ersetzt. (Quelle) Die OSI strebte eine neue Richtung der Open-Source-Bewegung an, um die Zusammenarbeit mit bekannten Softwareunternehmen zu fördern. Dabei sollte eine Software nur als 'Open-Source' gelten, wenn diese durch eine von der OSI anerkannte Lizenz geschützt ist und das Ziel verfolgt werden, eine einheitliche und neutrale Definition von OSS festzulegen. (Quelle)






