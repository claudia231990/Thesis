Der Aufbau dieser Arbeit ist in fünf wesentliche Kapitel unterteilt. Zunächst wird auf den theoretischen Hintergrund von DevOps und OSS eingegangen. Der theoretische Teil von DevOps fokussiert auf die Grundlagen und die Funktionsweise. Innerhalb der OSS-Theorie wird insbesondere auf die Grundlagen, Lizenzbedingungen anhand des Copyleft-Effekts und den rechtlichen Gesichtspunkt näher eingangen. Das dritte Kapitel fokussiert sich, auf die Anpassung des Softwareentwicklungsprozesses durch den Einsatz von OSS auf Basis der gewonnenen Informationen. Darüber hinaus wird eine manuelle Checkliste, für den präventiven Gebrauch vorgestellt. In Kapitel vier wird die Realisierbarkeit und Machbarkeit der erhobenen Konzepte anhand eines Proof of Concept dargestellt. In Kapitel fünf folgt eine Schlussfolgerung und ein entsprechender Ausblick. 