Durch die intensive Interaktion der Bereiche Entwicklung und Betrieb müssen unterschiedliche organisatorische und kulturelle Veränderungen durchgeführt werden, damit die DevOps-Kultur etabliert werden kann. 

In diesem Rahmen baut DevOps auf den Hauptprinzipien, als das Model der \textit{'The Three Ways'} auf, die von dem Autor Gene Kim \cite{kim_devops-handbuch_2017} definiert wurden. 

Dieses Model der drei Wege bilden die zugrunde liegenden Prinzipien von DevOps ab, indem das Verhalten und die Muster von DevOps näher beschrieben werden. \cite[S. 9 - 11]{kim_devops-handbuch_2017}, \cite{kim_three_2012}  

Der erste Weg bildet die Grundlage für DevOps ab und betont die Leistung des gesamten Systems, im Gegensatz zur Leistung eines besitmmten Teams oder einzelner Abteilungen. 

Der Fokus liegt auf einem schnellen Arbeitsfluss, der durch die IT ermöglicht wird. 

Den Anfang stellt die Entwicklung dar, über die Operations bis hin zum Kunden.

Dabei wird das Produkt, basierend auf den identifizierten Anforderungen, von der Entwicklung erstellt und in den Betrieb übergeben, wo dies dem Kunden ausgeliefert wird. 

Während dieses Schrittes sind alle Arbeiten sichtbar und in kleinen Aufgaben aufgeteilt, die in bestimmten Intervallen ausgeführt werden. 

In diesem Rahmen wird versucht, mehrere Ziele zu erreichen. 

Zunächst sollte das Verständnis des Arbeitssflusses sichergestellt werden. 

Unabhängig um welche Arbeit es sich handelt, sollte diese stillstehen oder zu einem negativene Ergebnis führen, ist dies fast immer ein Hinweis auf Probleme, die gelöst werden müssen. 




Zu den Ergebnissen der Umsetzung des Ersten Weges in der Praxis gehört, dass niemals ein bekannter Fehler an nachgelagerte Arbeitsplätze weitergegeben wird, dass eine lokale Optimierung niemals zu einer globalen Verschlechterung führt, dass immer versucht wird, den Fluss zu erhöhen, und dass immer versucht wird, ein tiefes Verständnis des Systems zu erlangen (im Sinne von Deming).


%Durch die Interaktion der beiden Bereiche Development und Operations gehen auch organisatorische Änderungen mit einher, damit eine umfassende DevOps-Kultur geschaffen werden kann. In diesem Abschnitt wird insbesondere auf die Ziele dieser Kultur eingegangen, damit die Zusammenarbeit zwischen den beiden Bereichen gelingt. In diesem Rahmen wird unter anderem auf die wesentlichen Aspekte/Werte wie Kontniuierliches Lernen, Experimentieren, Ingenieurskultur, Kultur der Effektivität, Produktdenken oder die Übernahme von Verantwortung eingegangen. Auch wird der Vergleich und die Nachteile des traditionellen Silodenkens aufgegriffen. 