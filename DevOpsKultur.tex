Durch die intensive Interaktion der Bereiche Entwicklung und Betrieb müssen unterschiedliche organisatorische und kulturelle Veränderungen durchgeführt werden, damit die DevOps-Kultur etabliert werden kann. 

In diesem Abschnitt wird zunächst auf die wesentlichen Bestandteile und die Ziele der Kultur eingegangen, damit die Zusammenarbeit zwischen den beiden Bereichen gelingt.

In diesem Rahmen baut DevOps auf den Hauptprinzipien, als das Model der \textit{'The Three Ways'} auf, die von dem Autor und Erfinder des DevOps-Ansatzes Gene Kim \cite{kim_devops-handbuch_2017} definiert wurden. 

Dieses Model der drei Wege bilden die zugrunde liegenden Prinzipien von DevOps ab, indem das Verhalten und die Muster von DevOps näher beschrieben werden. \cite[S. 9 - 11]{kim_devops-handbuch_2017}, \cite{kim_three_2012}  

Zunächst bildet der erste Weg die Grundlage für DevOps ab und betont die Leistung des gesamten Systems, im Gegensatz zur Leistung eines einzelnen Teams, Silos oder Abteilungen. 

Der Fokus liegt hierbei auf einem schnellen Arbeitsfluss des gesamten Systems, der durch die IT ermöglicht wird. 

Den Anfang stellt der Kunde dar, über die Entwicklung bis hin zu den Operations.

Dabei wird das Produkt, basierend auf den identifizierten Anforderungen, von der Entwicklung erstellt und in den Betrieb übergeben, wo dies dem Kunden ausgeliefert wird. 

Während dieses Schrittes sind alle Arbeiten sichtbar und in kleinen Aufgaben aufgeteilt, die in bestimmten Intervallen ausgeführt werden. 

Zu den Zielen des Ersten Weges gehört, dass bekannte Fehler an nachfolgende Arbeitsplätze nicht weitergegeben werden, dass eine lokale Optimierung niemals zu einer globalen Verschlechterung führt und dass versucht wird, ein tiefes Verständnis des gesamten Systems zu erlangen. 

\begin{figure}[h]
    \centering
    \includegraphics[scale=0.6]{Bilder/First Way.png}
    \caption{Der erste Weg: Systemdenken \cite{kim_three_2012}}
\end{figure}

Der zweite Weg beschreibt das Gestalten von effizienten Feedbackschleifen, um Fehler schnell und frühzeitig zu erkennen und zu beheben.

Dies sollte sowohl auf Seiten von einzelnen Teams untereinander als auch zwischen Entwicklung und Betrieb eingebettet werden.

Damit soll verhindert werden, dass Probleme kein zweites Mal auftreten, sich Kontniuierlich aufbauen können und sich möglicherweise auf neue Aufgaben auswirken. 

Zudem können neue oder geänderte Kundenanforderungen mittels schneller Feedbackschleifen an die Entwicklung weitergegeben werden, um möglichst zeitnah die geeigneten Maßnahmen treffen zu können.

Entscheidend ist, dass das Hinzufügen weiterer Kontrollschritte und schwerfällige Genehmigungsprozesse innerhalb großer Arbeitssyssteme die Wahrscheinlichkeit für zukünftige Fehler maßgeblich erhöht. 

Daher müssen DevOps-Teams die Freiheit haben, schnell Fehler zu beseitigen und ihre Arbeitsweise ohne organisatorische Hürden durchführen zu können.   

Die Ergebnisse des zweiten Wegs sind einerseits die Sicherstellung einer verbesserten Qualität und die Sicherheit des Arbeitssystems und andererseits die Möglichkeit des Aufbaus neuen Wissens.

\begin{figure}[h]
    \centering
    \includegraphics[scale=0.6]{Bilder/Second Way.png}
    \caption{Der zweite Weg: Feedback-Schleifen verstärken \cite{kim_three_2012}}
\end{figure}

Während sich der erste Weg technische und der zweite Weg organisatorische Aspekte beinhaltet, zielt der dritte Weg auf die Etablierung kultureller Änderungen ab. 

Dieser beschreibt die Schaffung einer Kultur die einen dynamischen Ansatz zum kontniuierlichen Experimentieren und firmenweites Lernen ermöglicht. 

\begin{figure}[h]
    \centering
    \includegraphics[scale=0.6]{Bilder/Third Way.png}
    \caption{Der dritte Weg: Kultur des kontniuierlichen Experimentierens und Lernens \cite{kim_three_2012}}
\end{figure}

Mittels dieses Ansatzes kann tiefer in die Materie und damit Risiken eingegangen werden, um schwierige oder versteckte Probleme identifizieren zu können. 

Imfolgedessen können die Fähigkeiten des DevOps-Teams insgesamt und jedes Teammitgliedes maßgeblich verbessert werden.

Für eine erfolgreiche Durchführung des dritten Weges, sollte die etablierte Kultur auf Vertrauen beruhen. 

Je höher der Grad des Vertrauens jedes Mitarbeiters, desto wahrscheinlicher und schneller ist die Weitergabe von gewonnenen Erkenntnissen.

Gemäß der DevOps-Kultur entfällt insbesondere die Bewertung von entdeckten Fehlern, sondern die Möglichkeit von diesen zu lernen und sich zu verbessern, rückt in den Vordergrund. 

Dieser Ansatz steigert die Motivitation jedes Mitgliedes eines DevOps-Teams und fördert das Experimentieren und daraus ableitende Eingehen eines Risikos, wodurch ein globales Lernen ermöglicht wird. 

Zu den Ergebnissen des Dritten Weges gehören die Einplanung von Zeit für die Verbesserung der täglichen Abläufe, die Schaffung von Routinen, die das Team für das Eingehen von Risiken honorieren und die Einbeziehung von Fehlern in das System, um die Belastbarkeit zu erhöhen.


%Durch die Interaktion der beiden Bereiche Development und Operations gehen auch organisatorische Änderungen mit einher, damit eine umfassende DevOps-Kultur geschaffen werden kann. In diesem Abschnitt wird insbesondere auf die Ziele dieser Kultur eingegangen, damit die Zusammenarbeit zwischen den beiden Bereichen gelingt. In diesem Rahmen wird unter anderem auf die wesentlichen Aspekte/Werte wie Kontniuierliches Lernen, Experimentieren, Ingenieurskultur, Kultur der Effektivität, Produktdenken oder die Übernahme von Verantwortung eingegangen. Auch wird der Vergleich und die Nachteile des traditionellen Silodenkens aufgegriffen. 