Unter diesen Gesichtspunkten lassen sich fünf Forschungsfragen für diese Arbeit wie folgt ableiten:\\ 

1. Welche praxisrelevanten Informationen können aus OSS extrahiert und verwendet werden?

%(Lizenzmodelle)
% was soll die Zielgruppe dieser Frage sein? --> Mehrheitlich die Devops-Umgebung (dh. Entwickler, Softwaredienstleister)
% zunächst grundlegende Informationen aufzuzeigen, danach den praxisrelevanten Kontext beachten (Überblick schaffen)


2. Welche Einschränkungen sind bei der Verwendung von OSS in Hinblick auf die verschiedenen Lizenzinformationen zu berücksichtigen? 

%(Haftungs-/Garantiebedingungen, Autorenhinweise, Copyleft-Bedinungen)
%Anhand von praxisrelevaten Lizemzmodellen, werden ausschließlich diese mit Einschränkungen versehen, die anderen eher erwähnen. 
%Praxisrelevanz wird anhand des beschriebenen Produktes bei msg festgelegt


3. Welche prozessualen Anpassungen müssen anhand der gewonnenen Erkenntnisse getroffen werden, um auf mögliche Veränderungen in den Lizenzinformationen reaktionsschnell vorbereitet zu sein? 

% GEnerell HAF-Projekt verwenden, kein exaktes Produkt da der Entwicklungspriozess gleich abläuft. 
% Ist-zustand muss exakt dargestellt werden, (Manuelle Aktualsierung aller FOSS-BEdinugngen und Deklarationen erforderlich) + gängiger Prozess muss designt werden!
% Ist-Zustand: OSS wird genommen und modifiziert und im Rahmen der Unternehmung ausgeliefert 
% --> Daraus ergibt sich der Soll- Zustand, indem die Einschränkungen basierend auf Frage 2, zum Tragen kommen und die Situationen aufzeigen, in dem sich der Ist-Zustand verändern muss --> Veränderungen am Prozessmodell angepasst 
% Welche Stakeholder sind beteiligt, wer hat welche Verantwortung/Pflichten/Aufgaben --> welche Informationen müssen weitergegeben werden
%Hieraus ergeben sich je nach beteiligten Stakeholder, unterschiedliche Verantwortungen, Pflichten und Aufgabengebiete.





%% (evtl.) Proof of Concept
%Prozess wird dargestellt --> Wie wird dieser durchgeführt/Wie wird erreicht?


4. Wie können relevante Lizenzinformationen und haftungsbedingte Anforderungen anhand eines Produktes automatisiert aus den verwendeten Lizenzmodellen aus OSS herausgearbeitet werden? 

%(Programm, Makros ???)\\
%Soll-Prozess wird implentiert --> Konkrektes Beispiel
%Automasierungsgrad muss hoch sein, in Sinne von Devops


5. Wie kann die Transparenz der gesammelten Informationen insgesamt sichergestellt werden? 

%(Wie kommen die Informationen schnell an das Devops-Team)\\ 
%Transparenz/Kommunikation muss gewährleistet werden --> Änderungen des Lizenzkonstellation muss tranzparent gestaltet werden (zb. Dashboards, Alters, REports)
%Dokumentation  

Ziel dieser Arbeit ist es, zunächst Informationen im Bereich DevOps und OSS zu gewinnen und dabei insbesondere die Einschränkungen der unterschiedlichen Lizenzmodelle zu beschreiben. Hierzu wird insbesondere auf die juristischen Regelungen der OS-Linzenzen und die rechtlichen Konsequenzen bei einem Nichteinhalten der gesetzlichen Schutzmaßnahmen, die sich für ein Unternehmen ergeben kann, näher eingegangen. In diesem Zuge werden prozessuale Anpassungen im DevOps-Prozess vorgenommen, die sich einerseits durch die Auslieferung des Softwareproduktes nach der Modifikation von OSS oder anderseits durch Änderungen von Lizenzbestimmungen während des Softwareentwicklungsprozesses ergeben. Ausgehend von den prozessualen Veränderung soll innerhalb der praxisorientierten Umsetzung, eine Möglichkeit dargestellt werden, wie die gewonnenen Ergebnisse automatisiert herausgearbeitet werden und der Informationsfluss innerhalb des ganzen DevOps-Teams sichergestellt werden kann.
 