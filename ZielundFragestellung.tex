In diesem Abschnitt wird das konkrete Ziel der Thesis festgelegt und die Forschungsfragen, an denen sich die Thesis ausrichtet und zu beantworten. Die Forschungsfragen wurden zunächst folgendermaßen festgelegt: \\


1. Welche relevanten Lizenzinformationen können aus FOSS automatisiert extrahiert und verwendet werden? (Lizenzmodelle)\\


2. Welche Einschränkungen sind bei der Verwendung von FOSS in Hinblick auf die verschiedenen Lizenzinformationen zu berücksichtigen? (Haftungs-/Garantiebedingungen, Autorenhinweise, Copyleft-Bedinungen)\\


3. Welche prozessualen Anpassungen müssen anhand der gewonnenen Erkenntnisse getroffen werden, um auf mögliche Veränderungen in den Lizenzinformationen reaktionsschnell vorbereitet zu sein?\\ 


4. Wie können relevante Lizenzinformationen und haftungsbedingte Anforderungen anhand eines Produktes automatisch aus den verwendeten Lizenzmodellen aus FOSS herausgearbeitet werden? (Programm, Makros ???)\\


5. Wie kann die Transparenz der gesammelten Informationen insgesamt sichergestellt werden? (Wie kommen die Informationen schnell an das Devops-Team)\\ 
