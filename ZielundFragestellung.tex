In diesem Abschnitt wird das konkrete Ziel der Thesis festgelegt und die ForsIhungsfragen, an denen sich die Thesis ausrichtet und zu beantworten. Die Forschungsfragen wurden zunächst folgendermaßen festgelegt: \\


1. Welche praxisrelevanten Informationen können aus FOSS extrahiert und verwendet werden? (Lizenzmodelle)\\

% was soll die Zielgruppe dieser Frage sein? --> Mehrheitlich die Devops-Umgebung (dh. Entwickler, Softwaredienstleister)
% zunächst grundlegende Informationen aufzuzeigen, danach den praxisrelevanten Kontext beachten (Überblick schaffen)


2. Welche Einschränkungen sind bei der Verwendung von FOSS in Hinblick auf die verschiedenen Lizenzinformationen zu berücksichtigen? (Haftungs-/Garantiebedingungen, Autorenhinweise, Copyleft-Bedinungen)\\

%Anhand von praxisrelevaten Lizemzmodellen, werden ausschließlich diese mit Einschränkungen versehen, die anderen eher erwähnen. 
%Praxisrelevanz wird anhand des beschriebenen Produktes bei msg festgelegt


3. Welche prozessualen Anpassungen müssen anhand der gewonnenen Erkenntnisse getroffen werden, um auf mögliche Veränderungen in den Lizenzinformationen reaktionsschnell vorbereitet zu sein?\\ 

% GEnerell HAF-Projekt verwenden, kein exaktes Produkt da der Entwicklungspriozess gleich abläuft. 
% Ist-zustand muss exakt dargestellt werden, (Manuelle Aktualsierung aller FOSS-BEdinugngen und Deklarationen erforderlich) + gängiger Prozess muss designt werden!
% Ist-Zustand: OSS wird genommen und modifiziert und im Rahmen der Unternehmung ausgeliefert 
% --> Daraus ergibt sich der Soll- Zustand, indem die Einschränkungen basierend auf Frage 2, zum Tragen kommen und die Situationen aufzeigen, in dem sich der Ist-Zustand verändern muss --> Veränderungen am Prozessmodell angepasst 
% Welche Stakeholder sind beteiligt, wer hat welche Verantwortung/Pflichten/Aufgaben --> welche Informationen müssen weitergegeben werden





%% (evtl.) Proof of Concept
%Prozess wird dargestellt --> Wie wird dieser durchgeführt/Wie wird erreicht?


4. Wie können relevante Lizenzinformationen und haftungsbedingte Anforderungen anhand eines Produktes automatisiert aus den verwendeten Lizenzmodellen aus FOSS herausgearbeitet werden? (Programm, Makros ???)\\

%Soll-Prozess wird implentiert --> Konkrektes Beispiel
%Automasierungsgrad muss hoch sein, in Sinne von Devops


5. Wie kann die Transparenz der gesammelten Informationen insgesamt sichergestellt werden? (Wie kommen die Informationen schnell an das Devops-Team)\\ 

%Transparenz/Kommunikation muss gewährleistet werden --> Änderungen des Lizenzkonstellation muss tranzparent gestaltet werden (zb. Dashboards, Alters, REports)
%Dokumentation  