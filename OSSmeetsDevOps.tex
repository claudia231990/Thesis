Wie bereits beschrieben, ist ein wesentliches Merkmal von Software die Wiederverwendung von Komponenten. 

Die Folge der massiven Wiederverwendung insbesondere durch OSS-Kompo

Die Verschachtelung hat in den letzten Jahren durch die massive Wiederverwendung von OSS-Komponenten zugenommen. 

Die Entwicklung eines Softwareprodukts ohne den Einsatz von OSS-Software ist heutzutage fast unmöglich. 

Wir befinden uns in einer neuen Software-Ära, in der es ohne den Einsatz von OSS-Komponenten nicht mehr möglich ist, die geforderte Time-to-Market zu erreichen und somit nicht mehr wettbewerbsfähig zu sein. 

Die starke Verschachtelung von OSS-Komponenten durch die vielen transitiven Abhängigkeiten erhöht die Komplexität erheblich. 

Jede dieser Komponenten hat eine oder mehrere Lizenzen, die die Nutzungsbedingungen der jeweiligen Komponente spezifizieren. Somit kann die Arbeit auch dem Bereich des Lizenzrechts zugeordnet werden.

In kleinen Unternehmen (50-100 Mitarbeiter, ein Standort) lassen sich diese Herausforderungen noch durch persönliche Kommunikation bewältigen. 

Mit zunehmender Anzahl von Mitarbeitern und verteilten Standorten wird die Bewältigung dieser Herausforderungen jedoch immer komplizierter. 

Daher ist eine zentrale Unternehmensinformationsbasis ratsam, um die Lösung der genannten Herausforderungen in der Praxis zu unterstützen.

Die Verantwortung für die Einhaltung der Lizenzbedingungen liegt bei den Nutzern selbst.



Nachdem der theoretische Rahmen und die Funktionsweise beider Bereiche beschrieben worden ist, wird in diesem Kapitel näher auf die Integration der OSS innerhalb des Prozesses der DevOps-Produktentwicklung, eingegangen.