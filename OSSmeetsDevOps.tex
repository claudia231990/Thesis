Nachdem der theoretische Rahmen und die Funktionsweise beider Bereiche beschrieben worden ist, wird in diesem Kapitel näher auf die Integration der OSS innerhalb des Prozesses der DevOps-Produktentwicklung, eingegangen. In den letzten Jahren hat sich die Nutzung der OSS, durch eine große Anzahl an Individualisierungsmöglichkeiten als ein Standard in der Softwareentwicklung innerhalb Unternehmen etabliert. Entwickler sind in der Lage je nach Projekt OSS mit wenig Ressourcen anpassen, zusätzliche Funktionen einbauen oder fehlerhafte Teile entfernen können. DevOps-Teams profitieren von dieser Vorgehensweise, da sich das Experimentieren mit OSS, wesentlicher einfacher gestaltet als mit kommerziellen Produkten. Gleichwohl der Vorteile die eine Einbindung von OSS befürwortet, fehlt in den meisten Unternehmen oftmals eine Strategie für die Nutzung von OSS, da der Einsatz mit wenig Hürden verbunden ist und die Konsequenzen von den damit verbundenen Lizenzmodellen nicht jedem bewusst sind.Die Folgen einer Verletzung der Nutzungsrechte Dritter an der OSS können über kostenintensive Abmahnungen, Unterlassungs- und Schadenersatzklagen oder kostspielige Patentrechtstretigkeiten führen. \textit{"Beliebtheit und zunehmende Verwendung stehen nach wie vor in Kontrast zu einer grundsätzlichen Unkenntnis: So selbstverständlich die Verwendung ist, so verbreitet sind oft auch Wissenslücken über Grundanforderungen im Umgang mit OSS, sei es als Basis für einen erfolgreichen Einsatz im Alltag oder als Voraussetzung für die Verwendung für eigene Softwareprojekte."} \cite{bitkom_ev_open_2016} In diesem Rahmen sollten sowohl manuelle als auch automatisierte Prüfungen der OSS-Lizenzen innerhalb des Softwareentwicklungsprozesses integriert werden, um mögliche rechtliche Schwierigkeiten und im schlimmsten Fall einen Rückgang in der Entwicklung zu vermeiden. In dem folgenden Kapitel wird zunächst die Einbindung eines OSS-Checks in den Entwicklungsprozess des DevOps-Teams näher beschrieben und der Umfang anhand festgelegter Szenarien innerhalb des Projekts bei der msg systems ag festgelegt. Die manuelle Überprüfung basiert auf einer Checkliste, die ebenfalls in diesem Kapitel näher erläutert wird. 



% Ferner können die Reduzierung der Gesamtkosten eines Unternehmens als ein essentieller Grund für die Verwendung von OSS-Komponeten angesehen werden. 

% Im Gegensatz zu proprietärer Software wird OSS meist kostenlos heruntergeladen und verlagert die Kostenstelle Betriebskosten im Bereich der Lizenzierung auf die Implementierung im Rahmen der Investitionskosten. \cite{augsten_10_2019} 

% Zwar fallen in der Regel auch mit Verwendung von OSS zusätzliche Kosten für Schulungen, Wartung und Support an, die als \emph{sunk costs} bezeichnet werden, wobei diese oft nicht ins Gewicht fallen, da Unternehmen diese Kostenpunkte unabhängig davon tragen, ob es sich um OSS oder Closed Software handelt. \cite{augsten_10_2019} 






















%In diesem Rahmen wird insbesondere auf drei Szenarien, die die Hauptanwendungsbereiche, der msg systems ag darstellen, näher beschrieben. 